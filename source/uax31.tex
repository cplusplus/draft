%!TEX root = std.tex
\infannex{uaxid}{Conformance with UAX \#31}

\subclause1[uaxid.general]{General}

\pnum
This Annex describes the choices made in application of
UAX \#31 (``Unicode Identifier and Pattern Syntax'')
to \Cpp{} in terms of the requirements from UAX \#31 and
how they do or do not apply to \Cpp{}.
In terms of UAX \#31,
\Cpp{} conforms by meeting the requirements
R1 ``Default Identifiers'' and
R4 ``Equivalent Normalized Identifiers''.
The other requirements, also listed below,
are either alternatives not taken or do not apply to \Cpp{}.

\subclause1[uaxid.def]{R1 Default identifiers}

\subclause2[uaxid.def.general]{General}

\pnum
UAX \#31 specifies a default syntax for identifiers
based on properties from the Unicode Character Database, UAX \#44.
The general syntax is
\begin{codeblock}
<Identifier> := <Start> <Continue>* (<Medial> <Continue>+)*
\end{codeblock}
where \tcode{<Start>} has the XID_Start property,
\tcode{<Continue>} has the XID_Continue property, and
\tcode{<Medial>} is a list of characters permitted between continue characters.
For \Cpp{} we add the character U+005F, LOW LINE, or \tcode{_},
to the set of permitted \tcode{<Start>} characters,
the \tcode{<Medial>} set is empty, and
the \tcode{<Continue>} characters are unmodified.
In the grammar used in UAX \#31, this is
\begin{codeblock}
<Identifier> := <Start> <Continue>*
<Start> := XID_Start + U+005F
<Continue> := <Start> + XID_Continue
\end{codeblock}

\pnum
This is described in the \Cpp{} grammar in \ref{lex.name},
where \grammarterm{identifier} is formed from
\grammarterm{identifier-start} or
\grammarterm{identifier} followed by \grammarterm{identifier-continue}.

\subclause2[uaxid.def.rfmt]{R1a Restricted format characters}

\pnum
If an implementation of UAX \#31 wishes to allow format characters
such as ZERO WIDTH JOINER or ZERO WIDTH NON-JOINER
it must define a profile allowing them, or
describe precisely which combinations are permitted.

\pnum
\Cpp{} does not allow format characters in identifiers, so this does not apply.

\subclause2[uaxid.def.stable]{R1b Stable identifiers}

\pnum
An implementation of UAX \#31 may choose to guarantee
that identifiers are stable across versions of the Unicode Standard.
Once a string qualifies as an identifier it does so in all future versions.

\pnum
\Cpp{} does not make this guarantee,
except to the extent that UAX \#31 guarantees
the stability of the XID_Start and XID_Continue properties.

\subclause1[uaxid.immutable]{R2 Immutable identifiers}

\pnum
An implementation may choose to guarantee that
the set of identifiers will never change
by fixing the set of code points allowed in identifiers forever.

\pnum
\Cpp{} does not choose to make this guarantee.
As scripts are added to Unicode,
additional characters in those scripts may become available
for use in identifiers.

\subclause1[uaxid.pattern]{R3 Pattern_White_Space and Pattern_Syntax characters}

\pnum
UAX \#31 describes how languages that use or interpret patterns of characters,
such as regular expressions or number formats,
may describe that syntax with Unicode properties.

\pnum
\Cpp{} does not do this as part of the language,
deferring to library components for such usage of patterns.
This requirement does not apply to \Cpp{}.

\subclause1[uaxid.eqn]{R4 Equivalent normalized identifiers}

\pnum
UAX \#31 requires that implementations describe
how identifiers are compared and considered equivalent.

\pnum
\Cpp{} requires that identifiers be in Normalization Form C and
therefore identifiers that compare the same under NFC are equivalent.
This is described in \ref{lex.name}.

\subclause1[uaxid.eqci]{R5 Equivalent case-insensitive identifiers}

\pnum
\Cpp{} considers case to be significant in identifier comparison, and
does not do any case folding.
This requirement does not apply to \Cpp{}.

\subclause1[uaxid.filter]{R6 Filtered normalized identifiers}

\pnum
If any characters are excluded from normalization,
UAX \#31 requires a precise specification of those exclusions.

\pnum
\Cpp{} does not make any such exclusions.

\subclause1[uaxid.filterci]{R7 Filtered case-insensitive identifiers}

\pnum
\Cpp{} identifiers are case sensitive, and
therefore this requirement does not apply.

\subclause1[uaxid.hashtag]{R8 Hashtag identifiers}

\pnum
There are no hashtags in \Cpp{}, so this requirement does not apply.
