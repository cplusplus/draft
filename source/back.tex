%!TEX root = std.tex

\chapter{Bibliography}

\begin{itemize}
\renewcommand{\labelitemi}{---}
% ISO documents in numerical order.
\item
  ISO 4217:2015,
  \doccite{Codes for the representation of currencies}
\item
  ISO/IEC 10967-1:2012,
  \doccite{Information technology --- Language independent arithmetic ---
    Part 1: Integer and floating point arithmetic}
\item
  ISO/IEC TS 18661-3:2015,
  \doccite{Information Technology ---
    Programming languages, their environments, and system software interfaces ---
    Floating-point extensions for C --- Part 3: Interchange and extended types}
% Other international standards.
\item
  IANA Time Zone Database.
  Available from: \url{https://www.iana.org/time-zones}
% Literature references.
\item
  Bjarne Stroustrup,
  \doccite{The \Cpp{} Programming Language, second edition}, Chapter R.
  Addison-Wesley Publishing Company, ISBN 0-201-53992-6, copyright \copyright 1991 AT\&T
\item
  Brian W. Kernighan and Dennis M. Ritchie,
  \doccite{The C Programming Language}, Appendix A.
  Prentice-Hall, 1978, ISBN 0-13-110163-3, copyright \copyright 1978 AT\&T
\item
  P.J. Plauger,
  \doccite{The Draft Standard \Cpp{} Library}.
  Prentice-Hall, ISBN 0-13-117003-1, copyright \copyright 1995 P.J. Plauger
\end{itemize}

The arithmetic specification described in ISO/IEC 10967-1:2012 is
called \defn{LIA-1} in this document.

% FIXME: For unknown reasons, hanging paragraphs are not indented within our
% glossaries by default.
\let\realglossitem\glossitem
\renewcommand{\glossitem}[4]{\hangpara{4em}{1}\realglossitem{#1}{#2}{#3}{#4}}

\clearpage
\renewcommand{\glossaryname}{Cross references}
\renewcommand{\preglossaryhook}{Each clause and subclause label is listed below along with the
corresponding clause or subclause number and page number, in alphabetical order by label.\\}
\twocolglossary
\renewcommand{\leftmark}{\glossaryname}
{
\raggedright
\printglossary[xrefindex]
}

\clearpage
%!TEX root = std.tex

\newcommand{\secref}[1]{\hyperref[\indexescape{#1}]{\indexescape{#1}}}

% Turn off page numbers for this glossary, they're not useful.
\newcommand{\swallow}[1]{}
\changeglossnumformat[xrefdelta]{|swallow}

\newcommand{\oldxref}[2]{\glossary[xrefdelta]{\indexescape{#1}}{#2}}
\newcommand{\removedxref}[1]{\oldxref{#1}{\textit{removed}}}
\newcommand{\movedxrefs}[2]{\oldxref{#1}{\textit{see} #2}}
\newcommand{\movedxref}[2]{\movedxrefs{#1}{\secref{#2}}}
\newcommand{\movedxrefii}[3]{\movedxrefs{#1}{\secref{#2}, \secref{#3}}}
\newcommand{\movedxrefiii}[4]{\movedxrefs{#1}{\secref{#2}, \secref{#3}, \secref{#4}}}
\newcommand{\deprxref}[1]{\oldxref{#1}{\textit{see} \secref{depr.#1}}}

% Removed features.
%\removedxref{removed.label}
\removedxref{variant.traits}

% Renamed sections.
%\movedxref{old.label}{new.label}
%\movedxrefii{old.label}{new.label.1}{new.label.2}
%\movedxrefiii{old.label}{new.label.1}{new.label.2}{new.label.3}
%\movedxrefs{old.label}{new place (eg Table~\ref{tab:blah})}

% Deprecated features.
%\deprxref{old.label}  (if moved to depr.old.label, otherwise use \movedxref)

\renewcommand{\glossaryname}{Cross references from ISO \CppXX{}}
\renewcommand{\preglossaryhook}{All clause and subclause labels from
ISO \CppXX{} (ISO/IEC 14882:2020, \doccite{Programming Languages --- \Cpp{}})
are present in this document, with the exceptions described below.\\}
\renewcommand{\leftmark}{\glossaryname}
{
\raggedright
\printglossary[xrefdelta]
}

\clearpage
\renewcommand{\leftmark}{\indexname}
\renewcommand{\preindexhook}{Constructions whose name appears in \exposid{monospaced italics} are for exposition only.\\}
{
\raggedright
\printindex[generalindex]
}

\clearpage
\renewcommand{\preindexhook}{The first bold page number for each entry is the page in the
general text where the grammar production is defined. The second bold page number is the
corresponding page in the Grammar summary\iref{gram}. Other page numbers refer to pages where the grammar production is mentioned in the general text.\\}
{
\raggedright
\printindex[grammarindex]
}

\clearpage
\renewcommand{\preindexhook}{The bold page number for each entry refers to
the page where the synopsis of the header is shown.\\}
{
\raggedright
\printindex[headerindex]
}

\clearpage
\renewcommand{\preindexhook}{Constructions whose name appears in \exposid{italics} are for exposition only.\\}
{
\raggedright
\printindex[libraryindex]
}

\clearpage
\renewcommand{\preindexhook}{The bold page number for each entry is the page
where the concept is defined.
Other page numbers refer to pages where the concept is mentioned in the general text.
Concepts whose name appears in \exposid{italics} are for exposition only.\\}
{
\raggedright
\printindex[conceptindex]
}

\clearpage
\renewcommand{\preindexhook}{The entries in this index are rough descriptions; exact
specifications are at the indicated page in the general text.\\}
\renewcommand{\leftmark}{Index of impl.-def. behavior}
{
\raggedright
\printindex[impldefindex]
}
