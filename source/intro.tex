%!TEX root = std.tex

\rSec0[intro.scope]{Scope}

\pnum
\indextext{scope|(}%
This document specifies requirements for implementations
of the \Cpp programming language. The first such requirement is that
they implement the language, so this document also
defines \Cpp. Other requirements and relaxations of the first
requirement appear at various places within this document.

\pnum
\Cpp is a general purpose programming language based on the C
programming language as described in ISO/IEC 9899:2011
\doccite{Programming languages --- C} (hereinafter referred to as the
\defnx{C standard}{C!standard}). In addition to
the facilities provided by C, \Cpp provides additional data types,
classes, templates, exceptions, namespaces, operator
overloading, function name overloading, references, free store
management operators, and additional library facilities.%
\indextext{scope|)}

\indextext{normative references|see{references, normative}}%
\rSec0[intro.refs]{Normative references}

\pnum
\indextext{references!normative|(}%
The following documents are referred to in the text
in such a way that some or all of their content
constitutes requirements
of this document. For dated references, only the edition cited applies.
For undated references, the latest edition of the referenced document
(including any amendments) applies.
\begin{itemize}
\item Ecma International, \doccite{ECMAScript Language Specification},
Standard Ecma-262, third edition, 1999.
\item ISO/IEC 2382 (all parts), \doccite{Information technology ---
Vocabulary}
\item ISO/IEC 9899:2011, \doccite{Programming languages --- C}
\item ISO/IEC 9899:2011/Cor.1:2012(E), \doccite{Programming languages --- C,
Technical Corrigendum 1}
\item ISO/IEC 9945:2003, \doccite{Information Technology --- Portable
Operating System Interface (POSIX)}
\item ISO/IEC 10646-1:1993, \doccite{Information technology ---
Universal Multiple-Octet Coded Character Set (UCS) --- Part 1:
Architecture and Basic Multilingual Plane}
\item ISO/IEC/IEEE 60559:2011, \doccite{Information technology ---
Microprocessor Systems --- Floating-Point arithmetic}
\item ISO 80000-2:2009, \doccite{Quantities and units ---
Part 2: Mathematical signs and symbols
to be used in the natural sciences and technology}
\end{itemize}

\pnum
The library described in Clause 7 of ISO/IEC 9899:2011
is hereinafter called the
\defnx{C standard library}{C!standard library}.%
\footnote{With the qualifications noted in \ref{\firstlibchapter}
through \ref{\lastlibchapter} and in \ref{diff.library}, the C standard
library is a subset of the \Cpp standard library.}

\pnum
The operating system interface described in ISO/IEC 9945:2003 is
hereinafter called \defn{POSIX}.

\pnum
The ECMAScript Language Specification described in Standard Ecma-262 is
hereinafter called \defn{ECMA-262}.
\indextext{references!normative|)}

\rSec0[intro.defs]{Terms and definitions}

\pnum
\indextext{definitions|(}%
For the purposes of this document,
the terms and definitions
given in ISO/IEC 2382-1:1993,
the terms, definitions, and symbols
given in ISO 80000-2:2009,
and the following apply.

\pnum
ISO and IEC maintain terminological databases
for use in standardization
at the following addresses:
\begin{itemize}
\item IEC Electropedia: available at \url{http://www.electropedia.org/}
\item ISO Online browsing platform: available at \url{http://www.iso.org/obp}
\end{itemize}

\pnum
\ref{definitions}
defines additional terms that are used only in \ref{library}
through \ref{\lastlibchapter} and \ref{depr}.

\pnum
Terms that are used only in a small portion of this document
are defined where they are used and italicized where they are
defined.

\def\definition{\definitionx{\section}}%

\indexdefn{access}%
\definition{access}{defns.access}
\defncontext{execution-time action} to read or modify the value of an object

\indexdefn{argument}%
\indexdefn{argument!function call expression}
\definition{argument}{defns.argument}
\defncontext{function call expression} expression in the
comma-separated list bounded by the parentheses\iref{expr.call}

\indexdefn{argument}%
\indexdefn{argument!function-like macro}%
\definition{argument}{defns.argument.macro}
\defncontext{function-like macro} sequence of preprocessing tokens in the
comma-separated list bounded by the parentheses\iref{cpp.replace}

\indexdefn{argument}%
\indexdefn{argument!throw expression}%
\definition{argument}{defns.argument.throw}
\defncontext{throw expression} the operand of \tcode{throw}\iref{expr.throw}

\indexdefn{argument}%
\indexdefn{argument!template instantiation}%
\definition{argument}{defns.argument.templ}
\defncontext{template instantiation}
\grammarterm{constant-expression},
\grammarterm{type-id}, or
\grammarterm{id-expression} in the comma-separated
list bounded by the angle brackets\iref{temp.arg}

\indexdefn{block}%
\definition{block}{defns.block}
a thread of execution that blocks is waiting for some condition (other than
for the implementation to execute its execution steps) to be satisfied before
it can continue execution past the blocking operation

\indexdefn{behavior!conditionally-supported}%
\definition{conditionally-supported}{defns.cond.supp}
program construct that an implementation is not required to support\\
\begin{note} Each implementation documents all conditionally-supported
constructs that it does not support.\end{note}

\indexdefn{message!diagnostic}%
\definition{diagnostic message}{defns.diagnostic}
message belonging to an \impldef{diagnostic message} subset of the
implementation's output messages

\indexdefn{type!dynamic}%
\definition{dynamic type}{defns.dynamic.type}
\defncontext{glvalue} type of the most derived object\iref{intro.object} to which the
glvalue refers\\
\begin{example}
If a pointer\iref{dcl.ptr} \tcode{p} whose static type is ``pointer to
class \tcode{B}'' is pointing to an object of class \tcode{D}, derived
from \tcode{B}\iref{class.derived}, the dynamic type of the
expression \tcode{*p} is ``\tcode{D}''. References\iref{dcl.ref} are
treated similarly.
\end{example}

\indexdefn{type!dynamic}%
\definition{dynamic type}{defns.dynamic.type.prvalue}
\defncontext{prvalue} static type of the prvalue expression

\indexdefn{program!ill-formed}%
\definition{ill-formed program}{defns.ill.formed}
program that is not well-formed\iref{defns.well.formed}

\indexdefn{behavior!implementation-defined}%
\definition{implementation-defined behavior}{defns.impl.defined}
behavior, for a well-formed program construct and correct data, that
depends on the implementation and that each implementation documents

\indexdefn{limits!implementation}%
\definition{implementation limits}{defns.impl.limits}
restrictions imposed upon programs by the implementation

\indexdefn{behavior!locale-specific}%
\definition{locale-specific behavior}{defns.locale.specific}
behavior that depends on local conventions of nationality, culture, and
language that each implementation documents

\indexdefn{character!multibyte}%
\definition{multibyte character}{defns.multibyte}
sequence of one or more bytes representing a member of the extended
character set of either the source or the execution environment\\
\begin{note} The
extended character set is a superset of the basic character
set\iref{lex.charset}.\end{note}

\indexdefn{parameter}%
\indexdefn{parameter!function}%
\indexdefn{parameter!catch clause}%
\definition{parameter}{defns.parameter}
\defncontext{function or catch clause} object or reference declared as part of a function declaration or
definition or in the catch clause of an exception handler that
acquires a value on entry to the function or handler

\indexdefn{parameter}%
\indexdefn{parameter!function-like macro}%
\definition{parameter}{defns.parameter.macro}
\defncontext{function-like macro} identifier from
the comma-separated list bounded by the parentheses immediately
following the macro name

\indexdefn{parameter}%
\indexdefn{parameter!template}%
\definition{parameter}{defns.parameter.templ}
\defncontext{template} member of a \grammarterm{template-parameter-list}

\indexdefn{signature}%
\definition{signature}{defns.signature}
\defncontext{function}
name,
parameter type list\iref{dcl.fct},
enclosing namespace (if any),
and
\grammarterm{requires-clause}\iref{temp.constr.decl} (if any)
\begin{note} Signatures are used as a basis for
name mangling and linking.\end{note}

\indexdefn{signature}%
\definition{signature}{defns.signature.templ}
\defncontext{function template}
name,
parameter type list\iref{dcl.fct},
enclosing namespace (if any),
return type,
\grammarterm{template-head},
and
\grammarterm{requires-clause}\iref{temp.constr.decl} (if any)

\indexdefn{signature}%
\definition{signature}{defns.signature.spec}
\defncontext{function template specialization} signature of the template of which it is a specialization
and its template arguments (whether explicitly specified or deduced)

\indexdefn{signature}%
\definition{signature}{defns.signature.member}
\defncontext{class member function}
name,
parameter type list\iref{dcl.fct},
class of which the function is a member,
\cv-qualifiers (if any),
\grammarterm{ref-qualifier} (if any),
and
\grammarterm{requires-clause}\iref{temp.constr.decl} (if any)

\indexdefn{signature}%
\definition{signature}{defns.signature.member.templ}
\defncontext{class member function template}
name,
parameter type list\iref{dcl.fct},
class of which the function is a member,
\cv-qualifiers (if any),
\grammarterm{ref-qualifier} (if any),
return type (if any),
\grammarterm{template-head},
and
\grammarterm{requires-clause}\iref{temp.constr.decl} (if any)

\indexdefn{signature}%
\definition{signature}{defns.signature.member.spec}
\defncontext{class member function template specialization} signature of the member function template
of which it is a specialization and its template arguments (whether explicitly specified or deduced)

\indexdefn{type!static}%
\definition{static type}{defns.static.type}
type of an expression\iref{basic.types} resulting from
analysis of the program without considering execution semantics\\
\begin{note} The
static type of an expression depends only on the form of the program in
which the expression appears, and does not change while the program is
executing. \end{note}

\indexdefn{unblock}%
\definition{unblock}{defns.unblock}
satisfy a condition that one or more blocked threads of execution are waiting for

\indexdefn{behavior!undefined}%
\definition{undefined behavior}{defns.undefined}
behavior for which this International Standard
imposes no requirements\\
\begin{note} Undefined behavior may be expected when
this International Standard omits any explicit
definition of behavior or when a program uses an erroneous construct or erroneous data.
Permissible undefined behavior ranges
from ignoring the situation completely with unpredictable results, to
behaving during translation or program execution in a documented manner
characteristic of the environment (with or without the issuance of a
diagnostic message), to terminating a translation or execution (with the
issuance of a diagnostic message). Many erroneous program constructs do
not engender undefined behavior; they are required to be diagnosed.
Evaluation of a constant expression never exhibits behavior explicitly
specified as undefined\iref{expr.const}.
\end{note}

\indexdefn{behavior!unspecified}%
\definition{unspecified behavior}{defns.unspecified}
behavior, for a well-formed program construct and correct data, that
depends on the implementation\\
\begin{note} The implementation is not required to
document which behavior occurs. The range of
possible behaviors is usually delineated by this International Standard.
\end{note}

\indexdefn{program!well-formed}%
\definition{well-formed program}{defns.well.formed}
\Cpp  program constructed according to the syntax rules, diagnosable
semantic rules, and the one-definition rule\iref{basic.def.odr}.%
\indextext{definitions|)}

\rSec0[intro]{General principles}

\indextext{diagnostic message|see{message, diagnostic}}%
\indexdefn{conditionally-supported behavior|see{behavior, con\-ditionally-supported}}%
\indextext{dynamic type|see{type, dynamic}}%
\indextext{static type|see{type, static}}%
\indextext{ill-formed program|see{program, ill-formed}}%
\indextext{well-formed program|see{program, well-formed}}%
\indextext{implementation-defined behavior|see{behavior, im\-plementation-defined}}%
\indextext{undefined behavior|see{behavior, undefined}}%
\indextext{unspecified behavior|see{behavior, unspecified}}%
\indextext{implementation limits|see{limits, implementation}}%
\indextext{locale-specific behavior|see{behavior, locale-spe\-cific}}%
\indextext{multibyte character|see{character, multibyte}}%
\indextext{object|seealso{object model}}%
\indextext{subobject|seealso{object model}}%
\indextext{derived class!most|see{most derived class}}%
\indextext{derived object!most|see{most derived object}}%
\indextext{program execution!as-if rule|see{as-if rule}}%
\indextext{observable behavior|see{behavior, observable}}%
\indextext{precedence of operator|see{operator, precedence of}}%
\indextext{order of evaluation in expression|see{expression, order of evaluation of}}%
\indextext{atomic operations|see{operation, atomic}}%
\indextext{multiple threads|see{threads, multiple}}%
\rSec1[intro.compliance]{Implementation compliance}

\pnum
\indextext{conformance requirements|(}%
\indextext{conformance requirements!general|(}%
The set of
\defn{diagnosable rules}
consists of all syntactic and semantic rules in this International
Standard except for those rules containing an explicit notation that
``no diagnostic is required'' or which are described as resulting in
``undefined behavior''.

\pnum
\indextext{conformance requirements!method of description}%
Although this International Standard states only requirements on \Cpp
implementations, those requirements are often easier to understand if
they are phrased as requirements on programs, parts of programs, or
execution of programs. Such requirements have the following meaning:
\begin{itemize}
\item
If a program contains no violations of the rules in this
International Standard, a conforming implementation shall,
within its resource limits, accept and correctly execute\footnote{``Correct execution'' can include undefined behavior, depending on
the data being processed; see \ref{intro.defs} and~\ref{intro.execution}.}
that program.
\item
\indextext{message!diagnostic}%
If a program contains a violation of any diagnosable rule or an occurrence
of a construct described in this International Standard as ``conditionally-supported'' when
the implementation does not support that construct, a conforming implementation
shall issue at least one diagnostic message.
\item
\indextext{behavior!undefined}%
If a program contains a violation of a rule for which no diagnostic
is required, this International Standard places no requirement on
implementations with respect to that program.
\end{itemize}
\begin{note}
During template argument deduction and substitution,
certain constructs that in other contexts require a diagnostic
are treated differently;
see~\ref{temp.deduct}.
\end{note}

\pnum
\indextext{conformance requirements!library|(}%
\indextext{conformance requirements!classes}%
\indextext{conformance requirements!class templates}%
For classes and class templates, the library Clauses specify partial
definitions. Private members\iref{class.access} are not
specified, but each implementation shall supply them to complete the
definitions according to the description in the library Clauses.

\pnum
For functions, function templates, objects, and values, the library
Clauses specify declarations. Implementations shall supply definitions
consistent with the descriptions in the library Clauses.

\pnum
The names defined in the library have namespace
scope\iref{basic.namespace}. A \Cpp  translation
unit\iref{lex.phases} obtains access to these names by including the
appropriate standard library header\iref{cpp.include}.

\pnum
The templates, classes, functions, and objects in the library have
external linkage\iref{basic.link}. The implementation provides
definitions for standard library entities, as necessary, while combining
translation units to form a complete \Cpp  program\iref{lex.phases}.%
\indextext{conformance requirements!library|)}

\pnum
Two kinds of implementations are defined: a \defn{hosted implementation} and a
\defn{freestanding implementation}. For a hosted implementation, this
International Standard defines the set of available libraries. A freestanding
implementation is one in which execution may take place without the benefit of
an operating system, and has an \impldef{required libraries for freestanding
implementation} set of libraries that includes certain language-support
libraries\iref{compliance}.

\pnum
A conforming implementation may have extensions (including
additional library functions), provided they do not alter the
behavior of any well-formed program.
Implementations are required to diagnose programs that use such
extensions that are ill-formed according to this International Standard.
Having done so, however, they can compile and execute such programs.

\pnum
Each implementation shall include documentation that identifies all
conditionally-supported constructs\indextext{behavior!conditionally-supported}
that it does not support and defines all locale-specific characteristics.\footnote{This documentation also defines implementation-defined behavior;
see~\ref{intro.execution}.}%
\indextext{conformance requirements!general|)}%
\indextext{conformance requirements|)}%

\rSec1[intro.structure]{Structure of this document}

\pnum
\indextext{standard!structure of|(}%
\indextext{standard!structure of}%
\ref{lex} through \ref{cpp} describe the \Cpp  programming
language. That description includes detailed syntactic specifications in
a form described in~\ref{syntax}. For convenience, \ref{gram}
repeats all such syntactic specifications.

\pnum
\ref{\firstlibchapter} through \ref{\lastlibchapter} and \ref{depr}
(the \defn{library clauses}) describe the \Cpp standard library.
That description includes detailed descriptions of the
entities and macros
that constitute the library, in a form described in \ref{library}.

\pnum
\ref{implimits} recommends lower bounds on the capacity of conforming
implementations.

\pnum
\ref{diff} summarizes the evolution of \Cpp  since its first
published description, and explains in detail the differences between
\Cpp  and C\@. Certain features of \Cpp  exist solely for compatibility
purposes; \ref{depr} describes those features.

\pnum
Throughout this document, each example is introduced by
``\noteintro{Example}'' and terminated by ``\noteoutro{example}''. Each note is
introduced by ``\noteintro{Note}'' and terminated by ``\noteoutro{note}''. Examples
and notes may be nested.%
\indextext{standard!structure of|)}

\rSec1[syntax]{Syntax notation}

\pnum
\indextext{notation!syntax|(}%
In the syntax notation used in this document, syntactic
categories are indicated by \grammarterm{italic} type, and literal words
and characters in \tcode{constant} \tcode{width} type. Alternatives are
listed on separate lines except in a few cases where a long set of
alternatives is marked by the phrase ``one of''. If the text of an alternative is too long to fit on a line, the text is continued on subsequent lines indented from the first one.
An optional terminal or non-terminal symbol is indicated by the subscript
``\opt'', so

\begin{ncbnf}
\terminal{\{} expression\opt{} \terminal{\}}
\end{ncbnf}

indicates an optional expression enclosed in braces.%

\pnum
Names for syntactic categories have generally been chosen according to
the following rules:
\begin{itemize}
\item \grammarterm{X-name} is a use of an identifier in a context that
determines its meaning (e.g., \grammarterm{class-name},
\grammarterm{typedef-name}).
\item \grammarterm{X-id} is an identifier with no context-dependent meaning
(e.g., \grammarterm{qualified-id}).
\item \grammarterm{X-seq} is one or more \grammarterm{X}'s without intervening
delimiters (e.g., \grammarterm{declaration-seq} is a sequence of
declarations).
\item \grammarterm{X-list} is one or more \grammarterm{X}'s separated by
intervening commas (e.g., \grammarterm{identifier-list} is a sequence of
identifiers separated by commas).
\end{itemize}%
\indextext{notation!syntax|)}

\rSec1[intro.memory]{The \Cpp memory model}

\pnum
\indextext{memory model|(}%
The fundamental storage unit in the \Cpp memory model is the
\defn{byte}.
A byte is at least large enough to contain any member of the basic
\indextext{character set!basic execution}%
execution character set\iref{lex.charset}
and the eight-bit code units of the Unicode UTF-8 encoding form
and is composed of a contiguous sequence of
bits,\footnote{The number of bits in a byte is reported by the macro
\tcode{CHAR_BIT} in the header \tcode{<climits>}.}
the number of which is \impldef{bits in a byte}. The least
significant bit is called the \defn{low-order bit}; the most
significant bit is called the \defn{high-order bit}. The memory
available to a \Cpp program consists of one or more sequences of
contiguous bytes. Every byte has a unique address.

\pnum
\begin{note} The representation of types is described
in~\ref{basic.types}. \end{note}

\pnum
A \defn{memory location} is either an object of scalar type or a maximal
sequence of adjacent bit-fields all having nonzero width. \begin{note} Various
features of the language, such as references and virtual functions, might
involve additional memory locations that are not accessible to programs but are
managed by the implementation. \end{note} Two or more threads of
execution\iref{intro.multithread} can access separate memory
locations without interfering with each other.

\pnum
\begin{note} Thus a bit-field and an adjacent non-bit-field are in separate memory
locations, and therefore can be concurrently updated by two threads of execution
without interference. The same applies to two bit-fields, if one is declared
inside a nested struct declaration and the other is not, or if the two are
separated by a zero-length bit-field declaration, or if they are separated by a
non-bit-field declaration. It is not safe to concurrently update two bit-fields
in the same struct if all fields between them are also bit-fields of nonzero
width. \end{note}

\pnum
\begin{example} A structure declared as

\begin{codeblock}
struct {
  char a;
  int b:5,
  c:11,
  :0,
  d:8;
  struct {int ee:8;} e;
}
\end{codeblock}

contains four separate memory locations: The member \tcode{a} and bit-fields
\tcode{d} and \tcode{e.ee} are each separate memory locations, and can be
modified concurrently without interfering with each other. The bit-fields
\tcode{b} and \tcode{c} together constitute the fourth memory location. The
bit-fields \tcode{b} and \tcode{c} cannot be concurrently modified, but
\tcode{b} and \tcode{a}, for example, can be. \end{example}%
\indextext{memory model|)}

\rSec1[intro.object]{The \Cpp object model}

\pnum
\indextext{object model|(}%
The constructs in a \Cpp program create, destroy, refer to, access, and
manipulate objects.
An \defn{object} is created
by a definition\iref{basic.def},
by a \grammarterm{new-expression}\iref{expr.new},
when implicitly changing the active member of a union\iref{class.union},
or
when a temporary object is created~(\ref{conv.rval}, \ref{class.temporary}).
An object occupies a region of storage
in its period of construction\iref{class.cdtor},
throughout its lifetime\iref{basic.life},
and
in its period of destruction\iref{class.cdtor}.
\begin{note} A function is not an object, regardless of whether or not it
occupies storage in the way that objects do. \end{note}
The properties of an
object are determined when the object is created. An object can have a
name\iref{basic}. An object has a storage
duration\iref{basic.stc} which influences its
lifetime\iref{basic.life}. An object has a
type\iref{basic.types}.
Some objects are
polymorphic\iref{class.virtual}; the implementation
generates information associated with each such object that makes it
possible to determine that object's type during program execution. For
other objects, the interpretation of the values found therein is
determined by the type of the \grammarterm{expression}{s}\iref{expr}
used to access them.

\pnum
\indextext{subobject}%
Objects can contain other objects, called \defnx{subobjects}{subobject}.
A subobject can be
a \defn{member subobject}\iref{class.mem}, a \defn{base class subobject}\iref{class.derived},
or an array element.
\indextext{object!complete}%
An object that is not a subobject of any other object is called a \defn{complete
object}.
If an object is created
in storage associated with a member subobject or array element \placeholder{e}
(which may or may not be within its lifetime),
the created object
is a subobject of \placeholder{e}'s containing object if:
\begin{itemize}
\item
the lifetime of \placeholder{e}'s containing object has begun and not ended, and
\item
the storage for the new object exactly overlays the storage location associated with \placeholder{e}, and
\item
the new object is of the same type as \placeholder{e} (ignoring cv-qualification).
\end{itemize}
\begin{note}
If the subobject contains a reference member or a \tcode{const} subobject,
the name of the original subobject cannot be used to access the new object\iref{basic.life}.
\end{note}
\begin{example}
\begin{codeblock}
struct X { const int n; };
union U { X x; float f; };
void tong() {
  U u = {{ 1 }};
  u.f = 5.f;                          // OK, creates new subobject of \tcode{u}\iref{class.union}
  X *p = new (&u.x) X {2};            // OK, creates new subobject of \tcode{u}
  assert(p->n == 2);                  // OK
  assert(*std::launder(&u.x.n) == 2); // OK
  assert(u.x.n == 2);                 // undefined behavior, \tcode{u.x} does not name new subobject
}
\end{codeblock}
\end{example}

\pnum
\indextext{object!providing storage for}%
If a complete object is created\iref{expr.new}
in storage associated with another object \placeholder{e}
of type ``array of $N$ \tcode{unsigned char}'' or
of type ``array of $N$ \tcode{std::byte}''\iref{cstddef.syn},
that array \defn{provides storage}
for the created object if:
\begin{itemize}
\item
the lifetime of \placeholder{e} has begun and not ended, and
\item
the storage for the new object fits entirely within \placeholder{e}, and
\item
there is no smaller array object that satisfies these constraints.
\end{itemize}
\begin{note}
If that portion of the array
previously provided storage for another object,
the lifetime of that object ends
because its storage was reused\iref{basic.life}.
\end{note}
\begin{example}
\begin{codeblock}
template<typename ...T>
struct AlignedUnion {
  alignas(T...) unsigned char data[max(sizeof(T)...)];
};
int f() {
  AlignedUnion<int, char> au;
  int *p = new (au.data) int;     // OK, \tcode{au.data} provides storage
  char *c = new (au.data) char(); // OK, ends lifetime of \tcode{*p}
  char *d = new (au.data + 1) char();
  return *c + *d; // OK
}

struct A { unsigned char a[32]; };
struct B { unsigned char b[16]; };
A a;
B *b = new (a.a + 8) B;      // \tcode{a.a} provides storage for \tcode{*b}
int *p = new (b->b + 4) int; // \tcode{b->b} provides storage for \tcode{*p}
                             // \tcode{a.a} does not provide storage for \tcode{*p} (directly),
                             // but \tcode{*p} is nested within \tcode{a} (see below)
\end{codeblock}
\end{example}

\pnum
\indextext{object!nested within}%
An object \placeholder{a} is \defn{nested within} another object \placeholder{b} if:
\begin{itemize}
\item
\placeholder{a} is a subobject of \placeholder{b}, or
\item
\placeholder{b} provides storage for \placeholder{a}, or
\item
there exists an object \placeholder{c}
where \placeholder{a} is nested within \placeholder{c},
and \placeholder{c} is nested within \placeholder{b}.
\end{itemize}

\pnum
For every object \tcode{x}, there is some object called the
\defn{complete object of} \tcode{x}, determined as follows:
\begin{itemize}
\item
If \tcode{x} is a complete object, then the complete object
of \tcode{x} is itself.

\item
Otherwise, the complete object of \tcode{x} is the complete object
of the (unique) object that contains \tcode{x}.
\end{itemize}

\pnum
If a complete object, a data member\iref{class.mem}, or an array element is of
class type, its type is considered the \defn{most derived
class}, to distinguish it from the class type of any base class subobject;
an object of a most derived class type or of a non-class type is called a
\defn{most derived object}.

\pnum
\indextext{most derived object!bit-field}%
Unless it is a bit-field\iref{class.bit}, a most derived object shall have a
nonzero size and shall occupy one or more bytes of storage. Base class
subobjects may have zero size. An object of trivially copyable or
standard-layout type\iref{basic.types} shall occupy contiguous bytes of
storage.

\pnum
\indextext{most derived object!bit-field}%
\indextext{most derived object!zero size subobject}%
Unless an object is a bit-field or a base class subobject of zero size, the
address of that object is the address of the first byte it occupies.
Two objects \placeholder{a} and \placeholder{b}
with overlapping lifetimes
that are not bit-fields
may have the same address
if one is nested within the other,
or
if at least one is a base class subobject of zero size
and they are of different types;
otherwise, they have distinct addresses.\footnote{Under the ``as-if'' rule an
implementation is allowed to store two objects at the same machine address or
not store an object at all if the program cannot observe the
difference\iref{intro.execution}.}

\begin{example}
\begin{codeblock}
static const char test1 = 'x';
static const char test2 = 'x';
const bool b = &test1 != &test2;      // always \tcode{true}
\end{codeblock}
\end{example}

\pnum
\begin{note}
\Cpp  provides a variety of fundamental types and several ways of composing
new types from existing types\iref{basic.types}.
\end{note}%
\indextext{object model|)}

\rSec1[intro.execution]{Program execution}

\pnum
\indextext{program execution|(}%
\indextext{program execution!abstract machine}%
The semantic descriptions in this International Standard define a
parameterized nondeterministic abstract machine. This International
Standard places no requirement on the structure of conforming
implementations. In particular, they need not copy or emulate the
structure of the abstract machine.
\indextext{as-if rule}%
\indextext{behavior!observable}%
Rather, conforming implementations are required to emulate (only) the observable
behavior of the abstract machine as explained below.\footnote{This provision is
sometimes called the ``as-if'' rule, because an implementation is free to
disregard any requirement of this International Standard as long as the result
is \emph{as if} the requirement had been obeyed, as far as can be determined
from the observable behavior of the program. For instance, an actual
implementation need not evaluate part of an expression if it can deduce that its
value is not used and that no
\indextext{side effects}%
side effects affecting the
observable behavior of the program are produced.}

\indextext{behavior!implementation-defined}%
\pnum
Certain aspects and operations of the abstract machine are described in this
International Standard as implementation-defined (for example,
\tcode{sizeof(int)}). These constitute the parameters of the abstract machine.
Each implementation shall include documentation describing its characteristics
and behavior in these respects.\footnote{This documentation also includes
conditionally-supported constructs and locale-specific behavior.
See~\ref{intro.compliance}.} Such documentation shall define the instance of the
abstract machine that corresponds to that implementation (referred to as the
``corresponding instance'' below).

\indextext{behavior!unspecified}%
\pnum
Certain other aspects and operations of the abstract machine are
described in this International Standard as unspecified (for example,
evaluation of expressions in a \grammarterm{new-initializer} if the allocation
function fails to allocate memory\iref{expr.new}). Where possible, this
International Standard defines a set of allowable behaviors. These
define the nondeterministic aspects of the abstract machine. An instance
of the abstract machine can thus have more than one possible execution
for a given program and a given input.

\indextext{behavior!undefined}%
\pnum
Certain other operations are described in this International Standard as
undefined (for example, the effect of
attempting to modify a \tcode{const} object).
\begin{note} This International Standard imposes no requirements on the
behavior of programs that contain undefined behavior. \end{note}

\indextext{program!well-formed}%
\indextext{behavior!observable}%
\pnum
A conforming implementation executing a well-formed program shall
produce the same observable behavior as one of the possible executions
of the corresponding instance of the abstract machine with the
same program and the same input.
\indextext{behavior!undefined}%
However, if any such execution contains an undefined operation, this International Standard places no
requirement on the implementation executing that program with that input
(not even with regard to operations preceding the first undefined
operation).

\pnum
An instance of each object with automatic storage
duration\iref{basic.stc.auto} is associated with each entry into its
block. Such an object exists and retains its last-stored value during
the execution of the block and while the block is suspended (by a call
of a function or receipt of a signal).

\indextext{conformance requirements}
\pnum
The least requirements on a conforming implementation are:
\begin{itemize}
\item
Accesses through volatile glvalues are evaluated strictly according to the
rules of the abstract machine.
\item
At program termination, all data written into files shall be
identical to one of the possible results that execution of the program
according to the abstract semantics would have produced.
\item
The input and output dynamics of interactive devices shall take
place in such a fashion that prompting output is actually delivered before a program waits for input. What constitutes an interactive device is
\impldef{interactive device}.
\end{itemize}

These collectively are referred to as the
\defnx{observable behavior}{behavior!observable} of the program.
\begin{note} More stringent correspondences between abstract and actual
semantics may be defined by each implementation. \end{note}

\pnum
\indextext{operator!precedence of}%
\indextext{expression!order of evaluation of}%
\begin{note} Operators can be regrouped according to the usual
mathematical rules only where the operators really are associative or
commutative.\footnote{Overloaded operators are never assumed to be associative or
commutative. }
For example, in the following fragment
\begin{codeblock}
int a, b;
@\commentellip@
a = a + 32760 + b + 5;
\end{codeblock}
the expression statement behaves exactly the same as
\begin{codeblock}
a = (((a + 32760) + b) + 5);
\end{codeblock}
due to the associativity and precedence of these operators. Thus, the
result of the sum \tcode{(a + 32760)} is next added to \tcode{b}, and
that result is then added to 5 which results in the value assigned to
\tcode{a}. On a machine in which overflows produce an exception and in
which the range of values representable by an \tcode{int} is
\crange{-32768}{+32767}, the implementation cannot rewrite this
expression as
\begin{codeblock}
a = ((a + b) + 32765);
\end{codeblock}
since if the values for \tcode{a} and \tcode{b} were, respectively,
-32754 and -15, the sum \tcode{a + b} would produce an exception while
the original expression would not; nor can the expression be rewritten
either as
\begin{codeblock}
a = ((a + 32765) + b);
\end{codeblock}
or
\begin{codeblock}
a = (a + (b + 32765));
\end{codeblock}
since the values for \tcode{a} and \tcode{b} might have been,
respectively, 4 and -8 or -17 and 12. However on a machine in which
overflows do not produce an exception and in which the results of
overflows are reversible, the above expression statement can be
rewritten by the implementation in any of the above ways because the
same result will occur. \end{note}

\pnum
A \defn{constituent expression} is defined as follows:
\begin{itemize}
\item
The constituent expression of an expression is that expression.
\item
The constituent expressions of a \grammarterm{braced-init-list} or
of a (possibly parenthesized) \grammarterm{expression-list}
are the constituent expressions of the elements of the respective list.
\item
The constituent expressions of a \grammarterm{brace-or-equal-initializer}
of the form \tcode{=}~\grammarterm{initializer-clause}
are the constituent expressions of the \grammarterm{initializer-clause}.
\end{itemize}
\begin{example}
\begin{codeblock}
struct A { int x; };
struct B { int y; struct A a; };
B b = { 5, { 1+1 } };
\end{codeblock}
The constituent expressions of the \grammarterm{initializer}
used for the initialization of \tcode{b} are \tcode{5} and \tcode{1+1}.
\end{example}

\pnum
The \defnx{immediate subexpressions}{immediate subexpression} of an expression \tcode{e} are
\begin{itemize}
\item
the constituent expressions of \tcode{e}'s operands\iref{expr},
\item
any function call that \tcode{e} implicitly invokes,
\item
if \tcode{e} is a \grammarterm{lambda-expression}\iref{expr.prim.lambda},
the initialization of the entities captured by copy and
the constituent expressions of the \grammarterm{initializer} of the \grammarterm{init-capture}{s},
\item
if \tcode{e} is a function call\iref{expr.call} or implicitly invokes a function,
the constituent expressions of each default argument\iref{dcl.fct.default}
used in the call, or
\item
if \tcode{e} creates an aggregate object\iref{dcl.init.aggr},
the constituent expressions of each default member initializer\iref{class.mem}
used in the initialization.
\end{itemize}

\pnum
A \defn{subexpression} of an expression \tcode{e} is
an immediate subexpression of \tcode{e} or
a subexpression of an immediate subexpression of \tcode{e}.
\begin{note}
Expressions appearing in the \grammarterm{compound-statement} of a \grammarterm{lambda-expression}
are not subexpressions of the \grammarterm{lambda-expression}.
\end{note}

\pnum
A \defn{full-expression} is
\begin{itemize}
\item
an unevaluated operand\iref{expr},
\item
a \grammarterm{constant-expression}\iref{expr.const},
\item
an \grammarterm{init-declarator}\iref{dcl.decl} or
a \grammarterm{mem-initializer}\iref{class.base.init},
including the constituent expressions of the initializer,
\item
an invocation of a destructor generated at the end of the lifetime
of an object other than a temporary object\iref{class.temporary}, or
\item
an expression that is not a subexpression of another expression and
that is not otherwise part of a full-expression.
\end{itemize}
If a language construct is defined to produce an implicit call of a function,
a use of the language construct is considered to be an expression
for the purposes of this definition.
Conversions applied to the result of an expression in order to satisfy the requirements
of the language construct in which the expression appears
are also considered to be part of the full-expression.
For an initializer, performing the initialization of the entity
(including evaluating default member initializers of an aggregate)
is also considered part of the full-expression.
\begin{example}
\begin{codeblock}
struct S {
  S(int i): I(i) { }       // full-expression is initialization of \tcode{I}
  int& v() { return I; }
  ~S() noexcept(false) { }
private:
  int I;
};

S s1(1);                   // full-expression is call of \tcode{S::S(int)}
void f() {
  S s2 = 2;                // full-expression is call of \tcode{S::S(int)}
  if (S(3).v())            // full-expression includes lvalue-to-rvalue and
                           // \tcode{int} to \tcode{bool} conversions, performed before
                           // temporary is deleted at end of full-expression
  { }
  bool b = noexcept(S());  // exception specification of destructor of \tcode{S}
                           // considered for \tcode{noexcept}
  // full-expression is destruction of \tcode{s2} at end of block
}
struct B {
      B(S = S(0));
   };
   B b[2] = { B(), B() };  // full-expression is the entire initialization
                           // including the destruction of temporaries
\end{codeblock}
\end{example}

\pnum
\begin{note} The evaluation of a full-expression can include the
evaluation of subexpressions that are not lexically part of the
full-expression. For example, subexpressions involved in evaluating
default arguments\iref{dcl.fct.default} are considered to
be created in the expression that calls the function, not the expression
that defines the default argument. \end{note}

\pnum
\indextext{value computation|(}%
Reading an object designated by a \tcode{volatile}
glvalue\iref{basic.lval}, modifying an object, calling a library I/O
function, or calling a function that does any of those operations are
all
\defn{side effects}, which are changes in the state of the execution
environment. \defnx{Evaluation}{evaluation} of an expression (or a
subexpression) in general includes both value computations (including
determining the identity of an object for glvalue evaluation and fetching
a value previously assigned to an object for prvalue evaluation) and
initiation of side effects. When a call to a library I/O function
returns or an access through a volatile glvalue is evaluated the side
effect is considered complete, even though some external actions implied
by the call (such as the I/O itself) or by the \tcode{volatile} access
may not have completed yet.

\pnum
\defnx{Sequenced before}{sequenced before} is an asymmetric, transitive, pair-wise relation between
evaluations executed by a single thread\iref{intro.multithread}, which induces
a partial order among those evaluations. Given any two evaluations \placeholder{A} and
\placeholder{B}, if \placeholder{A} is sequenced before \placeholder{B}
(or, equivalently, \placeholder{B} is \defn{sequenced after} \placeholder{A}),
then the execution of
\placeholder{A} shall precede the execution of \placeholder{B}. If \placeholder{A} is not sequenced
before \placeholder{B} and \placeholder{B} is not sequenced before \placeholder{A}, then \placeholder{A} and
\placeholder{B} are \defn{unsequenced}. \begin{note} The execution of unsequenced
evaluations can overlap. \end{note} Evaluations \placeholder{A} and \placeholder{B} are
\defn{indeterminately sequenced} when either \placeholder{A} is sequenced before
\placeholder{B} or \placeholder{B} is sequenced before \placeholder{A}, but it is unspecified which.
\begin{note} Indeterminately sequenced evaluations cannot overlap, but either
could be executed first. \end{note}
An expression \placeholder{X}
is said to be sequenced before
an expression \placeholder{Y} if
every value computation and every side effect
associated with the expression \placeholder{X}
is sequenced before
every value computation and every side effect
associated with the expression \placeholder{Y}.

\pnum
Every
\indextext{value computation}%
value computation and
\indextext{side effects}%
side effect associated with a full-expression is
sequenced before every value computation and side effect associated with the
next full-expression to be evaluated.\footnote{As specified
in~\ref{class.temporary}, after a full-expression is evaluated, a sequence of
zero or more invocations of destructor functions for temporary objects takes
place, usually in reverse order of the construction of each temporary object.}

\pnum
\indextext{evaluation!unspecified order of}%
Except where noted, evaluations of operands of individual operators and
of subexpressions of individual expressions are unsequenced. \begin{note}
In an expression that is evaluated more than once during the execution
of a program, unsequenced and indeterminately sequenced evaluations of
its subexpressions need not be performed consistently in different
evaluations. \end{note} The value computations of the operands of an
operator are sequenced before the value computation of the result of the
operator. If a
\indextext{side effects}%
side effect on a memory location\iref{intro.memory} is unsequenced
relative to either another side effect on the same memory location or
a value computation using the value of any object in the same memory location,
and they are not potentially concurrent\iref{intro.multithread},
the behavior is undefined.
\begin{note}
The next section imposes similar, but more complex restrictions on
potentially concurrent computations.
\end{note}

\begin{example}

\begin{codeblock}
void g(int i) {
  i = 7, i++, i++;    // \tcode{i} becomes \tcode{9}

  i = i++ + 1;        // the value of \tcode{i} is incremented
  i = i++ + i;        // the behavior is undefined
  i = i + 1;          // the value of \tcode{i} is incremented
}
\end{codeblock}
\end{example}

\pnum
When calling a function (whether or not the function is inline), every
\indextext{value computation}%
value computation and
\indextext{side effects}%
side effect associated with any argument
expression, or with the postfix expression designating the called
function, is sequenced before execution of every expression or statement
in the body of the called function.
For each function invocation \placeholder{F},
for every evaluation \placeholder{A} that occurs within \placeholder{F} and
every evaluation \placeholder{B} that does not occur within \placeholder{F} but
is evaluated on the same thread and as part of the same signal handler (if any),
either \placeholder{A} is sequenced before \placeholder{B} or
\placeholder{B} is sequenced before \placeholder{A}.\footnote{In other words,
function executions do not interleave with each other.}
\begin{note}
If \placeholder{A} and \placeholder{B} would not otherwise be sequenced then they are
indeterminately sequenced.
\end{note}
Several contexts in \Cpp  cause evaluation of a function call, even
though no corresponding function call syntax appears in the translation
unit.
\begin{example}
Evaluation of a \grammarterm{new-expression} invokes one or more allocation
and constructor functions; see~\ref{expr.new}. For another example,
invocation of a conversion function\iref{class.conv.fct} can arise in
contexts in which no function call syntax appears.
\end{example}
The sequencing constraints on the execution of the called function (as
described above) are features of the function calls as evaluated,
whatever the syntax of the expression that calls the function might be.%
\indextext{value computation|)}%

\indextext{behavior!on receipt of signal}%
\indextext{signal}%
\pnum
If a signal handler is executed as a result of a call to the \tcode{std::raise}
function, then the execution of the handler is sequenced after the invocation
of the \tcode{std::raise} function and before its return.
\begin{note} When a signal is received for another reason, the execution of the
signal handler is usually unsequenced with respect to the rest of the program.
\end{note}
\indextext{program execution|)}

\rSec1[intro.multithread]{Multi-threaded executions and data races}

\pnum
\indextext{threads!multiple|(}%
\indextext{operation!atomic|(}%
A \defn{thread of execution} (also known as a \defn{thread}) is a single flow of
control within a program, including the initial invocation of a specific
top-level function, and recursively including every function invocation
subsequently executed by the thread. \begin{note} When one thread creates another,
the initial call to the top-level function of the new thread is executed by the
new thread, not by the creating thread. \end{note} Every thread in a program can
potentially access every object and function in a program.\footnote{An object
with automatic or thread storage duration\iref{basic.stc} is associated with
one specific thread, and can be accessed by a different thread only indirectly
through a pointer or reference\iref{basic.compound}.} Under a hosted
implementation, a \Cpp program can have more than one thread running
concurrently. The execution of each thread proceeds as defined by the remainder
of this International Standard. The execution of the entire program consists of an execution
of all of its threads. \begin{note} Usually the execution can be viewed as an
interleaving of all its threads. However, some kinds of atomic operations, for
example, allow executions inconsistent with a simple interleaving, as described
below. \end{note} Under a freestanding implementation, it is \impldef{number of
threads in a program under a freestanding implementation} whether a program can
have more than one thread of execution.

\pnum
For a signal handler that is not executed as a result of a call to the
\tcode{std::raise} function, it is unspecified which thread of execution
contains the signal handler invocation.

\rSec2[intro.races]{Data races}

\pnum
The value of an object visible to a thread \placeholder{T} at a particular point is the
initial value of the object, a value assigned to the object by \placeholder{T}, or a
value assigned to the object by another thread, according to the rules below.
\begin{note} In some cases, there may instead be undefined behavior. Much of this
section is motivated by the desire to support atomic operations with explicit
and detailed visibility constraints. However, it also implicitly supports a
simpler view for more restricted programs. \end{note}

\pnum
Two expression evaluations \defn{conflict} if one of them modifies a memory
location\iref{intro.memory} and the other one reads or modifies the same
memory location.

\pnum
The library defines a number of atomic operations\iref{atomics} and
operations on mutexes\iref{thread} that are specially identified as
synchronization operations. These operations play a special role in making
assignments in one thread visible to another. A synchronization operation on one
or more memory locations is either a consume operation, an acquire operation, a
release operation, or both an acquire and release operation. A synchronization
operation without an associated memory location is a fence and can be either an
acquire fence, a release fence, or both an acquire and release fence. In
addition, there are relaxed atomic operations, which are not synchronization
operations, and atomic read-modify-write operations, which have special
characteristics. \begin{note} For example, a call that acquires a mutex will
perform an acquire operation on the locations comprising the mutex.
Correspondingly, a call that releases the same mutex will perform a release
operation on those same locations. Informally, performing a release operation on
\placeholder{A} forces prior
\indextext{side effects}%
side effects on other memory locations to become visible
to other threads that later perform a consume or an acquire operation on
\placeholder{A}. ``Relaxed'' atomic operations are not synchronization operations even
though, like synchronization operations, they cannot contribute to data races.
\end{note}

\pnum
All modifications to a particular atomic object \placeholder{M} occur in some
particular total order, called the \defn{modification order} of \placeholder{M}.
\begin{note} There is a separate order for each
atomic object. There is no requirement that these can be combined into a single
total order for all objects. In general this will be impossible since different
threads may observe modifications to different objects in inconsistent orders.
\end{note}

\pnum
A \defn{release sequence} headed by a release operation \placeholder{A} on an atomic object
\placeholder{M} is a maximal contiguous sub-sequence of
\indextext{side effects}%
side effects in the
modification order of \placeholder{M}, where the first operation is \tcode{A}, and
every subsequent operation

\begin{itemize}
\item is performed by the same thread that performed \tcode{A}, or
\item is an atomic read-modify-write operation.
\end{itemize}

\pnum
Certain library calls \defn{synchronize with} other library calls performed by
another thread. For example, an atomic store-release synchronizes with a
load-acquire that takes its value from the store\iref{atomics.order}.
\begin{note} Except in the specified cases, reading a later value does not
necessarily ensure visibility as described below. Such a requirement would
sometimes interfere with efficient implementation. \end{note} \begin{note} The
specifications of the synchronization operations define when one reads the value
written by another. For atomic objects, the definition is clear. All operations
on a given mutex occur in a single total order. Each mutex acquisition ``reads
the value written'' by the last mutex release. \end{note}

\pnum
An evaluation \placeholder{A} \defn{carries a dependency} to an evaluation \placeholder{B} if

\begin{itemize}
\item
the value of \placeholder{A} is used as an operand of \placeholder{B}, unless:
\begin{itemize}
\item
\placeholder{B} is an invocation of any specialization of
\tcode{std::kill_dependency}\iref{atomics.order}, or
\item
\placeholder{A} is the left operand of a built-in logical AND (\tcode{\&\&},
see~\ref{expr.log.and}) or logical OR (\tcode{||}, see~\ref{expr.log.or})
operator, or
\item
\placeholder{A} is the left operand of a conditional (\tcode{?:}, see~\ref{expr.cond})
operator, or
\item
\placeholder{A} is the left operand of the built-in comma (\tcode{,})
operator\iref{expr.comma}; \end{itemize} or
\item
\placeholder{A} writes a scalar object or bit-field \placeholder{M}, \placeholder{B} reads the value
written by \placeholder{A} from \placeholder{M}, and \placeholder{A} is sequenced before \placeholder{B}, or
\item
for some evaluation \placeholder{X}, \placeholder{A} carries a dependency to \placeholder{X}, and
\placeholder{X} carries a dependency to \placeholder{B}.
\end{itemize}
\begin{note} ``Carries a dependency to'' is a subset of ``is sequenced before'',
and is similarly strictly intra-thread. \end{note}

\pnum
An evaluation \placeholder{A} is \defn{dependency-ordered before} an evaluation
\placeholder{B} if
\begin{itemize}
\item
\placeholder{A} performs a release operation on an atomic object \placeholder{M}, and, in
another thread, \placeholder{B} performs a consume operation on \placeholder{M} and reads a
value written by any
\indextext{side effects}%
side effect in the release sequence headed by \placeholder{A}, or

\item
for some evaluation \placeholder{X}, \placeholder{A} is dependency-ordered before \placeholder{X} and
\placeholder{X} carries a dependency to \placeholder{B}.

\end{itemize}
\begin{note} The relation ``is dependency-ordered before'' is analogous to
``synchronizes with'', but uses release/consume in place of release/acquire.
\end{note}

\pnum
An evaluation \placeholder{A} \defn{inter-thread happens before} an evaluation \placeholder{B}
if
\begin{itemize}
\item
  \placeholder{A} synchronizes with \placeholder{B}, or
\item
  \placeholder{A} is dependency-ordered before \placeholder{B}, or
\item
  for some evaluation \placeholder{X}
  \begin{itemize}
  \item
    \placeholder{A} synchronizes with \placeholder{X} and \placeholder{X}
    is sequenced before \placeholder{B}, or
  \item
    \placeholder{A} is sequenced before \placeholder{X} and \placeholder{X}
    inter-thread happens before \placeholder{B}, or
  \item
    \placeholder{A} inter-thread happens before \placeholder{X} and \placeholder{X}
    inter-thread happens before \placeholder{B}.
  \end{itemize}
\end{itemize}
\begin{note} The ``inter-thread happens before'' relation describes arbitrary
concatenations of ``sequenced before'', ``synchronizes with'' and
``dependency-ordered before'' relationships, with two exceptions. The first
exception is that a concatenation is not permitted to end with
``dependency-ordered before'' followed by ``sequenced before''. The reason for
this limitation is that a consume operation participating in a
``dependency-ordered before'' relationship provides ordering only with respect
to operations to which this consume operation actually carries a dependency. The
reason that this limitation applies only to the end of such a concatenation is
that any subsequent release operation will provide the required ordering for a
prior consume operation. The second exception is that a concatenation is not
permitted to consist entirely of ``sequenced before''. The reasons for this
limitation are (1) to permit ``inter-thread happens before'' to be transitively
closed and (2) the ``happens before'' relation, defined below, provides for
relationships consisting entirely of ``sequenced before''. \end{note}

\pnum
An evaluation \placeholder{A} \defn{happens before} an evaluation \placeholder{B}
(or, equivalently, \placeholder{B} \defn{happens after} \placeholder{A}) if:
\begin{itemize}
\item \placeholder{A} is sequenced before \placeholder{B}, or
\item \placeholder{A} inter-thread happens before \placeholder{B}.
\end{itemize}
The implementation shall ensure that no program execution demonstrates a cycle
in the ``happens before'' relation. \begin{note} This cycle would otherwise be
possible only through the use of consume operations. \end{note}

\pnum
An evaluation \placeholder{A} \defn{strongly happens before} an evaluation \placeholder{B}
if either
\begin{itemize}
\item \placeholder{A} is sequenced before \placeholder{B}, or
\item \placeholder{A} synchronizes with \placeholder{B}, or
\item \placeholder{A} strongly happens before \placeholder{X} and \placeholder{X} strongly happens before \placeholder{B}.
\end{itemize}
\begin{note}
In the absence of consume operations,
the happens before and strongly happens before relations are identical.
Strongly happens before essentially excludes consume operations.
\end{note}

\pnum
A \defnx{visible side effect}{side effects!visible} \placeholder{A} on a scalar object or bit-field \placeholder{M}
with respect to a value computation \placeholder{B} of \placeholder{M} satisfies the
conditions:
\begin{itemize}
\item \placeholder{A} happens before \placeholder{B} and
\item there is no other
\indextext{side effects}%
side effect \placeholder{X} to \placeholder{M} such that \placeholder{A}
happens before \placeholder{X} and \placeholder{X} happens before \placeholder{B}.
\end{itemize}

The value of a non-atomic scalar object or bit-field \placeholder{M}, as determined by
evaluation \placeholder{B}, shall be the value stored by the
\indextext{side effects!visible}%
visible side effect
\placeholder{A}. \begin{note} If there is ambiguity about which side effect to a
non-atomic object or bit-field is visible, then the behavior is either
unspecified or undefined. \end{note} \begin{note} This states that operations on
ordinary objects are not visibly reordered. This is not actually detectable
without data races, but it is necessary to ensure that data races, as defined
below, and with suitable restrictions on the use of atomics, correspond to data
races in a simple interleaved (sequentially consistent) execution. \end{note}

\pnum
The value of an
atomic object \placeholder{M}, as determined by evaluation \placeholder{B}, shall be the value
stored by some
side effect \placeholder{A} that modifies \placeholder{M}, where \placeholder{B} does not happen
before \placeholder{A}.
\begin{note}
The set of such side effects is also restricted by the rest of the rules
described here, and in particular, by the coherence requirements below.
\end{note}

\pnum
If an operation \placeholder{A} that modifies an atomic object \placeholder{M} happens before
an operation \placeholder{B} that modifies \placeholder{M}, then \placeholder{A} shall be earlier
than \placeholder{B} in the modification order of \placeholder{M}. \begin{note} This requirement
is known as write-write coherence. \end{note}

\pnum
If a
\indextext{value computation}%
value computation \placeholder{A} of an atomic object \placeholder{M} happens before a
value computation \placeholder{B} of \placeholder{M}, and \placeholder{A} takes its value from a side
effect \placeholder{X} on \placeholder{M}, then the value computed by \placeholder{B} shall either be
the value stored by \placeholder{X} or the value stored by a
\indextext{side effects}%
side effect \placeholder{Y} on
\placeholder{M}, where \placeholder{Y} follows \placeholder{X} in the modification order of \placeholder{M}.
\begin{note} This requirement is known as read-read coherence. \end{note}

\pnum
If a
\indextext{value computation}%
value computation \placeholder{A} of an atomic object \placeholder{M} happens before an
operation \placeholder{B} that modifies \placeholder{M}, then \placeholder{A} shall take its value from a side
effect \placeholder{X} on \placeholder{M}, where \placeholder{X} precedes \placeholder{B} in the
modification order of \placeholder{M}. \begin{note} This requirement is known as
read-write coherence. \end{note}

\pnum
If a
\indextext{side effects}%
side effect \placeholder{X} on an atomic object \placeholder{M} happens before a value
computation \placeholder{B} of \placeholder{M}, then the evaluation \placeholder{B} shall take its
value from \placeholder{X} or from a
\indextext{side effects}%
side effect \placeholder{Y} that follows \placeholder{X} in the
modification order of \placeholder{M}. \begin{note} This requirement is known as
write-read coherence. \end{note}

\pnum
\begin{note} The four preceding coherence requirements effectively disallow
compiler reordering of atomic operations to a single object, even if both
operations are relaxed loads. This effectively makes the cache coherence
guarantee provided by most hardware available to \Cpp atomic operations.
\end{note}

\pnum
\begin{note} The value observed by a load of an atomic depends on the ``happens
before'' relation, which depends on the values observed by loads of atomics.
The intended reading is that there must exist an
association of atomic loads with modifications they observe that, together with
suitably chosen modification orders and the ``happens before'' relation derived
as described above, satisfy the resulting constraints as imposed here. \end{note}

\pnum
Two actions are \defn{potentially concurrent} if
\begin{itemize}
\item they are performed by different threads, or
\item they are unsequenced, at least one is performed by a signal handler, and
they are not both performed by the same signal handler invocation.
\end{itemize}
The execution of a program contains a \defn{data race} if it contains two
potentially concurrent conflicting actions, at least one of which is not atomic,
and neither happens before the other,
except for the special case for signal handlers described below.
Any such data race results in undefined
behavior. \begin{note} It can be shown that programs that correctly use mutexes
and \tcode{memory_order_seq_cst} operations to prevent all data races and use no
other synchronization operations behave as if the operations executed by their
constituent threads were simply interleaved, with each
\indextext{value computation}%
value computation of an
object being taken from the last
\indextext{side effects}%
side effect on that object in that
interleaving. This is normally referred to as ``sequential consistency''.
However, this applies only to data-race-free programs, and data-race-free
programs cannot observe most program transformations that do not change
single-threaded program semantics. In fact, most single-threaded program
transformations continue to be allowed, since any program that behaves
differently as a result must perform an undefined operation. \end{note}

\pnum
Two accesses to the same object of type \tcode{volatile std::sig_atomic_t} do not
result in a data race if both occur in the same thread, even if one or more
occurs in a signal handler. For each signal handler invocation, evaluations
performed by the thread invoking a signal handler can be divided into two
groups \placeholder{A} and \placeholder{B}, such that no evaluations in
\placeholder{B} happen before evaluations in \placeholder{A}, and the
evaluations of such \tcode{volatile std::sig_atomic_t} objects take values as though
all evaluations in \placeholder{A} happened before the execution of the signal
handler and the execution of the signal handler happened before all evaluations
in \placeholder{B}.

\pnum
\begin{note} Compiler transformations that introduce assignments to a potentially
shared memory location that would not be modified by the abstract machine are
generally precluded by this International Standard, since such an assignment might overwrite
another assignment by a different thread in cases in which an abstract machine
execution would not have encountered a data race. This includes implementations
of data member assignment that overwrite adjacent members in separate memory
locations. Reordering of atomic loads in cases in which the atomics in question
may alias is also generally precluded, since this may violate the coherence
rules. \end{note}

\pnum
\begin{note} Transformations that introduce a speculative read of a potentially
shared memory location may not preserve the semantics of the \Cpp program as
defined in this International Standard, since they potentially introduce a data race. However,
they are typically valid in the context of an optimizing compiler that targets a
specific machine with well-defined semantics for data races. They would be
invalid for a hypothetical machine that is not tolerant of races or provides
hardware race detection. \end{note}

\rSec2[intro.progress]{Forward progress}

\pnum
The implementation may assume that any thread will eventually do one of the
following:
\begin{itemize}
\item terminate,
\item make a call to a library I/O function,
\item perform an access through a volatile glvalue, or
\item perform a synchronization operation or an atomic operation.
\end{itemize}
\begin{note} This is intended to allow compiler transformations such as removal of
empty loops, even when termination cannot be proven. \end{note}

\pnum
Executions of atomic functions
that are either defined to be lock-free\iref{atomics.flag}
or indicated as lock-free\iref{atomics.lockfree}
are \defnx{lock-free executions}{lock-free execution}.
\begin{itemize}
\item
  If there is only one thread that is not blocked\iref{defns.block}
  in a standard library function,
  a lock-free execution in that thread shall complete.
  \begin{note}
    Concurrently executing threads
    may prevent progress of a lock-free execution.
    For example,
    this situation can occur
    with load-locked store-conditional implementations.
    This property is sometimes termed obstruction-free.
  \end{note}
\item
  When one or more lock-free executions run concurrently,
  at least one should complete.
  \begin{note}
    It is difficult for some implementations
    to provide absolute guarantees to this effect,
    since repeated and particularly inopportune interference
    from other threads
    may prevent forward progress,
    e.g.,
    by repeatedly stealing a cache line
    for unrelated purposes
    between load-locked and store-conditional instructions.
    Implementations should ensure
    that such effects cannot indefinitely delay progress
    under expected operating conditions,
    and that such anomalies
    can therefore safely be ignored by programmers.
    Outside this document,
    this property is sometimes termed lock-free.
  \end{note}
\end{itemize}

\pnum
During the execution of a thread of execution, each of the following is termed
an \defn{execution step}:
\begin{itemize}
\item termination of the thread of execution,
\item performing an access through a volatile glvalue, or
\item completion of a call to a library I/O function, a
      synchronization operation, or an atomic operation.
\end{itemize}

\pnum
An invocation of a standard library function that blocks\iref{defns.block}
is considered to continuously execute execution steps while waiting for the
condition that it blocks on to be satisfied.
\begin{example}
A library I/O function that blocks until the I/O operation is complete can
be considered to continuously check whether the operation is complete. Each
such check might consist of one or more execution steps, for example using
observable behavior of the abstract machine.
\end{example}

\pnum
\begin{note}
Because of this and the preceding requirement regarding what threads of execution
have to perform eventually, it follows that no thread of execution can execute
forever without an execution step occurring.
\end{note}

\pnum
A thread of execution \defnx{makes progress}{make progress!thread}
when an execution step occurs or a
lock-free execution does not complete because there are other concurrent threads
that are not blocked in a standard library function (see above).

\pnum
\indextext{forward progress guarantees!concurrent}%
For a thread of execution providing \defn{concurrent forward progress guarantees},
the implementation ensures that the thread will eventually make progress for as
long as it has not terminated.
\begin{note}
This is required regardless of whether or not other threads of executions (if any)
have been or are making progress. To eventually fulfill this requirement means that
this will happen in an unspecified but finite amount of time.
\end{note}

\pnum
It is \impldef{whether the thread that executes \tcode{main} and the threads created
by \tcode{std::thread} provide concurrent forward progress guarantees} whether the
implementation-created thread of execution that executes
\tcode{main}\iref{basic.start.main} and the threads of execution created by
\tcode{std::thread}\iref{thread.thread.class} provide concurrent forward progress
guarantees.
\begin{note}
General-purpose implementations should provide these guarantees.
\end{note}

\pnum
\indextext{forward progress guarantees!parallel}%
For a thread of execution providing \defn{parallel forward progress guarantees},
the implementation is not required to ensure that the thread will eventually make
progress if it has not yet executed any execution step; once this thread has
executed a step, it provides concurrent forward progress guarantees.

\pnum
\begin{note}
This does not specify a requirement for when to start this thread of execution,
which will typically be specified by the entity that creates this thread of
execution. For example, a thread of execution that provides concurrent forward
progress guarantees and executes tasks from a set of tasks in an arbitrary order,
one after the other, satisfies the requirements of parallel forward progress for
these tasks.
\end{note}

\pnum
\indextext{forward progress guarantees!weakly parallel}%
For a thread of execution providing \defn{weakly parallel forward progress
guarantees}, the implementation does not ensure that the thread will eventually
make progress.

\pnum
\begin{note}
Threads of execution providing weakly parallel forward progress guarantees cannot
be expected to make progress regardless of whether other threads make progress or
not; however, blocking with forward progress guarantee delegation, as defined below,
can be used to ensure that such threads of execution make progress eventually.
\end{note}

\pnum
Concurrent forward progress guarantees are stronger than parallel forward progress
guarantees, which in turn are stronger than weakly parallel forward progress
guarantees.
\begin{note}
For example, some kinds of synchronization between threads of execution may only
make progress if the respective threads of execution provide parallel forward progress
guarantees, but will fail to make progress under weakly parallel guarantees.
\end{note}

\pnum
\indextext{forward progress guarantees!delegation of}%
When a thread of execution \placeholder{P} is specified to \defn{block with forward
progress guarantee delegation} on the completion of a set \placeholder{S} of threads
of execution, then throughout the whole time of \placeholder{P} being blocked on
\placeholder{S}, the implementation shall ensure that the forward progress guarantees
provided by at least one thread of execution in \placeholder{S} is at least as strong
as \placeholder{P}'s forward progress guarantees.
\begin{note}
It is unspecified which thread or threads of execution in \placeholder{S} are chosen
and for which number of execution steps. The strengthening is not permanent and
not necessarily in place for the rest of the lifetime of the affected thread of
execution. As long as \placeholder{P} is blocked, the implementation has to eventually
select and potentially strengthen a thread of execution in \placeholder{S}.
\end{note}
Once a thread of execution in \placeholder{S} terminates, it is removed from \placeholder{S}.
Once \placeholder{S} is empty, \placeholder{P} is unblocked.

\pnum
\begin{note}
A thread of execution \placeholder{B} thus can temporarily provide an effectively
stronger forward progress guarantee for a certain amount of time, due to a
second thread of execution \placeholder{A} being blocked on it with forward
progress guarantee delegation. In turn, if \placeholder{B} then blocks with
forward progress guarantee delegation on \placeholder{C}, this may also temporarily
provide a stronger forward progress guarantee to \placeholder{C}.
\end{note}

\pnum
\begin{note}
If all threads of execution in \placeholder{S} finish executing (e.g., they terminate
and do not use blocking synchronization incorrectly), then \placeholder{P}'s execution
of the operation that blocks with forward progress guarantee delegation will not
result in \placeholder{P}'s progress guarantee being effectively weakened.
\end{note}

\pnum
\begin{note}
This does not remove any constraints regarding blocking synchronization for
threads of execution providing parallel or weakly parallel forward progress
guarantees because the implementation is not required to strengthen a particular
thread of execution whose too-weak progress guarantee is preventing overall progress.
\end{note}

\pnum
An implementation should ensure that the last value (in modification order)
assigned by an atomic or synchronization operation will become visible to all
other threads in a finite period of time.%
\indextext{operation!atomic|)}%
\indextext{threads!multiple|)}

\rSec1[intro.ack]{Acknowledgments}

\pnum
The \Cpp  programming language as described in this document
is based on the language as described in Chapter R (Reference
Manual) of Stroustrup: \doccite{The \Cpp  Programming Language} (second
edition, Addison-Wesley Publishing Company, ISBN 0-201-53992-6,
copyright \copyright 1991 AT\&T). That, in turn, is based on the C
programming language as described in Appendix A of Kernighan and
Ritchie: \doccite{The C Programming Language} (Prentice-Hall, 1978, ISBN
0-13-110163-3, copyright \copyright 1978 AT\&T).

\pnum
Portions of the library Clauses of this document are based
on work by P.J. Plauger, which was published as \doccite{The Draft
Standard \Cpp  Library} (Prentice-Hall, ISBN 0-13-117003-1, copyright
\copyright 1995 P.J. Plauger).

\pnum
POSIX\textregistered\ is a registered trademark of the Institute of Electrical and
Electronic Engineers, Inc.

\pnum
ECMAScript\textregistered\ is a registered trademark of Ecma International.

\pnum
All rights in these originals are reserved.
