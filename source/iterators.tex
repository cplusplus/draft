%!TEX root = std.tex
\rSec0[iterators]{Iterators library}

\rSec1[iterators.general]{General}

\pnum
This Clause describes components that \Cpp{} programs may use to perform
iterations over containers\iref{containers},
streams\iref{iostream.format},
and stream buffers\iref{stream.buffers}.

\pnum
The following subclauses describe
iterator requirements, and
components for
iterator primitives,
predefined iterators,
and stream iterators,
as summarized in \tref{iterators.lib.summary}.

\begin{libsumtab}{Iterators library summary}{tab:iterators.lib.summary}
\ref{iterator.requirements} & Requirements        &                           \\ \rowsep
\ref{iterator.primitives} & Iterator primitives   &   \tcode{<iterator>}      \\
\ref{predef.iterators} & Predefined iterators     &                           \\
\ref{stream.iterators} & Stream iterators         &                           \\
\end{libsumtab}


\rSec1[iterator.requirements]{Iterator requirements}

\rSec2[iterator.requirements.general]{In general}

\pnum
\indextext{requirements!iterator}%
Iterators are a generalization of pointers that allow a \Cpp{} program to work with different data structures
(containers) in a uniform manner.
To be able to construct template algorithms that work correctly and
efficiently on different types of data structures, the library formalizes not just the interfaces but also the
semantics and complexity assumptions of iterators.
An input iterator
\tcode{i}
supports the expression
\tcode{*i},
resulting in a value of some object type
\tcode{T},
called the
\term{value type}
of the iterator.
An output iterator \tcode{i} has a non-empty set of types that are
\defn{writable} to the iterator;
for each such type \tcode{T}, the expression \tcode{*i = o}
is valid where \tcode{o} is a value of type \tcode{T}.
An iterator
\tcode{i}
for which the expression
\tcode{(*i).m}
is well-defined supports the expression
\tcode{i->m}
with the same semantics as
\tcode{(*i).m}.
For every iterator type
\tcode{X}
for which
equality is defined, there is a corresponding signed integer type called the
\term{difference type}
of the iterator.

\pnum
Since iterators are an abstraction of pointers, their semantics is
a generalization of most of the semantics of pointers in \Cpp{}.
This ensures that every
function template
that takes iterators
works as well with regular pointers.
This document defines
five categories of iterators, according to the operations
defined on them:
\term{input iterators},
\term{output iterators},
\term{forward iterators},
\term{bidirectional iterators}
and
\term{random access iterators},
as shown in \tref{iterators.relations}.

\begin{floattable}{Relations among iterator categories}{tab:iterators.relations}
{llll}
\topline
\textbf{Random Access}          &   $\rightarrow$ \textbf{Bidirectional}    &
$\rightarrow$ \textbf{Forward}  &   $\rightarrow$ \textbf{Input}            \\
                        &   &   &   $\rightarrow$ \textbf{Output}           \\
\end{floattable}

\pnum
Forward iterators satisfy all the requirements of input
iterators and can be used whenever
an input iterator is specified;
Bidirectional iterators also satisfy all the requirements of
forward iterators and can be used whenever a forward iterator is specified;
Random access iterators also satisfy all the requirements of bidirectional
iterators and can be used whenever a bidirectional iterator is specified.

\pnum
Iterators that further satisfy the requirements of output iterators are
called \defnx{mutable iterators}{mutable iterator}. Nonmutable iterators are referred to
as \defnx{constant iterators}{constant iterator}.

\pnum
In addition to the requirements in this subclause,
the nested \grammarterm{typedef-name}{s} specified in \ref{iterator.traits}
shall be provided for the iterator type.
\begin{note} Either the iterator type must provide the \grammarterm{typedef-name}{s} directly
(in which case \tcode{iterator_traits} pick them up automatically), or
an \tcode{iterator_traits} specialization must provide them. \end{note}

\pnum
Iterators that further satisfy the requirement that,
for integral values \tcode{n} and
dereferenceable iterator values \tcode{a} and \tcode{(a + n)},
\tcode{*(a + n)} is equivalent to \tcode{*(addressof(*a) + n)},
are called \defn{contiguous iterators}.
\begin{note}
For example, the type ``pointer to \tcode{int}'' is a contiguous iterator,
but \tcode{reverse_iterator<int *>} is not.
For a valid iterator range $[$\tcode{a}$, $\tcode{b}$)$ with dereferenceable \tcode{a},
the corresponding range denoted by pointers is
$[$\tcode{addressof(*a)}$, $\tcode{addressof(*a) + (b - a)}$)$;
\tcode{b} might not be dereferenceable.
\end{note}

\pnum
Just as a regular pointer to an array guarantees that there is a pointer value pointing past the last element
of the array, so for any iterator type there is an iterator value that points past the last element of a
corresponding sequence.
These values are called
\term{past-the-end}
values.
Values of an iterator
\tcode{i}
for which the expression
\tcode{*i}
is defined are called
\term{dereferenceable}.
The library never assumes that past-the-end values are dereferenceable.
Iterators can also have singular values that are not associated with any
sequence.
\begin{example}
After the declaration of an uninitialized pointer
\tcode{x}
(as with
\tcode{int* x;}),
\tcode{x}
must always be assumed to have a singular value of a pointer.
\end{example}
Results of most expressions are undefined for singular values;
the only exceptions are destroying an iterator that holds a singular value,
the assignment of a non-singular value to
an iterator that holds a singular value, and, for iterators that satisfy the
\tcode{DefaultConstructible} requirements, using a value-initialized iterator
as the source of a copy or move operation. \begin{note} This guarantee is not
offered for default-initialization, although the distinction only matters for types
with trivial default constructors such as pointers or aggregates holding pointers.
\end{note}
In these cases the singular
value is overwritten the same way as any other value.
Dereferenceable
values are always non-singular.

\pnum
An iterator
\tcode{j}
is called
\term{reachable}
from an iterator
\tcode{i}
if and only if there is a finite sequence of applications of
the expression
\tcode{++i}
that makes
\tcode{i == j}.
If
\tcode{j}
is reachable from
\tcode{i},
they refer to elements of the same sequence.

\pnum
Most of the library's algorithmic templates that operate on data structures have interfaces that use ranges.
A
\term{range}
is a pair of iterators that designate the beginning and end of the computation.
A range \range{i}{i}
is an empty range;
in general, a range \range{i}{j}
refers to the elements in the data structure starting with the element
pointed to by
\tcode{i}
and up to but not including the element pointed to by
\tcode{j}.
Range \range{i}{j}
is valid if and only if
\tcode{j}
is reachable from
\tcode{i}.
The result of the application of functions in the library to invalid ranges is
undefined.

\pnum
All the categories of iterators require only those functions that are realizable for a given category in
constant time (amortized).
Therefore, requirement tables for the iterators do not have a complexity column.

\pnum
Destruction of an iterator may invalidate pointers and references
previously obtained from that iterator.

\pnum
An
\term{invalid}
iterator is an iterator that may be singular.\footnote{This definition applies to pointers, since pointers are iterators.
The effect of dereferencing an iterator that has been invalidated
is undefined.
}

\pnum
\indextext{iterator!constexpr}
Iterators are called \defn{constexpr iterators}
if all operations provided to satisfy iterator category operations
are constexpr functions, except for
\begin{itemize}
\item a pseudo-destructor call\iref{expr.pseudo}, and
\item the construction of an iterator with a singular value.
\end{itemize}
\begin{note}
For example, the types ``pointer to \tcode{int}'' and
\tcode{reverse_iterator<int*>} are constexpr iterators.
\end{note}

\pnum
In the following sections,
\tcode{a}
and
\tcode{b}
denote values of type
\tcode{X} or \tcode{const X},
\tcode{difference_type} and \tcode{reference} refer to the
types \tcode{iterator_traits<X>::difference_type} and
\tcode{iterator_traits<X>::reference}, respectively,
\tcode{n}
denotes a value of
\tcode{difference_type},
\tcode{u},
\tcode{tmp},
and
\tcode{m}
denote identifiers,
\tcode{r}
denotes a value of
\tcode{X\&},
\tcode{t}
denotes a value of value type
\tcode{T},
\tcode{o}
denotes a value of some type that is writable to the output iterator.
\begin{note} For an iterator type \tcode{X} there must be an instantiation
of \tcode{iterator_traits<X>}\iref{iterator.traits}. \end{note}

\rSec2[iterator.iterators]{Iterator}

\pnum
The \tcode{Iterator} requirements form the basis of the iterator
taxonomy; every iterator satisfies the \tcode{Iterator} requirements. This
set of requirements specifies operations for dereferencing and incrementing
an iterator. Most algorithms will require additional operations to
read\iref{input.iterators} or write\iref{output.iterators} values, or
to provide a richer set of iterator movements~(\ref{forward.iterators},
\ref{bidirectional.iterators}, \ref{random.access.iterators}).

\pnum
A type \tcode{X} satisfies the \tcode{Iterator} requirements if:

\begin{itemize}
\item \tcode{X} satisfies the \tcode{CopyConstructible}, \tcode{CopyAssignable}, and
\tcode{Destructible} requirements\iref{utility.arg.requirements} and lvalues
of type \tcode{X} are swappable\iref{swappable.requirements}, and

\item the expressions in \tref{iterator.requirements} are valid and have
the indicated semantics.
\end{itemize}

\begin{libreqtab4b}
{Iterator requirements}
{tab:iterator.requirements}
\\ \topline
\lhdr{Expression}   &   \chdr{Return type}  &   \chdr{Operational}  &   \rhdr{Assertion/note}       \\
                    &                       &   \chdr{semantics}    &   \rhdr{pre-/post-condition}   \\ \capsep
\endfirsthead
\continuedcaption\\
\hline
\lhdr{Expression}   &   \chdr{Return type}  &   \chdr{Operational}  &   \rhdr{Assertion/note}       \\
                    &                       &   \chdr{semantics}    &   \rhdr{pre-/post-condition}   \\ \capsep
\endhead

\tcode{*r}          &
  unspecified       &
                            &
  \requires \tcode{r} is dereferenceable.  \\ \rowsep

\tcode{++r}         &
  \tcode{X\&}       &
                            &
                    \\

\end{libreqtab4b}

\rSec2[input.iterators]{Input iterators}

\pnum
A class or pointer type
\tcode{X}
satisfies the requirements of an input iterator for the value type
\tcode{T}
if
\tcode{X} satisfies the \tcode{Iterator}\iref{iterator.iterators} and
\tcode{EqualityComparable} (\tref{equalitycomparable}) requirements and
the expressions in \tref{iterator.input.requirements} are valid and have
the indicated semantics.

\pnum
In \tref{iterator.input.requirements}, the term
\term{the domain of \tcode{==}}
is used in the ordinary mathematical sense to denote
the set of values over which
\tcode{==} is (required to be) defined.
This set can change over time.
Each algorithm places additional requirements on the domain of
\tcode{==} for the iterator values it uses.
These requirements can be inferred from the uses that algorithm
makes of \tcode{==} and \tcode{!=}.
\begin{example}
The call \tcode{find(a,b,x)}
is defined only if the value of \tcode{a}
has the property \textit{p}
defined as follows:
\tcode{b} has property \textit{p}
and a value \tcode{i}
has property \textit{p}
if
(\tcode{*i==x})
or if
(\tcode{*i!=x}
and
\tcode{++i}
has property
\textit{p}).
\end{example}

\begin{libreqtab4b}
{Input iterator requirements (in addition to Iterator)}
{tab:iterator.input.requirements}
\\ \topline
\lhdr{Expression}   &   \chdr{Return type}  &   \chdr{Operational}  &   \rhdr{Assertion/note}       \\
                    &                       &   \chdr{semantics}    &   \rhdr{pre-/post-condition}   \\ \capsep
\endfirsthead
\continuedcaption\\
\hline
\lhdr{Expression}   &   \chdr{Return type}  &   \chdr{Operational}  &   \rhdr{Assertion/note}       \\
                    &                       &   \chdr{semantics}    &   \rhdr{pre-/post-condition}   \\ \capsep
\endhead
\tcode{a != b}                  &
 contextually convertible to \tcode{bool}    &
 \tcode{!(a == b)}                              &
 \requires \orange{a}{b} is in the domain of \tcode{==}. \\ \rowsep

\tcode{*a}                      &
 \tcode{reference}, convertible to \tcode{T}       &
                                &
 \requires \tcode{a} is dereferenceable.\br
 The expression\br \tcode{(void)*a, *a} is equivalent to \tcode{*a}.\br
 If \tcode{a == b} and \orange{a}{b} is in the domain of \tcode{==}
 then \tcode{*a} is equivalent to \tcode{*b}.  \\ \rowsep
\tcode{a->m}                    &
                                &
 \tcode{(*a).m}                                &
 \requires \tcode{a} is dereferenceable. \\ \rowsep
\tcode{++r}                     &
 \tcode{X\&}                    &
                                &
 \requires \tcode{r} is dereferenceable.\br
 \postconditions \tcode{r} is dereferenceable or \tcode{r} is past-the-end;\br
 any copies of the previous value of \tcode{r} are no longer
 required either to be dereferenceable or to be in the domain of \tcode{==}.    \\ \rowsep

\tcode{(void)r++}               &
                                &
                                &
 equivalent to \tcode{(void)++r}    \\ \rowsep

\tcode{*r++}                    &
 convertible to \tcode{T}       &
 \tcode{\{ T tmp = *r;}\br
 \tcode{++r;}\br
 \tcode{return tmp; \}} & \\
\end{libreqtab4b}

\pnum
\begin{note}
For input iterators,
\tcode{a == b}
does not imply
\tcode{++a == ++b}.
(Equality does not guarantee the substitution property or referential transparency.)
Algorithms on input iterators should never attempt to pass through the same iterator twice.
They should be
\term{single pass}
algorithms.
Value type \tcode{T} is not required to be a \tcode{CopyAssignable} type (\tref{copyassignable}).
These algorithms can be used with istreams as the source of the input data through the
\tcode{istream_iterator}
class template.
\end{note}

\rSec2[output.iterators]{Output iterators}

\pnum
A class or pointer type
\tcode{X}
satisfies the requirements of an output iterator
if \tcode{X} satisfies the \tcode{Iterator} requirements\iref{iterator.iterators}
and the expressions in \tref{iterator.output.requirements}
are valid and have the indicated semantics.

\begin{libreqtab4b}
{Output iterator requirements (in addition to Iterator)}
{tab:iterator.output.requirements}
\\ \topline
\lhdr{Expression}   &   \chdr{Return type}  &   \chdr{Operational}  &   \rhdr{Assertion/note}       \\
                    &                       &   \chdr{semantics}    &   \rhdr{pre-/post-condition}   \\ \capsep
\endfirsthead
\continuedcaption\\
\hline
\lhdr{Expression}   &   \chdr{Return type}  &   \chdr{Operational}  &   \rhdr{Assertion/note}       \\
                    &                       &   \chdr{semantics}    &   \rhdr{pre-/post-condition}   \\ \capsep
\endhead
\tcode{*r = o}      &
 result is not used &
                    &
 \remarks\ After this operation \tcode{r} is not required to be dereferenceable.\br
 \postconditions \tcode{r} is incrementable. \\ \rowsep

\tcode{++r}         &
 \tcode{X\&}        &
                    &
 \tcode{\&r == \&++r}.\br
 \remarks\ After this operation \tcode{r} is not required to be dereferenceable.\br
 \postconditions \tcode{r} is incrementable. \\ \rowsep

\tcode{r++}         &
 convertible to \tcode{const X\&}   &
 \tcode{\{ X tmp = r;}\br
 \tcode{  ++r;}\br
 \tcode{  return tmp; \}}   &
 \remarks\ After this operation \tcode{r} is not required to be dereferenceable.\br
 \postconditions \tcode{r} is incrementable. \\ \rowsep

\tcode{*r++ = o}    &
 result is not used &&
 \remarks\ After this operation \tcode{r} is not required to be dereferenceable.\br
 \postconditions \tcode{r} is incrementable. \\
\end{libreqtab4b}

\pnum
\begin{note}
The only valid use of an
\tcode{operator*}
is on the left side of the assignment statement.
\textit{Assignment through the same value of the iterator happens only once.}
Algorithms on output iterators should never attempt to pass through the same iterator twice.
They should be
\term{single pass}
algorithms.
Equality and inequality might not be defined.
Algorithms that take output iterators can be used with ostreams as the destination
for placing data through the
\tcode{ostream_iterator}
class as well as with insert iterators and insert pointers.
\end{note}

\rSec2[forward.iterators]{Forward iterators}

\pnum
A class or pointer type
\tcode{X}
satisfies the requirements of a forward iterator if

\begin{itemize}
\item \tcode{X} satisfies the requirements of an input iterator\iref{input.iterators},

\item \tcode{X} satisfies the \tcode{DefaultConstructible}
requirements\iref{utility.arg.requirements},

\item if \tcode{X} is a mutable iterator, \tcode{reference} is a reference to \tcode{T};
if \tcode{X} is a constant iterator, \tcode{reference} is a reference to \tcode{const T},

\item the expressions in \tref{iterator.forward.requirements}
are valid and have the indicated semantics, and

\item objects of type \tcode{X} offer the multi-pass guarantee, described below.
\end{itemize}

\pnum
The domain of \tcode{==} for forward iterators is that of iterators over the same
underlying sequence. However, value-initialized iterators may be compared and
shall compare equal to other value-initialized iterators of the same type.
\begin{note} Value-initialized iterators behave as if they refer past the end of
the same empty sequence. \end{note}

\pnum
Two dereferenceable iterators \tcode{a} and \tcode{b} of type \tcode{X} offer the
\defn{multi-pass guarantee} if:

\begin{itemize}
\item \tcode{a == b} implies \tcode{++a == ++b} and
\item \tcode{X} is a pointer type or the expression
\tcode{(void)++X(a), *a} is equivalent to the expression \tcode{*a}.
\end{itemize}

\pnum
\begin{note}
The requirement that
\tcode{a == b}
implies
\tcode{++a == ++b}
(which is not true for input and output iterators)
and the removal of the restrictions on the number of the assignments through
a mutable iterator
(which applies to output iterators)
allows the use of multi-pass one-directional algorithms with forward iterators.
\end{note}

\begin{libreqtab4b}
{Forward iterator requirements (in addition to input iterator)}
{tab:iterator.forward.requirements}
\\ \topline
\lhdr{Expression}   &   \chdr{Return type}  &   \chdr{Operational}  &   \rhdr{Assertion/note}       \\
                    &                       &   \chdr{semantics}    &   \rhdr{pre-/post-condition}   \\ \capsep
\endfirsthead
\continuedcaption\\
\hline
\lhdr{Expression}   &   \chdr{Return type}  &   \chdr{Operational}  &   \rhdr{Assertion/note}       \\
                    &                       &   \chdr{semantics}    &   \rhdr{pre-/post-condition}   \\ \capsep
\endhead
\tcode{r++}         &
 convertible to \tcode{const X\&}   &
 \tcode{\{ X tmp = r;}\br
 \tcode{  ++r;}\br
 \tcode{  return tmp; \}}&  \\ \rowsep

\tcode{*r++}        &
 \tcode{reference}     &&  \\
\end{libreqtab4b}

\pnum
If \tcode{a} and \tcode{b} are equal, then either \tcode{a} and \tcode{b}
are both dereferenceable
or else neither is dereferenceable.

\pnum
If \tcode{a} and \tcode{b} are both dereferenceable, then \tcode{a == b}
if and only if
\tcode{*a} and \tcode{*b} are bound to the same object.

\rSec2[bidirectional.iterators]{Bidirectional iterators}

\pnum
A class or pointer type
\tcode{X}
satisfies the requirements of a bidirectional iterator if,
in addition to satisfying the requirements for forward iterators,
the following expressions are valid as shown in \tref{iterator.bidirectional.requirements}.

\begin{libreqtab4b}
{Bidirectional iterator requirements (in addition to forward iterator)}
{tab:iterator.bidirectional.requirements}
\\ \topline
\lhdr{Expression}   &   \chdr{Return type}  &   \chdr{Operational}  &   \rhdr{Assertion/note}       \\
                    &                       &   \chdr{semantics}    &   \rhdr{pre-/post-condition}   \\ \capsep
\endfirsthead
\continuedcaption\\
\hline
\lhdr{Expression}   &   \chdr{Return type}  &   \chdr{Operational}  &   \rhdr{Assertion/note}       \\
                    &                       &   \chdr{semantics}    &   \rhdr{pre-/post-condition}   \\ \capsep
\endhead
\tcode{\dcr r}      &
 \tcode{X\&}        &
                    &
 \requires there exists \tcode{s} such that \tcode{r == ++s}.\br
 \postconditions \tcode{r} is dereferenceable.\br
 \tcode{\dcr(++r) == r}.\br
 \tcode{\dcr r == \dcr s} implies \tcode{r == s}.\br
 \tcode{\&r == \&\dcr r}.   \\ \hline

\tcode{r\dcr}           &
 convertible to \tcode{const X\&}   &
 \tcode{\{ X tmp = r;}\br
 \tcode{  \dcr r;}\br
 \tcode{  return tmp; \}}&  \\ \rowsep

\tcode{*r\dcr}      &
 \tcode{reference}   &&  \\
\end{libreqtab4b}

\pnum
\begin{note}
Bidirectional iterators allow algorithms to move iterators backward as well as forward.
\end{note}

\rSec2[random.access.iterators]{Random access iterators}

\pnum
A class or pointer type
\tcode{X}
satisfies the requirements of a random access iterator if,
in addition to satisfying the requirements for bidirectional iterators,
the following expressions are valid as shown in \tref{iterator.random.access.requirements}.

\begin{libreqtab4b}
{Random access iterator requirements (in addition to bidirectional iterator)}
{tab:iterator.random.access.requirements}
\\ \topline
\lhdr{Expression}   &   \chdr{Return type}  &   \chdr{Operational}  &   \rhdr{Assertion/note}       \\
                    &                       &   \chdr{semantics}    &   \rhdr{pre-/post-condition}   \\ \capsep
\endfirsthead
\continuedcaption\\
\hline
\lhdr{Expression}   &   \chdr{Return type}  &   \chdr{Operational}  &   \rhdr{Assertion/note}       \\
                    &                       &   \chdr{semantics}    &   \rhdr{pre-/post-condition}   \\ \capsep
\endhead
\tcode{r += n}      &
 \tcode{X\&}        &
 \tcode{\{ difference_type m = n;}\br
 \tcode{  if (m >= 0)}\br
 \tcode{    while (m\dcr)}\br
 \tcode{      ++r;}\br
 \tcode{  else}\br
 \tcode{    while (m++)}\br
 \tcode{      \dcr r;}\br
 \tcode{  return r; \}}&    \\ \rowsep

\tcode{a + n}\br
\tcode{n + a}       &
 \tcode{X}          &
 \tcode{\{ X tmp = a;}\br
 \tcode{  return tmp += n; \}}  &
 \tcode{a + n == n + a}.        \\ \rowsep

\tcode{r -= n}      &
 \tcode{X\&}        &
 \tcode{return r += -n;}    &
 \requires the absolute value of \tcode{n} is in the range of
 representable values of \tcode{difference_type}.   \\ \rowsep

\tcode{a - n}       &
 \tcode{X}          &
 \tcode{\{ X tmp = a;}\br
 \tcode{  return tmp -= n; \}}  &   \\ \rowsep

\tcode{b - a}       &
 \tcode{difference_type}   &
 \tcode{return n}   &
 \requires there exists a value \tcode{n} of type \tcode{difference_type} such that \tcode{a + n == b}.\br
 \tcode{b == a + (b - a)}.  \\ \rowsep

\tcode{a[n]}        &
 convertible to \tcode{reference}  &
 \tcode{*(a + n)}   &   \\ \rowsep

\tcode{a < b}       &
 contextually
 convertible to \tcode{bool}    &
 \tcode{b - a > 0}  &
 \tcode{<} is a total ordering relation \\ \rowsep

\tcode{a > b}       &
 contextually
 convertible to \tcode{bool}    &
 \tcode{b < a}      &
 \tcode{>} is a total ordering relation opposite to \tcode{<}.  \\ \rowsep

\tcode{a >= b}      &
 contextually
 convertible to \tcode{bool}    &
 \tcode{!(a < b)}   &   \\ \rowsep

\tcode{a <= b}      &
 contextually
 convertible to \tcode{bool}.    &
 \tcode{!(a > b)}   &   \\
\end{libreqtab4b}

\rSec1[iterator.synopsis]{Header \tcode{<iterator>}\ synopsis}

\indexhdr{iterator}%
\begin{codeblock}
namespace std {
  // \ref{iterator.primitives}, primitives
  template<class Iterator> struct iterator_traits;
  template<class T> struct iterator_traits<T*>;

  struct input_iterator_tag { };
  struct output_iterator_tag { };
  struct forward_iterator_tag: public input_iterator_tag { };
  struct bidirectional_iterator_tag: public forward_iterator_tag { };
  struct random_access_iterator_tag: public bidirectional_iterator_tag { };

  // \ref{iterator.operations}, iterator operations
  template<class InputIterator, class Distance>
    constexpr void
      advance(InputIterator& i, Distance n);
  template<class InputIterator>
    constexpr typename iterator_traits<InputIterator>::difference_type
      distance(InputIterator first, InputIterator last);
  template<class InputIterator>
    constexpr InputIterator
      next(InputIterator x,
           typename iterator_traits<InputIterator>::difference_type n = 1);
  template<class BidirectionalIterator>
    constexpr BidirectionalIterator
      prev(BidirectionalIterator x,
           typename iterator_traits<BidirectionalIterator>::difference_type n = 1);

  // \ref{predef.iterators}, predefined iterators
  template<class Iterator> class reverse_iterator;

  template<class Iterator1, class Iterator2>
    constexpr bool operator==(
      const reverse_iterator<Iterator1>& x,
      const reverse_iterator<Iterator2>& y);
  template<class Iterator1, class Iterator2>
    constexpr bool operator!=(
      const reverse_iterator<Iterator1>& x,
      const reverse_iterator<Iterator2>& y);
  template<class Iterator1, class Iterator2>
    constexpr bool operator<(
      const reverse_iterator<Iterator1>& x,
      const reverse_iterator<Iterator2>& y);
  template<class Iterator1, class Iterator2>
    constexpr bool operator>(
      const reverse_iterator<Iterator1>& x,
      const reverse_iterator<Iterator2>& y);
  template<class Iterator1, class Iterator2>
    constexpr bool operator<=(
      const reverse_iterator<Iterator1>& x,
      const reverse_iterator<Iterator2>& y);
  template<class Iterator1, class Iterator2>
    constexpr bool operator>=(
      const reverse_iterator<Iterator1>& x,
      const reverse_iterator<Iterator2>& y);

  template<class Iterator1, class Iterator2>
    constexpr auto operator-(
      const reverse_iterator<Iterator1>& x,
      const reverse_iterator<Iterator2>& y) -> decltype(y.base() - x.base());
  template<class Iterator>
    constexpr reverse_iterator<Iterator>
      operator+(
    typename reverse_iterator<Iterator>::difference_type n,
    const reverse_iterator<Iterator>& x);

  template<class Iterator>
    constexpr reverse_iterator<Iterator> make_reverse_iterator(Iterator i);

  template<class Container> class back_insert_iterator;
  template<class Container>
    back_insert_iterator<Container> back_inserter(Container& x);

  template<class Container> class front_insert_iterator;
  template<class Container>
    front_insert_iterator<Container> front_inserter(Container& x);

  template<class Container> class insert_iterator;
  template<class Container>
    insert_iterator<Container> inserter(Container& x, typename Container::iterator i);

  template<class Iterator> class move_iterator;
  template<class Iterator1, class Iterator2>
    constexpr bool operator==(
      const move_iterator<Iterator1>& x, const move_iterator<Iterator2>& y);
  template<class Iterator1, class Iterator2>
    constexpr bool operator!=(
      const move_iterator<Iterator1>& x, const move_iterator<Iterator2>& y);
  template<class Iterator1, class Iterator2>
    constexpr bool operator<(
      const move_iterator<Iterator1>& x, const move_iterator<Iterator2>& y);
  template<class Iterator1, class Iterator2>
    constexpr bool operator>(
      const move_iterator<Iterator1>& x, const move_iterator<Iterator2>& y);
  template<class Iterator1, class Iterator2>
    constexpr bool operator<=(
      const move_iterator<Iterator1>& x, const move_iterator<Iterator2>& y);
  template<class Iterator1, class Iterator2>
    constexpr bool operator>=(
      const move_iterator<Iterator1>& x, const move_iterator<Iterator2>& y);

  template<class Iterator1, class Iterator2>
    constexpr auto operator-(
    const move_iterator<Iterator1>& x,
    const move_iterator<Iterator2>& y) -> decltype(x.base() - y.base());
  template<class Iterator>
    constexpr move_iterator<Iterator> operator+(
      typename move_iterator<Iterator>::difference_type n, const move_iterator<Iterator>& x);
  template<class Iterator>
    constexpr move_iterator<Iterator> make_move_iterator(Iterator i);

  // \ref{stream.iterators}, stream iterators
  template<class T, class charT = char, class traits = char_traits<charT>,
           class Distance = ptrdiff_t>
  class istream_iterator;
  template<class T, class charT, class traits, class Distance>
    bool operator==(const istream_iterator<T,charT,traits,Distance>& x,
            const istream_iterator<T,charT,traits,Distance>& y);
  template<class T, class charT, class traits, class Distance>
    bool operator!=(const istream_iterator<T,charT,traits,Distance>& x,
            const istream_iterator<T,charT,traits,Distance>& y);

  template<class T, class charT = char, class traits = char_traits<charT>>
      class ostream_iterator;

  template<class charT, class traits = char_traits<charT>>
    class istreambuf_iterator;
  template<class charT, class traits>
    bool operator==(const istreambuf_iterator<charT,traits>& a,
            const istreambuf_iterator<charT,traits>& b);
  template<class charT, class traits>
    bool operator!=(const istreambuf_iterator<charT,traits>& a,
            const istreambuf_iterator<charT,traits>& b);

  template<class charT, class traits = char_traits<charT>>
    class ostreambuf_iterator;

  // \ref{iterator.range}, range access
  template<class C> constexpr auto begin(C& c) -> decltype(c.begin());
  template<class C> constexpr auto begin(const C& c) -> decltype(c.begin());
  template<class C> constexpr auto end(C& c) -> decltype(c.end());
  template<class C> constexpr auto end(const C& c) -> decltype(c.end());
  template<class T, size_t N> constexpr T* begin(T (&array)[N]) noexcept;
  template<class T, size_t N> constexpr T* end(T (&array)[N]) noexcept;
  template<class C> constexpr auto cbegin(const C& c) noexcept(noexcept(std::begin(c)))
    -> decltype(std::begin(c));
  template<class C> constexpr auto cend(const C& c) noexcept(noexcept(std::end(c)))
    -> decltype(std::end(c));
  template<class C> constexpr auto rbegin(C& c) -> decltype(c.rbegin());
  template<class C> constexpr auto rbegin(const C& c) -> decltype(c.rbegin());
  template<class C> constexpr auto rend(C& c) -> decltype(c.rend());
  template<class C> constexpr auto rend(const C& c) -> decltype(c.rend());
  template<class T, size_t N> constexpr reverse_iterator<T*> rbegin(T (&array)[N]);
  template<class T, size_t N> constexpr reverse_iterator<T*> rend(T (&array)[N]);
  template<class E> constexpr reverse_iterator<const E*> rbegin(initializer_list<E> il);
  template<class E> constexpr reverse_iterator<const E*> rend(initializer_list<E> il);
  template<class C> constexpr auto crbegin(const C& c) -> decltype(std::rbegin(c));
  template<class C> constexpr auto crend(const C& c) -> decltype(std::rend(c));

  // \ref{iterator.container}, container access
  template<class C> constexpr auto size(const C& c) -> decltype(c.size());
  template<class T, size_t N> constexpr size_t size(const T (&array)[N]) noexcept;
  template<class C> [[nodiscard]] constexpr auto empty(const C& c) -> decltype(c.empty());
  template<class T, size_t N> [[nodiscard]] constexpr bool empty(const T (&array)[N]) noexcept;
  template<class E> [[nodiscard]] constexpr bool empty(initializer_list<E> il) noexcept;
  template<class C> constexpr auto data(C& c) -> decltype(c.data());
  template<class C> constexpr auto data(const C& c) -> decltype(c.data());
  template<class T, size_t N> constexpr T* data(T (&array)[N]) noexcept;
  template<class E> constexpr const E* data(initializer_list<E> il) noexcept;
}
\end{codeblock}

\rSec1[iterator.primitives]{Iterator primitives}

\pnum
To simplify the task of defining iterators, the library provides
several classes and functions:

\rSec2[iterator.traits]{Iterator traits}

\pnum
\indexlibrary{\idxcode{iterator_traits}}%
To implement algorithms only in terms of iterators, it is often necessary to
determine the value and
difference types that correspond to a particular iterator type.
Accordingly, it is required that if
\tcode{Iterator}
is the type of an iterator,
the types

\indexlibrarymember{difference_type}{iterator_traits}%
\indexlibrarymember{value_type}{iterator_traits}%
\indexlibrarymember{iterator_category}{iterator_traits}%
\begin{codeblock}
iterator_traits<Iterator>::difference_type
iterator_traits<Iterator>::value_type
iterator_traits<Iterator>::iterator_category
\end{codeblock}

be defined as the iterator's difference type, value type and iterator category, respectively.
In addition, the types

\indexlibrarymember{reference}{iterator_traits}%
\indexlibrarymember{pointer}{iterator_traits}%
\begin{codeblock}
iterator_traits<Iterator>::reference
iterator_traits<Iterator>::pointer
\end{codeblock}

shall be defined as the iterator's reference and pointer types, that is, for an
iterator object \tcode{a}, the same type as the type of \tcode{*a} and \tcode{a->},
respectively. In the case of an output iterator, the types

\begin{codeblock}
iterator_traits<Iterator>::difference_type
iterator_traits<Iterator>::value_type
iterator_traits<Iterator>::reference
iterator_traits<Iterator>::pointer
\end{codeblock}

may be defined as \tcode{void}.

\pnum
If \tcode{Iterator} has valid\iref{temp.deduct} member
types \tcode{difference_type}, \tcode{value_type}, \tcode{pointer},
\tcode{reference}, and \tcode{iterator_category},
\tcode{iterator_traits<Iterator>}
shall have the following as publicly accessible members:
\begin{codeblock}
  using difference_type   = typename Iterator::difference_type;
  using value_type        = typename Iterator::value_type;
  using pointer           = typename Iterator::pointer;
  using reference         = typename Iterator::reference;
  using iterator_category = typename Iterator::iterator_category;
\end{codeblock}
Otherwise, \tcode{iterator_traits<Iterator>}
shall have no members by any of the above names.

\pnum
It is specialized for pointers as

\begin{codeblock}
namespace std {
  template<class T> struct iterator_traits<T*> {
    using difference_type   = ptrdiff_t;
    using value_type        = remove_cv_t<T>;
    using pointer           = T*;
    using reference         = T&;
    using iterator_category = random_access_iterator_tag;
  };
}
\end{codeblock}

\pnum
\begin{example}
To implement a generic
\tcode{reverse}
function, a \Cpp{} program can do the following:

\begin{codeblock}
template<class BidirectionalIterator>
void reverse(BidirectionalIterator first, BidirectionalIterator last) {
  typename iterator_traits<BidirectionalIterator>::difference_type n =
    distance(first, last);
  --n;
  while(n > 0) {
    typename iterator_traits<BidirectionalIterator>::value_type
     tmp = *first;
    *first++ = *--last;
    *last = tmp;
    n -= 2;
  }
}
\end{codeblock}
\end{example}

\rSec2[std.iterator.tags]{Standard iterator tags}

\pnum
\indexlibrary{\idxcode{input_iterator_tag}}%
\indexlibrary{\idxcode{output_iterator_tag}}%
\indexlibrary{\idxcode{forward_iterator_tag}}%
\indexlibrary{\idxcode{bidirectional_iterator_tag}}%
\indexlibrary{\idxcode{random_access_iterator_tag}}%
It is often desirable for a
function template specialization
to find out what is the most specific category of its iterator
argument, so that the function can select the most efficient algorithm at compile time.
To facilitate this, the
library introduces
\term{category tag}
classes which are used as compile time tags for algorithm selection.
They are:
\tcode{input_iterator_tag},
\tcode{output_iterator_tag},
\tcode{forward_iterator_tag},
\tcode{bidirectional_iterator_tag}
and
\tcode{random_access_iterator_tag}.
For every iterator of type
\tcode{Iterator},
\tcode{iterator_traits<Iterator>::it\-er\-a\-tor_ca\-te\-go\-ry}
shall be defined to be the most specific category tag that describes the
iterator's behavior.

\begin{codeblock}
namespace std {
  struct input_iterator_tag { };
  struct output_iterator_tag { };
  struct forward_iterator_tag: public input_iterator_tag { };
  struct bidirectional_iterator_tag: public forward_iterator_tag { };
  struct random_access_iterator_tag: public bidirectional_iterator_tag { };
}
\end{codeblock}

\pnum
\indexlibrary{\idxcode{empty}}%
\indexlibrary{\idxcode{input_iterator_tag}}%
\indexlibrary{\idxcode{output_iterator_tag}}%
\indexlibrary{\idxcode{forward_iterator_tag}}%
\indexlibrary{\idxcode{bidirectional_iterator_tag}}%
\indexlibrary{\idxcode{random_access_iterator_tag}}%
\begin{example}
For a program-defined iterator
\tcode{BinaryTreeIterator},
it could be included
into the bidirectional iterator category by specializing the
\tcode{iterator_traits}
template:

\begin{codeblock}
template<class T> struct iterator_traits<BinaryTreeIterator<T>> {
  using iterator_category = bidirectional_iterator_tag;
  using difference_type   = ptrdiff_t;
  using value_type        = T;
  using pointer           = T*;
  using reference         = T&;
};
\end{codeblock}
\end{example}

\pnum
\begin{example}
If
\tcode{evolve()}
is well-defined for bidirectional iterators, but can be implemented more
efficiently for random access iterators, then the implementation is as follows:

\begin{codeblock}
template<class BidirectionalIterator>
inline void
evolve(BidirectionalIterator first, BidirectionalIterator last) {
  evolve(first, last,
    typename iterator_traits<BidirectionalIterator>::iterator_category());
}

template<class BidirectionalIterator>
void evolve(BidirectionalIterator first, BidirectionalIterator last,
  bidirectional_iterator_tag) {
  // more generic, but less efficient algorithm
}

template<class RandomAccessIterator>
void evolve(RandomAccessIterator first, RandomAccessIterator last,
  random_access_iterator_tag) {
  // more efficient, but less generic algorithm
}
\end{codeblock}
\end{example}

\rSec2[iterator.operations]{Iterator operations}

\pnum
Since only random access iterators provide
\tcode{+}
and
\tcode{-}
operators, the library provides two
function templates
\tcode{advance}
and
\tcode{distance}.
These
function templates
use
\tcode{+}
and
\tcode{-}
for random access iterators (and are, therefore, constant
time for them); for input, forward and bidirectional iterators they use
\tcode{++}
to provide linear time
implementations.

\indexlibrary{\idxcode{advance}}%
\begin{itemdecl}
template<class InputIterator, class Distance>
  constexpr void advance(InputIterator& i, Distance n);
\end{itemdecl}

\begin{itemdescr}
\pnum
\requires
\tcode{n}
shall be negative only for bidirectional and random access iterators.

\pnum
\effects
Increments (or decrements for negative
\tcode{n})
iterator reference
\tcode{i}
by
\tcode{n}.
\end{itemdescr}

\indexlibrary{\idxcode{distance}}%
\begin{itemdecl}
template<class InputIterator>
  constexpr typename iterator_traits<InputIterator>::difference_type
    distance(InputIterator first, InputIterator last);
\end{itemdecl}

\begin{itemdescr}
\pnum
\effects
If \tcode{InputIterator} meets the requirements of random access iterator,
returns \tcode{(last - first)}; otherwise, returns
the number of increments needed to get from
\tcode{first}
to
\tcode{last}.

\pnum
\requires
If \tcode{InputIterator} meets the requirements of random access iterator,
\tcode{last} shall be reachable from \tcode{first} or \tcode{first} shall be
reachable from \tcode{last}; otherwise,
\tcode{last}
shall be reachable from
\tcode{first}.
\end{itemdescr}

\indexlibrary{\idxcode{next}}%
\begin{itemdecl}
template<class InputIterator>
  constexpr InputIterator next(InputIterator x,
    typename iterator_traits<InputIterator>::difference_type n = 1);
\end{itemdecl}

\begin{itemdescr}
\pnum
\effects Equivalent to: \tcode{advance(x, n); return x;}
\end{itemdescr}

\indexlibrary{\idxcode{prev}}%
\begin{itemdecl}
template<class BidirectionalIterator>
  constexpr BidirectionalIterator prev(BidirectionalIterator x,
    typename iterator_traits<BidirectionalIterator>::difference_type n = 1);
\end{itemdecl}

\begin{itemdescr}
\pnum
\effects Equivalent to: \tcode{advance(x, -n); return x;}
\end{itemdescr}

\rSec1[predef.iterators]{Iterator adaptors}

\rSec2[reverse.iterators]{Reverse iterators}

\pnum
Class template \tcode{reverse_iterator} is an iterator adaptor that iterates from the end of the sequence defined by its underlying iterator to the beginning of that sequence.
The fundamental relation between a reverse iterator and its corresponding iterator
\tcode{i}
is established by the identity:
\tcode{\&*(reverse_iterator(i)) == \&*(i - 1)}.

\rSec3[reverse.iterator]{Class template \tcode{reverse_iterator}}

\indexlibrary{\idxcode{reverse_iterator}}%
\begin{codeblock}
namespace std {
  template<class Iterator>
  class reverse_iterator {
  public:
    using iterator_type     = Iterator;
    using iterator_category = typename iterator_traits<Iterator>::iterator_category;
    using value_type        = typename iterator_traits<Iterator>::value_type;
    using difference_type   = typename iterator_traits<Iterator>::difference_type;
    using pointer           = typename iterator_traits<Iterator>::pointer;
    using reference         = typename iterator_traits<Iterator>::reference;

    constexpr reverse_iterator();
    constexpr explicit reverse_iterator(Iterator x);
    template<class U> constexpr reverse_iterator(const reverse_iterator<U>& u);
    template<class U> constexpr reverse_iterator& operator=(const reverse_iterator<U>& u);

    constexpr Iterator base() const;      // explicit
    constexpr reference operator*() const;
    constexpr pointer   operator->() const;

    constexpr reverse_iterator& operator++();
    constexpr reverse_iterator  operator++(int);
    constexpr reverse_iterator& operator--();
    constexpr reverse_iterator  operator--(int);

    constexpr reverse_iterator  operator+ (difference_type n) const;
    constexpr reverse_iterator& operator+=(difference_type n);
    constexpr reverse_iterator  operator- (difference_type n) const;
    constexpr reverse_iterator& operator-=(difference_type n);
    constexpr @\unspecnc@ operator[](difference_type n) const;

  protected:
    Iterator current;
  };

  template<class Iterator1, class Iterator2>
    constexpr bool operator==(
      const reverse_iterator<Iterator1>& x,
      const reverse_iterator<Iterator2>& y);
  template<class Iterator1, class Iterator2>
    constexpr bool operator!=(
      const reverse_iterator<Iterator1>& x,
      const reverse_iterator<Iterator2>& y);
  template<class Iterator1, class Iterator2>
    constexpr bool operator<(
      const reverse_iterator<Iterator1>& x,
      const reverse_iterator<Iterator2>& y);
  template<class Iterator1, class Iterator2>
    constexpr bool operator>(
      const reverse_iterator<Iterator1>& x,
      const reverse_iterator<Iterator2>& y);
  template<class Iterator1, class Iterator2>
    constexpr bool operator<=(
      const reverse_iterator<Iterator1>& x,
      const reverse_iterator<Iterator2>& y);
  template<class Iterator1, class Iterator2>
    constexpr bool operator>=(
      const reverse_iterator<Iterator1>& x,
      const reverse_iterator<Iterator2>& y);
  template<class Iterator1, class Iterator2>
    constexpr auto operator-(
      const reverse_iterator<Iterator1>& x,
      const reverse_iterator<Iterator2>& y) -> decltype(y.base() - x.base());
  template<class Iterator>
    constexpr reverse_iterator<Iterator> operator+(
      typename reverse_iterator<Iterator>::difference_type n,
      const reverse_iterator<Iterator>& x);

  template<class Iterator>
    constexpr reverse_iterator<Iterator> make_reverse_iterator(Iterator i);
}
\end{codeblock}

\rSec3[reverse.iter.requirements]{\tcode{reverse_iterator} requirements}

\pnum
The template parameter
\tcode{Iterator}
shall satisfy all the requirements of a Bidirectional Iterator\iref{bidirectional.iterators}.

\pnum
Additionally,
\tcode{Iterator}
shall satisfy the requirements of a random access iterator\iref{random.access.iterators}
if any of the members
\tcode{operator+},
\tcode{operator-},
\tcode{operator+=},
\tcode{operator-=}\iref{reverse.iter.nav},
\tcode{operator[]}\iref{reverse.iter.elem},
or the non-member operators\iref{reverse.iter.cmp}
\tcode{operator<},
\tcode{operator>},
\tcode{operator<=},
\tcode{operator>=},
\tcode{operator-},
or
\tcode{operator+}\iref{reverse.iter.nonmember}
are referenced in a way that requires instantiation\iref{temp.inst}.

\rSec3[reverse.iter.cons]{\tcode{reverse_iterator} construction and assignment}

\indexlibrary{\idxcode{reverse_iterator}!constructor}%
\begin{itemdecl}
constexpr reverse_iterator();
\end{itemdecl}

\begin{itemdescr}
\pnum
\effects
Value-initializes
\tcode{current}.
Iterator operations applied to the resulting iterator have defined behavior
if and only if the corresponding operations are defined on a value-initialized iterator of type
\tcode{Iterator}.
\end{itemdescr}

\indexlibrary{\idxcode{reverse_iterator}!constructor}%
\begin{itemdecl}
constexpr explicit reverse_iterator(Iterator x);
\end{itemdecl}

\begin{itemdescr}
\pnum
\effects
Initializes
\tcode{current}
with \tcode{x}.
\end{itemdescr}

\indexlibrary{\idxcode{reverse_iterator}!constructor}%
\begin{itemdecl}
template<class U> constexpr reverse_iterator(const reverse_iterator<U>& u);
\end{itemdecl}

\begin{itemdescr}
\pnum
\effects
Initializes
\tcode{current}
with
\tcode{u.current}.
\end{itemdescr}

\indexlibrarymember{operator=}{reverse_iterator}%
\begin{itemdecl}
template<class U>
constexpr reverse_iterator&
  operator=(const reverse_iterator<U>& u);
\end{itemdecl}

\begin{itemdescr}
\pnum
\effects
Assigns \tcode{u.base()} to current.

\pnum
\returns
\tcode{*this}.
\end{itemdescr}

\rSec3[reverse.iter.conv]{Conversion}

\indexlibrarymember{base}{reverse_iterator}%
\begin{itemdecl}
constexpr Iterator base() const;          // explicit
\end{itemdecl}

\begin{itemdescr}
\pnum
\returns
\tcode{current}.
\end{itemdescr}

\rSec3[reverse.iter.elem]{\tcode{reverse_iterator} element access}

\indexlibrarymember{operator*}{reverse_iterator}%
\begin{itemdecl}
constexpr reference operator*() const;
\end{itemdecl}

\begin{itemdescr}
\pnum
\effects
As if by:
\begin{codeblock}
Iterator tmp = current;
return *--tmp;
\end{codeblock}

\end{itemdescr}

\indexlibrarymember{operator->}{reverse_iterator}%
\begin{itemdecl}
constexpr pointer operator->() const;
\end{itemdecl}

\begin{itemdescr}
\pnum
\returns \tcode{addressof(operator*())}.
\end{itemdescr}

\indexlibrarymember{operator[]}{reverse_iterator}%
\begin{itemdecl}
constexpr @\unspec@ operator[](difference_type n) const;
\end{itemdecl}

\begin{itemdescr}
\pnum
\returns
\tcode{current[-n-1]}.
\end{itemdescr}

\rSec3[reverse.iter.nav]{\tcode{reverse_iterator} navigation}

\indexlibrarymember{operator+}{reverse_iterator}%
\begin{itemdecl}
constexpr reverse_iterator operator+(difference_type n) const;
\end{itemdecl}

\begin{itemdescr}
\pnum
\returns
\tcode{reverse_iterator(current-n)}.
\end{itemdescr}

\indexlibrarymember{operator-}{reverse_iterator}%
\begin{itemdecl}
constexpr reverse_iterator operator-(difference_type n) const;
\end{itemdecl}

\begin{itemdescr}
\pnum
\returns
\tcode{reverse_iterator(current+n)}.
\end{itemdescr}

\indexlibrarymember{operator++}{reverse_iterator}%
\begin{itemdecl}
constexpr reverse_iterator& operator++();
\end{itemdecl}

\begin{itemdescr}
\pnum
\effects
As if by: \tcode{\dcr current;}

\pnum
\returns
\tcode{*this}.
\end{itemdescr}

\indexlibrarymember{operator++}{reverse_iterator}%
\begin{itemdecl}
constexpr reverse_iterator operator++(int);
\end{itemdecl}

\begin{itemdescr}
\pnum
\effects
As if by:
\begin{codeblock}
reverse_iterator tmp = *this;
--current;
return tmp;
\end{codeblock}
\end{itemdescr}

\indexlibrarymember{operator\dcr}{reverse_iterator}%
\begin{itemdecl}
constexpr reverse_iterator& operator--();
\end{itemdecl}

\begin{itemdescr}
\pnum
\effects
As if by \tcode{++current}.

\pnum
\returns
\tcode{*this}.
\end{itemdescr}

\indexlibrarymember{operator\dcr}{reverse_iterator}%
\begin{itemdecl}
constexpr reverse_iterator operator--(int);
\end{itemdecl}

\begin{itemdescr}
\pnum
\effects
As if by:
\begin{codeblock}
reverse_iterator tmp = *this;
++current;
return tmp;
\end{codeblock}
\end{itemdescr}

\indexlibrarymember{operator+=}{reverse_iterator}%
\begin{itemdecl}
constexpr reverse_iterator& operator+=(difference_type n);
\end{itemdecl}

\begin{itemdescr}
\pnum
\effects
As if by: \tcode{current -= n;}

\pnum
\returns
\tcode{*this}.
\end{itemdescr}

\indexlibrarymember{operator-=}{reverse_iterator}%
\begin{itemdecl}
constexpr reverse_iterator& operator-=(difference_type n);
\end{itemdecl}

\begin{itemdescr}
\pnum
\effects
As if by: \tcode{current += n;}

\pnum
\returns
\tcode{*this}.
\end{itemdescr}

\rSec3[reverse.iter.cmp]{\tcode{reverse_iterator} comparisons}

\indexlibrarymember{operator==}{reverse_iterator}%
\begin{itemdecl}
template<class Iterator1, class Iterator2>
  constexpr bool operator==(
    const reverse_iterator<Iterator1>& x,
    const reverse_iterator<Iterator2>& y);
\end{itemdecl}

\begin{itemdescr}
\pnum
\returns
\tcode{x.current == y.current}.
\end{itemdescr}

\indexlibrarymember{operator"!=}{reverse_iterator}%
\begin{itemdecl}
template<class Iterator1, class Iterator2>
  constexpr bool operator!=(
    const reverse_iterator<Iterator1>& x,
    const reverse_iterator<Iterator2>& y);
\end{itemdecl}

\begin{itemdescr}
\pnum
\returns
\tcode{x.current != y.current}.
\end{itemdescr}

\indexlibrarymember{operator<}{reverse_iterator}%
\begin{itemdecl}
template<class Iterator1, class Iterator2>
  constexpr bool operator<(
    const reverse_iterator<Iterator1>& x,
    const reverse_iterator<Iterator2>& y);
\end{itemdecl}

\begin{itemdescr}
\pnum
\returns
\tcode{x.current > y.current}.
\end{itemdescr}

\indexlibrarymember{operator>}{reverse_iterator}%
\begin{itemdecl}
template<class Iterator1, class Iterator2>
  constexpr bool operator>(
    const reverse_iterator<Iterator1>& x,
    const reverse_iterator<Iterator2>& y);
\end{itemdecl}

\begin{itemdescr}
\pnum
\returns
\tcode{x.current < y.current}.
\end{itemdescr}

\indexlibrarymember{operator<=}{reverse_iterator}%
\begin{itemdecl}
template<class Iterator1, class Iterator2>
  constexpr bool operator<=(
    const reverse_iterator<Iterator1>& x,
    const reverse_iterator<Iterator2>& y);
\end{itemdecl}

\begin{itemdescr}
\pnum
\returns
\tcode{x.current >= y.current}.
\end{itemdescr}

\indexlibrarymember{operator>=}{reverse_iterator}%
\begin{itemdecl}
template<class Iterator1, class Iterator2>
  constexpr bool operator>=(
    const reverse_iterator<Iterator1>& x,
    const reverse_iterator<Iterator2>& y);
\end{itemdecl}

\begin{itemdescr}
\pnum
\returns
\tcode{x.current <= y.current}.
\end{itemdescr}

\rSec3[reverse.iter.nonmember]{Non-member functions}

\indexlibrarymember{operator-}{reverse_iterator}%
\begin{itemdecl}
template<class Iterator1, class Iterator2>
    constexpr auto operator-(
    const reverse_iterator<Iterator1>& x,
    const reverse_iterator<Iterator2>& y) -> decltype(y.base() - x.base());
\end{itemdecl}

\begin{itemdescr}
\pnum
\returns
\tcode{y.current - x.current}.
\end{itemdescr}

\indexlibrarymember{operator+}{reverse_iterator}%
\begin{itemdecl}
template<class Iterator>
  constexpr reverse_iterator<Iterator> operator+(
    typename reverse_iterator<Iterator>::difference_type n,
    const reverse_iterator<Iterator>& x);
\end{itemdecl}

\begin{itemdescr}
\pnum
\returns
\tcode{reverse_iterator<Iterator> (x.current - n)}.
\end{itemdescr}

\indexlibrary{\idxcode{reverse_iterator}!\idxcode{make_reverse_iterator} non-member function}%
\indexlibrary{\idxcode{make_reverse_iterator}}%
\begin{itemdecl}
template<class Iterator>
  constexpr reverse_iterator<Iterator> make_reverse_iterator(Iterator i);
\end{itemdecl}

\begin{itemdescr}
\pnum
\returns
\tcode{reverse_iterator<Iterator>(i)}.
\end{itemdescr}

\rSec2[insert.iterators]{Insert iterators}

\pnum
To make it possible to deal with insertion in the same way as writing into an array, a special kind of iterator
adaptors, called
\term{insert iterators},
are provided in the library.
With regular iterator classes,

\begin{codeblock}
while (first != last) *result++ = *first++;
\end{codeblock}

causes a range \range{first}{last}
to be copied into a range starting with result.
The same code with
\tcode{result}
being an insert iterator will insert corresponding elements into the container.
This device allows all of the
copying algorithms in the library to work in the
\term{insert mode}
instead of the \term{regular overwrite} mode.

\pnum
An insert iterator is constructed from a container and possibly one of its iterators pointing to where
insertion takes place if it is neither at the beginning nor at the end of the container.
Insert iterators satisfy the requirements of output iterators.
\tcode{operator*}
returns the insert iterator itself.
The assignment
\tcode{operator=(const T\& x)}
is defined on insert iterators to allow writing into them, it inserts
\tcode{x}
right before where the insert iterator is pointing.
In other words, an insert iterator is like a cursor pointing into the
container where the insertion takes place.
\tcode{back_insert_iterator}
inserts elements at the end of a container,
\tcode{front_insert_iterator}
inserts elements at the beginning of a container, and
\tcode{insert_iterator}
inserts elements where the iterator points to in a container.
\tcode{back_inserter},
\tcode{front_inserter},
and
\tcode{inserter}
are three
functions making the insert iterators out of a container.

\rSec3[back.insert.iterator]{Class template \tcode{back_insert_iterator}}

\indexlibrary{\idxcode{back_insert_iterator}}%
\begin{codeblock}
namespace std {
  template<class Container>
  class back_insert_iterator {
  protected:
    Container* container;

  public:
    using iterator_category = output_iterator_tag;
    using value_type        = void;
    using difference_type   = void;
    using pointer           = void;
    using reference         = void;
    using container_type    = Container;

    explicit back_insert_iterator(Container& x);
    back_insert_iterator& operator=(const typename Container::value_type& value);
    back_insert_iterator& operator=(typename Container::value_type&& value);

    back_insert_iterator& operator*();
    back_insert_iterator& operator++();
    back_insert_iterator  operator++(int);
  };

  template<class Container>
    back_insert_iterator<Container> back_inserter(Container& x);
}
\end{codeblock}

\rSec4[back.insert.iter.ops]{\tcode{back_insert_iterator} operations}

\indexlibrary{\idxcode{back_insert_iterator}!constructor}%
\begin{itemdecl}
explicit back_insert_iterator(Container& x);
\end{itemdecl}

\begin{itemdescr}
\pnum
\effects
Initializes
\tcode{container}
with \tcode{addressof(x)}.
\end{itemdescr}

\indexlibrarymember{operator=}{back_insert_iterator}%
\begin{itemdecl}
back_insert_iterator& operator=(const typename Container::value_type& value);
\end{itemdecl}

\begin{itemdescr}
\pnum
\effects
As if by: \tcode{container->push_back(value);}

\pnum
\returns
\tcode{*this}.
\end{itemdescr}

\indexlibrarymember{operator=}{back_insert_iterator}%
\begin{itemdecl}
back_insert_iterator& operator=(typename Container::value_type&& value);
\end{itemdecl}

\begin{itemdescr}
\pnum
\effects
As if by: \tcode{container->push_back(std::move(value));}

\pnum
\returns
\tcode{*this}.
\end{itemdescr}

\indexlibrarymember{operator*}{back_insert_iterator}%
\begin{itemdecl}
back_insert_iterator& operator*();
\end{itemdecl}

\begin{itemdescr}
\pnum
\returns
\tcode{*this}.
\end{itemdescr}

\indexlibrarymember{operator++}{back_insert_iterator}%
\begin{itemdecl}
back_insert_iterator& operator++();
back_insert_iterator  operator++(int);
\end{itemdecl}

\begin{itemdescr}
\pnum
\returns
\tcode{*this}.
\end{itemdescr}

\rSec4[back.inserter]{ \tcode{back_inserter}}

\indexlibrary{\idxcode{back_inserter}}%
\begin{itemdecl}
template<class Container>
  back_insert_iterator<Container> back_inserter(Container& x);
\end{itemdecl}

\begin{itemdescr}
\pnum
\returns
\tcode{back_insert_iterator<Container>(x)}.
\end{itemdescr}

\rSec3[front.insert.iterator]{Class template \tcode{front_insert_iterator}}

\indexlibrary{\idxcode{front_insert_iterator}}%
\begin{codeblock}
namespace std {
  template<class Container>
  class front_insert_iterator {
  protected:
    Container* container;

  public:
    using iterator_category = output_iterator_tag;
    using value_type        = void;
    using difference_type   = void;
    using pointer           = void;
    using reference         = void;
    using container_type    = Container;

    explicit front_insert_iterator(Container& x);
    front_insert_iterator& operator=(const typename Container::value_type& value);
    front_insert_iterator& operator=(typename Container::value_type&& value);

    front_insert_iterator& operator*();
    front_insert_iterator& operator++();
    front_insert_iterator  operator++(int);
  };

  template<class Container>
    front_insert_iterator<Container> front_inserter(Container& x);
}
\end{codeblock}

\rSec4[front.insert.iter.ops]{\tcode{front_insert_iterator} operations}

\indexlibrary{\idxcode{front_insert_iterator}!constructor}%
\begin{itemdecl}
explicit front_insert_iterator(Container& x);
\end{itemdecl}

\begin{itemdescr}
\pnum
\effects
Initializes
\tcode{container}
with \tcode{addressof(x)}.
\end{itemdescr}

\indexlibrarymember{operator=}{front_insert_iterator}%
\begin{itemdecl}
front_insert_iterator& operator=(const typename Container::value_type& value);
\end{itemdecl}

\begin{itemdescr}
\pnum
\effects
As if by: \tcode{container->push_front(value);}

\pnum
\returns
\tcode{*this}.
\end{itemdescr}

\indexlibrarymember{operator=}{front_insert_iterator}%
\begin{itemdecl}
front_insert_iterator& operator=(typename Container::value_type&& value);
\end{itemdecl}

\begin{itemdescr}
\pnum
\effects
As if by: \tcode{container->push_front(std::move(value));}

\pnum
\returns
\tcode{*this}.
\end{itemdescr}

\indexlibrarymember{operator*}{front_insert_iterator}%
\begin{itemdecl}
front_insert_iterator& operator*();
\end{itemdecl}

\begin{itemdescr}
\pnum
\returns
\tcode{*this}.
\end{itemdescr}

\indexlibrarymember{operator++}{front_insert_iterator}%
\begin{itemdecl}
front_insert_iterator& operator++();
front_insert_iterator  operator++(int);
\end{itemdecl}

\begin{itemdescr}
\pnum
\returns
\tcode{*this}.
\end{itemdescr}

\rSec4[front.inserter]{\tcode{front_inserter}}

\indexlibrary{\idxcode{front_inserter}}%
\begin{itemdecl}
template<class Container>
  front_insert_iterator<Container> front_inserter(Container& x);
\end{itemdecl}

\begin{itemdescr}
\pnum
\returns
\tcode{front_insert_iterator<Container>(x)}.
\end{itemdescr}

\rSec3[insert.iterator]{Class template \tcode{insert_iterator}}

\indexlibrary{\idxcode{insert_iterator}}%
\begin{codeblock}
namespace std {
  template<class Container>
  class insert_iterator {
  protected:
    Container* container;
    typename Container::iterator iter;

  public:
    using iterator_category = output_iterator_tag;
    using value_type        = void;
    using difference_type   = void;
    using pointer           = void;
    using reference         = void;
    using container_type    = Container;

    insert_iterator(Container& x, typename Container::iterator i);
    insert_iterator& operator=(const typename Container::value_type& value);
    insert_iterator& operator=(typename Container::value_type&& value);

    insert_iterator& operator*();
    insert_iterator& operator++();
    insert_iterator& operator++(int);
  };

  template<class Container>
    insert_iterator<Container> inserter(Container& x, typename Container::iterator i);
}
\end{codeblock}

\rSec4[insert.iter.ops]{\tcode{insert_iterator} operations}

\indexlibrary{\idxcode{insert_iterator}!constructor}%
\begin{itemdecl}
insert_iterator(Container& x, typename Container::iterator i);
\end{itemdecl}

\begin{itemdescr}
\pnum
\effects
Initializes
\tcode{container}
with \tcode{addressof(x)} and
\tcode{iter}
with \tcode{i}.
\end{itemdescr}

\indexlibrarymember{operator=}{insert_iterator}%
\begin{itemdecl}
insert_iterator& operator=(const typename Container::value_type& value);
\end{itemdecl}

\begin{itemdescr}
\pnum
\effects
As if by:
\begin{codeblock}
iter = container->insert(iter, value);
++iter;
\end{codeblock}

\pnum
\returns
\tcode{*this}.
\end{itemdescr}

\indexlibrarymember{operator=}{insert_iterator}%
\begin{itemdecl}
insert_iterator& operator=(typename Container::value_type&& value);
\end{itemdecl}

\begin{itemdescr}
\pnum
\effects
As if by:
\begin{codeblock}
iter = container->insert(iter, std::move(value));
++iter;
\end{codeblock}

\pnum
\returns
\tcode{*this}.
\end{itemdescr}

\indexlibrarymember{operator*}{insert_iterator}%
\begin{itemdecl}
insert_iterator& operator*();
\end{itemdecl}

\begin{itemdescr}
\pnum
\returns
\tcode{*this}.
\end{itemdescr}

\indexlibrarymember{operator++}{insert_iterator}%
\begin{itemdecl}
insert_iterator& operator++();
insert_iterator& operator++(int);
\end{itemdecl}

\begin{itemdescr}
\pnum
\returns
\tcode{*this}.
\end{itemdescr}

\rSec4[inserter]{\tcode{inserter}}

\indexlibrary{\idxcode{inserter}}%
\begin{itemdecl}
template<class Container>
  insert_iterator<Container> inserter(Container& x, typename Container::iterator i);
\end{itemdecl}

\begin{itemdescr}
\pnum
\returns
\tcode{insert_iterator<Container>(x, i)}.
\end{itemdescr}

\rSec2[move.iterators]{Move iterators}

\pnum
Class template \tcode{move_iterator} is an iterator adaptor
with the same behavior as the underlying iterator except that its
indirection operator implicitly converts the value returned by the
underlying iterator's indirection operator to an rvalue.
Some generic algorithms can be called with move iterators to replace
copying with moving.

\pnum
\begin{example}

\begin{codeblock}
list<string> s;
// populate the list \tcode{s}
vector<string> v1(s.begin(), s.end());          // copies strings into \tcode{v1}
vector<string> v2(make_move_iterator(s.begin()),
                  make_move_iterator(s.end())); // moves strings into \tcode{v2}
\end{codeblock}

\end{example}

\rSec3[move.iterator]{Class template \tcode{move_iterator}}

\indexlibrary{\idxcode{move_iterator}}%
\begin{codeblock}
namespace std {
  template<class Iterator>
  class move_iterator {
  public:
    using iterator_type     = Iterator;
    using iterator_category = typename iterator_traits<Iterator>::iterator_category;
    using value_type        = typename iterator_traits<Iterator>::value_type;
    using difference_type   = typename iterator_traits<Iterator>::difference_type;
    using pointer           = Iterator;
    using reference         = @\seebelow@;

    constexpr move_iterator();
    constexpr explicit move_iterator(Iterator i);
    template<class U> constexpr move_iterator(const move_iterator<U>& u);
    template<class U> constexpr move_iterator& operator=(const move_iterator<U>& u);

    constexpr iterator_type base() const;
    constexpr reference operator*() const;
    constexpr pointer operator->() const;

    constexpr move_iterator& operator++();
    constexpr move_iterator operator++(int);
    constexpr move_iterator& operator--();
    constexpr move_iterator operator--(int);

    constexpr move_iterator operator+(difference_type n) const;
    constexpr move_iterator& operator+=(difference_type n);
    constexpr move_iterator operator-(difference_type n) const;
    constexpr move_iterator& operator-=(difference_type n);
    constexpr @\unspec@ operator[](difference_type n) const;

  private:
    Iterator current;   // \expos
  };

  template<class Iterator1, class Iterator2>
    constexpr bool operator==(
      const move_iterator<Iterator1>& x, const move_iterator<Iterator2>& y);
  template<class Iterator1, class Iterator2>
    constexpr bool operator!=(
      const move_iterator<Iterator1>& x, const move_iterator<Iterator2>& y);
  template<class Iterator1, class Iterator2>
    constexpr bool operator<(
      const move_iterator<Iterator1>& x, const move_iterator<Iterator2>& y);
  template<class Iterator1, class Iterator2>
    constexpr bool operator>(
      const move_iterator<Iterator1>& x, const move_iterator<Iterator2>& y);
  template<class Iterator1, class Iterator2>
    constexpr bool operator<=(
      const move_iterator<Iterator1>& x, const move_iterator<Iterator2>& y);
  template<class Iterator1, class Iterator2>
    constexpr bool operator>=(
      const move_iterator<Iterator1>& x, const move_iterator<Iterator2>& y);

  template<class Iterator1, class Iterator2>
    constexpr auto operator-(
      const move_iterator<Iterator1>& x,
      const move_iterator<Iterator2>& y) -> decltype(x.base() - y.base());
  template<class Iterator>
    constexpr move_iterator<Iterator> operator+(
      typename move_iterator<Iterator>::difference_type n, const move_iterator<Iterator>& x);
  template<class Iterator>
    constexpr move_iterator<Iterator> make_move_iterator(Iterator i);
}
\end{codeblock}

\pnum
Let \tcode{\placeholder{R}} denote \tcode{iterator_traits<Iterator>::reference}.
If \tcode{is_reference_v<\placeholder{R}>} is \tcode{true},
the template specialization \tcode{move_iterator<Iterator>} shall define
the nested type named \tcode{reference} as a synonym for
\tcode{remove_reference_t<\placeholder{R}>\&\&},
otherwise as a synonym for \tcode{\placeholder{R}}.

\rSec3[move.iter.requirements]{\tcode{move_iterator} requirements}

\pnum
The template parameter \tcode{Iterator} shall satisfy
the requirements of an input iterator\iref{input.iterators}.
Additionally, if any of the bidirectional or random access traversal
functions are instantiated, the template parameter shall satisfy the
requirements for a Bidirectional Iterator\iref{bidirectional.iterators}
or a Random Access Iterator\iref{random.access.iterators}, respectively.

\rSec3[move.iter.ops]{\tcode{move_iterator} operations}

\rSec4[move.iter.op.const]{\tcode{move_iterator} constructors}

\indexlibrary{\idxcode{move_iterator}!constructor}%
\begin{itemdecl}
constexpr move_iterator();
\end{itemdecl}

\begin{itemdescr}
\pnum
\effects Constructs a \tcode{move_iterator}, value-initializing
\tcode{current}. Iterator operations applied to the resulting
iterator have defined behavior if and only if the corresponding operations are defined
on a value-initialized iterator of type \tcode{Iterator}.
\end{itemdescr}


\indexlibrary{\idxcode{move_iterator}!constructor}%
\begin{itemdecl}
constexpr explicit move_iterator(Iterator i);
\end{itemdecl}

\begin{itemdescr}
\pnum
\effects Constructs a \tcode{move_iterator}, initializing
\tcode{current} with \tcode{i}.
\end{itemdescr}


\indexlibrary{\idxcode{move_iterator}!constructor}%
\begin{itemdecl}
template<class U> constexpr move_iterator(const move_iterator<U>& u);
\end{itemdecl}

\begin{itemdescr}
\pnum
\effects Constructs a \tcode{move_iterator}, initializing
\tcode{current} with \tcode{u.base()}.

\pnum
\requires \tcode{U} shall be convertible to
\tcode{Iterator}.
\end{itemdescr}

\rSec4[move.iter.op=]{\tcode{move_iterator::operator=}}

\indexlibrarymember{operator=}{move_iterator}%
\begin{itemdecl}
template<class U> constexpr move_iterator& operator=(const move_iterator<U>& u);
\end{itemdecl}

\begin{itemdescr}
\pnum
\effects Assigns \tcode{u.base()} to
\tcode{current}.

\pnum
\requires \tcode{U} shall be convertible to
\tcode{Iterator}.
\end{itemdescr}

\rSec4[move.iter.op.conv]{\tcode{move_iterator} conversion}

\indexlibrarymember{base}{move_iterator}%
\begin{itemdecl}
constexpr Iterator base() const;
\end{itemdecl}

\begin{itemdescr}
\pnum
\returns \tcode{current}.
\end{itemdescr}

\rSec4[move.iter.op.star]{\tcode{move_iterator::operator*}}

\indexlibrarymember{operator*}{move_iterator}%
\begin{itemdecl}
constexpr reference operator*() const;
\end{itemdecl}

\begin{itemdescr}
\pnum
\returns \tcode{static_cast<reference>(*current)}.
\end{itemdescr}

\rSec4[move.iter.op.ref]{\tcode{move_iterator::operator->}}

\indexlibrarymember{operator->}{move_iterator}%
\begin{itemdecl}
constexpr pointer operator->() const;
\end{itemdecl}

\begin{itemdescr}
\pnum
\returns \tcode{current}.
\end{itemdescr}

\rSec4[move.iter.op.incr]{\tcode{move_iterator::operator++}}

\indexlibrarymember{operator++}{move_iterator}%
\begin{itemdecl}
constexpr move_iterator& operator++();
\end{itemdecl}

\begin{itemdescr}
\pnum
\effects As if by \tcode{++current}.

\pnum
\returns \tcode{*this}.
\end{itemdescr}

\indexlibrarymember{operator++}{move_iterator}%
\begin{itemdecl}
constexpr move_iterator operator++(int);
\end{itemdecl}

\begin{itemdescr}
\pnum
\effects
As if by:
\begin{codeblock}
move_iterator tmp = *this;
++current;
return tmp;
\end{codeblock}
\end{itemdescr}

\rSec4[move.iter.op.decr]{\tcode{move_iterator::operator-{-}}}

\indexlibrarymember{operator\dcr}{move_iterator}%
\begin{itemdecl}
constexpr move_iterator& operator--();
\end{itemdecl}

\begin{itemdescr}
\pnum
\effects As if by \tcode{\dcr current}.

\pnum
\returns \tcode{*this}.
\end{itemdescr}

\indexlibrarymember{operator\dcr}{move_iterator}%
\begin{itemdecl}
constexpr move_iterator operator--(int);
\end{itemdecl}

\begin{itemdescr}
\pnum
\effects
As if by:
\begin{codeblock}
move_iterator tmp = *this;
--current;
return tmp;
\end{codeblock}
\end{itemdescr}

\rSec4[move.iter.op.+]{\tcode{move_iterator::operator+}}

\indexlibrarymember{operator+}{move_iterator}%
\begin{itemdecl}
constexpr move_iterator operator+(difference_type n) const;
\end{itemdecl}

\begin{itemdescr}
\pnum
\returns \tcode{move_iterator(current + n)}.
\end{itemdescr}

\rSec4[move.iter.op.+=]{\tcode{move_iterator::operator+=}}

\indexlibrarymember{operator+=}{move_iterator}%
\begin{itemdecl}
constexpr move_iterator& operator+=(difference_type n);
\end{itemdecl}

\begin{itemdescr}
\pnum
\effects As if by: \tcode{current += n;}

\pnum
\returns \tcode{*this}.
\end{itemdescr}

\rSec4[move.iter.op.-]{\tcode{move_iterator::operator-}}

\indexlibrarymember{operator-}{move_iterator}%
\begin{itemdecl}
constexpr move_iterator operator-(difference_type n) const;
\end{itemdecl}

\begin{itemdescr}
\pnum
\returns \tcode{move_iterator(current - n)}.
\end{itemdescr}

\rSec4[move.iter.op.-=]{\tcode{move_iterator::operator-=}}

\indexlibrarymember{operator-=}{move_iterator}%
\begin{itemdecl}
constexpr move_iterator& operator-=(difference_type n);
\end{itemdecl}

\begin{itemdescr}
\pnum
\effects As if by: \tcode{current -= n;}

\pnum
\returns \tcode{*this}.
\end{itemdescr}

\rSec4[move.iter.op.index]{\tcode{move_iterator::operator[]}}

\indexlibrarymember{operator[]}{move_iterator}%
\begin{itemdecl}
constexpr @\unspec@ operator[](difference_type n) const;
\end{itemdecl}

\begin{itemdescr}
\pnum
\returns \tcode{std::move(current[n])}.
\end{itemdescr}

\rSec4[move.iter.op.comp]{\tcode{move_iterator} comparisons}

\indexlibrarymember{operator==}{move_iterator}%
\begin{itemdecl}
template<class Iterator1, class Iterator2>
constexpr bool operator==(const move_iterator<Iterator1>& x, const move_iterator<Iterator2>& y);
\end{itemdecl}

\begin{itemdescr}
\pnum
\returns \tcode{x.base() == y.base()}.
\end{itemdescr}

\indexlibrarymember{operator"!=}{move_iterator}%
\begin{itemdecl}
template<class Iterator1, class Iterator2>
constexpr bool operator!=(const move_iterator<Iterator1>& x, const move_iterator<Iterator2>& y);
\end{itemdecl}

\begin{itemdescr}
\pnum
\returns \tcode{!(x == y)}.
\end{itemdescr}

\indexlibrarymember{operator<}{move_iterator}%
\begin{itemdecl}
template<class Iterator1, class Iterator2>
constexpr bool operator<(const move_iterator<Iterator1>& x, const move_iterator<Iterator2>& y);
\end{itemdecl}

\begin{itemdescr}
\pnum
\returns \tcode{x.base() < y.base()}.
\end{itemdescr}

\indexlibrarymember{operator>}{move_iterator}%
\begin{itemdecl}
template<class Iterator1, class Iterator2>
constexpr bool operator>(const move_iterator<Iterator1>& x, const move_iterator<Iterator2>& y);
\end{itemdecl}

\begin{itemdescr}
\pnum
\returns \tcode{y < x}.
\end{itemdescr}

\indexlibrarymember{operator<=}{move_iterator}%
\begin{itemdecl}
template<class Iterator1, class Iterator2>
constexpr bool operator<=(const move_iterator<Iterator1>& x, const move_iterator<Iterator2>& y);
\end{itemdecl}

\begin{itemdescr}
\pnum
\returns \tcode{!(y < x)}.
\end{itemdescr}

\indexlibrarymember{operator>=}{move_iterator}%
\begin{itemdecl}
template<class Iterator1, class Iterator2>
constexpr bool operator>=(const move_iterator<Iterator1>& x, const move_iterator<Iterator2>& y);
\end{itemdecl}

\begin{itemdescr}
\pnum
\returns \tcode{!(x < y)}.
\end{itemdescr}

\rSec4[move.iter.nonmember]{\tcode{move_iterator} non-member functions}

\indexlibrarymember{operator-}{move_iterator}%
\begin{itemdecl}
template<class Iterator1, class Iterator2>
    constexpr auto operator-(
    const move_iterator<Iterator1>& x,
    const move_iterator<Iterator2>& y) -> decltype(x.base() - y.base());
\end{itemdecl}

\begin{itemdescr}
\pnum
\returns \tcode{x.base() - y.base()}.
\end{itemdescr}

\indexlibrarymember{operator+}{move_iterator}%
\begin{itemdecl}
template<class Iterator>
  constexpr move_iterator<Iterator> operator+(
    typename move_iterator<Iterator>::difference_type n, const move_iterator<Iterator>& x);
\end{itemdecl}

\begin{itemdescr}
\pnum
\returns \tcode{x + n}.
\end{itemdescr}

\indexlibrary{\idxcode{make_move_iterator}}%
\begin{itemdecl}
template<class Iterator>
constexpr move_iterator<Iterator> make_move_iterator(Iterator i);
\end{itemdecl}

\begin{itemdescr}
\pnum
\returns \tcode{move_iterator<Iterator>(i)}.
\end{itemdescr}

\rSec1[stream.iterators]{Stream iterators}

\pnum
To make it possible for algorithmic templates to work directly with input/output streams, appropriate
iterator-like
class templates
are provided.

\begin{example}
\begin{codeblock}
partial_sum(istream_iterator<double, char>(cin),
  istream_iterator<double, char>(),
  ostream_iterator<double, char>(cout, "@\textbackslash@n"));
\end{codeblock}

reads a file containing floating-point numbers from
\tcode{cin},
and prints the partial sums onto
\tcode{cout}.
\end{example}

\rSec2[istream.iterator]{Class template \tcode{istream_iterator}}

\pnum
\indexlibrary{\idxcode{istream_iterator}}%
The class template
\tcode{istream_iterator}
is an input iterator\iref{input.iterators} that
reads (using
\tcode{operator>>})
successive elements from the input stream for which it was constructed.
After it is constructed, and every time
\tcode{++}
is used, the iterator reads and stores a value of
\tcode{T}.
If the iterator fails to read and store a value of \tcode{T}
(\tcode{fail()}
on the stream returns
\tcode{true}),
the iterator becomes equal to the
\term{end-of-stream}
iterator value.
The constructor with no arguments
\tcode{istream_iterator()}
always constructs
an end-of-stream input iterator object, which is the only legitimate iterator to be used
for the end condition.
The result of
\tcode{operator*}
on an end-of-stream iterator is not defined.
For any other iterator value a
\tcode{const T\&}
is returned.
The result of
\tcode{operator->}
on an end-of-stream iterator is not defined.
For any other iterator value a
\tcode{const T*}
is returned.
The behavior of a program that applies \tcode{operator++()} to an end-of-stream
iterator is undefined.
It is impossible to store things into istream iterators.
The type \tcode{T} shall satisfy the \tcode{DefaultConstructible},
\tcode{CopyConstructible}, and \tcode{CopyAssignable} requirements.

\pnum
Two end-of-stream iterators are always equal.
An end-of-stream iterator is not
equal to a non-end-of-stream iterator.
Two non-end-of-stream iterators are equal when they are constructed from the same stream.

\begin{codeblock}
namespace std {
  template<class T, class charT = char, class traits = char_traits<charT>,
           class Distance = ptrdiff_t>
  class istream_iterator {
  public:
    using iterator_category = input_iterator_tag;
    using value_type        = T;
    using difference_type   = Distance;
    using pointer           = const T*;
    using reference         = const T&;
    using char_type         = charT;
    using traits_type       = traits;
    using istream_type      = basic_istream<charT,traits>;

    constexpr istream_iterator();
    istream_iterator(istream_type& s);
    istream_iterator(const istream_iterator& x) = default;
    ~istream_iterator() = default;

    const T& operator*() const;
    const T* operator->() const;
    istream_iterator& operator++();
    istream_iterator  operator++(int);

  private:
    basic_istream<charT,traits>* in_stream; // \expos
    T value;                                // \expos
  };

  template<class T, class charT, class traits, class Distance>
    bool operator==(const istream_iterator<T,charT,traits,Distance>& x,
            const istream_iterator<T,charT,traits,Distance>& y);
  template<class T, class charT, class traits, class Distance>
    bool operator!=(const istream_iterator<T,charT,traits,Distance>& x,
            const istream_iterator<T,charT,traits,Distance>& y);
}
\end{codeblock}

\rSec3[istream.iterator.cons]{\tcode{istream_iterator} constructors and destructor}


\indexlibrary{\idxcode{istream_iterator}!constructor}%
\begin{itemdecl}
constexpr istream_iterator();
\end{itemdecl}

\begin{itemdescr}
\pnum
\effects
Constructs the end-of-stream iterator.
If \tcode{is_trivially_default_constructible_v<T>} is \tcode{true},
then this constructor is a constexpr constructor.

\pnum
\postconditions \tcode{in_stream == 0}.
\end{itemdescr}


\indexlibrary{\idxcode{istream_iterator}!constructor}%
\begin{itemdecl}
istream_iterator(istream_type& s);
\end{itemdecl}

\begin{itemdescr}
\pnum
\effects
Initializes \tcode{in_stream} with \tcode{addressof(s)}.
\tcode{value} may be initialized during
construction or the first time it is referenced.

\pnum
\postconditions \tcode{in_stream == addressof(s)}.
\end{itemdescr}


\indexlibrary{\idxcode{istream_iterator}!constructor}%
\begin{itemdecl}
istream_iterator(const istream_iterator& x) = default;
\end{itemdecl}

\begin{itemdescr}
\pnum
\effects
Constructs a copy of \tcode{x}.
If \tcode{is_trivially_copy_constructible_v<T>} is \tcode{true},
then this constructor is a trivial copy constructor.

\pnum
\postconditions \tcode{in_stream == x.in_stream}.
\end{itemdescr}

\indexlibrary{\idxcode{istream_iterator}!destructor}%
\begin{itemdecl}
~istream_iterator() = default;
\end{itemdecl}

\begin{itemdescr}
\pnum
\effects
The iterator is destroyed.
If \tcode{is_trivially_destructible_v<T>} is \tcode{true},
then this destructor is trivial.
\end{itemdescr}

\rSec3[istream.iterator.ops]{\tcode{istream_iterator} operations}

\indexlibrarymember{operator*}{istream_iterator}%
\begin{itemdecl}
const T& operator*() const;
\end{itemdecl}

\begin{itemdescr}
\pnum
\returns
\tcode{value}.
\end{itemdescr}

\indexlibrarymember{operator->}{istream_iterator}%
\begin{itemdecl}
const T* operator->() const;
\end{itemdecl}

\begin{itemdescr}
\pnum
\returns
\tcode{addressof(operator*())}.
\end{itemdescr}

\indexlibrarymember{operator++}{istream_iterator}%
\begin{itemdecl}
istream_iterator& operator++();
\end{itemdecl}

\begin{itemdescr}
\pnum
\requires \tcode{in_stream != 0}.

\pnum
\effects
As if by: \tcode{*in_stream >> value;}

\pnum
\returns
\tcode{*this}.
\end{itemdescr}

\indexlibrarymember{operator++}{istream_iterator}%
\begin{itemdecl}
istream_iterator operator++(int);
\end{itemdecl}

\begin{itemdescr}
\pnum
\requires \tcode{in_stream != 0}.

\pnum
\effects
As if by:
\begin{codeblock}
istream_iterator tmp = *this;
*in_stream >> value;
return (tmp);
\end{codeblock}
\end{itemdescr}

\indexlibrarymember{operator==}{istream_iterator}%
\begin{itemdecl}
template<class T, class charT, class traits, class Distance>
  bool operator==(const istream_iterator<T,charT,traits,Distance>& x,
                  const istream_iterator<T,charT,traits,Distance>& y);
\end{itemdecl}

\begin{itemdescr}
\pnum
\returns
\tcode{x.in_stream == y.in_stream}.%
\end{itemdescr}

\indexlibrarymember{operator"!=}{istream_iterator}%
\begin{itemdecl}
template<class T, class charT, class traits, class Distance>
  bool operator!=(const istream_iterator<T,charT,traits,Distance>& x,
                  const istream_iterator<T,charT,traits,Distance>& y);
\end{itemdecl}

\begin{itemdescr}
\pnum
\returns
\tcode{!(x == y)}
\end{itemdescr}

\rSec2[ostream.iterator]{Class template \tcode{ostream_iterator}}

\pnum
\indexlibrary{\idxcode{ostream_iterator}}%
\tcode{ostream_iterator}
writes (using
\tcode{operator<<})
successive elements onto the output stream from which it was constructed.
If it was constructed with
\tcode{charT*}
as a constructor argument, this string, called a
\term{delimiter string},
is written to the stream after every
\tcode{T}
is written.
It is not possible to get a value out of the output iterator.
Its only use is as an output iterator in situations like

\begin{codeblock}
while (first != last)
  *result++ = *first++;
\end{codeblock}

\pnum
\tcode{ostream_iterator}
is defined as:

\begin{codeblock}
namespace std {
  template<class T, class charT = char, class traits = char_traits<charT>>
  class ostream_iterator {
  public:
    using iterator_category = output_iterator_tag;
    using value_type        = void;
    using difference_type   = void;
    using pointer           = void;
    using reference         = void;
    using char_type         = charT;
    using traits_type       = traits;
    using ostream_type      = basic_ostream<charT,traits>;

    ostream_iterator(ostream_type& s);
    ostream_iterator(ostream_type& s, const charT* delimiter);
    ostream_iterator(const ostream_iterator& x);
    ~ostream_iterator();
    ostream_iterator& operator=(const T& value);

    ostream_iterator& operator*();
    ostream_iterator& operator++();
    ostream_iterator& operator++(int);

  private:
    basic_ostream<charT,traits>* out_stream;  // \expos
    const charT* delim;                       // \expos
  };
}
\end{codeblock}

\rSec3[ostream.iterator.cons.des]{\tcode{ostream_iterator} constructors and destructor}


\indexlibrary{\idxcode{ostream_iterator}!constructor}%
\begin{itemdecl}
ostream_iterator(ostream_type& s);
\end{itemdecl}

\begin{itemdescr}
\pnum
\effects
Initializes \tcode{out_stream} with \tcode{addressof(s)} and
\tcode{delim} with null.
\end{itemdescr}


\indexlibrary{\idxcode{ostream_iterator}!constructor}%
\begin{itemdecl}
ostream_iterator(ostream_type& s, const charT* delimiter);
\end{itemdecl}

\begin{itemdescr}
\pnum
\effects
Initializes \tcode{out_stream} with \tcode{addressof(s)} and
\tcode{delim} with \tcode{delimiter}.
\end{itemdescr}


\indexlibrary{\idxcode{ostream_iterator}!constructor}%
\begin{itemdecl}
ostream_iterator(const ostream_iterator& x);
\end{itemdecl}

\begin{itemdescr}
\pnum
\effects
Constructs a copy of \tcode{x}.
\end{itemdescr}

\indexlibrary{\idxcode{ostream_iterator}!destructor}%
\begin{itemdecl}
~ostream_iterator();
\end{itemdecl}

\begin{itemdescr}
\pnum
\effects
The iterator is destroyed.
\end{itemdescr}

\rSec3[ostream.iterator.ops]{\tcode{ostream_iterator} operations}

\indexlibrarymember{operator=}{ostream_iterator}%
\begin{itemdecl}
ostream_iterator& operator=(const T& value);
\end{itemdecl}

\begin{itemdescr}
\pnum
\effects
As if by:
\begin{codeblock}
*out_stream << value;
if (delim != 0)
  *out_stream << delim;
return *this;
\end{codeblock}
\end{itemdescr}

\indexlibrarymember{operator*}{ostream_iterator}%
\begin{itemdecl}
ostream_iterator& operator*();
\end{itemdecl}

\begin{itemdescr}
\pnum
\returns
\tcode{*this}.
\end{itemdescr}

\indexlibrarymember{operator++}{ostream_iterator}%
\begin{itemdecl}
ostream_iterator& operator++();
ostream_iterator& operator++(int);
\end{itemdecl}

\begin{itemdescr}
\pnum
\returns
\tcode{*this}.
\end{itemdescr}

\rSec2[istreambuf.iterator]{Class template \tcode{istreambuf_iterator}}

\pnum
The
class template
\tcode{istreambuf_iterator}
defines an input iterator\iref{input.iterators} that
reads successive
\textit{characters}
from the streambuf for which it was constructed.
\tcode{operator*}
provides access to the current input character, if any.
Each time
\tcode{operator++}
is evaluated, the iterator advances to the next input character.
If the end of stream is reached (\tcode{streambuf_type::sgetc()} returns
\tcode{traits::eof()}),
the iterator becomes equal to the
\term{end-of-stream}
iterator value.
The default constructor
\tcode{istreambuf_iterator()}
and the constructor
\tcode{istreambuf_iterator(0)}
both construct an end-of-stream iterator object suitable for use
as an end-of-range.
All specializations of \tcode{istreambuf_iterator} shall have a trivial copy
constructor, a \tcode{constexpr} default constructor, and a trivial destructor.

\pnum
The result of
\tcode{operator*()}
on an end-of-stream iterator is undefined.
\indextext{behavior!undefined}%
For any other iterator value a
\tcode{char_type}
value is returned.
It is impossible to assign a character via an input iterator.

\indexlibrary{\idxcode{istreambuf_iterator}}%
\begin{codeblock}
namespace std {
  template<class charT, class traits = char_traits<charT>>
  class istreambuf_iterator {
  public:
    using iterator_category = input_iterator_tag;
    using value_type        = charT;
    using difference_type   = typename traits::off_type;
    using pointer           = @\unspec@;
    using reference         = charT;
    using char_type         = charT;
    using traits_type       = traits;
    using int_type          = typename traits::int_type;
    using streambuf_type    = basic_streambuf<charT,traits>;
    using istream_type      = basic_istream<charT,traits>;

    class proxy;                          // \expos

    constexpr istreambuf_iterator() noexcept;
    istreambuf_iterator(const istreambuf_iterator&) noexcept = default;
    ~istreambuf_iterator() = default;
    istreambuf_iterator(istream_type& s) noexcept;
    istreambuf_iterator(streambuf_type* s) noexcept;
    istreambuf_iterator(const proxy& p) noexcept;
    charT operator*() const;
    istreambuf_iterator& operator++();
    proxy operator++(int);
    bool equal(const istreambuf_iterator& b) const;

  private:
    streambuf_type* sbuf_;                // \expos
  };

  template<class charT, class traits>
    bool operator==(const istreambuf_iterator<charT,traits>& a,
            const istreambuf_iterator<charT,traits>& b);
  template<class charT, class traits>
    bool operator!=(const istreambuf_iterator<charT,traits>& a,
            const istreambuf_iterator<charT,traits>& b);
}
\end{codeblock}

\rSec3[istreambuf.iterator.proxy]{Class template \tcode{istreambuf_iterator::proxy}}

\indexlibrary{\idxcode{proxy}!\idxcode{istreambuf_iterator}}%
\begin{codeblock}
namespace std {
  template<class charT, class traits = char_traits<charT>>
  class istreambuf_iterator<charT, traits>::proxy { // \expos
    charT keep_;
    basic_streambuf<charT,traits>* sbuf_;
    proxy(charT c, basic_streambuf<charT,traits>* sbuf)
      : keep_(c), sbuf_(sbuf) { }
  public:
    charT operator*() { return keep_; }
  };
}
\end{codeblock}

\pnum
Class
\tcode{istreambuf_iterator<charT,traits>::proxy}
is for exposition only.
An implementation is permitted to provide equivalent functionality without
providing a class with this name.
Class
\tcode{istreambuf_iterator<charT, traits>::proxy}
provides a temporary
placeholder as the return value of the post-increment operator
(\tcode{operator++}).
It keeps the character pointed to by the previous value
of the iterator for some possible future access to get the character.

\rSec3[istreambuf.iterator.cons]{\tcode{istreambuf_iterator} constructors}

\pnum
For each \tcode{istreambuf_iterator} constructor in this subclause,
an end-of-stream iterator is constructed if and only if
the exposition-only member \tcode{sbuf_} is initialized with a null pointer value.


\indexlibrary{\idxcode{istreambuf_iterator}!constructor}%
\begin{itemdecl}
constexpr istreambuf_iterator() noexcept;
\end{itemdecl}

\begin{itemdescr}
\pnum
\effects
Initializes \tcode{sbuf_} with \tcode{nullptr}.
\end{itemdescr}


\indexlibrary{\idxcode{istreambuf_iterator}!constructor}%
\begin{itemdecl}
istreambuf_iterator(istream_type& s) noexcept;
\end{itemdecl}

\begin{itemdescr}
\pnum
\effects
Initializes \tcode{sbuf_} with \tcode{s.rdbuf()}.
\end{itemdescr}


\indexlibrary{\idxcode{istreambuf_iterator}!constructor}%
\begin{itemdecl}
istreambuf_iterator(streambuf_type* s) noexcept;
\end{itemdecl}

\begin{itemdescr}
\pnum
\effects
Initializes \tcode{sbuf_} with \tcode{s}.
\end{itemdescr}


\indexlibrary{\idxcode{istreambuf_iterator}!constructor}%
\begin{itemdecl}
istreambuf_iterator(const proxy& p) noexcept;
\end{itemdecl}

\begin{itemdescr}
\pnum
\effects
Initializes \tcode{sbuf_} with \tcode{p.sbuf_}.
\end{itemdescr}

\rSec3[istreambuf.iterator.ops]{\tcode{istreambuf_iterator} operations}

\indexlibrarymember{operator*}{istreambuf_iterator}%
\begin{itemdecl}
charT operator*() const
\end{itemdecl}

\begin{itemdescr}
\pnum
\returns
The character obtained via the
\tcode{streambuf}
member
\tcode{sbuf_->sgetc()}.
\end{itemdescr}

\indexlibrarymember{operator++}{istreambuf_iterator}%
\begin{itemdecl}
istreambuf_iterator& operator++();
\end{itemdecl}

\begin{itemdescr}
\pnum
\effects
As if by \tcode{sbuf_->sbumpc()}.

\pnum
\returns
\tcode{*this}.
\end{itemdescr}

\indexlibrarymember{operator++}{istreambuf_iterator}%
\begin{itemdecl}
proxy operator++(int);
\end{itemdecl}

\begin{itemdescr}
\pnum
\returns
\tcode{proxy(sbuf_->sbumpc(), sbuf_)}.
\end{itemdescr}

\indexlibrarymember{equal}{istreambuf_iterator}%
\begin{itemdecl}
bool equal(const istreambuf_iterator& b) const;
\end{itemdecl}

\begin{itemdescr}
\pnum
\returns
\tcode{true}
if and only if both iterators are at end-of-stream,
or neither is at end-of-stream, regardless of what
\tcode{streambuf}
object they use.
\end{itemdescr}

\indexlibrarymember{operator==}{istreambuf_iterator}%
\begin{itemdecl}
template<class charT, class traits>
  bool operator==(const istreambuf_iterator<charT,traits>& a,
                  const istreambuf_iterator<charT,traits>& b);
\end{itemdecl}

\begin{itemdescr}
\pnum
\returns
\tcode{a.equal(b)}.
\end{itemdescr}

\indexlibrarymember{operator"!=}{istreambuf_iterator}%
\begin{itemdecl}
template<class charT, class traits>
  bool operator!=(const istreambuf_iterator<charT,traits>& a,
                  const istreambuf_iterator<charT,traits>& b);
\end{itemdecl}

\begin{itemdescr}
\pnum
\returns
\tcode{!a.equal(b)}.
\end{itemdescr}

\rSec2[ostreambuf.iterator]{Class template \tcode{ostreambuf_iterator}}

\indexlibrary{\idxcode{ostreambuf_iterator}}%
\begin{codeblock}
namespace std {
  template<class charT, class traits = char_traits<charT>>
  class ostreambuf_iterator {
  public:
    using iterator_category = output_iterator_tag;
    using value_type        = void;
    using difference_type   = void;
    using pointer           = void;
    using reference         = void;
    using char_type         = charT;
    using traits_type       = traits;
    using streambuf_type    = basic_streambuf<charT,traits>;
    using ostream_type      = basic_ostream<charT,traits>;

    ostreambuf_iterator(ostream_type& s) noexcept;
    ostreambuf_iterator(streambuf_type* s) noexcept;
    ostreambuf_iterator& operator=(charT c);

    ostreambuf_iterator& operator*();
    ostreambuf_iterator& operator++();
    ostreambuf_iterator& operator++(int);
    bool failed() const noexcept;

  private:
    streambuf_type* sbuf_;                // \expos
  };
}
\end{codeblock}

\pnum
The
class template
\tcode{ostreambuf_iterator}
writes successive
\textit{characters}
onto the output stream from which it was constructed.
It is not possible to get a character value out of the output iterator.

\rSec3[ostreambuf.iter.cons]{\tcode{ostreambuf_iterator} constructors}


\indexlibrary{\idxcode{ostreambuf_iterator}!constructor}%
\begin{itemdecl}
ostreambuf_iterator(ostream_type& s) noexcept;
\end{itemdecl}

\begin{itemdescr}
\pnum
\requires
\tcode{s.rdbuf()}
shall not be a null pointer.

\pnum
\effects
Initializes \tcode{sbuf_} with \tcode{s.rdbuf()}.
\end{itemdescr}


\indexlibrary{\idxcode{ostreambuf_iterator}!constructor}%
\begin{itemdecl}
ostreambuf_iterator(streambuf_type* s) noexcept;
\end{itemdecl}

\begin{itemdescr}
\pnum
\requires
\tcode{s}
shall not be a null pointer.

\pnum
\effects
Initializes \tcode{sbuf_} with \tcode{s}.
\end{itemdescr}

\rSec3[ostreambuf.iter.ops]{\tcode{ostreambuf_iterator} operations}

\indexlibrarymember{operator=}{ostreambuf_iterator}%
\begin{itemdecl}
ostreambuf_iterator& operator=(charT c);
\end{itemdecl}

\begin{itemdescr}
\pnum
\effects
If
\tcode{failed()}
yields
\tcode{false},
calls
\tcode{sbuf_->sputc(c)};
otherwise has no effect.

\pnum
\returns
\tcode{*this}.
\end{itemdescr}

\indexlibrarymember{operator*}{ostreambuf_iterator}%
\begin{itemdecl}
ostreambuf_iterator& operator*();
\end{itemdecl}

\begin{itemdescr}
\pnum
\returns
\tcode{*this}.
\end{itemdescr}

\indexlibrarymember{operator++}{ostreambuf_iterator}%
\begin{itemdecl}
ostreambuf_iterator& operator++();
ostreambuf_iterator& operator++(int);
\end{itemdecl}

\begin{itemdescr}
\pnum
\returns
\tcode{*this}.
\end{itemdescr}

\indexlibrarymember{failed}{ostreambuf_iterator}%
\begin{itemdecl}
bool failed() const noexcept;
\end{itemdecl}

\begin{itemdescr}
\pnum
\returns
\tcode{true}
if in any prior use of member
\tcode{operator=},
the call to
\tcode{sbuf_->sputc()}
returned
\tcode{traits::eof()};
or
\tcode{false}
otherwise.
\end{itemdescr}

\rSec1[iterator.range]{Range access}

\pnum
In addition to being available via inclusion of the \tcode{<iterator>} header,
the function templates in \ref{iterator.range} are available when any of the following
headers are included: \tcode{<array>}, \tcode{<deque>}, \tcode{<forward_list>},
\tcode{<list>}, \tcode{<map>}, \tcode{<regex>}, \tcode{<set>}, \tcode{<span>}, \tcode{<string>},
\tcode{<string_view>}, \tcode{<unordered_map>}, \tcode{<unordered_set>}, and \tcode{<vector>}.
Each of these templates
is a designated customization point\iref{namespace.std}.

\indexlibrary{\idxcode{begin(C\&)}}%
\begin{itemdecl}
template<class C> constexpr auto begin(C& c) -> decltype(c.begin());
template<class C> constexpr auto begin(const C& c) -> decltype(c.begin());
\end{itemdecl}

\begin{itemdescr}
\pnum
\returns \tcode{c.begin()}.
\end{itemdescr}

\indexlibrary{\idxcode{end(C\&)}}%
\begin{itemdecl}
template<class C> constexpr auto end(C& c) -> decltype(c.end());
template<class C> constexpr auto end(const C& c) -> decltype(c.end());
\end{itemdecl}

\begin{itemdescr}
\pnum
\returns \tcode{c.end()}.
\end{itemdescr}

\indexlibrary{\idxcode{begin(T (\&)[N])}}%
\begin{itemdecl}
template<class T, size_t N> constexpr T* begin(T (&array)[N]) noexcept;
\end{itemdecl}

\begin{itemdescr}
\pnum
\returns \tcode{array}.
\end{itemdescr}

\indexlibrary{\idxcode{end(T (\&)[N])}}%
\begin{itemdecl}
template<class T, size_t N> constexpr T* end(T (&array)[N]) noexcept;
\end{itemdecl}

\begin{itemdescr}
\pnum
\returns \tcode{array + N}.
\end{itemdescr}

\indexlibrary{\idxcode{cbegin(const C\&)}}%
\begin{itemdecl}
template<class C> constexpr auto cbegin(const C& c) noexcept(noexcept(std::begin(c)))
  -> decltype(std::begin(c));
\end{itemdecl}
\begin{itemdescr}
\pnum \returns \tcode{std::begin(c)}.
\end{itemdescr}

\indexlibrary{\idxcode{cend(const C\&)}}%
\begin{itemdecl}
template<class C> constexpr auto cend(const C& c) noexcept(noexcept(std::end(c)))
  -> decltype(std::end(c));
\end{itemdecl}
\begin{itemdescr}
\pnum \returns \tcode{std::end(c)}.
\end{itemdescr}

\indexlibrary{\idxcode{rbegin(C\&)}}%
\begin{itemdecl}
template<class C> constexpr auto rbegin(C& c) -> decltype(c.rbegin());
template<class C> constexpr auto rbegin(const C& c) -> decltype(c.rbegin());
\end{itemdecl}
\begin{itemdescr}
\pnum \returns \tcode{c.rbegin()}.
\end{itemdescr}

\indexlibrary{\idxcode{rend(C\&)}}%
\begin{itemdecl}
template<class C> constexpr auto rend(C& c) -> decltype(c.rend());
template<class C> constexpr auto rend(const C& c) -> decltype(c.rend());
\end{itemdecl}
\begin{itemdescr}
\pnum \returns \tcode{c.rend()}.
\end{itemdescr}

\indexlibrary{\idxcode{rbegin(T (\&array)[N])}}%
\begin{itemdecl}
template<class T, size_t N> constexpr reverse_iterator<T*> rbegin(T (&array)[N]);
\end{itemdecl}
\begin{itemdescr}
\pnum \returns \tcode{reverse_iterator<T*>(array + N)}.
\end{itemdescr}

\indexlibrary{\idxcode{rend(T (\&array)[N])}}%
\begin{itemdecl}
template<class T, size_t N> constexpr reverse_iterator<T*> rend(T (&array)[N]);
\end{itemdecl}
\begin{itemdescr}
\pnum \returns \tcode{reverse_iterator<T*>(array)}.
\end{itemdescr}

\indexlibrary{\idxcode{rbegin(initializer_list<E>)}}%
\begin{itemdecl}
template<class E> constexpr reverse_iterator<const E*> rbegin(initializer_list<E> il);
\end{itemdecl}
\begin{itemdescr}
\pnum \returns \tcode{reverse_iterator<const E*>(il.end())}.
\end{itemdescr}

\indexlibrary{\idxcode{rend(initializer_list<E>)}}%
\begin{itemdecl}
template<class E> constexpr reverse_iterator<const E*> rend(initializer_list<E> il);
\end{itemdecl}
\begin{itemdescr}
\pnum \returns \tcode{reverse_iterator<const E*>(il.begin())}.
\end{itemdescr}

\indexlibrary{\idxcode{crbegin(const C\& c)}}%
\begin{itemdecl}
template<class C> constexpr auto crbegin(const C& c) -> decltype(std::rbegin(c));
\end{itemdecl}
\begin{itemdescr}
\pnum \returns \tcode{std::rbegin(c)}.
\end{itemdescr}

\indexlibrary{\idxcode{crend(const C\& c)}}%
\begin{itemdecl}
template<class C> constexpr auto crend(const C& c) -> decltype(std::rend(c));
\end{itemdecl}
\begin{itemdescr}
\pnum \returns \tcode{std::rend(c)}.
\end{itemdescr}

\rSec1[iterator.container]{Container and view access}

\pnum
In addition to being available via inclusion of the \tcode{<iterator>} header,
the function templates in \ref{iterator.container} are available
when any of the following headers are included:
\tcode{<array>}, \tcode{<deque>}, \tcode{<forward_list>}, \tcode{<list>},
\tcode{<map>}, \tcode{<regex>}, \tcode{<set>}, \tcode{<span>}, \tcode{<string>},
\tcode{<string_view>}, \tcode{<unordered_map>}, \tcode{<unordered_set>}, and \tcode{<vector>}.
Each of these templates
is a designated customization point\iref{namespace.std}.

\indexlibrary{\idxcode{size(C\& c)}}%
\begin{itemdecl}
template<class C> constexpr auto size(const C& c) -> decltype(c.size());
\end{itemdecl}
\begin{itemdescr}
\pnum \returns \tcode{c.size()}.
\end{itemdescr}

\indexlibrary{\idxcode{size(T (\&array)[N])}}%
\begin{itemdecl}
template<class T, size_t N> constexpr size_t size(const T (&array)[N]) noexcept;
\end{itemdecl}
\begin{itemdescr}
\pnum \returns \tcode{N}.
\end{itemdescr}

\indexlibrary{\idxcode{empty(C\& c)}}%
\begin{itemdecl}
template<class C> [[nodiscard]] constexpr auto empty(const C& c) -> decltype(c.empty());
\end{itemdecl}
\begin{itemdescr}
\pnum \returns \tcode{c.empty()}.
\end{itemdescr}

\indexlibrary{\idxcode{empty(T (\&array)[N])}}%
\begin{itemdecl}
template<class T, size_t N> [[nodiscard]] constexpr bool empty(const T (&array)[N]) noexcept;
\end{itemdecl}
\begin{itemdescr}
\pnum \returns \tcode{false}.
\end{itemdescr}

\indexlibrary{\idxcode{empty(initializer_list<E>)}}%
\begin{itemdecl}
template<class E> [[nodiscard]] constexpr bool empty(initializer_list<E> il) noexcept;
\end{itemdecl}
\begin{itemdescr}
\pnum \returns \tcode{il.size() == 0}.
\end{itemdescr}

\indexlibrary{\idxcode{data(C\& c)}}%
\begin{itemdecl}
template<class C> constexpr auto data(C& c) -> decltype(c.data());
template<class C> constexpr auto data(const C& c) -> decltype(c.data());
\end{itemdecl}
\begin{itemdescr}
\pnum \returns \tcode{c.data()}.
\end{itemdescr}

\indexlibrary{\idxcode{data(T (\&array)[N])}}%
\begin{itemdecl}
template<class T, size_t N> constexpr T* data(T (&array)[N]) noexcept;
\end{itemdecl}
\begin{itemdescr}
\pnum \returns \tcode{array}.
\end{itemdescr}

\indexlibrary{\idxcode{data(initializer_list<E>)}}%
\begin{itemdecl}
template<class E> constexpr const E* data(initializer_list<E> il) noexcept;
\end{itemdecl}
\begin{itemdescr}
\pnum \returns \tcode{il.begin()}.
\end{itemdescr}
