%!TEX root = std.tex
\rSec0[basic]{Basics}

\gramSec[gram.basic]{Basics}

\rSec1[basic.pre]{Preamble}
\indextext{type}%
\indextext{object}%
\indextext{storage class}%
\indextext{scope}%
\indextext{linkage}%

\pnum
\begin{note}
This Clause presents the basic concepts of the \Cpp{} language.
It explains the difference between an object and a
name and how they relate to the value categories for expressions.
It introduces the concepts of a
declaration and a definition and presents \Cpp{}'s
notion of type, scope, linkage, and
storage duration. The mechanisms for starting and
terminating a program are discussed. Finally, this Clause presents the
fundamental types of the language and lists the ways of constructing
compound types from these.
\end{note}

\pnum
\begin{note}
This Clause does not cover concepts that affect only a single
part of the language. Such concepts are discussed in the relevant
Clauses.
\end{note}

\pnum
A \defn{name} is an \grammarterm{identifier}\iref{lex.name},
\grammarterm{conversion-function-id}\iref{class.conv.fct},
\grammarterm{operator-function-id}\iref{over.oper}, or
\grammarterm{literal-operator-id}\iref{over.literal}.

\pnum
Two names are \defnx{the same}{name!same} if
\begin{itemize}
\item they are \grammarterm{identifier}{s} composed of the same character sequence, or
\item they are \grammarterm{conversion-function-id}{s} formed with
equivalent\iref{temp.over.link} types, or
\item they are \grammarterm{operator-function-id}{s} formed with
the same operator, or
\item they are \grammarterm{literal-operator-id}{s} formed with
the same literal suffix identifier.
\end{itemize}

\pnum
Every name is introduced by a \defn{declaration}, which is a
\begin{itemize}
\item
\grammarterm{name-declaration},
\grammarterm{block-declaration}, or
\grammarterm{member-declaration}\iref{dcl.pre,class.mem},
\item
\grammarterm{init-declarator}\iref{dcl.decl},
\item
\grammarterm{identifier}
in a structured binding declaration\iref{dcl.struct.bind},
\item
\grammarterm{identifier}
in a \grammarterm{result-name-introducer}
in a postcondition assertion\iref{dcl.contract.res},
\item
\grammarterm{init-capture}\iref{expr.prim.lambda.capture},
\item
\grammarterm{condition} with a \grammarterm{declarator}\iref{stmt.pre},
\item
\grammarterm{member-declarator}\iref{class.mem},
\item
\grammarterm{using-declarator}\iref{namespace.udecl},
\item
\grammarterm{parameter-declaration}\iref{dcl.fct,temp.param},
\item
\grammarterm{type-parameter}\iref{temp.param},
\item
\grammarterm{type-tt-parameter}\iref{temp.param},
\item
\grammarterm{variable-tt-parameter}\iref{temp.param},
\item
\grammarterm{concept-tt-parameter}\iref{temp.param},
\item
\grammarterm{elaborated-type-specifier}
that introduces a name\iref{dcl.type.elab},
\item
\grammarterm{class-specifier}\iref{class.pre},
\item
\grammarterm{enum-specifier} or
\grammarterm{enumerator-definition}\iref{dcl.enum},
\item
\grammarterm{exception-declaration}\iref{except.pre}, or
\item
implicit declaration of an injected-class-name\iref{class.pre}.
\end{itemize}
\begin{note}
The term declaration is not a synonym for the grammar non-terminal \grammarterm{declaration}\iref{dcl.pre}.
\end{note}
\begin{note}
The interpretation of a \grammarterm{for-range-declaration} produces
one or more of the above\iref{stmt.ranged}.
\end{note}

\pnum
\begin{note}
Some names denote types or templates.
In general, whenever a name is encountered
it is necessary to look it up\iref{basic.lookup}
to determine whether that name denotes one of these entities
before continuing to parse the program that contains it.
\end{note}

\pnum
A \defn{variable} is introduced by the
declaration of
\begin{itemize}
\item
a reference other than a non-static data member or
\item
an object.
\end{itemize}

\pnum
An \defn{entity} is a
variable,
structured binding,
result binding,
function,
enumerator,
type,
type alias,
non-static data member,
bit-field,
template,
namespace,
namespace alias,
template parameter,
function parameter, or
\grammarterm{init-capture}.
The \defnadj{underlying}{entity} of an entity is that entity
unless otherwise specified.
A name \defnx{denotes}{denote} the underlying entity of
the entity declared by each declaration that introduces the name.
\begin{note}
Type aliases and namespace aliases have underlying entities
that are distinct from themselves.
\end{note}

\pnum
A \defnadj{local}{entity} is a variable with
automatic storage duration\iref{basic.stc.auto},
a structured binding\iref{dcl.struct.bind}
whose corresponding variable is such an entity,
a result binding\iref{dcl.contract.res},
or the \tcode{*\keyword{this}} object\iref{expr.prim.this}.

\pnum
\indextext{translation unit!name and}%
\indextext{linkage}%
A name used in more than one translation unit can potentially
refer to the same entity in these translation units depending on the
linkage\iref{basic.link} of the name specified in each
translation unit.

\rSec1[basic.def]{Declarations and definitions}

\pnum
\indextext{declaration!definition versus}%
\indextext{declaration}%
\indextext{declaration!name}%
A declaration\iref{basic.pre} may (re)introduce
one or more names and/or entities into a translation
unit.
If so, the
declaration specifies the interpretation and semantic properties of these names.
A declaration of an entity $X$ is
a redeclaration of $X$
if another declaration of $X$ is reachable from it\iref{module.reach};
otherwise, it is a \defnadj{first}{declaration}.
A declaration may also have effects including:
\begin{itemize}
\item a static assertion\iref{dcl.pre},
\item controlling template instantiation\iref{temp.explicit},
\item guiding template argument deduction for constructors\iref{temp.deduct.guide},
\item use of attributes\iref{dcl.attr}, and
\item nothing (in the case of an \grammarterm{empty-declaration}).
\end{itemize}

\pnum
\indextext{declaration!function}%
\indextext{definition}%
Each entity declared by a declaration is
also \defnx{defined}{define} by that declaration unless:
\begin{itemize}
\item
it declares a function
without specifying the function's body\iref{dcl.fct.def},
\item
it contains
the
\indextext{declaration!\idxcode{extern}}%
\keyword{extern} specifier\iref{dcl.stc} or a
\grammarterm{linkage-specification}
\begin{footnote}
Appearing inside the brace-enclosed
\grammarterm{declaration-seq} in a \grammarterm{linkage-specification} does
not affect whether a declaration is a definition.
\end{footnote}\iref{dcl.link}
and neither an \grammarterm{initializer} nor a
\grammarterm{function-body},
\item
\indextext{declaration!static member@\tcode{static} member}%
it declares a non-inline static data member in a class
definition\iref{class.mem,class.static},
\item
it declares a static data member outside a class definition
and the variable was defined within the class with the \keyword{constexpr}
specifier\iref{class.static.data} (this usage is deprecated; see \ref{depr.static.constexpr}),
\item
\indextext{declaration!class name}%
it is an \grammarterm{elaborated-type-specifier}\iref{class.name},
\item
it is an
\indextext{declaration!opaque enum}%
\grammarterm{opaque-enum-declaration}\iref{dcl.enum},
\item
it is a
\indextext{parameter!template}\indextext{template parameter}%
\grammarterm{template-parameter}\iref{temp.param},
\item
it is a
\indextext{declaration!parameter}\indextext{parameter declaration}%
\grammarterm{parameter-declaration}\iref{dcl.fct} in a function
\indextext{declarator}%
declarator that is not the \grammarterm{declarator} of a
\grammarterm{function-definition},
\item
it is a
\indextext{declaration!\idxcode{typedef}}%
\keyword{typedef} declaration\iref{dcl.typedef},
\item it is
an \grammarterm{alias-declaration}\iref{dcl.typedef},
\item it is
a \grammarterm{namespace-alias-definition}\iref{namespace.alias},
\item it is
a \grammarterm{using-declaration}\iref{namespace.udecl},
\item it is
a \grammarterm{deduction-guide}\iref{temp.deduct.guide},
\item it is
a \grammarterm{static_assert-declaration}\iref{dcl.pre},
\item it is
a \grammarterm{consteval-block-declaration},
\item
it is an
\grammarterm{attribute-declaration}\iref{dcl.pre},
\item
it is an
\grammarterm{empty-declaration}\iref{dcl.pre},
\item it is
a \grammarterm{using-directive}\iref{namespace.udir},
\item it is
a \grammarterm{using-enum-declaration}\iref{enum.udecl},
\item it is
a \grammarterm{template-declaration}\iref{temp.pre}
whose \grammarterm{template-head} is not followed by
either a \grammarterm{concept-definition} or a \grammarterm{declaration}
that defines a function, a class, a variable, or a static data member,
\item it is
an explicit instantiation declaration\iref{temp.explicit}, or
\item it is
an explicit specialization\iref{temp.expl.spec} whose
\grammarterm{declaration} is not a definition.
\end{itemize}
A declaration is said to be a \defn{definition} of each entity that it defines.
\begin{example}
All but one of the following are definitions:
\begin{codeblock}
int a;                          // defines \tcode{a}
extern const int c = 1;         // defines \tcode{c}
int f(int x) { return x+a; }    // defines \tcode{f} and defines \tcode{x}
struct S { int a; int b; };     // defines \tcode{S}, \tcode{S::a}, and \tcode{S::b}
struct X {                      // defines \tcode{X}
  int x;                        // defines non-static data member \tcode{x}
  static int y;                 // declares static data member \tcode{y}
  X(): x(0) { }                 // defines a constructor of \tcode{X}
};
int X::y = 1;                   // defines \tcode{X::y}
enum { up, down };              // defines \tcode{up} and \tcode{down}
namespace N { int d; }          // defines \tcode{N} and \tcode{N::d}
X anX;                          // defines \tcode{anX}

\end{codeblock}
whereas these are just declarations:
\begin{codeblock}
extern int a;                   // declares \tcode{a}
extern const int c;             // declares \tcode{c}
int f(int);                     // declares \tcode{f}
struct S;                       // declares \tcode{S}
typedef int Int;                // declares \tcode{Int}
namespace N1 = N;               // declares \tcode{N1}
extern X anotherX;              // declares \tcode{anotherX}
using N::d;                     // declares \tcode{d}
\end{codeblock}
\end{example}

\pnum
\begin{note}
\indextext{implementation-generated}%
In some circumstances, \Cpp{} implementations implicitly define the
default constructor\iref{class.default.ctor},
copy constructor, move constructor\iref{class.copy.ctor},
copy assignment operator, move assignment operator\iref{class.copy.assign},
or destructor\iref{class.dtor} member functions.
\end{note}
\begin{example}
Given
\begin{codeblock}
#include <string>

struct C {
  std::string s;                // \tcode{std::string} is the standard library class\iref{string.classes}
};

int main() {
  C a;
  C b = a;
  b = a;
}
\end{codeblock}
the implementation will implicitly define functions to make the
definition of \tcode{C} equivalent to
\begin{codeblock}
struct C {
  std::string s;
  C() : s() { }
  C(const C& x): s(x.s) { }
  C(C&& x): s(static_cast<std::string&&>(x.s)) { }
      @\rlap{\textnormal{\textit{//}}}@    : s(std::move(x.s)) { }
  C& operator=(const C& x) { s = x.s; return *this; }
  C& operator=(C&& x) { s = static_cast<std::string&&>(x.s); return *this; }
      @\rlap{\textnormal{\textit{//}}}@                { s = std::move(x.s); return *this; }
  ~C() { }
};
\end{codeblock}
\end{example}

\pnum
\begin{note}
A class name can also be implicitly declared by an
\grammarterm{elaborated-type-specifier}\iref{dcl.type.elab}.
\end{note}

\pnum
\indextext{type!incomplete}%
In the definition of an object,
the type of that object shall not be
an incomplete type\iref{term.incomplete.type},
an abstract class type\iref{class.abstract}, or
a (possibly multidimensional) array thereof.

\rSec1[basic.def.odr]{One-definition rule}%
\indextext{object!definition}%
\indextext{function!definition}%
\indextext{class!definition}%
\indextext{enumerator!definition}%
\indextext{one-definition rule|(}%

\pnum
Each of the following is termed a \defnadj{definable}{item}:
\begin{itemize}
\item a class type\iref{class},
\item an enumeration type\iref{dcl.enum},
\item a function\iref{dcl.fct},
\item a variable\iref{basic.pre},
\item a templated entity\iref{temp.pre},
\item a default argument for a parameter
(for a function in a given scope)\iref{dcl.fct.default}, or
\item a default template argument\iref{temp.param}.
\end{itemize}

\pnum
No translation unit shall contain more than one definition of any
definable item.

\pnum
\indextext{expression!potentially evaluated}%
An expression or conversion is \defn{potentially evaluated} unless it is
an unevaluated operand\iref{expr.context},
a subexpression thereof, or
a conversion in an initialization or conversion sequence in such a context.
The set of \defn{potential results} of an expression $E$ is
defined as follows:
\begin{itemize}
\item If $E$ is
an \grammarterm{id-expression}\iref{expr.prim.id} or
a \grammarterm{splice-expression}\iref{expr.prim.splice}, the set
contains only $E$.
\item If $E$ is a subscripting operation\iref{expr.sub} with
an array operand, the set contains the potential results of that operand.
\item If $E$ is a class member access
expression\iref{expr.ref} of the form
$E_1$ \tcode{.} \opt{\keyword{template}} $E_2$
naming a non-static data member,
the set contains the potential results of $E_1$.
\item If $E$ is a class member access expression
naming a static data member,
the set contains the \grammarterm{id-expression} designating the data member.
\item If $E$ is a pointer-to-member
expression\iref{expr.mptr.oper} of the form
$E_1$ \tcode{.*} $E_2$,
the set contains the potential results of $E_1$.
\item If $E$ has the form \tcode{($E_1$)}, the set contains the
potential results of $E_1$.
\item If $E$ is a glvalue conditional
expression\iref{expr.cond}, the set is the union of the sets of
potential results of the second and third operands.
\item If $E$ is a comma expression\iref{expr.comma}, the set
contains the potential results of the right operand.
\item Otherwise, the set is empty.
\end{itemize}
\begin{note}
This set is a (possibly-empty) set of
\grammarterm{id-expression}{s} and \grammarterm{splice-expression}s,
each of which is either $E$ or a subexpression of $E$.
\begin{example}
In the following example, the set of potential results of the initializer
of \tcode{n} contains the first \tcode{S::x} subexpression, but not the second
\tcode{S::x} subexpression.
The set of potential results of the initializer of \tcode{o} contains
the subexpression \tcode{[:\caret\caret S::x:]}.
\begin{codeblock}
struct S { static const int x = 0; };
const int &f(const int &r);
int n = b ? (1, S::x)           // \tcode{S::x} is not odr-used here
          : f(S::x);            // \tcode{S::x} is odr-used here, so a definition is required
int o = [:^^S::x:];
\end{codeblock}
\end{example}
\end{note}

\pnum
A function is \defnx{named by}{function!named by expression or conversion}
an expression or conversion as follows:
\begin{itemize}
\item
  A function is named by an expression or conversion
  if it is the selected member
  of an overload set\iref{basic.lookup,over.match,over.over}
  in an overload resolution performed
  as part of forming that expression or conversion,
  and either
  it is not a pure virtual function or
  the expression is an \grammarterm{id-expression} naming the function with
  an explicitly qualified name
  that does not form a pointer to member\iref{expr.unary.op}.
  \begin{note}
This covers
  taking the address of functions\iref{conv.func,expr.unary.op},
  calls to named functions\iref{expr.call},
  operator overloading\iref{over},
  user-defined conversions\iref{class.conv.fct},
  allocation functions for \grammarterm{new-expression}{s}\iref{expr.new}, as well as
  non-default initialization\iref{dcl.init}.
  A constructor selected to copy or move an object of class type
  is considered to be named by an expression or conversion
  even if the call is actually elided by the implementation\iref{class.copy.elision}.
\end{note}
\item
  A deallocation function for a class
  is named by a \grammarterm{new-expression}
  if it is the single matching deallocation function
  for the allocation function selected by overload resolution,
  as specified in~\ref{expr.new}.
\item
  A deallocation function for a class
  is named by a \grammarterm{delete-expression}
  if it is the selected usual deallocation function
  as specified in~\ref{expr.delete} and~\ref{class.free}.
\end{itemize}

\pnum
\label{term.odr.use}%
A variable is named by an expression
if the expression is an \grammarterm{id-expression} or
\grammarterm{splice-expression}\iref{expr.prim.splice}
that designates it.
A variable \tcode{x} that is named by a
potentially-evaluated expression $N$
that appears at a point $P$
is \defnx{odr-used}{odr-use} by $N$ unless
\begin{itemize}
\item
\tcode{x} is a reference
that is usable in constant expressions at $P$\iref{expr.const} or
\item
$N$ is an element of the set of potential results of an expression $E$, where
\begin{itemize}
\item
$E$ is a discarded-value expression\iref{expr.context}
to which the lvalue-to-rvalue conversion is not applied or
\item
\tcode{x} is a non-volatile object
that is usable in constant expressions at $P$ and
has no mutable subobjects and
\begin{itemize}
\item
$E$ is a class member access expression\iref{expr.ref}
naming a non-static data member of reference type and
whose object expression has non-volatile-qualified type or
\item
the lvalue-to-rvalue conversion\iref{conv.lval} is applied to $E$ and
$E$ has non-volatile-qualified non-class type
\end{itemize}
\end{itemize}
\end{itemize}
\begin{example}
\begin{codeblock}
int f(int);
int g(int&);
struct A {
  int x;
};
struct B {
  int& r;
};
int h(bool cond) {
  constexpr A a = {1};
  constexpr const volatile A& r = a;    // odr-uses \tcode{a}
  int _ = f(cond ? a.x : r.x);          // does not odr-use \tcode{a} or \tcode{r}
  int x, y;
  constexpr B b1 = {x}, b2 = {y};       // odr-uses \tcode{x} and \tcode{y}
  int _ = g(cond ? b1.r : b2.r);        // does not odr-use \tcode{b1} or \tcode{b2}
  int _ = ((cond ? x : y), 0);          // does not odr-use \tcode{x} or \tcode{y}
  return [] {
    return b1.r;                        // error: \tcode{b1} is odr-used here because the object
                                        // referred to by \tcode{b1.r} is not constexpr-referenceable here
  }();
}
\end{codeblock}
\end{example}

\pnum
A structured binding is named by an expression
if that expression is either an \grammarterm{id-expression} or
a \grammarterm{splice-expression}
that designates that structured binding.
A structured binding is odr-used
if it is named by a potentially-evaluated expression.

\pnum
\tcode{*\keyword{this}} is odr-used if \keyword{this} appears as a potentially-evaluated expression
(including as the result of any implicit transformation to
a class member access expression\iref{expr.prim.id.general}).

\pnum
A virtual member
function is odr-used if it is not pure.
A function is odr-used if it is named by
a potentially-evaluated expression or conversion.
A non-placement allocation or deallocation
function for a class is odr-used by the definition of a constructor of that
class. A non-placement deallocation function for a class is odr-used by the
definition of the destructor of that class, or by being selected by the
lookup at the point of definition of a virtual
destructor\iref{class.dtor}.
\begin{footnote}
An implementation is not required
to call allocation and
deallocation functions from constructors or destructors; however, this
is a permissible implementation technique.
\end{footnote}

\pnum
An assignment operator function in a class is odr-used by an
implicitly-defined
copy assignment or move assignment function for another class as specified
in~\ref{class.copy.assign}.
A constructor for a class is odr-used as specified
in~\ref{dcl.init}. A destructor for a class is odr-used if it is potentially
invoked\iref{class.dtor}.

\pnum
A local entity\iref{basic.pre}
is \defn{odr-usable} in a scope\iref{basic.scope.scope} if
\begin{itemize}
\item either the local entity is not \tcode{*\keyword{this}}, or
an enclosing class or non-lambda function parameter scope exists and,
if the innermost such scope is a function parameter scope,
it corresponds to a non-static member function, and
\item
for each intervening scope\iref{basic.scope.scope}
between the point at which the entity is introduced and the scope
(where \tcode{*\keyword{this}} is considered to be introduced
within the innermost enclosing class or non-lambda function definition scope),
either
\begin{itemize}
\item the intervening scope is a block scope,
\item the intervening scope is a contract-assertion scope\iref{basic.scope.contract},
\item the intervening scope is the function parameter scope of
a \grammarterm{lambda-expression} or \grammarterm{requires-expression}, or
\item the intervening scope is the lambda scope of
a \grammarterm{lambda-expression}
that has a \grammarterm{simple-capture}
naming the entity or has a \grammarterm{capture-default}, and
the block scope of the \grammarterm{lambda-expression}
is also an intervening scope.
\end{itemize}
\end{itemize}

If a local entity is odr-used
in a scope in which it is not odr-usable,
the program is ill-formed.
\begin{example}
\begin{codeblock}
void f(int n) {
  [] { n = 1; };                // error: \tcode{n} is not odr-usable due to intervening lambda-expression
  struct A {
    void f() { n = 2; }         // error: \tcode{n} is not odr-usable due to intervening function definition scope
  };
  void g(int = n);              // error: \tcode{n} is not odr-usable due to intervening function parameter scope
  [=](int k = n) {};            // error: \tcode{n} is not odr-usable due to being
                                // outside the block scope of the \grammarterm{lambda-expression}
  [&] { [n]{ return n; }; };    // OK
}
\end{codeblock}
\end{example}

\pnum
\begin{example}
\begin{codeblock}
void g() {
  constexpr int x = 1;
  auto lambda = [] <typename T, int = ((T)x, 0)> {};    // OK
  lambda.operator()<int, 1>();          // OK, does not consider \tcode{x} at all
  lambda.operator()<int>();             // OK, does not odr-use \tcode{x}
  lambda.operator()<const int&>();      // error: odr-uses \tcode{x} from a context where \tcode{x} is not odr-usable
}

void h() {
  constexpr int x = 1;
  auto lambda = [] <typename T> { (T)x; };      // OK
  lambda.operator()<int>();             // OK, does not odr-use \tcode{x}
  lambda.operator()<void>();            // OK, does not odr-use \tcode{x}
  lambda.operator()<const int&>();      // error: odr-uses \tcode{x} from a context where \tcode{x} is not odr-usable
}
\end{codeblock}
\end{example}

\pnum
Every program shall contain at least one definition of every
function or variable that is odr-used in that program
outside of a discarded statement\iref{stmt.if}; no diagnostic required.
The definition can appear explicitly in the program, it can be found in
the standard or a user-defined library, or (when appropriate) it is
implicitly defined (see~\ref{class.default.ctor}, \ref{class.copy.ctor},
\ref{class.dtor}, and \ref{class.copy.assign}).
\begin{example}
\begin{codeblock}
auto f() {
  struct A {};
  return A{};
}
decltype(f()) g();
auto x = g();
\end{codeblock}
A program containing this translation unit is ill-formed
because \tcode{g} is odr-used but not defined,
and cannot be defined in any other translation unit
because the local class \tcode{A} cannot be named outside this
translation unit.
\end{example}

\pnum
A \defn{definition domain} is
a \grammarterm{private-module-fragment} or
the portion of a translation unit
excluding its \grammarterm{private-module-fragment} (if any).
A definition of an inline function or variable shall be reachable
from the end of every definition domain
in which it is odr-used outside of a discarded statement.

\pnum
\indextext{type!incomplete}%
A definition of a class shall be reachable in every context in which
the class is used in a way that requires the class type to be complete.
\begin{example}
The following complete translation unit is well-formed,
even though it never defines \tcode{X}:
\begin{codeblock}
struct X;                       // declare \tcode{X} as a struct type
struct X* x1;                   // use \tcode{X} in pointer formation
X* x2;                          // use \tcode{X} in pointer formation
\end{codeblock}
\end{example}
\begin{note}
The rules for declarations and expressions
describe in which contexts complete class types are required. A class
type \tcode{T} must be complete if
\begin{itemize}
\item an object of type \tcode{T} is defined\iref{basic.def}, or
\item a non-static class data member of type \tcode{T} is
declared\iref{class.mem}, or
\item \tcode{T} is used as the allocated type or array element type in a
\grammarterm{new-expression}\iref{expr.new}, or
\item an lvalue-to-rvalue conversion is applied to
a glvalue referring
to an object of type \tcode{T}\iref{conv.lval}, or
\item an expression is converted (either implicitly or explicitly) to
type \tcode{T}\iref{conv,expr.type.conv,
expr.dynamic.cast,expr.static.cast,expr.cast}, or
\item an expression that is not a null pointer constant, and has type
other than \cv{}~\tcode{\keyword{void}*}, is converted to the type pointer to \tcode{T}
or reference to \tcode{T} using a standard conversion\iref{conv},
a \keyword{dynamic_cast}\iref{expr.dynamic.cast} or
a \keyword{static_cast}\iref{expr.static.cast}, or
\item a class member access operator is applied to an expression of type
\tcode{T}\iref{expr.ref}, or
\item the \keyword{typeid} operator\iref{expr.typeid} or the
\keyword{sizeof} operator\iref{expr.sizeof} is applied to an operand of
type \tcode{T}, or
\item a function with a return type or argument type of type \tcode{T}
is defined\iref{basic.def} or called\iref{expr.call}, or
\item a class with a base class of type \tcode{T} is
defined\iref{class.derived}, or
\item an lvalue of type \tcode{T} is assigned to\iref{expr.assign}, or
\item the type \tcode{T} is the subject of an
\keyword{alignof} expression\iref{expr.alignof}, or
\item an \grammarterm{exception-declaration} has type \tcode{T}, reference to
\tcode{T}, or pointer to \tcode{T}\iref{except.handle}.
\end{itemize}
\end{note}

\pnum
If a definable item \tcode{D} is defined in a translation unit
by an injected declaration $X$\iref{expr.const} and
another translation unit contains a definition of \tcode{D},
that definition shall be an injected declaration
having the same characteristic sequence as $X$;
a diagnostic is required only if \tcode{D} is attached to a named module and
a prior definition is reachable at the point where a later definition occurs.

\pnum
For any other definable item \tcode{D} with definitions in multiple translation units,
\begin{itemize}
\item
if \tcode{D} is a non-inline non-templated function or variable, or
\item
if the definitions in different translation units
do not satisfy the following requirements,
\end{itemize}
the program is ill-formed;
a diagnostic is required only
if the definable item is attached to a named module and
a prior definition is reachable at the point where a later definition occurs.
Given such an item,
for all definitions of \tcode{D}, or,
if \tcode{D} is an unnamed enumeration,
for all definitions of \tcode{D} that are reachable at any given program point,
the following requirements shall be satisfied.
\begin{itemize}
\item Each such definition
shall not be attached to a named module\iref{module.unit}.
\item Each such definition shall consist of
the same sequence of tokens,
where the definition of a closure type
is considered to consist of the sequence of tokens of
the corresponding \grammarterm{lambda-expression}.
\item In each such definition, corresponding names, looked up
according to~\ref{basic.lookup}, shall denote the same entity, after
overload resolution\iref{over.match} and after matching of partial
template specializations\iref{temp.spec.partial.match}, except that a name can refer to
\begin{itemize}
\item
a non-volatile const object with internal or no linkage if the object
\begin{itemize}
\item has the same literal type in all definitions of \tcode{D},
\item is initialized with a constant expression\iref{expr.const},
\item is not odr-used in any definition of \tcode{D}, and
\item has the same value in all definitions of \tcode{D},
\end{itemize}
or
\item
a reference with internal or no linkage
initialized with a constant expression such that
the reference refers to the same object or function in all definitions of \tcode{D}.
\end{itemize}

\item In each such definition, except within
the default arguments and default template arguments of \tcode{D},
corresponding \grammarterm{lambda-expression}{s} shall have
the same closure type (see below).

\item In each such definition, corresponding entities shall have the
same language linkage.

\item In each such definition,
const objects with static or thread storage duration
shall be constant-initialized if
the object is constant-initialized in any such definition.

\item In each such definition,
corresponding manifestly constant-evaluated expressions
that are not value-dependent
shall have the same value\iref{expr.const,temp.dep.constexpr}.

\item In each such definition, the overloaded operators referred
to, the implicit calls to conversion functions, constructors, operator
new functions and operator delete functions, shall refer to the same
function.

\item In each such definition,
a default argument used by an (implicit or explicit) function call or
a default template argument used by an (implicit or explicit)
\grammarterm{template-id},
\grammarterm{simple-template-id}, or
\grammarterm{splice-specialization-specifier}
is treated as if its token sequence
were present in the definition of \tcode{D};
that is, the default argument or default template argument
is subject to the requirements described in this paragraph (recursively).

\item In each such definition,
corresponding \grammarterm{reflect-expression}s\iref{expr.reflect}
compute equivalent values\iref{expr.eq}.
\end{itemize}

\pnum
For the purposes of the preceding requirements:

\begin{itemize}
\item If \tcode{D} is a class with an implicitly-declared
constructor\iref{class.default.ctor,class.copy.ctor},
it is as if the constructor was
implicitly defined in every translation unit where it is odr-used, and the
implicit definition in every translation unit shall call the same
constructor for a subobject of \tcode{D}.
\begin{example}
\begin{codeblock}
// translation unit 1:
struct X {
  X(int, int);
  X(int, int, int);
};
X::X(int, int = 0) { }
class D {
  X x = 0;
};
D d1;                           // \tcode{X(int, int)} called by \tcode{D()}

// translation unit 2:
struct X {
  X(int, int);
  X(int, int, int);
};
X::X(int, int = 0, int = 0) { }
class D {
  X x = 0;
};
D d2;                           // \tcode{X(int, int, int)} called by \tcode{D()};
                                // \tcode{D()}'s implicit definition violates the ODR
\end{codeblock}
\end{example}

\item If \tcode{D} is a class with
a defaulted three-way comparison operator function\iref{class.spaceship},
it is as if the operator was
implicitly defined in every translation unit where it is odr-used, and the
implicit definition in every translation unit shall call the same
comparison operators for each subobject of \tcode{D}.

\item
If \tcode{D} is a template and is defined in more than one
translation unit, the requirements
apply both to names from the template's enclosing scope used in the
template definition, and also to dependent names at
the point of instantiation\iref{temp.dep}.
\end{itemize}

\pnum
These requirements also apply to corresponding entities
defined within each definition of \tcode{D}
(including the closure types of \grammarterm{lambda-expression}{s},
but excluding entities defined within default arguments or
default template arguments of either \tcode{D} or
an entity not defined within \tcode{D}).
For each such entity and for \tcode{D} itself,
the behavior is as if there is a single entity with a single definition,
including in the application of these requirements to other entities.
\begin{note}
The entity is still declared in multiple translation units, and \ref{basic.link}
still applies to these declarations. In particular,
\grammarterm{lambda-expression}{s}\iref{expr.prim.lambda}
appearing in the type of \tcode{D} can result
in the different declarations having distinct types, and
\grammarterm{lambda-expression}{s} appearing in a default argument of \tcode{D}
might still denote different types in different translation units.
\end{note}

\pnum
\begin{example}
\begin{codeblock}
inline void f(bool cond, void (*p)()) {
  if (cond) f(false, []{});
}
inline void g(bool cond, void (*p)() = []{}) {
  if (cond) g(false);
}
struct X {
  void h(bool cond, void (*p)() = []{}) {
    if (cond) h(false);
  }
};
\end{codeblock}

If the definition of \tcode{f} appears in multiple translation units,
the behavior of the program is as if
there is only one definition of \tcode{f}.
If the definition of \tcode{g} appears in multiple translation units,
the program is ill-formed (no diagnostic required) because
each such definition uses a default argument that
refers to a distinct \grammarterm{lambda-expression} closure type.
The definition of \tcode{X} can appear
in multiple translation units of a valid program;
the \grammarterm{lambda-expression}{s} defined within
the default argument of \tcode{X::h} within the definition of \tcode{X}
denote the same closure type in each translation unit.
\end{example}

\pnum
If, at any point in the program,
there is more than one
reachable unnamed enumeration definition in the same scope
that have the same first enumerator name and
do not have typedef names for linkage purposes\iref{dcl.enum},
those unnamed enumeration types shall be the same; no diagnostic required.
\indextext{one-definition rule|)}

\rSec1[basic.scope]{Scope}%
\indextext{scope|(}

\rSec2[basic.scope.scope]{General}

\pnum
The declarations in a program appear in a number of \defnx{scopes}{scope}
that are in general discontiguous.
The \defnadj{global}{scope} contains the entire program;
every other scope $S$ is introduced by a
declaration,
\grammarterm{parameter-declaration-clause},
\grammarterm{statement},
\grammarterm{handler}, or
contract assertion
(as described in the following subclauses of \ref{basic.scope})
appearing in another scope, which thereby contains $S$.
An \defnadj{enclosing}{scope} at a program point is any scope that contains it;
the smallest such scope is said to be the \defnadj{immediate}{scope}
at that point.
A scope \defnx{intervenes}{scope!intervene}
between a program point $P$ and a scope $S$
(that does not contain $P$) if it is or contains $S$ but does not contain $P$.

\pnum
Unless otherwise specified:
\begin{itemize}
\item
The smallest scope that contains a scope $S$ is
the \defnadj{parent}{scope} of $S$.
\item
No two declarations (re)introduce the same entity.
\item
A declaration \defnx{inhabits}{scope!inhabit}
the immediate scope at its locus\iref{basic.scope.pdecl}.
\item
A declaration's \defnadj{target}{scope} is the scope it inhabits.
\item
Any names (re)introduced by a declaration are \defnx{bound}{name!bound} to it
in its target scope.
\end{itemize}
The \defnadj{host}{scope} of a declaration is
the inhabited scope if that scope is a block scope and
the target scope otherwise.
An entity \defnx{belongs}{entity!belong} to a scope $S$
if $S$ is the target scope of a declaration of the entity.
\begin{note}
Special cases include that:
\begin{itemize}
\item
Template parameter scopes are parents
only to other template parameter scopes\iref{basic.scope.temp}.
\item
Corresponding declarations with appropriate linkage
declare the same entity\iref{basic.link}.
\item
The declaration in a \grammarterm{template-declaration}
inhabits the same scope as the \grammarterm{template-declaration}.
\item
Friend declarations and
declarations of template specializations do not bind names\iref{dcl.meaning};
those with qualified names target a specified scope, and
other friend declarations and
certain \grammarterm{elaborated-type-specifier}s\iref{dcl.type.elab}
target a larger enclosing scope.
\item
Block-scope extern or function declarations target a larger enclosing scope
but bind a name in their immediate scope\iref{dcl.meaning.general}.
\item
The names of unscoped enumerators are bound
in the two innermost enclosing scopes\iref{dcl.enum}.
\item
A class's name is also bound in its own scope\iref{class.pre}.
\item
The names of the members of an anonymous union are bound in
the union's parent scope\iref{class.union.anon}.
\end{itemize}
\end{note}

\pnum
Two non-static member functions have
\defnadjx{corresponding}{object parameters}{object parameter} if
\begin{itemize}
\item
exactly one is an implicit object member function
with no \grammarterm{ref-qualifier} and
the types of their object parameters\iref{dcl.fct},
after removing references,
are the same, or
\item
their object parameters have the same type.
\end{itemize}
\indextext{template!function!corresponding object parameter}%
Two non-static member function templates have
\defnadjx{corresponding}{object parameters}{object parameter} if
\begin{itemize}
\item
exactly one is an implicit object member function
with no \grammarterm{ref-qualifier} and
the types of their object parameters,
after removing any references,
are equivalent, or
\item
the types of their object parameters are equivalent.
\end{itemize}
\indextext{template!function!corresponding signature}%
Two function templates have
\defnadjx{corresponding}{signatures}{signature} if
their \grammarterm{template-parameter-list}{s}
have the same length,
their corresponding \grammarterm{template-parameter}{s} are equivalent,
they have equivalent non-object-parameter-type-lists and return types (if any), and,
if both are non-static members, they have corresponding object parameters.

\pnum
Two declarations \defn{correspond}
if they (re)introduce the same name,
both declare constructors, or
both declare destructors,
unless
\begin{itemize}
\item
either is a \grammarterm{using-declarator}, or
\item
one declares a type (not a type alias) and the other declares a
variable,
non-static data member other than of an anonymous union\iref{class.union.anon},
enumerator,
function, or
function template, or
\item
each declares a function or function template
and they do not declare corresponding overloads.
\end{itemize}
Two function or function template declarations declare
\defn{corresponding overloads} if
\begin{itemize}
\item
both declare functions with the same non-object-parameter-type-list,
\begin{footnote}
An implicit object parameter\iref{over.match.funcs}
is not part of the parameter-type-list.
\end{footnote}
equivalent\iref{temp.over.link} trailing \grammarterm{requires-clause}s
(if any, except as specified in \ref{temp.friend}), and,
if both are non-static members,
they have corresponding object parameters, or
\item
both declare function templates with corresponding signatures and equivalent
\grammarterm{template-head}s and
trailing \grammarterm{requires-clause}s (if any).
\end{itemize}
\begin{note}
Declarations can correspond even if neither binds a name.
\begin{example}
\begin{codeblock}
struct A {
  friend void f();      // \#1
};
struct B {
  friend void f() {}    // corresponds to, and defines, \#1
};
\end{codeblock}
\end{example}
\end{note}
\begin{example}
\begin{codeblock}
typedef int Int;
enum E : int { a };
void f(int);                    // \#1
void f(Int) {}                  // defines \#1
void f(E) {}                    // OK, another overload

struct X {
  static void f();
  void f() const;               // error: redeclaration
  void g();
  void g() const;               // OK
  void g() &;                   // error: redeclaration

  void h(this X&, int);
  void h(int) &&;               // OK, another overload
  void j(this const X&);
  void j() const &;             // error: redeclaration
  void k();
  void k(this X&);              // error: redeclaration
};
\end{codeblock}
\end{example}

\pnum
A declaration is \defnx{name-independent}{declaration!name-independent}
if its name is \tcode{_} (\unicode{005f}{low line}) and it declares
\begin{itemize}
\item
a variable with automatic storage duration,
\item
a structured binding
%FIXME: "and" is strange below; maybe reword to something like:
%FIXME: "that has no \grammarterm{storage-class-specifier} and
%FIXME:  that is not inhabiting a namespace scope,"
with no \grammarterm{storage-class-specifier} and
not inhabiting a namespace scope,
\item
a result binding\iref{dcl.contract.res},
\item
the variable introduced by an \grammarterm{init-capture}, or
\item
%FIXME: "of" is strange below; remove it?
a non-static data member of other than an anonymous union.
\end{itemize}

\recommended
Implementations should not emit a warning
that a name-independent declaration is used or unused.

\pnum
\indextext{declaration!potentially conflict}%
Two declarations \defn{potentially conflict}
if they correspond and
cause their shared name to denote different entities\iref{basic.link}.
The program is ill-formed
if, in any scope, a name is bound to two declarations $A$ and $B$
that potentially conflict and $A$ precedes $B$\iref{basic.lookup},
unless $B$ is name-independent.
\begin{note}
An \grammarterm{id-expression} that names a unique name-independent declaration
is usable until an additional declaration of the same name
is introduced in the same scope\iref{basic.lookup.general}.
\end{note}
\begin{note}
Overload resolution can consider potentially conflicting declarations
found in multiple scopes
(e.g., via \grammarterm{using-directive}s or for operator functions),
in which case it is often ambiguous.
\end{note}
\begin{example}
\begin{codeblock}
void f() {
  int x,y;
  void x();             // error: different entity for \tcode{x}
  int y;                // error: redefinition
}
enum { f };             // error: different entity for \tcode{::f}
namespace A {}
namespace B = A;
namespace B = A;        // OK, no effect
namespace B = B;        // OK, no effect
namespace A = B;        // OK, no effect
namespace B {}          // error: different entity for \tcode{B}

void g() {
  int _;
  _ = 0;                // OK
  int _;                // OK, name-independent declaration
  _ = 0;                // error: two non-function declarations in the lookup set
}
void h () {
  int _;                // \#1
  _ ++;                 // OK
  static int _;         // error: conflicts with \#1 because static variables are not name-independent
}
\end{codeblock}
\end{example}

\pnum
\indextext{declaration!nominable}%
A declaration is \defn{nominable}
in a class, class template, or namespace $E$ at a point $P$ if
it precedes $P$,
it does not inhabit a block scope, and
its target scope is the scope associated with $E$ or,
if $E$ is a namespace,
any element of the inline namespace set of $E$\iref{namespace.def}.
\begin{example}
\begin{codeblock}
namespace A {
  void f() {void g();}
  inline namespace B {
    struct S {
      friend void h();
      static int i;
    };
  }
}
\end{codeblock}
At the end of this example,
the declarations of \tcode{f}, \tcode{B}, \tcode{S}, and \tcode{h}
are nominable in \tcode{A}, but those of \tcode{g} and \tcode{i} are not.
\end{example}

\pnum
When instantiating a templated entity\iref{temp.pre},
any scope $S$ introduced by any part of the template definition is considered
to be introduced by the instantiated entity and
to contain the instantiations of any declarations that inhabit $S$.

\rSec2[basic.scope.pdecl]{Point of declaration}

\indextext{declaration!point of|see{locus}}%
\indextext{scope!declarations and|see{locus}}%
\indextext{locus|(}%

\pnum
\indextext{declaration!locus|see{locus}}%
The \defn{locus} of a declaration\iref{basic.pre} that is a declarator
is immediately after the complete declarator\iref{dcl.decl}.
\begin{example}
\begin{codeblock}
unsigned char x = 12;
{ unsigned char x = x; }
\end{codeblock}
Here, the initialization of the second \tcode{x} has undefined behavior,
because the initializer accesses the second \tcode{x}
outside its lifetime\iref{basic.life}.
\end{example}

\pnum
\begin{note}
\indextext{name hiding}%
A name from an outer scope remains visible up to
the locus of the declaration that hides it.
\begin{example}
\begin{codeblock}
const int i = 2;
{ int i[i]; }
\end{codeblock}
declares a block-scope array of two integers.
\end{example}
\end{note}

\pnum
The locus of a \grammarterm{class-specifier} is immediately after
the \grammarterm{identifier} or \grammarterm{simple-template-id} (if any)
in its \grammarterm{class-head}\iref{class.pre}.
The locus of an \grammarterm{enum-specifier}
is immediately after
its \grammarterm{enum-head};
the locus of an \grammarterm{opaque-enum-declaration}
is immediately after it\iref{dcl.enum}.
%FIXME: What's "it" below? What's "it" above?
The locus of an \grammarterm{alias-declaration} is immediately after it.

\pnum
The locus of a \grammarterm{using-declarator}
that does not name a constructor
is immediately after the \grammarterm{using-declarator}\iref{namespace.udecl}.

\pnum
The locus of an \grammarterm{enumerator-definition} is immediately after it.
\begin{example}
\begin{codeblock}
const int x = 12;
{ enum { x = x }; }
\end{codeblock}
Here, the enumerator \tcode{x} is initialized with the value of the
constant \tcode{x}, namely 12.
\end{example}

\pnum
\begin{note}
\indextext{type!incomplete}%
After the declaration of a class member,
the member name can be found in the scope of its class
even if the class is an incomplete class.
\begin{example}
\begin{codeblock}
struct X {
  enum E { z = 16 };
  int b[X::z];          // OK
};
\end{codeblock}
\end{example}
\end{note}

\pnum
The locus of an \grammarterm{elaborated-type-specifier}
that is a declaration\iref{dcl.type.elab} is immediately after it.

\pnum
The locus of an injected-class-name declaration\iref{class.pre}
is immediately following the opening brace of the class definition.

\pnum
The locus of the implicit declaration of
a function-local predefined variable\iref{dcl.fct.def.general}
is immediately before
the \grammarterm{function-body} of its function's definition.

\pnum
The locus of the declaration of a structured binding\iref{dcl.struct.bind}
is immediately after
the \grammarterm{identifier-list} of the structured binding declaration.

\pnum
The locus of a \grammarterm{for-range-declaration}
of a range-based \keyword{for} statement\iref{stmt.ranged}
is immediately after the \grammarterm{for-range-initializer}.
The locus of a \grammarterm{for-range-declaration}
of an expansion statement\iref{stmt.expand}
is immediately after the \grammarterm{expansion-initializer}.

\pnum
The locus of a \grammarterm{template-parameter} is immediately after it.
\begin{example}
\begin{codeblock}
typedef unsigned char T;
template<class T
  = T               // lookup finds the \grammarterm{typedef-name}
  , T               // lookup finds the template parameter
    N = 0> struct A { };
\end{codeblock}
\end{example}

\pnum
The locus of a \grammarterm{result-name-introducer}\iref{dcl.contract.res}
is immediately after it.

\pnum
The locus of a \grammarterm{concept-definition}
is immediately after its \grammarterm{concept-name}\iref{temp.concept}.
\begin{note}
The \grammarterm{constraint-expression} cannot use
the \grammarterm{concept-name}.
\end{note}

\pnum
The locus of a \grammarterm{namespace-definition}
with an \grammarterm{identifier}
is immediately after the \grammarterm{identifier}.
\begin{note}
An identifier is invented
for an \grammarterm{unnamed-namespace-definition}\iref{namespace.unnamed}.
\end{note}

\pnum
\begin{note}
Friend declarations can introduce functions or classes
that belong to the nearest enclosing namespace or block scope,
but they do not bind names anywhere\iref{class.friend}.
Function declarations at block scope and
variable declarations with the \keyword{extern} specifier at block scope
declare entities
that belong to the nearest enclosing namespace,
but they do not bind names in it.
\end{note}

\pnum
\begin{note}
For point of instantiation of a template, see~\ref{temp.point}.
\end{note}
\indextext{locus|)}

\rSec2[basic.scope.block]{Block scope}

\indextext{local scope|see{scope, block}}%

\pnum
Each
\begin{itemize}
\item
selection, iteration, or expansion statement\iref{stmt.select,stmt.iter,stmt.expand},
\item
substatement of such a statement,
\item
\indextext{scope!\idxgram{handler}}%
\grammarterm{handler}\iref{except.pre}, or
\item
compound statement\iref{stmt.block}
that is not the \grammarterm{compound-statement} of a \grammarterm{handler}
\end{itemize}
introduces a \defnadj{block}{scope}
that includes that statement or \grammarterm{handler}.
\begin{note}
A substatement that is also a block has only one scope.
\end{note}
A variable that belongs to a block scope is a \defn{block variable}.
\begin{example}
\begin{codeblock}
int i = 42;
int a[10];

for (int i = 0; i < 10; i++)
  a[i] = i;

int j = i;          // \tcode{j = 42}
\end{codeblock}
\end{example}

\pnum
If a declaration
that is not a name-independent declaration and
that binds a name in the block scope $S$ of a
\begin{itemize}
\item
\grammarterm{compound-statement} of a \grammarterm{lambda-expression},
\grammarterm{function-body}, or \grammarterm{function-try-block},
\item
substatement of a selection or iteration statement
that is not itself a selection or iteration statement, or
\item
\grammarterm{handler} of a \grammarterm{function-try-block}
\end{itemize}
potentially conflicts with a declaration
whose target scope is the parent scope of $S$,
the program is ill-formed.
\begin{example}
\begin{codeblock}
if (int x = f()) {
  int x;            // error: redeclaration of \tcode{x}
}
else {
  int x;            // error: redeclaration of \tcode{x}
}
\end{codeblock}
\end{example}

\rSec2[basic.scope.param]{Function parameter scope}

\indextext{scope!function prototype|see{scope, function parameter}}%
\indextext{parameter!scope of}%

\pnum
A \grammarterm{parameter-declaration-clause} $P$ introduces
a \defnadj{function parameter}{scope} that includes $P$.
\begin{note}
A function parameter cannot be used for its value
within the \grammarterm{parameter-declaration-clause}\iref{dcl.fct.default}.
\end{note}
\begin{itemize}
\item
If $P$ is associated with a \grammarterm{declarator} and
is preceded by a (possibly-parenthesized) \grammarterm{noptr-declarator} of
the form
\grammarterm{declarator-id} \opt{\grammarterm{attribute-specifier-seq}},
its scope extends to the end of the nearest enclosing
\grammarterm{init-declarator},
\grammarterm{member-declarator},
\grammarterm{declarator} of a \grammarterm{parameter-declaration} or
a \grammarterm{nodeclspec-function-declaration}, or
\grammarterm{function-definition},
but does not include the locus of the associated \grammarterm{declarator}.
\begin{note}
In this case, $P$ declares the parameters of a function
(or a function or template parameter declared with function type).
A member function's parameter scope is nested within its class's scope.
\end{note}
\item
If $P$ is associated with a \grammarterm{lambda-declarator},
its scope extends to the end of the \grammarterm{compound-statement}
in the \grammarterm{lambda-expression}.
\item
If $P$ is associated with a \grammarterm{requirement-parameter-list},
its scope extends to the end of the \grammarterm{requirement-body} of the \grammarterm{requires-expression}.
\item
If $P$ is associated with a \grammarterm{deduction-guide},
its scope extends to the end of the \grammarterm{deduction-guide}.
\end{itemize}

\rSec2[basic.scope.lambda]{Lambda scope}

A \grammarterm{lambda-expression} \tcode{E} introduces a \defnadj{lambda}{scope}
that starts immediately after the \grammarterm{lambda-introducer} of \tcode{E}
and extends to the end of the \grammarterm{compound-statement} of \tcode{E}.

\rSec2[basic.scope.namespace]{Namespace scope}

\pnum
Any \grammarterm{namespace-definition} for a namespace $N$ introduces
a \defnadj{namespace}{scope}
that includes the \grammarterm{namespace-body}
for every \grammarterm{namespace-definition} for $N$.
For each non-friend redeclaration or specialization
whose target scope is or is contained by the scope,
the portion after the
\grammarterm{declarator-id},
\grammarterm{class-head-name}, or
\grammarterm{enum-head-name}
is also included in the scope.
The global scope is
the namespace scope of the global namespace\iref{basic.namespace}.
\begin{example}
\begin{codeblock}
namespace Q {
  namespace V { void f(); }
  void V::f() {         // in the scope of \tcode{V}
    void h();           // declares \tcode{Q::V::h}
  }
}
\end{codeblock}
\end{example}

\rSec2[basic.scope.class]{Class scope}

\pnum
Any declaration of a class or class template $C$ introduces
a \defnadj{class}{scope}
that includes the \grammarterm{member-specification} of
the \grammarterm{class-specifier} for $C$ (if any).
For each non-friend redeclaration or specialization
whose target scope is or is contained by the scope,
the portion after the
\grammarterm{declarator-id},
\grammarterm{class-head-name}, or
\grammarterm{enum-head-name} is also included in the scope.
\begin{note}
Lookup from a program point
before the \grammarterm{class-specifier} of a class
will find no bindings in the class scope.
\begin{example}
\begin{codeblock}
template<class D>
struct B {
  D::type x;            // \#1
};

struct A { using type = int; };
struct C : A, B<C> {};  // error at \#1: \tcode{C::type} not found
\end{codeblock}
\end{example}
\end{note}

\rSec2[basic.scope.enum]{Enumeration scope}%

\pnum
Any declaration of an enumeration $E$ introduces
an \defnadj{enumeration}{scope}
that includes the \grammarterm{enumerator-list} of
the \grammarterm{enum-specifier} for $E$ (if any).

\rSec2[basic.scope.temp]{Template parameter scope}%

\pnum
Each
\grammarterm{type-tt-parameter},
\grammarterm{variable-tt-parameter}, and
\grammarterm{concept-tt-parameter}
introduces
a \defnadj{template parameter}{scope}
that includes the \grammarterm{template-head} of
the \grammarterm{template-parameter}.

\pnum
Each \grammarterm{template-declaration} $D$ introduces
a template parameter scope
that extends from the beginning of its \grammarterm{template-parameter-list}
to the end of the \grammarterm{template-declaration}.
Any declaration outside the \grammarterm{template-parameter-list}
that would inhabit that scope instead inhabits the same scope as $D$.
The parent scope of any scope $S$ that is not a template parameter scope
is the smallest scope that contains $S$ and is not a template parameter scope.
\begin{note}
Therefore, only template parameters belong to a template parameter scope, and
only template parameter scopes have
a template parameter scope as a parent scope.
\end{note}

\rSec2[basic.scope.contract]{Contract-assertion scope}%

\pnum
Each contract assertion\iref{basic.contract}
$C$ introduces a \defnadj{contract-assertion}{scope}
that includes $C$.

\pnum
If a \grammarterm{result-name-introducer}\iref{dcl.contract.res}
that is not name-independent\iref{basic.scope.scope}
and whose enclosing postcondition assertion
is associated with a function \tcode{F}
potentially conflicts with
a declaration whose target scope is
\begin{itemize}
\item
the function parameter scope of \tcode{F} or
\item
if associated with a \grammarterm{lambda-declarator},
the nearest enclosing lambda scope
of the precondition assertion\iref{expr.prim.lambda},
\end{itemize}
the program is ill-formed.

\indextext{scope|)}

\rSec1[basic.lookup]{Name lookup}%

\indextext{lookup!name|(}

\rSec2[basic.lookup.general]{General}%
\indextext{scope!name lookup and|see{lookup, name}}%

\pnum
\defnx{Name lookup}{lookup!name} associates the use of a name
with a set of declarations\iref{basic.def} of that name.
The name lookup rules apply uniformly to all names (including
\grammarterm{typedef-name}{s}\iref{dcl.typedef},
\grammarterm{namespace-name}{s}\iref{basic.namespace}, and
\grammarterm{class-name}{s}\iref{class.name}) wherever the grammar allows
such names in the context discussed by a particular rule.
Unless otherwise specified,
the program is ill-formed if no declarations are found.
If the declarations found by name lookup
all denote functions or function templates,
the declarations are said to form an \defn{overload set}.
Otherwise,
if the declarations found by name lookup do not all denote the same entity,
\indextext{lookup!ambiguous}%
they are \defn{ambiguous} and the program is ill-formed.
Overload resolution\iref{over.match,over.over}
takes place after name lookup has succeeded. The access rules\iref{class.access}
are considered only once name lookup and
function overload resolution (if applicable) have succeeded. Only after
name lookup, function overload resolution (if applicable) and access
checking have succeeded
are the semantic properties introduced by the declarations
used in further processing.

\pnum
A program point $P$ is said to follow
any declaration in the same translation unit
whose locus\iref{basic.scope.pdecl} is before $P$.
\begin{note}
The declaration might appear in a scope that does not contain $P$.
\end{note}
\indextext{precede|see{declaration, precede}}%
A declaration $X$ \defnx{precedes}{declaration!precede}
a program point $P$ in a translation unit $L$
if $P$ follows $X$, $X$ inhabits a class scope and is reachable from $P$, or
else $X$ appears in a translation unit $D$ and
\begin{itemize}
\item
$P$ follows
a \grammarterm{module-import-declaration} or \grammarterm{module-declaration}
that imports $D$ (directly or indirectly), and
\item
$X$ appears after the \grammarterm{module-declaration} in $D$ (if any) and
before the \grammarterm{private-module-fragment} in $D$ (if any), and
\item
either $X$ is exported or else $D$ and $L$ are part of the same module and
$X$ does not inhabit a namespace with internal linkage or
declare a name with internal linkage.
\begin{note}
Names declared by a \grammarterm{using-declaration} have no linkage.
\end{note}
\end{itemize}
\begin{note}
A \grammarterm{module-import-declaration} imports both
the named translation unit(s) and
any modules named by exported
\grammarterm{module-import-declaration}{s} within them,
recursively.
\begin{example}
\begin{codeblocktu}{Translation unit \#1}
export module Q;
export int sq(int i) { return i*i; }
\end{codeblocktu}

\begin{codeblocktu}{Translation unit \#2}
export module R;
export import Q;
\end{codeblocktu}

\begin{codeblocktu}{Translation unit \#3}
import R;
int main() { return sq(9); }    // OK, \tcode{sq} from module \tcode{Q}
\end{codeblocktu}
\end{example}
\end{note}

\pnum
\indextext{scope!search!single}%
A \defnadj{single}{search} in a scope $S$
for a name $N$ from a program point $P$
finds all declarations that precede $P$
to which any name that is the same as $N$\iref{basic.pre} is bound in $S$.
If any such declaration is a \grammarterm{using-declarator}
whose terminal name\iref{expr.prim.id.unqual}
is not dependent\iref{temp.dep.type},
it is replaced by the declarations named by
the \grammarterm{using-declarator}\iref{namespace.udecl}.

\pnum
In certain contexts, only certain kinds of declarations are included.
After any such restriction, any declarations of classes or enumerations are discarded if any other declarations are found.
\begin{note}
A type (but not a type alias or template)
is therefore hidden by any other entity in its scope.
\end{note}
\indextext{type-only!lookup|see{lookup, type-only}}%
However, if a lookup is \defnx{type-only}{lookup!type-only},
only declarations of
types and templates whose specializations are types are considered;
furthermore, if declarations
of a type alias and of its underlying entity are found,
the declaration of the type alias is discarded
instead of the type declaration.

\rSec2[class.member.lookup]{Member name lookup}%
\indextext{lookup!member name}%
\indextext{ambiguity!base class member}%
\indextext{ambiguity!member access}

\pnum
\indextext{scope!search}%
A \defn{search} in a scope $X$ for a name $M$ from a program point $P$
is a single search in $X$ for $M$ from $P$
unless $X$ is the scope of a class or class template $T$, in which case the
following steps define the result of the search.
\begin{note}
The result differs only
if $M$ is a \grammarterm{conversion-function-id} or
if the single search would find nothing.
\end{note}

\pnum
The \defn{lookup set} for a name $N$ in a class or class template $C$, called $S(N,C)$,
consists of two component sets:
the \term{declaration set}, a set of members named $N$; and
the \term{subobject set},
a set of subobjects where declarations of these members were found
(possibly via \grammarterm{using-declaration}{s}).
In the declaration set, type declarations (including injected-class-names)
are replaced by the types they designate. $S(N,C)$ is calculated as follows:

\pnum
The declaration set is the result of
a single search in the scope of $C$ for $N$
from immediately after the \grammarterm{class-specifier} of $C$
if $P$ is in a complete-class context of $C$ or
from $P$ otherwise.
If the resulting declaration set is not empty, the subobject set
contains $C$ itself, and calculation is complete.

\pnum
Otherwise (i.e., $C$ does not contain a declaration of $N$
or the resulting declaration set is empty), $S(N,C)$ is initially empty.
Calculate the lookup set for $N$
in each direct non-dependent\iref{temp.dep.type} base class subobject $B_i$, and
merge each such lookup set $S(N,B_i)$ in turn into $S(N,C)$.
\begin{note}
If $C$ is incomplete,
only base classes whose \grammarterm{base-specifier} appears before $P$
are considered.
If $C$ is an instantiated class, its base classes are not dependent.
\end{note}

\pnum
The following steps define the result of merging lookup set $S(N,B_i)$
into the intermediate $S(N,C)$:

\begin{itemize}
\item If each of the subobject members of $S(N,B_i)$ is a base class
subobject of at least one of the subobject members of $S(N,C)$, or if
$S(N,B_i)$ is empty, $S(N,C)$ is unchanged and the merge is complete.
Conversely, if each of the subobject members of $S(N,C)$ is a base class
subobject of at least one of the subobject members of $S(N,B_i)$, or if
$S(N,C)$ is empty, the new $S(N,C)$ is a copy of $S(N,B_i)$.

\item Otherwise, if the declaration sets of $S(N,B_i)$ and $S(N,C)$
differ, the merge is ambiguous: the new $S(N,C)$ is a lookup set with an
invalid declaration set and the union of the subobject sets. In
subsequent merges, an invalid declaration set is considered different
from any other.

\item Otherwise, the new $S(N,C)$ is a lookup set with the shared set of
declarations and the union of the subobject sets.
\end{itemize}

\pnum
The result of the search is the declaration set of $S(M,T)$.
If it is an invalid set, the program is ill-formed.
If it differs from the result of a search in $T$ for $M$
in a complete-class context\iref{class.mem} of $T$,
the program is ill-formed, no diagnostic required.
\begin{example}
\begin{codeblock}
struct A { int x; };                    // S(x,A) = \{ \{ \tcode{A::x} \}, \{ \tcode{A} \} \}
struct B { float x; };                  // S(x,B) = \{ \{ \tcode{B::x} \}, \{ \tcode{B} \} \}
struct C: public A, public B { };       // S(x,C) = \{ invalid, \{ \tcode{A} in \tcode{C}, \tcode{B} in \tcode{C} \} \}
struct D: public virtual C { };         // S(x,D) = S(x,C)
struct E: public virtual C { char x; }; // S(x,E) = \{ \{ \tcode{E::x} \}, \{ \tcode{E} \} \}
struct F: public D, public E { };       // S(x,F) = S(x,E)
int main() {
  F f;
  f.x = 0;                              // OK, lookup finds \tcode{E::x}
}
\end{codeblock}

$S(\tcode{x},\tcode{F})$ is unambiguous because the \tcode{A} and \tcode{B} base
class subobjects of \tcode{D} are also base class subobjects of \tcode{E}, so
$S(\tcode{x},\tcode{D})$ is discarded in the first merge step.
\end{example}

\pnum
If $M$ is a non-dependent \grammarterm{conversion-function-id},
conversion function templates that are members of $T$ are considered.
For each such template $F$, the lookup set $S(t,T)$ is constructed,
considering a function template declaration to have the name $t$
only if it corresponds to a declaration of $F$\iref{basic.scope.scope}.
The members of the declaration set of each such lookup set,
which shall not be an invalid set, are included in the result.
\begin{note}
Overload resolution will discard those
that cannot convert to the type specified by $M$\iref{temp.over}.
\end{note}

\pnum
\begin{note}
A static member, a nested type or an enumerator defined in a base class
\tcode{T} can unambiguously be found even if an object has more than one
base class subobject of type \tcode{T}. Two base class subobjects share
the non-static member subobjects of their common virtual base classes.
\end{note}
\begin{example}
\begin{codeblock}
struct V {
  int v;
};
struct A {
  int a;
  static int s;
  enum { e };
};
struct B : A, virtual V { };
struct C : A, virtual V { };
struct D : B, C { };

void f(D* pd) {
  pd->v++;          // OK, only one \tcode{v} (virtual)
  pd->s++;          // OK, only one \tcode{s} (static)
  int i = pd->e;    // OK, only one \tcode{e} (enumerator)
  pd->a++;          // error: ambiguous: two \tcode{a}{s} in \tcode{D}
}
\end{codeblock}
\end{example}

\pnum
\begin{note}
\indextext{dominance!virtual base class}%
When virtual base classes are used, a hidden declaration can be reached
along a path through the subobject lattice that does not pass through
the hiding declaration. This is not an ambiguity. The identical use with
non-virtual base classes is an ambiguity; in that case there is no
unique instance of the name that hides all the others.
\end{note}
\begin{example}
\begin{codeblock}
struct V { int f();  int x; };
struct W { int g();  int y; };
struct B : virtual V, W {
  int f();  int x;
  int g();  int y;
};
struct C : virtual V, W { };

struct D : B, C { void glorp(); };
\end{codeblock}

\begin{importgraphic}
{Name lookup}
{class.lookup}
{figname.pdf}
\end{importgraphic}

As illustrated in \fref{class.lookup},
the names declared in \tcode{V} and the left-hand instance of \tcode{W}
are hidden by those in \tcode{B}, but the names declared in the
right-hand instance of \tcode{W} are not hidden at all.
\begin{codeblock}
void D::glorp() {
  x++;              // OK, \tcode{B::x} hides \tcode{V::x}
  f();              // OK, \tcode{B::f()} hides \tcode{V::f()}
  y++;              // error: \tcode{B::y} and \tcode{C}'s \tcode{W::y}
  g();              // error: \tcode{B::g()} and \tcode{C}'s \tcode{W::g()}
}
\end{codeblock}
\end{example}
\indextext{ambiguity!class conversion}%

\pnum
An explicit or implicit conversion from a pointer to or
an expression designating an object
of a
derived class to a pointer or reference to one of its base classes shall
unambiguously refer to a unique object representing the base class.
\begin{example}
\begin{codeblock}
struct V { };
struct A { };
struct B : A, virtual V { };
struct C : A, virtual V { };
struct D : B, C { };

void g() {
  D d;
  B* pb = &d;
  A* pa = &d;       // error: ambiguous: \tcode{C}'s \tcode{A} or \tcode{B}'s \tcode{A}?
  V* pv = &d;       // OK, only one \tcode{V} subobject
}
\end{codeblock}
\end{example}

\pnum
\begin{note}
Even if the result of name lookup is unambiguous, use of a name found in
multiple subobjects might still be
ambiguous\iref{conv.mem,expr.ref,class.access.base}.
\end{note}
\begin{example}
\begin{codeblock}
struct B1 {
  void f();
  static void f(int);
  int i;
};
struct B2 {
  void f(double);
};
struct I1: B1 { };
struct I2: B1 { };

struct D: I1, I2, B2 {
  using B1::f;
  using B2::f;
  void g() {
    f();                        // Ambiguous conversion of \keyword{this}
    f(0);                       // Unambiguous (static)
    f(0.0);                     // Unambiguous (only one \tcode{B2})
    int B1::* mpB1 = &D::i;     // Unambiguous
    int D::* mpD = &D::i;       // Ambiguous conversion
  }
};
\end{codeblock}
\end{example}

\rSec2[basic.lookup.unqual]{Unqualified name lookup}

\indextext{name!unqualified}%

\pnum
A \grammarterm{using-directive} is
\term{active} in a scope $S$ at a program point $P$
if it precedes $P$ and inhabits either $S$ or
the scope of a namespace nominated by a \grammarterm{using-directive}
that is active in $S$ at $P$.

\pnum
An \term{unqualified search} in a scope $S$ from a program point $P$
includes the results of searches from $P$ in
\begin{itemize}
\item
$S$, and
\item
for any scope $U$ that contains $P$ and is or is contained by $S$,
each namespace contained by $S$ that is nominated by
a \grammarterm{using-directive} that is active in $U$ at $P$.
\end{itemize}
If no declarations are found,
the results of the unqualified search are
the results of an unqualified search in the parent scope of $S$, if any,
from $P$.
\begin{note}
When a class scope is searched,
the scopes of its base classes are also searched\iref{class.member.lookup}.
If it inherits from a single base,
it is as if the scope of the base immediately contains
the scope of the derived class.
Template parameter scopes
that are associated with one scope in the chain of parents
are also considered\iref{temp.local}.
\end{note}

\pnum
\defnx{Unqualified name lookup}{lookup!unqualified name}
from a program point performs an unqualified search in its immediate scope.

\pnum
An \defnadj{unqualified}{name} is a name
that does not immediately follow a \grammarterm{nested-name-specifier} or
the \tcode{.} or \tcode{->} in a class member access expression\iref{expr.ref},
possibly after a \keyword{template} keyword or \tcode{\~}.
Unless otherwise specified,
such a name undergoes unqualified name lookup from the point where it appears.

\pnum
An unqualified name that is a component name\iref{expr.prim.id.unqual} of
a \grammarterm{type-specifier} or \grammarterm{ptr-operator} of
a \grammarterm{conversion-type-id} is looked up in the same fashion
as the \grammarterm{conversion-function-id} in which it appears.
If that lookup finds nothing, it undergoes unqualified name lookup;
in each case, only names
that denote types or templates whose specializations are types are considered.
\begin{example}
\begin{codeblock}
struct T1 { struct U { int i; }; };
struct T2 { };
struct U1 {};
struct U2 {};

struct B {
  using T = T1;
  using U = U1;
  operator U1 T1::*();
  operator U1 T2::*();
  operator U2 T1::*();
  operator U2 T2::*();
};

template<class X, class T>
int g() {
  using U = U2;
  X().operator U T::*();                // \#1, searches for \tcode{T} in the scope of \tcode{X} first
  X().operator U decltype(T())::*();    // \#2
  return 0;
}
int x = g<B, T2>();                     // \#1 calls \tcode{B::operator U1 T1::*}
                                        // \#2 calls \tcode{B::operator U1 T2::*}
\end{codeblock}
\end{example}

\pnum
In a friend declaration \grammarterm{declarator}
whose \grammarterm{declarator-id} is a \grammarterm{qualified-id}
whose lookup context\iref{basic.lookup.qual} is a class or namespace $S$,
lookup for an unqualified name
that appears after the \grammarterm{declarator-id}
performs a search in the scope associated with $S$.
If that lookup finds nothing, it undergoes unqualified name lookup.
\begin{example}
\begin{codeblock}
using I = int;
using D = double;
namespace A {
  inline namespace N {using C = char; }
  using F = float;
  void f(I);
  void f(D);
  void f(C);
  void f(F);
}
struct X0 {using F = float; };
struct W {
  using D = void;
  struct X : X0 {
    void g(I);
    void g(::D);
    void g(F);
  };
};
namespace B {
  typedef short I, F;
  class Y {
    friend void A::f(I);        // error: no \tcode{void A::f(short)}
    friend void A::f(D);        // OK
    friend void A::f(C);        // error: \tcode{A::N::C} not found
    friend void A::f(F);        // OK
    friend void W::X::g(I);     // error: no \tcode{void X::g(short)}
    friend void W::X::g(D);     // OK
    friend void W::X::g(F);     // OK
  };
}
\end{codeblock}
\end{example}

\rSec2[basic.lookup.argdep]{Argument-dependent name lookup}%
\indextext{lookup!argument-dependent}

\pnum
When the \grammarterm{postfix-expression} in
a function call\iref{expr.call} is an \grammarterm{unqualified-id},
and unqualified lookup\iref{basic.lookup.unqual}
for the name in the \grammarterm{unqualified-id} does not find any
\begin{itemize}
\item
declaration of a class member, or
\item
function declaration inhabiting a block scope, or
\item
declaration not of a function or function template
\end{itemize}
then lookup for the name also includes the result of
\defnadj{argument-dependent}{lookup} in a set of associated namespaces
that depends on the types of the arguments
(and for type template template arguments, the namespace of the template argument),
as specified below.
\begin{example}
\begin{codeblock}
namespace N {
  struct S { };
  void f(S);
}

void g() {
  N::S s;
  f(s);             // OK, calls \tcode{N::f}
  (f)(s);           // error: \tcode{N::f} not considered; parentheses prevent argument-dependent lookup
}
\end{codeblock}
\end{example}

\pnum
\begin{note}
For purposes of determining
(during parsing) whether an expression is a
\grammarterm{postfix-expression} for a function call, the usual name lookup
rules apply.
In some cases
a name followed by \tcode{<} is treated as a \grammarterm{template-name}
even though name lookup did not find a \grammarterm{template-name}
(see \ref{temp.names}).
For example,
\begin{codeblock}
int h;
void g();
namespace N {
  struct A {};
  template <class T> int f(T);
  template <class T> int g(T);
  template <class T> int h(T);
}

int x = f<N::A>(N::A());        // OK, lookup of \tcode{f} finds nothing, \tcode{f} treated as template name
int y = g<N::A>(N::A());        // OK, lookup of \tcode{g} finds a function, \tcode{g} treated as template name
int z = h<N::A>(N::A());        // error: \tcode{h<} does not begin a \grammarterm{template-id}
\end{codeblock}

The rules have no effect on the syntactic interpretation of an expression.
For example,
\begin{codeblock}
typedef int f;
namespace N {
  struct A {
    friend void f(A &);
    operator int();
    void g(A a) {
      int i = f(a);             // \tcode{f} is the typedef, not the friend function: equivalent to \tcode{int(a)}
    }
  };
}
\end{codeblock}
Because the expression is not a function call,
argument-dependent name lookup does not apply and
the friend function \tcode{f} is not found.
\end{note}

\pnum
For each argument type \tcode{T} in the function call,
there is a set of zero or more \defnx{associated entities}{entity!associated}
to be considered.
The set of entities is determined entirely by
the types of the function arguments
(and any type template template arguments).
Any \grammarterm{typedef-name}s and \grammarterm{using-declaration}{s}
used to specify the types
do not contribute to this set.
The set of entities
is determined in the following way:
\begin{itemize}
\item If \tcode{T} is \tcode{std::meta::info}\iref{meta.syn},
its associated set of entities is the singleton containing
the enumeration type \tcode{std::meta::operators}\iref{meta.reflection.operators}.
\begin{note}
The \tcode{std::meta::info} type is a type alias,
so an explicit rule is needed to associate calls
whose arguments are reflections with the namespace \tcode{std::meta}.
\end{note}

\item If \tcode{T} is any other fundamental type, its associated set of
entities is empty.

\item If \tcode{T} is a class type (including unions),
its associated entities are:
the class itself;
the class of which it is a member, if any;
and, if it is a complete type, its direct and indirect base classes.
Furthermore, if \tcode{T} is a class template specialization,
its associated entities also include:
the entities
associated with the types of the template arguments
provided for template type parameters;
the templates used as type template template arguments; and
the classes of which any member templates used as type template template
arguments are members.
\begin{note}
Constant template arguments,
variable template template arguments, and
concept template arguments
do not
contribute to the set of associated entities.
\end{note}

\item If \tcode{T} is an enumeration type,
its associated entities are \tcode{T}
and, if it is a class member, the member's class.

\item If \tcode{T} is a pointer to \tcode{U} or an array of \tcode{U},
its associated entities are those associated with \tcode{U}.

\item If \tcode{T} is a function type, its associated
entities are those associated with the function parameter types and those
associated with the return type.

\item If \tcode{T} is a pointer to a member function of a class
\tcode{X}, its associated entities are those associated
with the function parameter types and return type, together with those
associated with \tcode{X}.

\item If \tcode{T} is a pointer to a data member of class \tcode{X}, its
associated entities are those associated with the member
type together with those associated with \tcode{X}.
\end{itemize}
In addition, if the argument is an overload set or the address of such a set,
its associated entities
are the union of those associated with each of the
members of the set, i.e., the entities associated with its
parameter types and return type.
Additionally, if the aforementioned overload set is named with
a \grammarterm{template-id}, its associated entities also include
its template template arguments and
those associated with its type template arguments.

\pnum
The \term{associated namespaces} for a call are
the innermost enclosing non-inline namespaces for its associated entities
as well as every element of the inline namespace set\iref{namespace.def}
of those namespaces.
Argument-dependent lookup finds
all declarations of functions and function templates that
\begin{itemize}
\item
are found by a search of any associated namespace, or
\item
are declared as a friend\iref{class.friend} of any class
with a reachable definition in the set of associated entities, or
\item
are exported,
are attached to a named module \tcode{M}\iref{module.interface},
do not appear in the translation unit containing the point of the lookup, and
have the same innermost enclosing non-inline namespace scope as
a declaration of an associated entity attached to \tcode{M}\iref{basic.link}.
\end{itemize}
If the lookup is for a dependent name\iref{temp.dep,temp.dep.candidate},
the above lookup is also performed
from each point in the instantiation context\iref{module.context} of the lookup,
additionally ignoring any declaration that
appears in another translation unit,
is attached to the global module, and
is either discarded\iref{module.global.frag} or has internal linkage.

\pnum
\begin{example}
\begin{codeblocktu}{Translation unit \#1}
export module M;
namespace R {
  export struct X {};
  export void f(X);
}
namespace S {
  export void f(R::X, R::X);
}
\end{codeblocktu}

\begin{codeblocktu}{Translation unit \#2}
export module N;
import M;
export R::X make();
namespace R { static int g(X); }
export template<typename T, typename U> void apply(T t, U u) {
  f(t, u);
  g(t);
}
\end{codeblocktu}

\begin{codeblocktu}{Translation unit \#3}
module Q;
import N;
namespace S {
  struct Z { template<typename T> operator T(); };
}
void test() {
  auto x = make();              // OK, \tcode{decltype(x)} is \tcode{R::X} in module \tcode{M}
  R::f(x);                      // error: \tcode{R} and \tcode{R::f} are not visible here
  f(x);                         // OK, calls \tcode{R::f} from interface of \tcode{M}
  f(x, S::Z());                 // error: \tcode{S::f} in module \tcode{M} not considered
                                // even though \tcode{S} is an associated namespace
  apply(x, S::Z());             // error: \tcode{S::f} is visible in instantiation context, but
                                // \tcode{R::g} has internal linkage and cannot be used outside TU \#2
}
\end{codeblocktu}
\end{example}

\pnum
\begin{note}
The associated namespace can include namespaces
already considered by ordinary unqualified lookup.
\end{note}
\begin{example}
\begin{codeblock}
namespace NS {
  class T { };
  void f(T);
  void g(T, int);
}
NS::T parm;
void g(NS::T, float);
int main() {
  f(parm);                      // OK, calls \tcode{NS::f}
  extern void g(NS::T, float);
  g(parm, 1);                   // OK, calls \tcode{g(NS::T, float)}
}
\end{codeblock}
\end{example}

\rSec2[basic.lookup.qual]{Qualified name lookup}

\rSec3[basic.lookup.qual.general]{General}

\pnum
\indextext{lookup!qualified name|(}%
\indextext{name!qualified}%
\indextext{qualification!explicit}%
Lookup of an \grammarterm{identifier}
followed by a \tcode{::} scope resolution operator
considers only
namespaces, types, and templates whose specializations are types.
If a
name,
\grammarterm{template-id},
\grammarterm{splice-scope-specifier}, or
\grammarterm{computed-type-specifier}
is followed by a \tcode{::},
it shall either be
a dependent \grammarterm{splice-scope-specifier}\iref{temp.dep.splice} or
it shall designate a namespace, class, enumeration, or dependent type,
and the \tcode{::} is never interpreted as
a complete \grammarterm{nested-name-specifier}.
\begin{example}
\begin{codeblock}
class A {
public:
  static int n;
};
int main() {
  int A;
  A::n = 42;            // OK
  A b;                  // error: \tcode{A} does not name a type
}
template<int> struct B : A {};
namespace N {
  template<int> void B();
  int f() {
    return B<0>::n;     // error: \tcode{N::B<0>} is not a type
  }
}
\end{codeblock}
\end{example}

\indextext{operator!scope resolution}%
\indextext{scope resolution operator|see{operator, scope resolution}}%

\pnum
A member-qualified name is
the (unique) component name\iref{expr.prim.id.unqual}, if any, of
\begin{itemize}
\item
an \grammarterm{unqualified-id} or
\item
a \grammarterm{nested-name-specifier} of the form
\grammarterm{type-name} \tcode{::} or \grammarterm{namespace-name} \tcode{::}
\end{itemize}
in the \grammarterm{id-expression} of a class member access expression\iref{expr.ref}.
A \defnadj{qualified}{name} is
\begin{itemize}
\item a member-qualified name or
\item the terminal name of
\begin{itemize}
\item a \grammarterm{qualified-id},
\item a \grammarterm{using-declarator},
\item a \grammarterm{typename-specifier},
\item a \grammarterm{qualified-namespace-specifier}, or
\item a \grammarterm{nested-name-specifier},
\grammarterm{reflection-name},
\grammarterm{elaborated-type-specifier}, or
\grammarterm{class-or-decltype}
that has a \grammarterm{nested-name-specifier}\iref{expr.prim.id.qual}.
\end{itemize}
\end{itemize}
The \defn{lookup context} of a member-qualified name is
the type of its associated object expression
(considered dependent if the object expression is type-dependent).
The lookup context of any other qualified name is
the type, template, or namespace
nominated by the preceding \grammarterm{nested-name-specifier}.
\begin{note}
When parsing a class member access,
the name following the \tcode{->} or \tcode{.} is
a qualified name even though it is not yet known of which kind.
\end{note}
\begin{example}
In
\begin{codeblock}
  N::C::m.Base::f()
\end{codeblock}
\tcode{Base} is a member-qualified name;
the other qualified names are \tcode{C}, \tcode{m}, and \tcode{f}.
\end{example}

\pnum
\defnx{Qualified name lookup}{lookup!qualified name}
in a class, namespace, or enumeration performs
a search of the scope associated with it\iref{class.member.lookup}
except as specified below.
Unless otherwise specified,
a qualified name undergoes qualified name lookup in its lookup context
from the point where it appears
unless the lookup context either
is dependent and is not the current instantiation\iref{temp.dep.type} or
is not a class or class template.
If nothing is found by qualified lookup for a member-qualified name
that is the terminal name\iref{expr.prim.id.unqual} of
a \grammarterm{nested-name-specifier} and
is not dependent, it undergoes unqualified lookup.
\begin{note}
During lookup for a template specialization, no names are dependent.
\end{note}
\begin{example}
\begin{codeblock}
int f();
struct A {
  int B, C;
  template<int> using D = void;
  using T = void;
  void f();
};
using B = A;
template<int> using C = A;
template<int> using D = A;
template<int> using X = A;

template<class T>
void g(T *p) {                  // as instantiated for \tcode{g<A>}:
  p->X<0>::f();                 // error: \tcode{A::X} not found in \tcode{((p->X) < 0) > ::f()}
  p->template X<0>::f();        // OK, \tcode{::X} found in definition context
  p->B::f();                    // OK, non-type \tcode{A::B} ignored
  p->template C<0>::f();        // error: \tcode{A::C} is not a template
  p->template D<0>::f();        // error: \tcode{A::D<0>} is not a class type
  p->T::f();                    // error: \tcode{A::T} is not a class type
}
template void g(A*);
\end{codeblock}
\end{example}

\pnum
If a qualified name $Q$ follows a \tcode{\~}:
\begin{itemize}
\item
If $Q$ is a member-qualified name,
it undergoes unqualified lookup as well as qualified lookup.
\item
Otherwise, its \grammarterm{nested-name-specifier} $N$ shall nominate a type.
If $N$ has another \grammarterm{nested-name-specifier} $S$,
$Q$ is looked up as if its lookup context were that nominated by $S$.
\item
Otherwise, if the terminal name of $N$ is a member-qualified name $M$,
$Q$ is looked up as if $\tcode{\~}Q$ appeared in place of $M$ (as above).
\item
Otherwise, $Q$ undergoes unqualified lookup.
\item
Each lookup for $Q$ considers only
types (if $Q$ is not followed by a \tcode{<}) and
templates whose specializations are types.
If it finds nothing or is ambiguous, it is discarded.
\item
The \grammarterm{type-name} that is or contains $Q$
shall refer to its (original) lookup context (ignoring cv-qualification) under
the interpretation established by at least one (successful) lookup performed.
\end{itemize}
\begin{example}
\begin{codeblock}
struct C {
  typedef int I;
};
typedef int I1, I2;
extern int* p;
extern int* q;
void f() {
  p->C::I::~I();        // \tcode{I} is looked up in the scope of \tcode{C}
  q->I1::~I2();         // \tcode{I2} is found by unqualified lookup
}
struct A {
  ~A();
};
typedef A AB;
int main() {
  AB* p;
  p->AB::~AB();         // explicitly calls the destructor for \tcode{A}
}
\end{codeblock}
\end{example}

\rSec3[class.qual]{Class members}

\indextext{lookup!class member}%

\pnum
In a lookup for a qualified name $N$ whose lookup context is a class $C$
in which function names are not ignored,
\begin{footnote}
Lookups in which
function names are ignored include names appearing in a
\grammarterm{nested-name-specifier}, an
\grammarterm{elaborated-type-specifier}, or a \grammarterm{base-specifier}.
\end{footnote}
\begin{itemize}
\item
if the search finds the injected-class-name of \tcode{C}\iref{class.pre}, or
\item
if $N$ is dependent and
is the terminal name of a \grammarterm{using-declarator}\iref{namespace.udecl}
that names a constructor,
\end{itemize}
$N$ is instead considered to name the constructor of class \tcode{C}.
Such a constructor name shall be used only
in the \grammarterm{declarator-id} of a (friend) declaration of a constructor or
in a \grammarterm{using-declaration}.
\begin{example}
\begin{codeblock}
struct A { A(); };
struct B: public A { B(); };

A::A() { }
B::B() { }

B::A ba;            // object of type \tcode{A}
A::A a;             // error: \tcode{A::A} is not a type name
struct A::A a2;     // object of type \tcode{A}
\end{codeblock}
\end{example}

\rSec3[namespace.qual]{Namespace members}

\pnum
Qualified name lookup in a namespace $N$ additionally searches
every element of the inline namespace set of $N$\iref{namespace.def}.
If nothing is found,
the results of the lookup are the results of qualified name lookup
in each namespace nominated by a \grammarterm{using-directive}
that precedes the point of the lookup and
inhabits $N$ or an element of $N$'s inline namespace set.
\begin{note}
If a \grammarterm{using-directive} refers to a namespace
that has already been considered, it does not affect the result.
\end{note}
\begin{example}
\begin{codeblock}
int x;
namespace Y {
  void f(float);
  void h(int);
}

namespace Z {
  void h(double);
}

namespace A {
  using namespace Y;
  void f(int);
  void g(int);
  int i;
}

namespace B {
  using namespace Z;
  void f(char);
  int i;
}

namespace AB {
  using namespace A;
  using namespace B;
  void g();
}

void h()
{
  AB::g();          // \tcode{g} is declared directly in \tcode{AB}, therefore \tcode{S} is $\{ \tcode{AB::g()} \}$ and \tcode{AB::g()} is chosen

  AB::f(1);         // \tcode{f} is not declared directly in \tcode{AB} so the rules are applied recursively to \tcode{A} and \tcode{B};
                    // namespace \tcode{Y} is not searched and \tcode{Y::f(float)} is not considered;
                    // \tcode{S} is $\{ \tcode{A::f(int)}, \tcode{B::f(char)} \}$ and overload resolution chooses \tcode{A::f(int)}

  AB::f('c');       // as above but resolution chooses \tcode{B::f(char)}

  AB::x++;          // \tcode{x} is not declared directly in \tcode{AB}, and is not declared in \tcode{A} or \tcode{B}, so the rules
                    // are applied recursively to \tcode{Y} and \tcode{Z}, \tcode{S} is $\{ \}$ so the program is ill-formed

  AB::i++;          // \tcode{i} is not declared directly in \tcode{AB} so the rules are applied recursively to \tcode{A} and \tcode{B},
                    // \tcode{S} is $\{ \tcode{A::i}, \tcode{B::i} \}$ so the use is ambiguous and the program is ill-formed

  AB::h(16.8);      // \tcode{h} is not declared directly in \tcode{AB} and not declared directly in \tcode{A} or \tcode{B} so the rules
                    // are applied recursively to \tcode{Y} and \tcode{Z}, \tcode{S} is $\{ \tcode{Y::h(int)}, \tcode{Z::h(double)} \}$ and
                    // overload resolution chooses \tcode{Z::h(double)}
}
\end{codeblock}
\end{example}

\pnum
\begin{note}
The same declaration found more than once is not an ambiguity (because
it is still a unique declaration).
\begin{example}
\begin{codeblock}
namespace A {
  int a;
}

namespace B {
  using namespace A;
}

namespace C {
  using namespace A;
}

namespace BC {
  using namespace B;
  using namespace C;
}

void f()
{
  BC::a++;          // OK, \tcode{S} is $\{ \tcode{A::a}, \tcode{A::a} \}$
}

namespace D {
  using A::a;
}

namespace BD {
  using namespace B;
  using namespace D;
}

void g()
{
  BD::a++;          // OK, \tcode{S} is $\{ \tcode{A::a}, \tcode{A::a} \}$
}
\end{codeblock}
\end{example}
\end{note}

\pnum
\begin{example}
Because each referenced namespace is searched at most once, the
following is well-defined:
\begin{codeblock}
namespace B {
  int b;
}

namespace A {
  using namespace B;
  int a;
}

namespace B {
  using namespace A;
}

void f()
{
  A::a++;           // OK, \tcode{a} declared directly in \tcode{A}, \tcode{S} is $\{ \tcode{A::a} \}$
  B::a++;           // OK, both \tcode{A} and \tcode{B} searched (once), \tcode{S} is $\{ \tcode{A::a} \}$
  A::b++;           // OK, both \tcode{A} and \tcode{B} searched (once), \tcode{S} is $\{ \tcode{B::b} \}$
  B::b++;           // OK, \tcode{b} declared directly in \tcode{B}, \tcode{S} is $\{ \tcode{B::b} \}$
}
\end{codeblock}
\end{example}

\pnum
\begin{note}
Class and enumeration declarations are not discarded
because of other declarations found in other searches.
\end{note}
\begin{example}
\begin{codeblock}
namespace A {
  struct x { };
  int x;
  int y;
}

namespace B {
  struct y { };
}

namespace C {
  using namespace A;
  using namespace B;
  int i = C::x;     // OK, \tcode{A::x} (of type \tcode{int})
  int j = C::y;     // ambiguous, \tcode{A::y} or \tcode{B::y}
}
\end{codeblock}
\end{example}

\indextext{lookup!qualified name|)}%

\rSec2[basic.lookup.elab]{Elaborated type specifiers}%
\indextext{lookup!elaborated type specifier|(}%
\indextext{type specifier!elaborated}

\pnum
If the \grammarterm{class-key} or \keyword{enum} keyword
in an \grammarterm{elaborated-type-specifier}
is followed by an \grammarterm{identifier}
that is not followed by \tcode{::},
lookup for the \grammarterm{identifier} is type-only\iref{basic.lookup.general}.
\begin{note}
In general, the recognition of an \grammarterm{elaborated-type-specifier}
depends on the following tokens.
If the \grammarterm{identifier} is followed by \tcode{::},
see \ref{basic.lookup.qual}.
\end{note}

\pnum
If the terminal name of the \grammarterm{elaborated-type-specifier}
is a qualified name,
lookup for it is type-only.
If the name lookup does not find a previously declared \grammarterm{type-name},
the \grammarterm{elaborated-type-specifier} is ill-formed.

\pnum
\begin{example}
\begin{codeblock}
struct Node {
  struct Node* Next;            // OK, refers to injected-class-name \tcode{Node}
  struct Data* Data;            // OK, declares type \tcode{Data} at global scope and member \tcode{Data}
};

struct Data {
  struct Node* Node;            // OK, refers to \tcode{Node} at global scope
  friend struct ::Glob;         // error: \tcode{Glob} is not declared, cannot introduce a qualified type\iref{dcl.type.elab}
  friend struct Glob;           // OK, refers to (as yet) undeclared \tcode{Glob} at global scope.
  @\commentellip@
};

struct Base {
  struct Data;                  // OK, declares nested \tcode{Data}
  struct ::Data*     thatData;  // OK, refers to \tcode{::Data}
  struct Base::Data* thisData;  // OK, refers to nested \tcode{Data}
  friend class ::Data;          // OK, global \tcode{Data} is a friend
  friend class Data;            // OK, nested \tcode{Data} is a friend
  struct Data { @\commentellip@ };    // Defines nested \tcode{Data}
};

struct Data;                    // OK, redeclares \tcode{Data} at global scope
struct ::Data;                  // error: cannot introduce a qualified type\iref{dcl.type.elab}
struct Base::Data;              // error: cannot introduce a qualified type\iref{dcl.type.elab}
struct Base::Datum;             // error: \tcode{Datum} undefined
struct Base::Data* pBase;       // OK, refers to nested \tcode{Data}
\end{codeblock}
\end{example}
\indextext{lookup!elaborated type specifier|)}%

\rSec2[basic.lookup.udir]{Using-directives and namespace aliases}

\pnum
\indextext{lookup!using-directives and}%
\indextext{lookup!namespace aliases and}%
In a \grammarterm{using-directive} or \grammarterm{namespace-alias-definition},
during the lookup for a \grammarterm{namespace-name} or for a name in a
\grammarterm{nested-name-specifier}{}
only namespace names are considered.%
\indextext{lookup!name|)}%

\rSec1[basic.splice]{Splice specifiers}
\indextext{splice|(}%

\begin{bnf}
\nontermdef{splice-specifier}\br
  \terminal{[:} constant-expression \terminal{:]}
\end{bnf}

\begin{bnf}
\nontermdef{splice-specialization-specifier}\br
  splice-specifier \terminal{<} \opt{template-argument-list} \terminal{>}
\end{bnf}

\pnum
The \grammarterm{constant-expression} of a \grammarterm{splice-specifier}
shall be a converted constant expression of
type \tcode{std::meta::info}\iref{expr.const}.
A \grammarterm{splice-specifier}
whose converted \grammarterm{constant-expression} represents
a construct $X$ is said to \defn{designate} either
\begin{itemize}
\item the underlying entity of $X$ if $X$ is an entity\iref{basic.pre}, or
\item $X$ otherwise.
\end{itemize}
\begin{note}
A \grammarterm{splice-specifier} is dependent
if the converted \grammarterm{constant-expression} is
value-dependent\iref{temp.dep.splice}.
\end{note}

\pnum
A non-dependent \grammarterm{splice-specifier} of
a \grammarterm{splice-specialization-specifier} shall designate a template.

\pnum
\begin{note}
A \tcode{<} following a \grammarterm{splice-specifier} is interpreted as
the delimiter of a \grammarterm{template-argument-list}
when the \grammarterm{splice-specifier} is preceded by
the keyword \keyword{template} or the keyword \keyword{typename}, or
when it appears in a type-only context\iref{temp.names}.
\begin{example}
\begin{codeblock}
constexpr int v = 1;
template<int V> struct TCls {
  static constexpr int s = V + 1;
};

using alias = [:^^TCls:]<([:^^v:])>;
  // OK, a \grammarterm{splice-specialization-specifier} with a parenthesized \grammarterm{splice-expression} as a template argument

static_assert(alias::s == 2);

auto o1 = [:^^TCls:]<([:^^v:])>();              // error: < means less than
auto o2 = typename [:^^TCls:]<([:^^v:])>();     // OK, \tcode{o2} is an object of type \tcode{TCls<1>}

consteval int bad_splice(std::meta::info v) {
  return [:v:];                                 // error: \tcode{v} is not constant
}
\end{codeblock}
\end{example}
\end{note}
\indextext{splice|)}

\rSec1[basic.link]{Program and linkage}%
\indextext{linkage|(}

\pnum
\indextext{program}%
\indextext{linking}%
A \defn{program} consists of one or more translation units\iref{lex.separate}
linked together. A translation unit consists
of a sequence of declarations.

\begin{bnf}
\nontermdef{translation-unit}\br
    \opt{declaration-seq}\br
    \opt{global-module-fragment} module-declaration \opt{declaration-seq} \opt{private-module-fragment}
\end{bnf}

\pnum
\indextext{translation unit}%
A name has
\defnadj{external}{linkage},
\defnadj{module}{linkage},
\defnadj{internal}{linkage}, or
\defnadj{no}{linkage},
as determined by the rules below.
\begin{note}
All declarations of an entity with a name with internal linkage
appear in the same translation unit.
All declarations of an entity with module linkage
are attached to the same module.
\end{note}

\pnum
\indextext{linkage!\idxcode{static} and}%
\indextext{\idxcode{static}!linkage of}%
\indextext{linkage!\idxcode{const} and}%
\indextext{\idxcode{const}!linkage of}%
\indextext{linkage!\idxcode{inline} and}%
\indextext{\idxcode{inline}!linkage of}%
The name of an entity
that belongs to a namespace scope\iref{basic.scope.namespace}
has internal linkage if it is the name of
\begin{itemize}
\item
  a variable, variable template, function, or function template that is
  explicitly declared \keyword{static}; or
\item
  a non-template variable of non-volatile const-qualified type, unless
  \begin{itemize}
  \item it is declared in the purview of a module interface unit
    (outside the \grammarterm{private-module-fragment}, if any) or
    module partition, or
  \item it is explicitly declared \keyword{extern}, or
  \item it is inline, or
  \item it was previously declared and the prior declaration did
  not have internal linkage; or
  \end{itemize}
\item
  a data member of an anonymous union.
\end{itemize}
\begin{note}
An instantiated variable template that has const-qualified type
can have external or module linkage, even if not declared \keyword{extern}.
\end{note}

\pnum
An unnamed namespace or a namespace declared directly or indirectly within an
unnamed namespace has internal linkage. All other namespaces have external linkage.
The name of an entity that belongs to a namespace scope,
that has not been given internal linkage above,
and that is the name of
\begin{itemize}
\item a variable; or
\item a function; or
\item
\indextext{class!linkage of}%
a named class\iref{class.pre}, or an unnamed class defined in a
typedef declaration in which the class has the typedef name for linkage
purposes\iref{dcl.typedef}; or
\item
\indextext{enumeration!linkage of}%
a named enumeration\iref{dcl.enum}, or an unnamed enumeration defined
in a typedef declaration in which the enumeration has the typedef name
for linkage purposes\iref{dcl.typedef}; or
\item an unnamed enumeration
that has an enumerator as a name for linkage purposes\iref{dcl.enum}; or
\item a template
\end{itemize}
has its linkage determined as follows:
\begin{itemize}
\item
\indextext{friend function!linkage of}%
if the entity is a function or function template
first declared in a friend declaration and
that declaration is a definition and
the enclosing class is defined within an \grammarterm{export-declaration},
the name has the same linkage, if any,
as the name of the enclosing class\iref{class.friend};
\item
otherwise,
\indextext{friend function!linkage of}%
if the entity is a function or function template
declared in a friend declaration and
a corresponding non-friend declaration is reachable,
%FIXME: Which declaration is "that prior declaration"?
%FIXME: "prior" with respect to what? And what about dependent lookup?
the name has the linkage determined from that prior declaration,
\item
otherwise,
if the enclosing namespace has internal linkage,
the name has internal linkage;
\item
otherwise,
if the declaration of the name is
attached to a named module\iref{module.unit}
and is not exported\iref{module.interface},
the name has module linkage;
\item
otherwise,
the name has external linkage.
\end{itemize}

\pnum
In addition,
a member function,
a static data member,
a named class or enumeration that inhabits a class scope, or
an unnamed class or enumeration defined in a typedef declaration
that inhabits a class scope
such that the class or enumeration
has the typedef name for linkage purposes\iref{dcl.typedef},
has the same linkage, if any, as the name of the class of which it is a member.

\pnum
\begin{example}
\begin{codeblock}
static void f();
extern "C" void h();
static int i = 0;               // \#1
void q() {
  extern void f();              // internal linkage
  extern void g();              // \tcode{::g}, external linkage
  extern void h();              // C language linkage
  int i;                        // \#2: \tcode{i} has no linkage
  {
    extern void f();            // internal linkage
    extern int i;               // \#3: internal linkage
  }
}
\end{codeblock}
Even though the declaration at line \#2 hides the declaration at line \#1,
the declaration at line \#3 still redeclares \#1 and receives internal linkage.
\end{example}

\pnum
\indextext{linkage!no}%
Names not covered by these rules have no linkage. Moreover, except as
noted, a name declared at block scope\iref{basic.scope.block} has no
linkage.

\pnum
Two declarations of entities declare the same entity
if, considering declarations of unnamed types to introduce their names
for linkage purposes, if any\iref{dcl.typedef,dcl.enum},
they correspond\iref{basic.scope.scope},
have the same target scope that is not a function or template parameter scope,
neither is a name-independent declaration,
and either
\begin{itemize}
\item
they appear in the same translation unit, or
\item
they both declare type aliases or namespace aliases that have the same underlying entity, or
\item
they both declare names with module linkage and are attached to the same module, or
\item
they both declare names with external linkage.
\end{itemize}
\begin{note}
There are other circumstances in which declarations declare
the same entity\iref{dcl.link,temp.type,temp.spec.partial}.
\end{note}

\pnum
If a declaration $H$ that declares a name with internal linkage
precedes a declaration $D$ in another translation unit $U$ and
would declare the same entity as $D$ if it appeared in $U$,
the program is ill-formed.
\begin{note}
Such an $H$ can appear only in a header unit.
\end{note}

\pnum
If two declarations of an entity are
attached to different modules, the program is ill-formed;
no diagnostic is required if neither is reachable from the other.
\begin{example}
\begin{codeblocktu}{\tcode{"decls.h"}}
int f();            // \#1, attached to the global module
int g();            // \#2, attached to the global module
\end{codeblocktu}

\begin{codeblocktu}{Module interface of \tcode{M}}
module;
#include "decls.h"
export module M;
export using ::f;   // OK, does not declare an entity, exports \#1
int g();            // error: matches \#2, but attached to \tcode{M}
export int h();     // \#3
export int k();     // \#4
\end{codeblocktu}

\begin{codeblocktu}{Other translation unit}
import M;
static int h();     // error: matches \#3
int k();            // error: matches \#4
\end{codeblocktu}
\end{example}
As a consequence of these rules,
all declarations of an entity are attached to the same module;
the entity is said to be \defnx{attached}{attached!entity} to that module.

\pnum
\indextext{consistency!type declaration}%
\indextext{declaration!multiple}%
For any two declarations of an entity $E$:
\begin{itemize}
\item
If one declares $E$ to be a variable or function,
the other shall declare $E$ as one of the same type.
\item
If one declares $E$ to be an enumerator, the other shall do so.
\item
If one declares $E$ to be a namespace, the other shall do so.
\item
If one declares $E$ to be a type,
the other shall declare $E$ to be a type of the same kind\iref{dcl.type.elab}.
\item
If one declares $E$ to be a class template,
the other shall do so with the same kind and
an equivalent \grammarterm{template-head}\iref{temp.over.link}.
\begin{note}
The declarations can supply different default template arguments.
\end{note}
\item
If one declares $E$ to be a function template or
a (partial specialization of a) variable template,
the other shall declare $E$ to be one
with an equivalent \grammarterm{template-head} and type.
\item
If one declares $E$ to be an alias template,
the other shall declare $E$ to be one with
an equivalent \grammarterm{template-head} and \grammarterm{defining-type-id}.
\item
If one declares $E$ to be a concept, the other shall do so.
\end{itemize}
Types are compared after all adjustments of types (during which
typedefs\iref{dcl.typedef} are replaced by their definitions);
declarations for an array
object can specify array types that differ by the presence or absence of
a major array bound\iref{dcl.array}.
No diagnostic is required if neither declaration is reachable from the other.
\begin{example}
\begin{codeblock}
int f(int x, int x);    // error: different entities for \tcode{x}
void g();               // \#1
void g(int);            // OK, different entity from \#1
int g();                // error: same entity as \#1 with different type
void h();               // \#2
namespace h {}          // error: same entity as \#2, but not a function
\end{codeblock}
\end{example}

\pnum
\begin{note}
Linkage to non-\Cpp{} declarations can be achieved using a
\grammarterm{linkage-specification}\iref{dcl.link}.
\end{note}
\indextext{linkage|)}

\pnum
A declaration $D$ \defnx{names}{name} an entity $E$ if
\begin{itemize}
\item
$D$ contains a \grammarterm{lambda-expression} whose closure type is $E$,
\item
$D$ contains
a \grammarterm{reflect-expression} or a \grammarterm{splice-specifier}
that, respectively, represents or designates $E$,
\item
$D$ is an injected declaration\iref{expr.const}
whose characteristic sequence contains a reflection
that represents
a data member description ($T$, $N$, $A$, $W$, $\mathit{NUA}$)\iref{class.mem.general}
for which $T$ is $E$,
\item
$E$ is not a function or function template and $D$ contains an
\grammarterm{id-expression},
\grammarterm{type-specifier},
\grammarterm{nested-name-specifier},
\grammarterm{template-name}, or
\grammarterm{concept-name}
denoting $E$, or
\item
$E$ is a function or function template and
$D$ contains an expression that names $E$\iref{basic.def.odr} or
an \grammarterm{id-expression}
that refers to a set of overloads that contains $E$.
\begin{note}
Non-dependent names in an instantiated declaration
do not refer to a set of overloads\iref{temp.res}.
\end{note}
\end{itemize}

\pnum
A declaration is an \defn{exposure}
if it either names a TU-local entity (defined below), ignoring
\begin{itemize}
\item
the \grammarterm{function-body}
for a non-inline function or function template
(but not the deduced return type
for a (possibly instantiated) definition of a function
with a declared return type that uses a placeholder type\iref{dcl.spec.auto}),
\item
the \grammarterm{initializer}
for a variable or variable template (but not the variable's type),
\item
friend declarations in a class definition, and
\item
any reference to a non-volatile const object or reference
with internal or no linkage initialized with a constant expression
that is not an odr-use\iref{term.odr.use},
\end{itemize}
or defines a constexpr variable initialized to a TU-local value (defined below).
\begin{note}
An inline function template can be an exposure even though
certain explicit specializations of it would be usable in other translation units.
\end{note}

\pnum
An entity is \defnx{TU-local}{TU-local!entity} if it is
\begin{itemize}
\item
a type, type alias, namespace, namespace alias, function, variable, or template that
\begin{itemize}
\item
has a name with internal linkage, or
\item
does not have a name with linkage and is declared,
or introduced by a \grammarterm{lambda-expression},
within the definition of a TU-local entity,
\end{itemize}
\item
a type with no name that is defined outside a
\grammarterm{class-specifier},
function body, or
\grammarterm{initializer}
or is introduced by a \grammarterm{defining-type-specifier}
that is used to declare only TU-local entities,
\item
a specialization of a TU-local template,
\item
a specialization of a template with any TU-local template argument, or
\item
a specialization of a template
whose (possibly instantiated) declaration is an exposure.
\begin{note}
A specialization can be produced by implicit or explicit instantiation.
\end{note}
\end{itemize}

\pnum
A value or object is \defnx{TU-local}{TU-local!value or object} if either
\begin{itemize}
\item
it is of TU-local type,
\item
it is, or is a pointer to,
a TU-local function or the object associated with a TU-local variable,
\item
it is an object of class or array type and
any of its subobjects or
any of the objects or functions
to which its non-static data members of reference type refer
is TU-local and is usable in constant expressions, or
\item
it is a reflection value\iref{basic.fundamental} that represents
\begin{itemize}
\item
an entity, value, or object that is TU-local,
\item
a direct base class relationship (\tcode{D}, \tcode{B})\iref{class.derived.general}
for which either \tcode{D} or \tcode{B} is TU-local, or
\item
a data member description ($T$, $N$, $A$, $W$, $\mathit{NUA}$)\iref{class.mem.general}
for which $T$ is TU-local.
\end{itemize}
\end{itemize}

\pnum
If a (possibly instantiated) declaration of, or a deduction guide for,
a non-TU-local entity in a module interface unit
(outside the \grammarterm{private-module-fragment}, if any) or
module partition\iref{module.unit} is an exposure,
the program is ill-formed.
Such a declaration in any other context is deprecated\iref{depr.local}.

\pnum
If a declaration that appears in one translation unit
names a TU-local entity declared
in another translation unit that is not a header unit,
the program is ill-formed.
A declaration instantiated for a template specialization\iref{temp.spec}
appears at the point of instantiation of the specialization\iref{temp.point}.

\pnum
\begin{example}
\begin{codeblocktu}{Translation unit \#1}
export module A;
static void f() {}
inline void it() { f(); }           // error: is an exposure of \tcode{f}
static inline void its() { f(); }   // OK
template<int> void g() { its(); }   // OK
template void g<0>();

decltype(f) *fp;                    // error: \tcode{f} (though not its type) is TU-local
auto &fr = f;                       // OK
constexpr auto &fr2 = fr;           // error: is an exposure of \tcode{f}
constexpr static auto fp2 = fr;     // OK

struct S { void (&ref)(); } s{f};               // OK, value is TU-local
constexpr extern struct W { S &s; } wrap{s};    // OK, value is not TU-local

static auto x = []{f();};           // OK
auto x2 = x;                        // error: the closure type is TU-local
int y = ([]{f();}(),0);             // error: the closure type is not TU-local
int y2 = (x,0);                     // OK

namespace N {
  struct A {};
  void adl(A);
  static void adl(int);
}
void adl(double);

inline void h(auto x) { adl(x); }   // OK, but certain specializations are exposures

constexpr std::meta::info r1 = ^^g<0>;  // OK
namespace N2 {
  static constexpr std::meta::info r2 = ^^g<1>;     // OK, \tcode{r2} is TU-local
}
constexpr std::meta::info r3 = ^^f;                 // error: \tcode{r3} is an exposure of \tcode{f}

constexpr auto ctx = std::meta::access_context::current();
constexpr std::meta::info r4 =
  std::meta::members_of(^^N2, ctx)[0];              // error: \tcode{r4} is an exposure of \tcode{N2::r2}
\end{codeblocktu}
\begin{codeblocktu}{Translation unit \#2}
module A;
void other() {
  g<0>();                           // OK, specialization is explicitly instantiated
  g<1>();                           // error: instantiation uses TU-local \tcode{its}
  h(N::A{});                        // error: overload set contains TU-local \tcode{N::adl(int)}
  h(0);                             // OK, calls \tcode{adl(double)}
  adl(N::A{});                      // OK; \tcode{N::adl(int)} not found, calls \tcode{N::adl(N::A)}
  fr();                             // OK, calls \tcode{f}
  constexpr auto ptr = fr;          // error: \tcode{fr} is not usable in constant expressions here
}
\end{codeblocktu}
\end{example}

\rSec1[basic.memobj]{Memory and objects}

\rSec2[intro.memory]{Memory model}

\pnum
\indextext{memory model|(}%
The fundamental storage unit in the \Cpp{} memory model is the
\defn{byte}.
A byte is at least large enough to contain
the ordinary literal encoding of any element of the basic
\indextext{character set!basic literal}%
literal character set\iref{lex.charset}
and the eight-bit code units of the Unicode
\indextext{UTF-8}%
UTF-8 encoding form
and is composed of a contiguous sequence of
bits,
\begin{footnote}
The number of bits in a byte is reported by the macro
\tcode{CHAR_BIT} in the header \libheaderref{climits}.
\end{footnote}
the number of which is \impldef{bits in a byte}.
The memory available to a \Cpp{} program consists of one or more sequences of
contiguous bytes.
Every byte has a unique address.

\pnum
\begin{note}
The representation of types is described
in~\ref{basic.types.general}.
\end{note}

\pnum
A \defn{memory location} is
the storage occupied by the object representation of
either an object of scalar type that is not a bit-field
or a maximal sequence of adjacent bit-fields all having nonzero width.
\begin{note}
Various
features of the language, such as references and virtual functions, might
involve additional memory locations that are not accessible to programs but are
managed by the implementation.
\end{note}
Two or more threads of
execution\iref{intro.multithread} can access separate memory
locations without interfering with each other.

\pnum
\begin{note}
Thus a bit-field and an adjacent non-bit-field are in separate memory
locations, and therefore can be concurrently updated by two threads of execution
without interference. The same applies to two bit-fields, if one is declared
inside a nested struct declaration and the other is not, or if the two are
separated by a zero-length bit-field declaration, or if they are separated by a
non-bit-field declaration. It is not safe to concurrently update two bit-fields
in the same struct if all fields between them are also bit-fields of nonzero
width.
\end{note}

\pnum
\begin{example}
A class declared as
\begin{codeblock}
struct {
  char a;
  int b:5,
  c:11,
  :0,
  d:8;
  struct {int ee:8;} e;
};
\end{codeblock}
contains four separate memory locations: The member \tcode{a} and bit-fields
\tcode{d} and \tcode{e.ee} are each separate memory locations, and can be
modified concurrently without interfering with each other. The bit-fields
\tcode{b} and \tcode{c} together constitute the fourth memory location. The
bit-fields \tcode{b} and \tcode{c} cannot be concurrently modified, but
\tcode{b} and \tcode{a}, for example, can be.
\end{example}
\indextext{memory model|)}

\rSec2[intro.object]{Object model}

\pnum
\indextext{object model|(}%
The constructs in a \Cpp{} program create, destroy, refer to, access, and
manipulate objects.
An \defn{object} is created
by a definition\iref{basic.def},
by a \grammarterm{new-expression}\iref{expr.new},
by an operation that implicitly creates objects (see below),
when implicitly changing the active member of a union\iref{class.union},
or
when a temporary object is created\iref{conv.rval,class.temporary}.
An object occupies a region of storage
in its period of construction\iref{class.cdtor},
throughout its lifetime\iref{basic.life},
and
in its period of destruction\iref{class.cdtor}.
\begin{note}
A function is not an object, regardless of whether or not it
occupies storage in the way that objects do.
\end{note}
The properties of an
object are determined when the object is created. An object can have a
name\iref{basic.pre}. An object has a storage
duration\iref{basic.stc} which influences its
lifetime\iref{basic.life}. An object has a
type\iref{basic.types}.
\begin{note}
Some objects are
polymorphic\iref{class.virtual}; the implementation
generates information associated with each such object that makes it
possible to determine that object's type during program execution.
\end{note}

\pnum
\indextext{subobject}%
Objects can contain other objects, called \defnx{subobjects}{subobject}.
A subobject can be
a \defn{member subobject}\iref{class.mem}, a \defn{base class subobject}\iref{class.derived},
or an array element.
\indextext{object!complete}%
An object that is not a subobject of any other object is called a \defn{complete
object}.
If an object is created
in storage associated with a member subobject or array element \placeholder{e}
(which may or may not be within its lifetime),
the created object
is a subobject of \placeholder{e}'s containing object if
\begin{itemize}
\item
the lifetime of \placeholder{e}'s containing object has begun and not ended, and
\item
the storage for the new object exactly overlays the storage location associated with \placeholder{e}, and
\item
the new object is of the same type as \placeholder{e} (ignoring cv-qualification).
\end{itemize}

\pnum
\indextext{object!providing storage for}%
If a complete object is created\iref{expr.new}
in storage associated with another object \placeholder{e}
of type ``array of $N$ \tcode{\keyword{unsigned} \keyword{char}}'' or
of type ``array of $N$ \tcode{std::byte}''\iref{cstddef.syn},
that array \defn{provides storage}
for the created object if
\begin{itemize}
\item
the lifetime of \placeholder{e} has begun and not ended, and
\item
the storage for the new object fits entirely within \placeholder{e}, and
\item
there is no array object that satisfies these constraints nested within \placeholder{e}.
\end{itemize}
\begin{note}
If that portion of the array
previously provided storage for another object,
the lifetime of that object ends
because its storage was reused\iref{basic.life}.
\end{note}
\begin{example}
\begin{codeblock}
// assumes that \tcode{sizeof(int)} is equal to 4

template<typename ...T>
struct AlignedUnion {
  alignas(T...) unsigned char data[max(sizeof(T)...)];
};
int f() {
  AlignedUnion<int, char> au;
  int *p = new (au.data) int;           // OK, \tcode{au.data} provides storage
  char *c = new (au.data) char();       // OK, ends lifetime of \tcode{*p}
  char *d = new (au.data + 1) char();
  return *c + *d;                       // OK
}

struct A { unsigned char a[32]; };
struct B { unsigned char b[16]; };
alignas(int) A a;
B *b = new (a.a + 8) B;                 // \tcode{a.a} provides storage for \tcode{*b}
int *p = new (b->b + 4) int;            // \tcode{b->b} provides storage for \tcode{*p}
                                        // \tcode{a.a} does not provide storage for \tcode{*p} (directly),
                                        // but \tcode{*p} is nested within \tcode{a} (see below)
\end{codeblock}
\end{example}

\pnum
\indextext{object!nested within}%
An object \placeholder{a} is \defn{nested within} another object \placeholder{b} if
\begin{itemize}
\item
\placeholder{a} is a subobject of \placeholder{b}, or
\item
\placeholder{b} provides storage for \placeholder{a}, or
\item
there exists an object \placeholder{c}
where \placeholder{a} is nested within \placeholder{c},
and \placeholder{c} is nested within \placeholder{b}.
\end{itemize}

\pnum
For every object \tcode{x}, there is some object called the
\defn{complete object of} \tcode{x}, determined as follows:
\begin{itemize}
\item
If \tcode{x} is a complete object, then the complete object
of \tcode{x} is itself.

\item
Otherwise, the complete object of \tcode{x} is the complete object
of the (unique) object that contains \tcode{x}.
\end{itemize}

\pnum
If a complete object, a member subobject, or an array element is of
class type, its type is considered the \defn{most derived
class}, to distinguish it from the class type of any base class subobject;
an object of a most derived class type or of a non-class type is called a
\defn{most derived object}.

\pnum
A \defn{potentially-overlapping subobject} is either:
\begin{itemize}
\item a base class subobject, or
\item a non-static data member
declared with the \tcode{no_unique_address} attribute\iref{dcl.attr.nouniqueaddr}.
\end{itemize}

\pnum
\indextext{object!zero size}%
\indextext{object!nonzero size}%
An object has nonzero size if it
\begin{itemize}
\item is not a potentially-overlapping subobject, or
\item is not of class type, or
\item is of a class type with virtual member functions or virtual base classes, or
\item has subobjects of nonzero size or unnamed bit-fields of nonzero length.
\end{itemize}
Otherwise, if the object is a base class subobject
of a standard-layout class type
with no non-static data members,
it has zero size.
Otherwise, the circumstances under which the object has zero size
are \impldef{which non-standard-layout objects
containing no data are considered empty}.
\indextext{most derived object!bit-field}%
Unless it is a bit-field\iref{class.bit},
an object with nonzero size
shall occupy one or more bytes of storage,
including every byte that is occupied in full or in part
by any of its subobjects.
An object of trivially copyable or
standard-layout type\iref{basic.types.general} shall occupy contiguous bytes of
storage.

\pnum
An object is a \defnadj{potentially non-unique}{object} if it is
\begin{itemize}
\item
  a string literal object\iref{lex.string},
\item
  the backing array of an initializer list\iref{dcl.init.ref}, or
\item
  the object introduced by a call to \tcode{std::meta::reflect_constant_array}
  or \tcode{std::meta::\brk{}reflect_con\-stant_string}\iref{meta.reflection.array}, or
\item
  a subobject thereof.
\end{itemize}

\pnum
\indextext{most derived object!bit-field}%
\indextext{most derived object!zero size subobject}%
Unless an object is a bit-field or a subobject of zero size, the
address of that object is the address of the first byte it occupies.
Two objects
with overlapping lifetimes
that are not bit-fields
may have the same address if
\begin{itemize}
\item one is nested within the other,
\item at least one is a subobject of zero size and they are not of similar types\iref{conv.qual},
or
\item they are both potentially non-unique objects;
\end{itemize}
otherwise, they have distinct addresses
and occupy disjoint bytes of storage.
\begin{footnote}
Under the ``as-if'' rule an
implementation is allowed to store two objects at the same machine address or
not store an object at all if the program cannot observe the
difference\iref{intro.execution}.
\end{footnote}
\begin{example}
\begin{codeblock}
static const char test1 = 'x';
static const char test2 = 'x';
const bool b = &test1 != &test2;        // always \tcode{true}

static const char (&r) [] = "x";
static const char *s = "x";
static std::initializer_list<char> il = { 'x' };
const bool b2 = r != il.begin();        // unspecified result
const bool b3 = r != s;                 // unspecified result
const bool b4 = il.begin() != &test1;   // always \tcode{true}
const bool b5 = r != &test1;            // always \tcode{true}
\end{codeblock}
\end{example}
The address of a non-bit-field subobject of zero size is
the address of an unspecified byte of storage
occupied by the complete object of that subobject.

\pnum
Some operations are described as
\defnx{implicitly creating objects}{object!implicit creation}
within a specified region of storage.
For each operation that is specified as implicitly creating objects,
that operation implicitly creates and starts the lifetime of
zero or more objects of implicit-lifetime types\iref{term.implicit.lifetime.type}
in its specified region of storage
if doing so would result in the program having defined behavior.
If no such set of objects would give the program defined behavior,
the behavior of the program is undefined.
If multiple such sets of objects would give the program defined behavior,
it is unspecified which such set of objects is created.
\begin{note}
Such operations do not start the lifetimes of subobjects of such objects
that are not themselves of implicit-lifetime types.
\end{note}

\pnum
Further, after implicitly creating objects within a specified region of storage,
some operations are described as producing a pointer to a
\defnadj{suitable created}{object}.
These operations select one of the implicitly-created objects
whose address is the address of the start of the region of storage,
and produce a pointer value that points to that object,
if that value would result in the program having defined behavior.
If no such pointer value would give the program defined behavior,
the behavior of the program is undefined.
If multiple such pointer values would give the program defined behavior,
it is unspecified which such pointer value is produced.

\pnum
\begin{example}
\begin{codeblock}
#include <cstdlib>
struct X { int a, b; };
X *make_x() {
  // The call to \tcode{std::malloc} implicitly creates an object of type \tcode{X}
  // and its subobjects \tcode{a} and \tcode{b}, and returns a pointer to that \tcode{X} object
  // (or an object that is pointer-interconvertible\iref{basic.compound} with it),
  // in order to give the subsequent class member access operations
  // defined behavior.
  X *p = (X*)std::malloc(sizeof(struct X));
  p->a = 1;
  p->b = 2;
  return p;
}
\end{codeblock}
\end{example}

\pnum
Except during constant evaluation,
an operation that begins the lifetime of
an array of \tcode{unsigned char} or \tcode{std::byte}
implicitly creates objects within the region of storage occupied by the array.
\begin{note}
The array object provides storage for these objects.
\end{note}
Except during constant evaluation,
any implicit or explicit invocation of a function
named \tcode{\keyword{operator} \keyword{new}} or \tcode{\keyword{operator} \keyword{new}[]}
implicitly creates objects in the returned region of storage and
returns a pointer to a suitable created object.
\begin{note}
Some functions in the \Cpp{} standard library implicitly create
objects\iref{obj.lifetime,c.malloc,mem.res.public,bit.cast,cstring.syn}.
\end{note}
\indextext{object model|)}

\rSec2[basic.align]{Alignment}

\pnum
Object types have \defnx{alignment requirements}{alignment requirement!implementation-defined}\iref{basic.fundamental,basic.compound}
which place restrictions on the addresses at which an object of that type
may be allocated. An \defn{alignment} is an \impldef{alignment}
integer value representing the number of bytes between successive addresses
at which a given object can be allocated. An object type imposes an alignment
requirement on every object of that type; stricter alignment can be requested
using the alignment specifier\iref{dcl.align}.
Attempting to create an object\iref{intro.object} in storage that
does not meet the alignment requirements of the object's type
is undefined behavior.

\pnum
A \defnadj{fundamental}{alignment} is represented by an alignment
less than or equal to the greatest alignment supported by the implementation in
all contexts, which is equal to
\tcode{\keyword{alignof}(std::max_align_t)}\iref{support.types}.
The alignment required for a type may be different when it is used as the type
of a complete object and when it is used as the type of a subobject.
\begin{example}
\begin{codeblock}
struct B { long double d; };
struct D : virtual B { char c; };
\end{codeblock}

When \tcode{D} is the type of a complete object, it will have a subobject of
type \tcode{B}, so it must be aligned appropriately for a \tcode{\keyword{long} \keyword{double}}.
If \tcode{D} appears as a subobject of another object that also has \tcode{B}
as a virtual base class, the \tcode{B} subobject might be part of a different
subobject, reducing the alignment requirements on the \tcode{D} subobject.
\end{example}
The result of the \keyword{alignof} operator reflects the alignment
requirement of the type in the complete-object case.

\pnum
An \defnadj{extended}{alignment} is represented by an alignment
greater than \tcode{\keyword{alignof}(std::max_align_t)}. It is \impldef{support for extended alignments}
whether any extended alignments are supported and the contexts in which they are
supported\iref{dcl.align}. A type having an extended alignment
requirement is an \defnadj{over-aligned}{type}.
\begin{note}
Every over-aligned type is or contains a class type
to which extended alignment applies (possibly through a non-static data member).
\end{note}
A \defnadj{new-extended}{alignment} is represented by
an alignment greater than \mname{STDCPP_DEFAULT_NEW_ALIGNMENT}\iref{cpp.predefined}.

\pnum
Alignments are represented as values of the type \tcode{std::size_t}.
Valid alignments include only those values returned by an \keyword{alignof}
expression for the fundamental types plus an additional \impldef{alignment additional
values}
set of values, which may be empty.
Every alignment value shall be a non-negative integral power of two.

\pnum
Alignments have an order from \defnx{weaker}{alignment!weaker} to
\defnx{stronger}{alignment!stronger} or \defnx{stricter}{alignment!stricter} alignments. Stricter
alignments have larger alignment values. An address that satisfies an alignment
requirement also satisfies any weaker valid alignment requirement.

\pnum
The alignment requirement of a complete type can be queried using an
\keyword{alignof} expression\iref{expr.alignof}. Furthermore,
the narrow character types\iref{basic.fundamental} shall have the weakest
alignment requirement.
\begin{note}
This enables the ordinary character types to be used as the
underlying type for an aligned memory area\iref{dcl.align}.
\end{note}

\pnum
Comparing alignments is meaningful and provides the obvious results:

\begin{itemize}
\item Two alignments are equal when their numeric values are equal.
\item Two alignments are different when their numeric values are not equal.
\item When an alignment is larger than another it represents a stricter alignment.
\end{itemize}

\pnum
\begin{note}
The runtime pointer alignment function\iref{ptr.align}
can be used to obtain an aligned pointer within a buffer;
an \grammarterm{alignment-specifier}\iref{dcl.align}
can be used to align storage explicitly.
\end{note}

\pnum
If a request for a specific extended alignment in a specific context is not
supported by an implementation, the program is ill-formed.

\rSec2[basic.life]{Lifetime}

\pnum
In this subclause, ``before'' and ``after'' refer to the ``happens before''
relation\iref{intro.multithread}.

\pnum
\indextext{object lifetime|(}%
The \defn{lifetime} of an object or reference is a runtime property of the
object or reference.
A variable is said to have \defnadj{vacuous}{initialization}
if it is default-initialized, no other initialization is performed, and,
if it is of class type or a (possibly multidimensional) array thereof,
a trivial constructor of that class type is selected for
the default-initialization.
The lifetime of an object of type \tcode{T} begins when:
\begin{itemize}
\item storage with the proper alignment and size
  for type \tcode{T} is obtained, and
\item its initialization (if any) is complete
  (including vacuous initialization)\iref{dcl.init},
\end{itemize}
except that if the object is a union member or subobject thereof,
its lifetime only begins if that union member is the
initialized member in the union\iref{dcl.init.aggr,class.base.init},
or as described in
\ref{class.union}, \ref{class.copy.ctor}, and \ref{class.copy.assign},
and except as described in \ref{allocator.members}.
The lifetime of an object \placeholder{o} of type \tcode{T} ends when:
\begin{itemize}
\item if \tcode{T} is a non-class type, the object is destroyed, or
\item if \tcode{T} is a class type, the destructor call starts, or
\item the storage which the object occupies is released,
or is reused by an object that is not nested within \placeholder{o}\iref{intro.object}.
\end{itemize}
When evaluating a \grammarterm{new-expression},
storage is considered reused after it is returned from the allocation function,
but before the evaluation of the \grammarterm{new-initializer}\iref{expr.new}.
\begin{example}
\begin{codeblock}
struct S {
  int m;
};

void f() {
  S x{1};
  new(&x) S(x.m);   // undefined behavior
}
\end{codeblock}
\end{example}

\pnum
\indextext{reference lifetime}%
The lifetime of a reference begins when its initialization is complete.
The lifetime of a reference ends as if it were a scalar object requiring storage.

\pnum
\begin{note}
\ref{class.base.init}
describes the lifetime of base and member subobjects.
\end{note}

\pnum
The properties ascribed to objects and references throughout this document
apply for a given object or reference only during its lifetime.
\begin{note}
In particular, before the lifetime of an object starts and after its
lifetime ends there are significant restrictions on the use of the
object, as described below, in~\ref{class.base.init}, and
in~\ref{class.cdtor}. Also, the behavior of an object under construction
and destruction can differ from the behavior of an object whose
lifetime has started and not ended. \ref{class.base.init}
and~\ref{class.cdtor} describe the behavior of an object during its periods
of construction and destruction.
\end{note}

\pnum
A program may end the lifetime of an object of class type without invoking the
destructor, by reusing or releasing the storage as described above.
\begin{note}
A \grammarterm{delete-expression}\iref{expr.delete} invokes the destructor
prior to releasing the storage.
\end{note}
In this case, the destructor is not implicitly invoked.
\begin{note}
The correct behavior of a program often depends on
the destructor being invoked for each object of class type.
\end{note}

\pnum
Before the lifetime of an object has started but after the storage which
the object will occupy has been allocated
\begin{footnote}
For example, before the dynamic initialization of
an object with static storage duration\iref{basic.start.dynamic}.
\end{footnote}
or after the lifetime of an object has ended and before the storage
which the object occupied is reused or released, any pointer that represents the address of
the storage location where the object will be or was located may be
used but only in limited ways.
For an object under construction or destruction, see~\ref{class.cdtor}.
Otherwise, such
a pointer refers to allocated
storage\iref{basic.stc.dynamic.allocation}, and using the pointer as
if the pointer were of type \tcode{\keyword{void}*} is
well-defined. Indirection through such a pointer is permitted but the resulting lvalue may only be used in
limited ways, as described below. The
program has undefined behavior if
\begin{itemize}
\item
  the pointer is used as the operand of a \grammarterm{delete-expression},
\item
  the pointer is used to access a non-static data member or call a
  non-static member function of the object, or
\item
  the pointer is implicitly converted\iref{conv.ptr} to a pointer
  to a virtual base class, or
\item
  the pointer is used as the operand of a
  \keyword{static_cast}\iref{expr.static.cast}, except when the conversion
  is to pointer to \cv{}~\keyword{void}, or to pointer to \cv{}~\keyword{void}
  and subsequently to pointer to
  \cv{}~\keyword{char},
  \cv{}~\tcode{\keyword{unsigned} \keyword{char}}, or
  \cv{}~\tcode{std::byte}\iref{cstddef.syn}, or
\item
  the pointer is used as the operand of a
  \keyword{dynamic_cast}\iref{expr.dynamic.cast}.
\end{itemize}
\begin{example}
\begin{codeblock}
#include <cstdlib>

struct B {
  virtual void f();
  void mutate();
  virtual ~B();
};

struct D1 : B { void f(); };
struct D2 : B { void f(); };

void B::mutate() {
  new (this) D2;    // reuses storage --- ends the lifetime of \tcode{*this}
  f();              // undefined behavior
  ... = this;       // OK, \keyword{this} points to valid memory
}

void g() {
  void* p = std::malloc(sizeof(D1) + sizeof(D2));
  B* pb = new (p) D1;
  pb->mutate();
  *pb;              // OK, \tcode{pb} points to valid memory
  void* q = pb;     // OK, \tcode{pb} points to valid memory
  pb->f();          // undefined behavior: lifetime of \tcode{*pb} has ended
}
\end{codeblock}
\end{example}

\pnum
Similarly, before the lifetime of an object has started but after the
storage which the object will occupy has been allocated or after the
lifetime of an object has ended and before the storage which the object
occupied is reused or released, any glvalue that refers to the original
object may be used but only in limited ways.
For an object under construction or destruction, see~\ref{class.cdtor}.
Otherwise, such
a glvalue refers to
allocated storage\iref{basic.stc.dynamic.allocation}, and using the
properties of the glvalue that do not depend on its value is
well-defined. The program has undefined behavior if
\begin{itemize}
\item the glvalue is used to access the object, or
\item the glvalue is used to call a non-static member function of the object, or
\item the glvalue is bound to a reference to a virtual base class\iref{dcl.init.ref}, or
\item the glvalue is used as the operand of a
\keyword{dynamic_cast}\iref{expr.dynamic.cast} or as the operand of
\keyword{typeid}.
\end{itemize}

\begin{note}
Therefore, undefined behavior results
if an object that is being constructed in one thread is referenced from another
thread without adequate synchronization.
\end{note}

\pnum
An object $o_1$ is \defn{transparently replaceable} by an object $o_2$ if
\begin{itemize}
\item the storage that $o_2$ occupies exactly overlays
the storage that $o_1$ occupied, and

\item $o_1$ and $o_2$ are of the same type
(ignoring the top-level cv-qualifiers), and

\item $o_1$ is not a const, complete object, and

\item neither $o_1$ nor $o_2$
is a potentially-overlapping subobject\iref{intro.object}, and

\item either $o_1$ and $o_2$ are both complete objects, or
$o_1$ and $o_2$ are direct subobjects of objects $p_1$ and $p_2$, respectively,
and $p_1$ is transparently replaceable by $p_2$.
\end{itemize}

\pnum
After the lifetime of an object has ended and before the storage which the
object occupied is reused or released, if a new object is created at the
storage location which the original object occupied and the original object was
transparently replaceable by the new object, a pointer that pointed to the
original object, a reference that referred to the original object, or the name
of the original object will automatically refer to the new object and, once the
lifetime of the new object has started, can be used to manipulate the new
object.

\begin{example}
\begin{codeblock}
struct C {
  int i;
  void f();
  const C& operator=( const C& );
};

const C& C::operator=( const C& other) {
  if ( this != &other ) {
    this->~C();                 // lifetime of \tcode{*this} ends
    new (this) C(other);        // new object of type \tcode{C} created
    f();                        // well-defined
  }
  return *this;
}

C c1;
C c2;
c1 = c2;                        // well-defined
c1.f();                         // well-defined; \tcode{c1} refers to a new object of type \tcode{C}
\end{codeblock}
\end{example}
\begin{note}
If these conditions are not met,
a pointer to the new object can be obtained from
a pointer that represents the address of its storage
by calling \tcode{std::launder}\iref{ptr.launder}.
\end{note}

\pnum
If a program ends the lifetime of an object of type \tcode{T} with
static\iref{basic.stc.static}, thread\iref{basic.stc.thread},
or automatic\iref{basic.stc.auto}
storage duration and if \tcode{T} has a non-trivial destructor,
\begin{footnote}
That
is, an object for which a destructor will be called
implicitly---upon exit from the block for an object with
automatic storage duration, upon exit from the thread for an object with
thread storage duration, or upon exit from the program for an object
with static storage duration.
\end{footnote}
and another object of the original type does not occupy
that same storage location when the implicit destructor call takes
place, the behavior of the program is undefined. This is true
even if the block is exited with an exception.
\begin{example}
\begin{codeblock}
class T { };
struct B {
  ~B();
};

void h() {
  B b;
  new (&b) T;
}                               // undefined behavior at block exit
\end{codeblock}
\end{example}

\pnum
Creating a new object within the storage that a const, complete
object with static, thread, or automatic storage duration occupies,
or within the storage that such a const object used to occupy before
its lifetime ended, results in undefined behavior.
\begin{example}
\begin{codeblock}
struct B {
  B();
  ~B();
};

const B b;

void h() {
  b.~B();
  new (const_cast<B*>(&b)) const B;     // undefined behavior
}
\end{codeblock}
\end{example}
\indextext{object lifetime|)}

\rSec2[basic.indet]{Indeterminate and erroneous values}

\pnum
When storage for an object with automatic or dynamic storage duration
is obtained,
the bytes comprising the storage for the object
have the following initial value:
\begin{itemize}
\item
If the object has dynamic storage duration, or
is the object associated with a variable or function parameter
whose first declaration is marked with
the \tcode{[[indeterminate]]} attribute\iref{dcl.attr.indet},
the bytes have \defnadjx{indeterminate}{values}{value};
\item
otherwise, the bytes have \defnadjx{erroneous}{values}{value},
where each value is determined by the implementation
independently of the state of the program.
\end{itemize}
If no initialization is performed for an object (including subobjects),
such a byte retains its initial value
until that value is replaced\iref{dcl.init.general,expr.assign}.
If any bit in the value representation has an indeterminate value,
the object has an indeterminate value;
otherwise, if any bit in the value representation has an erroneous value,
the object has an erroneous value\iref{conv.lval}.
\begin{note}
Objects with static or thread storage duration are zero-initialized,
see~\ref{basic.start.static}.
\end{note}

\pnum
Except in the following cases,
if an indeterminate value is produced by an evaluation,
the behavior is undefined, and
if an erroneous value is produced by an evaluation,
the behavior is erroneous and
the result of the evaluation is the value so produced but is not erroneous:
\begin{itemize}
\item
  If an indeterminate or erroneous value of
  unsigned ordinary character type\iref{basic.fundamental}
  or \tcode{std::byte} type\iref{cstddef.syn}
  is produced by the evaluation of:
  \begin{itemize}
  \item
    the second or third operand of a conditional expression\iref{expr.cond},
  \item
    the right operand of a comma expression\iref{expr.comma},
  \item
    the operand of a cast or conversion\iref{conv.integral,
    expr.type.conv,expr.static.cast,expr.cast}
    to an unsigned ordinary character type
    or \tcode{std::byte} type\iref{cstddef.syn}, or
  \item
    a discarded-value expression\iref{expr.context},
  \end{itemize}
  then the result of the operation is an indeterminate value or
  that erroneous value, respectively.
\item
  If an indeterminate or erroneous value of
  unsigned ordinary character type or \tcode{std::byte} type
  is produced by the evaluation of
  the right operand of a simple assignment operator\iref{expr.assign}
  whose first operand is an lvalue of
  unsigned ordinary character type or \tcode{std::byte} type,
  an indeterminate value or that erroneous value, respectively, replaces
  the value of the object referred to by the left operand.
\item
  If an indeterminate or erroneous value of unsigned ordinary character type
  is produced by the evaluation of the initialization expression
  when initializing an object of unsigned ordinary character type,
  that object is initialized to an indeterminate
  value or that erroneous value, respectively.
\item
  If an indeterminate value of
  unsigned ordinary character type or \tcode{std::byte} type
  is produced by the evaluation of the initialization expression
  when initializing an object of \tcode{std::byte} type,
  that object is initialized to an indeterminate value or
  that erroneous value, respectively.
\end{itemize}
Converting an indeterminate or erroneous value of
unsigned ordinary character type or \tcode{std::byte} type
produces an indeterminate or erroneous value, respectively.
In the latter case,
the result of the conversion is the value of the converted operand.
\begin{example}
\begin{codeblock}
int f(bool b) {
  unsigned char *c = new unsigned char;
  unsigned char d = *c;         // OK, \tcode{d} has an indeterminate value
  int e = d;                    // undefined behavior
  return b ? d : 0;             // undefined behavior if \tcode{b} is \tcode{true}
}

int g(bool b) {
  unsigned char c;
  unsigned char d = c;          // no erroneous behavior, but \tcode{d} has an erroneous value

  assert(c == d);               // holds, both integral promotions have erroneous behavior

  int e = d;                    // erroneous behavior
  return b ? d : 0;             // erroneous behavior if \tcode{b} is \tcode{true}
}

void h() {
  int d1, d2;

  int e1 = d1;                  // erroneous behavior
  int e2 = d1;                  // erroneous behavior

  assert(e1 == e2);             // holds
  assert(e1 == d1);             // holds, erroneous behavior
  assert(e2 == d1);             // holds, erroneous behavior

  std::memcpy(&d2, &d1, sizeof(int));   // no erroneous behavior, but \tcode{d2} has an erroneous value
  assert(e1 == d2);             // holds, erroneous behavior
  assert(e2 == d2);             // holds, erroneous behavior
}
\end{codeblock}
\end{example}

\rSec2[basic.stc]{Storage duration}

\rSec3[basic.stc.general]{General}

\pnum
\indextext{storage duration|(}%
The \defn{storage duration} is the property of an object that defines the minimum
potential lifetime of the storage containing the object. The storage
duration is determined by the construct used to create the object and is
one of the following:
\begin{itemize}
\item static storage duration
\item thread storage duration
\item automatic storage duration
\item dynamic storage duration
\end{itemize}
\begin{note}
After the duration of a region of storage has ended,
the use of pointers to that region of storage is limited\iref{basic.compound}.
\end{note}

\pnum
\indextext{storage duration!static}%
\indextext{storage duration!thread}%
\indextext{storage duration!automatic}%
\indextext{storage duration!dynamic}%
Static, thread, and automatic storage durations are associated with objects
introduced by declarations\iref{basic.def} and
with temporary objects\iref{class.temporary}.
The dynamic storage duration
is associated with objects created by a
\grammarterm{new-expression}\iref{expr.new} or
with implicitly created objects\iref{intro.object}.

\pnum
The storage duration categories apply to references as well.

\pnum
\indextext{storage duration!class member}%
The storage duration of subobjects and reference members
is that of their complete object\iref{intro.object}.
\indextext{storage duration|)}%

\rSec3[basic.stc.static]{Static storage duration}

\pnum
All variables which
\begin{itemize}
\item
do not have thread storage duration and
\item
belong to a namespace scope\iref{basic.scope.namespace} or
are first declared with
the \keyword{static} or \keyword{extern} keywords\iref{dcl.stc}
\end{itemize}
have \defnadj{static}{storage duration}.
The storage for these entities lasts for the duration of the
program\iref{basic.start.static,basic.start.term}.

\pnum
If a variable with static storage duration has initialization or a
destructor with side effects, it shall not be eliminated even if it
appears to be unused, except that a class object or its copy/move may be
eliminated as specified in~\ref{class.copy.elision}.

\pnum
\begin{note}
\indextext{object!local static@local \tcode{static}}%
The keyword \keyword{static} can be used to declare
a block variable\iref{basic.scope.block} with static storage duration;
\ref{stmt.dcl} and \ref{basic.start.term} describe the
initialization and destruction of such variables.
\indextext{member!class static@class \tcode{static}}%
The keyword \keyword{static} applied to
a class data member in a class definition
gives the data member static storage duration\iref{class.static.data}.
\end{note}

\rSec3[basic.stc.thread]{Thread storage duration}

\pnum
All variables declared with the \keyword{thread_local} keyword have
\defnadj{thread}{storage duration}.
The storage for these entities lasts for the duration of
the thread in which they are created. There is a distinct object or reference
per thread, and use of the declared name refers to the entity associated with
the current thread.

\pnum
\begin{note}
A variable with thread storage duration is initialized as specified
in~\ref{basic.start.static}, \ref{basic.start.dynamic}, and \ref{stmt.dcl}
and, if constructed, is destroyed on thread exit\iref{basic.start.term}.
\end{note}

\rSec3[basic.stc.auto]{Automatic storage duration}

\pnum
\indextext{storage duration!local object}%
Variables that belong to a block scope and are
not explicitly declared \keyword{static}, \keyword{thread_local}, or \keyword{extern} have
\defnadj{automatic}{storage duration}. The storage
for such variables lasts until the block in which they are created exits.
\begin{note}
These variables are initialized and destroyed as described in~\ref{stmt.dcl}.
\end{note}
Variables that belong to a parameter scope also have automatic storage duration.
The storage for a function parameter lasts until
immediately after its destruction\iref{expr.call}.

\pnum
If a variable with automatic storage duration has initialization or a destructor with side
effects, an implementation shall not destroy it before the end of its block
nor eliminate it as an optimization, even if it appears to be
unused, except that a class object or its copy/move may be eliminated as
specified in~\ref{class.copy.elision}.

\rSec3[basic.stc.dynamic]{Dynamic storage duration}%

\rSec4[basic.stc.dynamic.general]{General}%
\indextext{storage duration!dynamic|(}

\pnum
Objects can be created dynamically during program
execution\iref{intro.execution}, using
\indextext{\idxcode{new}}%
\grammarterm{new-expression}{s}\iref{expr.new}, and destroyed using
\indextext{\idxcode{delete}}%
\grammarterm{delete-expression}{s}\iref{expr.delete}. A \Cpp{} implementation
provides access to, and management of, dynamic storage via
the global \defnx{allocation functions}{allocation function}
\tcode{\keyword{operator} \keyword{new}} and
\tcode{\keyword{operator} \keyword{new}[]} and
the global \defnx{deallocation functions}{deallocation function}
\tcode{\keyword{operator} \keyword{delete}} and
\tcode{\keyword{operator} \keyword{delete}[]}.
\begin{note}
The non-allocating forms described in \ref{new.delete.placement}
do not perform allocation or deallocation.
\end{note}

\pnum
The library provides default definitions for the global allocation and
deallocation functions. Some global allocation and deallocation
functions are replaceable\iref{term.replaceable.function}.
The following allocation and deallocation functions\iref{support.dynamic}
are implicitly declared in global scope in each translation unit of a
program.

\begin{codeblock}
void* operator new(std::size_t);
void* operator new(std::size_t, std::align_val_t);

void operator delete(void*) noexcept;
void operator delete(void*, std::size_t) noexcept;
void operator delete(void*, std::align_val_t) noexcept;
void operator delete(void*, std::size_t, std::align_val_t) noexcept;

void* operator new[](std::size_t);
void* operator new[](std::size_t, std::align_val_t);

void operator delete[](void*) noexcept;
void operator delete[](void*, std::size_t) noexcept;
void operator delete[](void*, std::align_val_t) noexcept;
void operator delete[](void*, std::size_t, std::align_val_t) noexcept;
\end{codeblock}

These implicit declarations introduce only the function names
\tcode{\keyword{operator} \keyword{new}},
\tcode{\keyword{operator} \keyword{new}[]},
\tcode{\keyword{operator} \keyword{delete}}, and
\tcode{\keyword{operator} \keyword{delete}[]}.
\begin{note}
The implicit declarations do not introduce
the names \tcode{std},
\tcode{std::size_t},
\tcode{std::align_val_t},
or any other names that the library uses to
declare these names. Thus, a \grammarterm{new-expression},
\grammarterm{delete-expression}, or function call that refers to one of
these functions without importing or including the header \libheaderref{new}
or importing a \Cpp{} library module\iref{std.modules}
is well-formed. However, referring to \tcode{std}
or \tcode{std::size_t}
or \tcode{std::align_val_t}
is ill-formed unless
a standard library declaration\iref{cstddef.syn,new.syn,std.modules}
of that name precedes\iref{basic.lookup.general} the use of that name.
\end{note}
Allocation and/or
deallocation functions may also be declared and defined for any
class\iref{class.free}.

\pnum
If the behavior of an allocation or deallocation function
does not satisfy the semantic constraints
specified in~\ref{basic.stc.dynamic.allocation}
and~\ref{basic.stc.dynamic.deallocation},
the behavior is undefined.

\indextext{storage duration!dynamic|)}

\rSec4[basic.stc.dynamic.allocation]{Allocation functions}

\pnum
\indextext{function!allocation}%
An allocation function that is not a class member function
shall belong to the global scope and not have a name with internal linkage.
The return type shall be \tcode{\keyword{void}*}. The first
parameter shall have type \tcode{std::size_t}\iref{support.types}. The
first parameter shall not have an associated default
argument\iref{dcl.fct.default}. The value of the first parameter
is interpreted as the requested size of the allocation. An allocation
function can be a function template. Such a template shall declare its
return type and first parameter as specified above (that is, template
parameter types shall not be used in the return type and first parameter
type). Allocation function templates shall have two or more parameters.

\pnum
An allocation function attempts to allocate the requested amount of
storage. If it is successful, it returns the address of the start
of a block of storage whose length in bytes is at least as large
as the requested size.
The order,
contiguity, and initial value of storage allocated by successive calls
to an allocation function are unspecified.
Even if the size of the space
requested is zero, the request can fail. If the request succeeds, the
value returned by a replaceable allocation function
is a non-null pointer value\iref{basic.compound}
\tcode{p0} different from any previously returned value \tcode{p1},
unless that value \tcode{p1} was subsequently passed to a
replaceable deallocation function.
Furthermore, for the library allocation functions
in~\ref{new.delete.single} and~\ref{new.delete.array},
\tcode{p0} represents the address of a block of storage disjoint from the storage
for any other object accessible to the caller.
The effect of indirecting through a pointer
returned from a request for zero size is undefined.
\begin{footnote}
The intent is
to have \tcode{\keyword{operator} \keyword{new}()} implementable by
calling \tcode{std::malloc()} or \tcode{std::calloc()}, so the rules are
substantially the same. \Cpp{} differs from C in requiring a zero request
to return a non-null pointer.
\end{footnote}

\pnum
For an allocation function other than
a reserved placement allocation function\iref{new.delete.placement},
the pointer returned on a successful call
shall represent the address of storage that is aligned as follows:
\begin{itemize}
\item
  If the allocation function takes an argument
  of type \tcode{std::align_val_t},
  the storage will have the alignment specified
  by the value of this argument.
\item
  Otherwise, if the allocation function is named \tcode{\keyword{operator} \keyword{new}[]},
  the storage is aligned for any object that
  does not have new-extended alignment\iref{basic.align} and
  is no larger than the requested size.
\item
  Otherwise, the storage is aligned for any object that
  does not have new-extended alignment and is of the requested size.
\end{itemize}

\pnum
An allocation function that fails to allocate storage can invoke the
currently installed new-handler function\iref{new.handler}, if any.
\begin{note}
\indextext{\idxcode{new_handler}}%
A program-supplied allocation function can obtain the
currently installed \tcode{new_handler} using the
\tcode{std::get_new_handler} function\iref{get.new.handler}.
\end{note}
An allocation function that has a non-throwing
exception specification\iref{except.spec}
indicates failure by returning
a null pointer value.
Any other allocation function
never returns a null pointer value and
indicates failure only by throwing an exception\iref{except.throw} of a type
that would match a handler\iref{except.handle} of type
\tcode{std::bad_alloc}\iref{bad.alloc}.

\pnum
A global allocation function is only called as the result of a new
expression\iref{expr.new}, or called directly using the function call
syntax\iref{expr.call}, or called indirectly to allocate storage for
a coroutine state\iref{dcl.fct.def.coroutine},
or called indirectly through calls to the
functions in the \Cpp{} standard library.
\begin{note}
In particular, a
global allocation function is not called to allocate storage
for objects with static storage duration\iref{basic.stc.static},
for objects or references with thread storage duration\iref{basic.stc.thread},
for objects of type \tcode{std::type_info}\iref{expr.typeid},
for an object of type \tcode{std::contracts::contract_violation}
when a contract violation occurs\iref{basic.contract.eval}, or
for an exception object\iref{except.throw}.
\end{note}

\rSec4[basic.stc.dynamic.deallocation]{Deallocation functions}

\pnum
\indextext{function!deallocation}%
A deallocation function that is not a class member function
shall belong to the global scope and not have a name with internal linkage.

\pnum
A deallocation function
is a \defnadj{destroying}{operator delete}
if it has at least two parameters
and its second parameter
is of type \tcode{std::destroying_delete_t}.
A destroying operator delete
shall be a class member function named \tcode{\keyword{operator} \keyword{delete}}.
\begin{note}
Array deletion cannot use a destroying operator delete.
\end{note}

\pnum
\indextext{\idxcode{delete}!overloading and}%
Each deallocation function shall return \keyword{void}.
If the function is a destroying operator delete
declared in class type \tcode{C},
the type of its first parameter shall be \tcode{C*};
otherwise, the type of its first
parameter shall be \tcode{\keyword{void}*}. A deallocation function may have more
than one parameter.
\indextext{deallocation function!usual}%
A \defn{usual deallocation function} is a deallocation function
whose parameters after the first are
\begin{itemize}
\item
optionally, a parameter of type \tcode{std::destroying_delete_t}, then
\item
optionally, a parameter of type \tcode{std::size_t},
\begin{footnote}
The global \tcode{\keyword{operator} \keyword{delete}(\keyword{void}*, std::size_t)}
precludes use of an
allocation function \tcode{\keyword{void} \keyword{operator} \keyword{new}(std::size_t, std::size_t)} as a placement
allocation function\iref{diff.cpp11.basic}.
\end{footnote}
then
\item
optionally, a parameter of type \tcode{std::align_val_t}.
\end{itemize}
A destroying operator delete shall be a usual deallocation function.
A deallocation function may be an instance of a function
template. Neither the first parameter nor the return type shall depend
on a template parameter.
A deallocation
function template shall have two or more function parameters. A template
instance is never a usual deallocation function, regardless of its
signature.

\pnum
If a deallocation function terminates by throwing an exception, the behavior is undefined.
The value of the first argument supplied to a deallocation function may
be a null pointer value; if so, and if the deallocation function is one
supplied in the standard library, the call has no effect.

\pnum
If the argument given to a deallocation function in the standard library
is a pointer that is not the null pointer value\iref{basic.compound}, the
deallocation function shall deallocate the storage referenced by the
pointer, ending the duration of the region of storage.

\rSec2[class.temporary]{Temporary objects}

\pnum
\indextext{temporary}%
\indextext{optimization of temporary|see{temporary, elimination of}}%
\indextext{temporary!elimination of}%
\indextext{temporary!implementation-defined generation of}%
A \defnadj{temporary}{object} is an object created
\begin{itemize}
\item
when a prvalue is converted to an xvalue\iref{conv.rval} and
\item
when needed by the implementation to pass or return an object of suitable type (see below).
\end{itemize}
Even when the creation of the temporary object is
unevaluated\iref{expr.context},
all the semantic restrictions shall be respected as if the temporary object
had been created and later destroyed.
\begin{note}
This includes accessibility\iref{class.access} and whether it is deleted,
for the constructor selected and for the destructor. However, in the special
case of the operand of a
\grammarterm{decltype-specifier}\iref{dcl.type.decltype}, no temporary is introduced,
so the foregoing does not apply to such a prvalue.
\end{note}

\pnum
The materialization of a temporary object is generally
delayed as long as possible
in order to avoid creating unnecessary temporary objects.
\begin{note}
Temporary objects are materialized:
\begin{itemize}
\item
when binding a reference to a prvalue\iref{dcl.init.ref,expr.type.conv,
expr.dynamic.cast,expr.static.cast,expr.const.cast,expr.cast},
\item
when performing certain member accesses on a class prvalue\iref{expr.ref,expr.mptr.oper},
\item
when invoking an implicit object member function on a class prvalue\iref{expr.call},
\item
when performing an array-to-pointer conversion or subscripting on an array prvalue\iref{conv.array,expr.sub},
\item
when initializing an object of type \tcode{std::initializer_list<T>} from a \grammarterm{braced-init-list}\iref{dcl.init.list},
\item
for certain unevaluated operands\iref{expr.typeid,expr.sizeof}, and
\item
when a prvalue that has type other than \cv{}~\keyword{void} appears as a discarded-value expression\iref{expr.context}.
\end{itemize}
\end{note}
\begin{example}
Consider the following code:
\begin{codeblock}
class X {
public:
  X(int);
  X(const X&);
  X& operator=(const X&);
  ~X();
};

class Y {
public:
  Y(int);
  Y(Y&&);
  ~Y();
};

X f(X);
Y g(Y);

void h() {
  X a(1);
  X b = f(X(2));
  Y c = g(Y(3));
  a = f(a);
}
\end{codeblock}

\indextext{class object copy|see{constructor, copy}}%
\indextext{constructor!copy}%
\tcode{X(2)} is constructed in the space used to hold \tcode{f()}'s argument and
\tcode{Y(3)} is constructed in the space used to hold \tcode{g()}'s argument.
Likewise,
\tcode{f()}'s result is constructed directly in \tcode{b} and
\tcode{g()}'s result is constructed directly in \tcode{c}.
On the other hand, the expression
\tcode{a = f(a)}
requires a temporary for
the result of \tcode{f(a)},
which is materialized so that the reference parameter
of \tcode{X::operator=(const X\&)} can bind to it.
\end{example}

\pnum
When an object of class type \tcode{X}
is passed to or returned from a potentially-evaluated function call,
if \tcode{X} is
\begin{itemize}
\item
a scalar type or
\item
a class type that
has at least one eligible copy or move constructor\iref{special},
where each such constructor is trivial,
and the destructor of \tcode{X} is either trivial or deleted,
\end{itemize}
implementations are permitted
to create temporary objects
to hold the function parameter or result object,
as follows:
\begin{itemize}
\item
The first such temporary object
is constructed from the function argument or return value, respectively.
\item
Each successive temporary object
is initialized from the previous one
as if by direct-initialization if \tcode{X} is a scalar type,
otherwise by using an eligible trivial constructor.
\item
The function parameter or return object is initialized
from the final temporary
as if by direct-initialization if \tcode{X} is a scalar type,
otherwise by using an eligible trivial constructor.
\end{itemize}
(In all cases, the eligible constructor is used
even if that constructor is inaccessible
or would not be selected by overload resolution
to perform a copy or move of the object).
\begin{note}
This latitude is granted to allow objects
to be passed to or returned from functions in registers.
\end{note}

\pnum
\indextext{temporary!constructor for}%
\indextext{temporary!destructor for}%
\indextext{temporary!destruction of}%
Temporary objects are destroyed as the last step
in evaluating
the full-expression\iref{intro.execution}
that (lexically) contains the point where
they were created.
This is true even if that evaluation ends in throwing an exception.
The
\indextext{value computation}%
value computations and
\indextext{side effects}%
side effects of destroying a temporary object
are associated only with the full-expression, not with any specific
subexpression.

\pnum
\indextext{initializer!temporary and declarator}%
\indextext{temporary!order of destruction of}%
There are several contexts in which temporaries are destroyed at a different
point than the end of the full-expression.
The first context is when a default constructor is called to initialize
an element of an array with no corresponding initializer\iref{dcl.init}.
The second context is when a copy constructor is called to copy an element of
an array while the entire array is copied\iref{expr.prim.lambda.capture,class.copy.ctor}.
In either case, if the constructor has one or more default arguments,
the destruction of every temporary created in a default argument is
sequenced before the construction of the next array element, if any.

\pnum
The third context is when a reference binds to a temporary object.
\begin{footnote}
The same rules apply to initialization of an
  \tcode{initializer_list} object\iref{dcl.init.list} with its
  underlying temporary array.
\end{footnote}
The temporary object to which the reference is bound or the temporary object
that is the complete object of a subobject to which the reference is bound
persists for the lifetime of the reference if the glvalue
to which the reference is bound
was obtained through one of the following:
\begin{itemize}
\item
  a temporary materialization conversion\iref{conv.rval},
\item
  \tcode{(} \grammarterm{expression} \tcode{)},
  where \grammarterm{expression} is one of these expressions,
\item
  subscripting\iref{expr.sub} of an array operand,
  where that operand is one of these expressions,
\item
  a class member access\iref{expr.ref} using the \tcode{.} operator
  where the left operand is one of these expressions and
  the right operand designates a non-static data member of non-reference type,
\item
  a pointer-to-member operation\iref{expr.mptr.oper} using the \tcode{.*} operator
  where the left operand is one of these expressions and
  the right operand is a pointer to data member of non-reference type,
\item
  a
  \begin{itemize}
  \item \keyword{const_cast}\iref{expr.const.cast},
  \item \keyword{static_cast}\iref{expr.static.cast},
  \item \keyword{dynamic_cast}\iref{expr.dynamic.cast}, or
  \item \keyword{reinterpret_cast}\iref{expr.reinterpret.cast}
  \end{itemize}
  converting, without a user-defined conversion,
  a glvalue operand that is one of these expressions
  to a glvalue that refers
  to the object designated by the operand, or
  to its complete object or a subobject thereof,
\item
  a conditional expression\iref{expr.cond} that is a glvalue
  where the second or third operand is one of these expressions, or
\item
  a comma expression\iref{expr.comma} that is a glvalue
  where the right operand is one of these expressions.
\end{itemize}
\begin{example}
\begin{codeblock}
template<typename T> using id = T;

int i = 1;
int&& a = id<int[3]>{1, 2, 3}[i];           // temporary array has same lifetime as \tcode{a}
const int& b = static_cast<const int&>(0);  // temporary \tcode{int} has same lifetime as \tcode{b}
int&& c = cond ? id<int[3]>{1, 2, 3}[i] : static_cast<int&&>(0);
                                            // exactly one of the two temporaries is lifetime-extended
\end{codeblock}
\end{example}
\begin{note}
An explicit type conversion\iref{expr.type.conv,expr.cast}
is interpreted as
a sequence of elementary casts,
covered above.
\begin{example}
\begin{codeblock}
const int& x = (const int&)1;   // temporary for value 1 has same lifetime as x
\end{codeblock}
\end{example}
\end{note}
\begin{note}
If a temporary object has a reference member initialized by another temporary object,
lifetime extension applies recursively to such a member's initializer.
\begin{example}
\begin{codeblock}
struct S {
  const int& m;
};
const S& s = S{1};              // both \tcode{S} and \tcode{int} temporaries have lifetime of \tcode{s}
\end{codeblock}
\end{example}
\end{note}

The exceptions to this lifetime rule are:
\begin{itemize}
\item A temporary object bound to a reference parameter in a function call\iref{expr.call}
persists until the completion of the full-expression containing the call.

\item A temporary object bound to a reference element of
an aggregate of class type initialized from
a parenthesized \grammarterm{expression-list}\iref{dcl.init}
persists until the completion of the full-expression
containing the \grammarterm{expression-list}.

\item A temporary bound to a reference in a \grammarterm{new-initializer}\iref{expr.new} persists until the completion of the full-expression containing the \grammarterm{new-initializer}.
\begin{note}
This might introduce a dangling reference.
\end{note}
\begin{example}
\begin{codeblock}
struct S { int mi; const std::pair<int,int>& mp; };
S a { 1, {2,3} };
S* p = new S{ 1, {2,3} };       // creates dangling reference
\end{codeblock}
\end{example}
\end{itemize}

\pnum
The fourth context is when a temporary object
is created in the \grammarterm{for-range-initializer} of
either a range-based \keyword{for} statement
or an enumerating expansion statement\iref{stmt.expand}.
If such a temporary object would otherwise be destroyed
at the end of the \grammarterm{for-range-initializer} full-expression,
the object persists for the lifetime of the reference
initialized by the \grammarterm{for-range-initializer}.

\pnum
The fifth context is when a temporary object is created
in the \grammarterm{expansion-initializer}
of an iterating or destructuring expansion statement.
If such a temporary object would otherwise be destroyed
at the end of that \grammarterm{expansion-initializer},
the object persists for the lifetime of the reference
initialized by the \grammarterm{expansion-initializer}, if any.

\pnum
The sixth context is when a temporary object
is created in a structured binding declaration\iref{dcl.struct.bind}.
Any temporary objects introduced by
the \grammarterm{initializer}{s} for the variables
with unique names
are destroyed at the end of the structured binding declaration.

\pnum
Let \tcode{x} and \tcode{y} each be either
a temporary object whose lifetime is not extended, or
a function parameter.
If the lifetimes of \tcode{x} and \tcode{y} end at
the end of the same full-expression, and
\tcode{x} is initialized before \tcode{y}, then
the destruction of \tcode{y} is sequenced before that of \tcode{x}.
If the lifetime of two or more temporaries
with lifetimes extending beyond the full-expressions in which they were created
ends at the same point,
these temporaries are destroyed at that point in the reverse order of the
completion of their construction.
In addition, the destruction of such temporaries shall
take into account the ordering of destruction of objects with static, thread, or
automatic storage duration\iref{basic.stc.static,basic.stc.thread,basic.stc.auto};
that is, if
\tcode{obj1}
is an object with the same storage duration as the temporary and
created before the temporary is created
the temporary shall be destroyed before
\tcode{obj1}
is destroyed;
if
\tcode{obj2}
is an object with the same storage duration as the temporary and
created after the temporary is created
the temporary shall be destroyed after
\tcode{obj2}
is destroyed.

\pnum
\begin{example}
\begin{codeblock}
struct S {
  S();
  S(int);
  friend S operator+(const S&, const S&);
  ~S();
};
S obj1;
const S& cr = S(16)+S(23);
S obj2;
\end{codeblock}

The expression
\tcode{S(16) + S(23)}
creates three temporaries:
a first temporary
\tcode{T1}
to hold the result of the expression
\tcode{S(16)},
a second temporary
\tcode{T2}
to hold the result of the expression
\tcode{S(23)},
and a third temporary
\tcode{T3}
to hold the result of the addition of these two expressions.
The temporary
\tcode{T3}
is then bound to the reference
\tcode{cr}.
It is unspecified whether
\tcode{T1}
or
\tcode{T2}
is created first.
On an implementation where
\tcode{T1}
is created before
\tcode{T2},
\tcode{T2}
shall be destroyed before
\tcode{T1}.
The temporaries
\tcode{T1}
and
\tcode{T2}
are bound to the reference parameters of
\tcode{\keyword{operator}+};
these temporaries are destroyed at the end of the full-expression
containing the call to
\tcode{\keyword{operator}+}.
The temporary
\tcode{T3}
bound to the reference
\tcode{cr}
is destroyed at the end of
\tcode{cr}'s
lifetime, that is, at the end of the program.
In addition, the order in which
\tcode{T3}
is destroyed takes into account the destruction order of other objects with
static storage duration.
That is, because
\tcode{obj1}
is constructed before
\tcode{T3},
and
\tcode{T3}
is constructed before
\tcode{obj2},
\tcode{obj2}
shall be destroyed before
\tcode{T3},
and
\tcode{T3}
shall be destroyed before
\tcode{obj1}.
\end{example}

\rSec1[basic.types]{Types}%

\rSec2[basic.types.general]{General}%
\indextext{type|(}

\pnum
\begin{note}
\ref{basic.types} and the subclauses thereof
impose requirements on implementations regarding the representation
of types.
There are two kinds of types: fundamental types and compound types.
Types describe objects\iref{intro.object},
references\iref{dcl.ref},
or functions\iref{dcl.fct}.
\end{note}

\pnum
\indextext{object!byte copying and|(}%
\indextext{type!trivially copyable}%
For any object (other than a potentially-overlapping subobject) of trivially copyable type
\tcode{T}, whether or not the object holds a valid value of type
\tcode{T}, the underlying bytes\iref{intro.memory} making up the
object can be copied into an array of
\keyword{char},
\tcode{\keyword{unsigned} \keyword{char}}, or
\tcode{std::byte}\iref{cstddef.syn}.
\begin{footnote}
By using, for example, the library
functions\iref{headers} \tcode{std::memcpy} or \tcode{std::memmove}.
\end{footnote}
If the content of that array
is copied back into the object, the object shall
subsequently hold its original value.
\begin{example}
\begin{codeblock}
constexpr std::size_t N = sizeof(T);
char buf[N];
T obj;                          // \tcode{obj} initialized to its original value
std::memcpy(buf, &obj, N);      // between these two calls to \tcode{std::memcpy}, \tcode{obj} might be modified
std::memcpy(&obj, buf, N);      // at this point, each subobject of \tcode{obj} of scalar type holds its original value
\end{codeblock}
\end{example}

\pnum
For two distinct objects \tcode{obj1} and \tcode{obj2}
of trivially copyable type \tcode{T},
where neither \tcode{obj1} nor \tcode{obj2} is a potentially-overlapping subobject,
if the underlying bytes\iref{intro.memory} making up
\tcode{obj1} are copied into \tcode{obj2},
\begin{footnote}
By using, for example,
the library functions\iref{headers} \tcode{std::memcpy} or \tcode{std::memmove}.
\end{footnote}
 \tcode{obj2} shall subsequently hold the same value as
\tcode{obj1}.
\begin{example}
\begin{codeblock}
T* t1p;
T* t2p;
    // provided that \tcode{t2p} points to an initialized object ...
std::memcpy(t1p, t2p, sizeof(T));
    // at this point, every subobject of trivially copyable type in \tcode{*t1p} contains
    // the same value as the corresponding subobject in \tcode{*t2p}
\end{codeblock}
\end{example}
\indextext{object!byte copying and|)}

\pnum
\label{term.object.representation}%
The \defnx{object representation}{representation!object}
of a complete object type \tcode{T} is the
sequence of \placeholder{N} \tcode{\keyword{unsigned} \keyword{char}} objects taken up
by a non-bit-field complete object of type \tcode{T},
where \placeholder{N} equals
\tcode{\keyword{sizeof}(T)}.
The \defnx{value representation}{representation!value}
of a type \tcode{T} is the set of bits
in the object representation of \tcode{T}
that participate in representing a value of type \tcode{T}.
The object and value representation of
a non-bit-field complete object of type \tcode{T} are
the bytes and bits, respectively, of
the object corresponding to the object and value representation of its type.
The object representation of a bit-field object is
the sequence of \placeholder{N} bits taken up by the object,
where \placeholder{N} is the width of the bit-field\iref{class.bit}.
The value representation of a bit-field object is
the set of bits in the object representation that
participate in representing its value.
\label{term.padding.bits}%
Bits in the object representation of a type or object that are
not part of the value representation
are \defn{padding bits}.
For trivially copyable types, the value representation is
a set of bits in the object representation that determines a
\defn{value}, which is one discrete element of an
\impldef{values of a trivially copyable type} set of values.
\begin{footnote}
The intent is that the memory model of \Cpp{} is compatible
with that of the C programming language.
\end{footnote}

\pnum
\indextext{type!incompletely-defined object}%
A class that has been declared but not defined, an enumeration type in certain
contexts\iref{dcl.enum}, or an array of unknown
bound or of incomplete element type, is an
\defnadj{incompletely-defined}{object type}.
\begin{footnote}
The size and layout of an instance of an incompletely-defined
object type is unknown.
\end{footnote}
\label{term.incomplete.type}%
Incompletely-defined object types and \cv{}~\keyword{void} are
\defnadjx{incomplete}{types}{type}\iref{basic.fundamental}.
\begin{note}
Objects cannot be defined to have an incomplete type\iref{basic.def}.
\end{note}

\pnum
A class type (such as ``\tcode{\keyword{class} X}'') can be incomplete at one
point in a translation unit and complete later on; the type
``\tcode{\keyword{class} X}'' is the same type at both points. The declared type
of an array object can be an array of incomplete class type and
therefore incomplete; if the class type is completed later on in the
translation unit, the array type becomes complete; the array type at
those two points is the same type. The declared type of an array object
can be an array of unknown bound and therefore be incomplete at one
point in a translation unit and complete later on; the array types at
those two points (``array of unknown bound of \tcode{T}'' and ``array of
\tcode{N} \tcode{T}'') are different types.
\begin{note}
The type of a pointer or reference to array of unknown bound
permanently points to or refers to an incomplete type.
An array of unknown bound named by a \keyword{typedef} declaration
permanently refers to an incomplete type.
In either case, the array type cannot be completed.
\end{note}
\begin{example}
\indextext{type!example of incomplete}%
\begin{codeblock}
class X;                        // \tcode{X} is an incomplete type
extern X* xp;                   // \tcode{xp} is a pointer to an incomplete type
extern int arr[];               // the type of arr is incomplete
typedef int UNKA[];             // \tcode{UNKA} is an incomplete type
UNKA* arrp;                     // \tcode{arrp} is a pointer to an incomplete type
UNKA** arrpp;

void foo() {
  xp++;                         // error: \tcode{X} is incomplete
  arrp++;                       // error: incomplete type
  arrpp++;                      // OK, sizeof \tcode{UNKA*} is known
}

struct X { int i; };            // now \tcode{X} is a complete type
int arr[10];                    // now the type of \tcode{arr} is complete

X x;
void bar() {
  xp = &x;                      // OK; type is ``pointer to \tcode{X}''
  arrp = &arr;                  // OK; qualification conversion\iref{conv.qual}
  xp++;                         // OK, \tcode{X} is complete
  arrp++;                       // error: \tcode{UNKA} can't be completed
}
\end{codeblock}
\end{example}

\pnum
\begin{note}
The rules for declarations and expressions describe in which
contexts incomplete types are prohibited.
\end{note}

\pnum
\label{term.object.type}%
An \defn{object type} is a (possibly cv-qualified) type that is not
a function type, not a reference type, and not \cv{}~\keyword{void}.

\pnum
\indextext{class!trivial}%
\indextext{class!trivially copyable}%
\indextext{class!standard-layout}%
\label{term.scalar.type}%
Arithmetic types\iref{basic.fundamental}, enumeration types,
pointer types, pointer-to-member types\iref{basic.compound},
\tcode{std::meta::\brk{}info}, \tcode{std::nullptr_t},
and
cv-qualified\iref{basic.type.qualifier} versions of these
types are collectively called
\defnadjx{scalar}{types}{type}.
\label{term.trivially.copyable.type}%
Scalar types, trivially copyable class types\iref{class.prop},
arrays of such types, and cv-qualified versions of these
types are collectively called \defnadjx{trivially copyable}{types}{type}.
\label{term.trivially.relocatable.type}%
Scalar types, trivially relocatable class types\iref{class.prop},
arrays of such types, and cv-qualified versions of these
types are collectively called \defnadjx{trivially relocatable}{types}{type}.
\label{term.replaceable.type}%
\indextext{replaceable!type|see{type, replaceable}}%
Cv-unqualified scalar types, replaceable class types\iref{class.prop}, and
arrays of such types are collectively called
\defnx{replaceable types}{type!replaceable}.
\label{term.standard.layout.type}%
Scalar types, standard-layout class
types\iref{class.prop}, arrays of such types, and
cv-qualified versions of these types
are collectively called \defnadjx{standard-layout}{types}{type}.
\label{term.implicit.lifetime.type}%
Scalar types, implicit-lifetime class types\iref{class.prop},
array types, and cv-qualified versions of these types
are collectively called \defnadjx{implicit-lifetime}{types}{type}.

\pnum
\label{term.literal.type}%
A type is a \defnadj{literal}{type} if it is:
\begin{itemize}
\item \cv{}~\keyword{void}; or
\item a scalar type; or
\item a reference type; or
\item an array of literal type; or
\item a possibly cv-qualified class type\iref{class} that
has all of the following properties:
\begin{itemize}
\item it has a constexpr destructor\iref{dcl.constexpr},
\item all of its non-variant non-static data members and base classes are of non-volatile literal types, and
\item it
\begin{itemize}
\item is a closure type\iref{expr.prim.lambda.closure},
\item is an aggregate union type that has either
no variant members or
at least one variant member of non-volatile literal type,
\item is a non-union aggregate type for which
each of its anonymous union members
satisfies the above requirements for an aggregate union type, or
\item has at least one constexpr constructor or constructor template
(possibly inherited\iref{namespace.udecl} from a base class)
that is not a copy or move constructor.
\end{itemize}
\end{itemize}
\end{itemize}
\begin{note}
A literal type is one for which
it might be possible to create an object
within a constant expression.
It is not a guarantee that it is possible to create such an object,
nor is it a guarantee that any object of that type
will be usable in a constant expression.
\end{note}

\pnum
\label{term.layout.compatible.type}%
Two types \cvqual{cv1} \tcode{T1} and \cvqual{cv2} \tcode{T2} are
\defnadjx{layout-compatible}{types}{type}
if \tcode{T1} and \tcode{T2} are the same type,
layout-compatible enumerations\iref{dcl.enum}, or
layout-compatible standard-layout class types\iref{class.mem}.

\pnum
A type is \defn{consteval-only} if it is either
\tcode{std::meta::info} or
a type compounded from a consteval-only type\iref{basic.compound}.
Every object of consteval-only type shall be
\begin{itemize}
\item
the object associated with a constexpr variable or a subobject thereof,
\item
a template parameter object\iref{temp.param} or a subobject thereof, or
\item
an object whose lifetime begins and ends
during the evaluation of a core constant expression.
\end{itemize}

\rSec2[basic.fundamental]{Fundamental types}

\pnum
\indextext{type!implementation-defined \tcode{sizeof}}%
There are five \defnadjx{standard signed integer}{types}{type}:
\indextext{type!\idxcode{signed char}}%
\indextext{type!\idxcode{short}}%
\indextext{type!\idxcode{int}}%
\indextext{type!\idxcode{long}}%
\indextext{type!\idxcode{long long}}%
``\tcode{\keyword{signed} \keyword{char}}'', ``\tcode{\keyword{short} \keyword{int}}'', ``\keyword{int}'',
``\tcode{\keyword{long} \keyword{int}}'', and ``\tcode{\keyword{long} \keyword{long} \keyword{int}}''. In
this list, each type provides at least as much storage as those
preceding it in the list.
There may also be \impldef{extended signed integer types}
\defnadjx{extended signed integer}{types}{type}.
The standard and extended signed integer types are collectively called
\defnadjx{signed integer}{types}{type}.
The range of representable values for a signed integer type is
$-2^{N-1}$ to $2^{N-1}-1$ (inclusive),
where $N$ is called the \defn{width} of the type.
\indextext{integral type!implementation-defined \tcode{sizeof}}%
\begin{note}
Plain \tcode{int}s are intended to have
the natural width suggested by the architecture of the execution environment;
the other signed integer types are provided to meet special needs.
\end{note}

\pnum
\indextext{type!\idxcode{unsigned}}%
For each of the standard signed integer types,
there exists a corresponding (but different)
\defnadj{standard unsigned integer}{type}:
\indextext{type!\idxcode{unsigned char}}%
\indextext{type!\idxcode{unsigned short}}%
\indextext{type!\idxcode{unsigned int}}%
\indextext{type!\idxcode{unsigned long}}%
\indextext{type!\idxcode{unsigned long long}}%
``\tcode{\keyword{unsigned} \keyword{char}}'', ``\tcode{\keyword{unsigned} \keyword{short} \keyword{int}}'',
``\tcode{\keyword{unsigned} \keyword{int}}'', ``\tcode{\keyword{unsigned} \keyword{long} \keyword{int}}'', and
``\tcode{\keyword{unsigned} \keyword{long} \keyword{long} \keyword{int}}''.
Likewise, for each of the extended signed integer types,
there exists a corresponding \defnadj{extended unsigned integer}{type}.
The standard and extended unsigned integer types
are collectively called \defnadjx{unsigned integer}{types}{type}.
An unsigned integer type has the same width \placeholder{N}
as the corresponding signed integer type.
\indextext{arithmetic!\idxcode{unsigned}}%
The range of representable values for the unsigned type is
$0$ to $2^N-1$ (inclusive);
arithmetic for the unsigned type is performed modulo $2^N$.
\begin{note}
Unsigned arithmetic does not overflow.
Overflow for signed arithmetic yields undefined behavior\iref{expr.pre}.
\end{note}

\pnum
\indextext{signed integer representation!two's complement}%
An unsigned integer type has the same
object representation,
value representation, and
alignment requirements\iref{basic.align}
as the corresponding signed integer type.
For each value $x$ of a signed integer type,
the value of the corresponding unsigned integer type
congruent to $x$ modulo $2^N$ has the same value
of corresponding bits in its value representation.
\begin{footnote}
This
is also known as two's complement representation.
\end{footnote}
\begin{example}
The value $-1$ of a signed integer type has the same representation as
the largest value of the corresponding unsigned type.
\end{example}

\begin{floattable}{Minimum width}{basic.fundamental.width}{lc}
\topline
\lhdr{Type} & \rhdr{Minimum width $N$} \\
\capsep
\tcode{\keyword{signed} \keyword{char}} & 8 \\
\tcode{\keyword{short} \keyword{int}} & 16 \\
\keyword{int} & 16 \\
\tcode{\keyword{long} \keyword{int}} & 32 \\
\tcode{\keyword{long} \keyword{long} \keyword{int}} & 64 \\
\end{floattable}

\pnum
The width of each standard signed integer type
shall not be less than the values specified in \tref{basic.fundamental.width}.
The value representation of a signed or unsigned integer type
comprises $N$ bits, where N is the respective width.
Each set of values for any padding bits\iref{basic.types.general}
in the object representation are
alternative representations of the value specified by the value representation.
\begin{note}
Padding bits have unspecified value, but cannot cause traps.
In contrast, see \IsoC{} 6.2.6.2.
\end{note}
\begin{note}
The signed and unsigned integer types satisfy
the constraints given in \IsoC{} 5.3.5.3.2.
\end{note}
Except as specified above,
the width of a signed or unsigned integer type is
\impldef{width of integral type}.

\pnum
Each value $x$ of an unsigned integer type with width $N$ has
a unique representation $x = x_0 2^0 + x_1 2^1 + \ldots + x_{N-1} 2^{N-1}$,
where each coefficient $x_i$ is either 0 or 1;
this is called the \defn{base-2 representation} of $x$.
The base-2 representation of a value of signed integer type is
the base-2 representation of the congruent value
of the corresponding unsigned integer type.
The standard signed integer types and standard unsigned integer types
are collectively called the \defnadjx{standard integer}{types}{type}, and the extended
signed integer types and extended
unsigned integer types are collectively called the
\defnadjx{extended integer}{types}{type}.

\pnum
A fundamental type specified to have
a signed or unsigned integer type as its \defnadj{underlying}{type} has
the same object representation,
value representation,
alignment requirements\iref{basic.align}, and
range of representable values as the underlying type.
Further, each value has the same representation in both types.

\pnum
\indextext{type!\idxcode{char}}%
\indextext{type!character}%
\indextext{\idxcode{char}!implementation-defined sign of}%
\indextext{type!\idxcode{signed char}}%
\indextext{character!\idxcode{signed}}%
\indextext{type!\idxcode{unsigned char}}%
Type \keyword{char} is a distinct type
that has an \impldef{underlying type of \tcode{char}} choice of
``\tcode{\keyword{signed} \keyword{char}}'' or ``\tcode{\keyword{unsigned} \keyword{char}}'' as its underlying type.
The three types \keyword{char}, \tcode{\keyword{signed} \keyword{char}}, and \tcode{\keyword{unsigned} \keyword{char}}
are collectively called
\defnadjx{ordinary character}{types}{type}.
The ordinary character types and \keyword{char8_t}
are collectively called \defnadjx{narrow character}{types}{type}.
For narrow character types,
each possible bit pattern of the object representation represents
a distinct value.
\begin{note}
This requirement does not hold for other types.
\end{note}
\begin{note}
A bit-field of narrow character type whose width is larger than
the width of that type has padding bits; see \ref{basic.types.general}.
\end{note}

\pnum
\indextext{\idxcode{wchar_t}|see{type, \tcode{wchar_t}}}%
\indextext{type!\idxcode{wchar_t}}%
\indextext{type!underlying!\idxcode{wchar_t}}%
Type \keyword{wchar_t} is a distinct type that has
an \impldef{underlying type of \tcode{wchar_t}}
signed or unsigned integer type as its underlying type.

\pnum
\indextext{\idxcode{char8_t}|see{type, \tcode{char8_t}}}%
\indextext{type!\idxcode{char8_t}}%
\indextext{type!underlying!\idxcode{char8_t}}%
Type \keyword{char8_t} denotes a distinct type
whose underlying type is \tcode{\keyword{unsigned} \keyword{char}}.
\indextext{\idxcode{char16_t}|see{type, \tcode{char16_t}}}%
\indextext{\idxcode{char32_t}|see{type, \tcode{char32_t}}}%
\indextext{type!\idxcode{char16_t}}%
\indextext{type!\idxcode{char32_t}}%
\indextext{type!underlying!\idxcode{char16_t}}%
\indextext{type!underlying!\idxcode{char32_t}}%
Types \keyword{char16_t} and \keyword{char32_t} denote distinct types
whose underlying types are \tcode{uint_least16_t} and \tcode{uint_least32_t},
respectively, in \libheaderref{cstdint}.

\pnum
\indextext{Boolean type}%
\indextext{type!Boolean}%
Type \tcode{bool} is a distinct type that has
the same object representation,
value representation, and
alignment requirements as
an \impldef{underlying type of \tcode{bool}} unsigned integer type.
The values of type \keyword{bool} are
\keyword{true} and \keyword{false}.
\begin{note}
There are no \keyword{signed}, \keyword{unsigned},
\keyword{short}, or \tcode{\keyword{long} \keyword{bool}} types or values.
\end{note}

\pnum
\indextext{type!integral}%
The types \keyword{char}, \keyword{wchar_t},
\keyword{char8_t}, \keyword{char16_t}, and \keyword{char32_t}
are collectively called \defnadjx{character}{types}{type}.
The character types, \keyword{bool},
the signed and unsigned integer types,
and cv-qualified versions\iref{basic.type.qualifier} thereof,
are collectively termed
\defnx{integral types}{integral type}.
A synonym for integral type is \defn{integer type}.
\begin{note}
Enumerations\iref{dcl.enum} are not integral;
however, unscoped enumerations can be promoted to integral types
as specified in \ref{conv.prom}.
\end{note}

\pnum
\indextext{floating-point type|see{type, floating-point}}%
The three distinct types
\indextext{type!\idxcode{float}}%
\keyword{float},
\indextext{type!\idxcode{double}}%
\keyword{double},
and
\indextext{type!\idxcode{long double}}%
\tcode{\keyword{long} \keyword{double}}
can represent floating-point numbers.
The type \keyword{double} provides at least as much
precision as \keyword{float}, and the type \tcode{\keyword{long} \keyword{double}} provides at
least as much precision as \keyword{double}. The set of values of the type
\keyword{float} is a subset of the set of values of the type
\keyword{double}; the set of values of the type \keyword{double} is a subset
of the set of values of the type \tcode{\keyword{long} \keyword{double}}.
The types
\keyword{float}, \keyword{double}, and \tcode{\keyword{long} \keyword{double}},
and cv-qualified versions\iref{basic.type.qualifier} thereof,
are collectively termed
\defnx{standard floating-point types}{type!floating-point!standard}.
An implementation may also provide additional types
that represent floating-point values and define them (and cv-qualified versions thereof) to be
\defnx{extended floating-point types}{type!floating-point!extended}.
The standard and extended floating-point types
are collectively termed \defnx{floating-point types}{type!floating-point}.
\begin{note}
Any additional implementation-specific types representing floating-point values
that are not defined by the implementation to be extended floating-point types
are not considered to be floating-point types, and
this document imposes no requirements on them or
their interactions with floating-point types.
\end{note}
Except as specified in \ref{basic.extended.fp},
the object and value representations and accuracy of operations
of floating-point types are \impldef{representation of floating-point types}.

\pnum
The minimum range of representable values for a floating-point type is
the most negative finite floating-point number representable
in that type through
the most positive finite floating-point number representable in that type.
In addition, if negative infinity is representable in a type,
the range of that type is extended to all negative real numbers;
likewise, if positive infinity is representable in a type,
the range of that type is extended to all positive real numbers.
\begin{note}
Since negative and positive infinity are representable
in \IsoFloatUndated{} formats,
all real numbers lie within the range of representable values of
a floating-point type adhering to \IsoFloatUndated{}.
\end{note}

\pnum
Integral and floating-point types are collectively
termed \defnx{arithmetic types}{type!arithmetic}.
\begin{note}
Properties of the arithmetic types,
such as their minimum and maximum representable value,
can be queried using the facilities in the standard library headers
\libheaderref{limits},
\libheaderref{climits}, and
\libheaderref{cfloat}.
\end{note}

\pnum
\indextext{type!\idxcode{void}}%
A type \cv{}~\keyword{void}
is an incomplete type that cannot be completed; such a type has
an empty set of values. It is used as the return
type for functions that do not return a value.
An expression of type \cv{}~\keyword{void} shall
be used only as
\begin{itemize}
\item an expression statement\iref{stmt.expr},
\item the expression in a \keyword{return} statement\iref{stmt.return}
for a function with the return type \cv{}~\keyword{void},
\item an operand of a comma expression\iref{expr.comma},
\item the second or third operand of \tcode{?:}\iref{expr.cond},
\item the operand of a \keyword{typeid} expression\iref{expr.typeid},
\item the operand of a \keyword{noexcept} operator\iref{expr.unary.noexcept},
\item the operand of a \keyword{decltype} specifier\iref{dcl.type.decltype}, or
\item the operand of an explicit conversion to type
\cv{}~\keyword{void}\iref{expr.type.conv,expr.static.cast,expr.cast}.
\end{itemize}

\pnum
The types denoted by \cv~\tcode{std::nullptr_t} are distinct types.
A prvalue of type \tcode{std::nullptr_t} is a null pointer
constant\iref{conv.ptr}. Such values participate in the pointer and the
pointer-to-member conversions\iref{conv.ptr,conv.mem}.
\tcode{\keyword{sizeof}(std::nullptr_t)} shall be equal to \tcode{\keyword{sizeof}(\keyword{void}*)}.

\pnum
A value of type \tcode{std::meta::info} is called a \defn{reflection}.
There exists a unique \defnadj{null}{reflection};
every other reflection is a representation of
\begin{itemize}
\item a value of scalar type\iref{temp.param},
\item an object with static storage duration\iref{basic.stc},
\item a variable\iref{basic.pre},
\item a structured binding\iref{dcl.struct.bind},
\item a function\iref{dcl.fct},
\item a function parameter,
\item an enumerator\iref{dcl.enum},
\item an annotation\iref{dcl.attr.grammar},
\item a type alias\iref{dcl.typedef},
\item a type\iref{basic.types},
\item a class member\iref{class.mem},
\item an unnamed bit-field\iref{class.bit},
\item a class template\iref{temp.pre},
\item a function template,
\item a variable template,
\item an alias template\iref{temp.alias},
\item a concept\iref{temp.concept},
\item a namespace alias\iref{namespace.alias},
\item a namespace\iref{basic.namespace.general},
\item a direct base class relationship\iref{class.derived.general}, or
\item a data member description\iref{class.mem.general}.
\end{itemize}
A reflection is said to \defn{represent} the corresponding construct.
\begin{note}
A reflection of a value can be produced by library functions such as
\tcode{std::meta::constant_of} and \tcode{std::meta::reflect_constant}.
\end{note}
\begin{example}
\begin{codeblock}
int arr[] = {1, 2, 3};
auto [a1, a2, a3] = arr;
void fn();
enum Enum { A };
using Alias = int;
struct B {};
struct S : B {
  int mem;
  int : 0;
};
template<auto> struct TCls {};
template<auto> void TFn();
template<auto> int TVar;
template<auto> concept Concept = requires { true; };
namespace NS {};
namespace NSAlias = NS;

constexpr auto ctx = std::meta::access_context::current();

constexpr auto r1 = std::meta::reflect_constant(42);        // represents \tcode{int} value of \tcode{42}
constexpr auto r2 = std::meta::reflect_object(arr[1]);      // represents \tcode{int} object
constexpr auto r3 = ^^arr;                                  // represents a variable
constexpr auto r4 = ^^a3;                                   // represents a structured binding
constexpr auto r5 = ^^fn;                                   // represents a function
constexpr auto r6 = ^^Enum::A;                              // represents an enumerator
constexpr auto r7 = ^^Alias;                                // represents a type alias
constexpr auto r8 = ^^S;                                    // represents a type
constexpr auto r9 = ^^S::mem;                               // represents a class member
constexpr auto r10 = std::meta::members_of(^^S, ctx)[1];    // represents an unnamed bit-field
constexpr auto r11 = ^^TCls;                                // represents a class template
constexpr auto r12 = ^^TFn;                                 // represents a function template
constexpr auto r13 = ^^TVar;                                // represents a variable template
constexpr auto r14 = ^^Concept;                             // represents a concept
constexpr auto r15 = ^^NSAlias;                             // represents a namespace alias
constexpr auto r16 = ^^NS;                                  // represents a namespace
constexpr auto r17 = std::meta::bases_of(^^S, ctx)[0];      // represents a direct base class relationship
constexpr auto r18 =
  std::meta::data_member_spec(^^int, {.name="member"});     // represents a data member description
\end{codeblock}
\end{example}

\pnum
\recommended
Implementations should not represent other constructs
specified in this document, such as
\grammarterm{using-declarator}s,
partial template specializations,
attributes, placeholder types,
statements, or
expressions,
as values of type \tcode{std::meta::info}.
\begin{note}
Future revisions of this document can specify semantics for reflections
representing any such constructs.
\end{note}

\pnum
\indextext{type!fundamental}%
The types described in this subclause
are called \defnx{fundamental types}{fundamental type}.
\begin{note}
Even if the implementation defines two or more fundamental types to have the
same value representation, they are nevertheless different types.
\end{note}

\rSec2[basic.extended.fp]{Optional extended floating-point types}

\pnum
If the implementation supports an extended floating-point type\iref{basic.fundamental}
whose properties are specified by
the \IsoFloatUndated{} floating-point interchange format binary16,
then the \grammarterm{typedef-name} \tcode{std::float16_t}
is declared in the header \libheaderref{stdfloat} and names such a type,
the macro \mname{STDCPP_FLOAT16_T} is defined\iref{cpp.predefined}, and
the floating-point literal suffixes \tcode{f16} and \tcode{F16}
are supported\iref{lex.fcon}.

\pnum
If the implementation supports an extended floating-point type
whose properties are specified by
the \IsoFloatUndated{} floating-point interchange format binary32,
then the \grammarterm{typedef-name} \tcode{std::float32_t}
is declared in the header \libheader{stdfloat} and names such a type,
the macro \mname{STDCPP_FLOAT32_T} is defined, and
the floating-point literal suffixes \tcode{f32} and \tcode{F32} are supported.

\pnum
If the implementation supports an extended floating-point type
whose properties are specified by
the \IsoFloatUndated{} floating-point interchange format binary64,
then the \grammarterm{typedef-name} \tcode{std::float64_t}
is declared in the header \libheader{stdfloat} and names such a type,
the macro \mname{STDCPP_FLOAT64_T} is defined, and
the floating-point literal suffixes \tcode{f64} and \tcode{F64} are supported.

\pnum
If the implementation supports an extended floating-point type
whose properties are specified by
the \IsoFloatUndated{} floating-point interchange format binary128,
then the \grammarterm{typedef-name} \tcode{std::float128_t}
is declared in the header \libheader{stdfloat} and names such a type,
the macro \mname{STDCPP_FLOAT128_T} is defined, and
the floating-point literal suffixes \tcode{f128} and \tcode{F128} are supported.

\pnum
If the implementation supports an extended floating-point type
with the properties, as specified by \IsoFloatUndated{}, of
radix ($b$) of 2,
storage width in bits ($k$) of 16,
precision in bits ($p$) of 8,
maximum exponent ($emax$) of 127, and
exponent field width in bits ($w$) of 8, then
the \grammarterm{typedef-name} \tcode{std::bfloat16_t}
is declared in the header \libheader{stdfloat} and names such a type,
the macro \mname{STDCPP_BFLOAT16_T} is defined, and
the floating-point literal suffixes \tcode{bf16} and \tcode{BF16} are supported.

\pnum
\begin{note}
A summary of the parameters for each type is given in \tref{basic.extended.fp}.
The precision $p$ includes the implicit 1 bit at the beginning of the significand,
so the storage used for the significand is $p-1$ bits.
\IsoFloatUndated{} does not assign a name for a type
having the parameters specified for \tcode{std::bfloat16_t}.
\end{note}
\begin{floattable}
{Properties of named extended floating-point types}{basic.extended.fp}{llllll}
\topline
\lhdr{Parameter} & \chdr{\tcode{float16_t}} & \chdr{\tcode{float32_t}} &
\chdr{\tcode{float64_t}} & \chdr{\tcode{float128_t}} &
\rhdr{\tcode{bfloat16_t}} \\
\capsep
\IsoFloatUndated{} name & binary16 & binary32 & binary64 & binary128 & \\
$k$, storage width in bits & 16 & 32 & 64 & 128 & 16 \\
$p$, precision in bits & 11 & 24 & 53 & 113 & 8 \\
$emax$, maximum exponent & 15 & 127 & 1023 & 16383 & 127 \\
$w$, exponent field width in bits & 5 & 8 & 11 & 15 & 8 \\
\end{floattable}

\pnum
\recommended
Any names that the implementation provides for
the extended floating-point types described in this subsection
that are in addition to the names declared in the \libheader{stdfloat} header
should be chosen to increase compatibility and interoperability
with the interchange types
\tcode{_Float16}, \tcode{_Float32}, \tcode{_Float64}, and \tcode{_Float128}
defined in ISO/IEC TS 18661-3 and with future versions of \IsoCUndated{}.

\rSec2[basic.compound]{Compound types}

\pnum
\indextext{type!compound}%
Compound types can be constructed in the following ways:
\begin{itemize}
\item \defnx{arrays}{type!array} of objects of a given type, \ref{dcl.array};

\item \defnx{functions}{type!function}, which have parameters of given types and return
\keyword{void} or a result of a given type, \ref{dcl.fct};

\item \defnx{pointers}{type!pointer} to \cv{}~\keyword{void} or objects or functions (including
static members of classes) of a given type, \ref{dcl.ptr};

\item
\indextext{reference!lvalue}%
\indextext{reference!rvalue}%
\defnx{references}{reference} to objects or functions of a given
type, \ref{dcl.ref}. There are two types of references:
\begin{itemize}
\item lvalue reference
\item rvalue reference
\end{itemize}

\item
\defnx{classes}{class} containing a sequence of class members\iref{class,class.mem},
and a set of restrictions
on the access to these entities\iref{class.access};

\item
\defnx{unions}{\idxcode{union}}, which are classes capable of containing objects of
different types at different times, \ref{class.union};

\item
\defnx{enumerations}{\idxcode{enum}},
which comprise a set of named constant values, \ref{dcl.enum};

\item
\indextext{member pointer to|see{pointer to member}}%
\defnx{pointers to non-static class members}{pointer to member},%
\begin{footnote}
Static class members are objects or functions, and pointers to them are
ordinary pointers to objects or functions.
\end{footnote}
which identify members of a given
type within objects of a given class, \ref{dcl.mptr}.
Pointers to data members and pointers to member functions are collectively
called \term{pointer-to-member} types.
\end{itemize}

\pnum
These methods of constructing types can be applied recursively;
restrictions are mentioned in~\ref{dcl.meaning}.
Constructing a type such that the number of
bytes in its object representation exceeds the maximum value representable in
the type \tcode{std::size_t}\iref{support.types} is ill-formed.

\pnum
\indextext{terminology!pointer}%
The type of a pointer to \cv{}~\keyword{void} or a pointer to an object type is
called an \defn{object pointer type}.
\begin{note}
A pointer to \keyword{void}
does not have a pointer-to-object type, however, because \keyword{void} is not
an object type.
\end{note}
The type of a pointer that can designate a function
is called a \defn{function pointer type}.
A pointer to an object of type \tcode{T} is referred to as a ``pointer to
\tcode{T}''.
\begin{example}
A pointer to an object of type \keyword{int} is
referred to as ``pointer to \keyword{int}'' and a pointer to an object of
class \tcode{X} is called a ``pointer to \tcode{X}''.
\end{example}
Except for pointers to static members, text referring to ``pointers''
does not apply to pointers to members. Pointers to incomplete types are
allowed although there are restrictions on what can be done with
them\iref{basic.types.general}.
\indextext{address}%
Every value of pointer type is one of the following:
\begin{itemize}
\item
a \defn{pointer to} an object or function (the pointer is said to \defn{point} to the object or function), or
\item
a \defn{pointer past the end of} an object\iref{expr.add}, or
\item
\indextext{null pointer value|see{value, null pointer}}
the \defnx{null pointer value}{value!null pointer} for that type, or
\item
\indextext{invalid pointer value|see{value, invalid pointer}}
an \defnx{invalid pointer value}{value!invalid pointer}.
\end{itemize}
A value of a
pointer type
that is a pointer to or past the end of an object
\defn{represents the address} of
the first byte in memory\iref{intro.memory} occupied by the object
\begin{footnote}
For an object that is not within its lifetime,
this is the first byte in memory that it will occupy or used to occupy.
\end{footnote}
or the first byte in memory
after the end of the storage occupied by the object,
respectively.
\begin{note}
A pointer past the end of an object\iref{expr.add}
is not considered to point to an unrelated object
of the object's type,
even if the unrelated object is located at that address.
\end{note}
For purposes of pointer arithmetic\iref{expr.add}
and comparison\iref{expr.rel,expr.eq},
a pointer past the end of the last element of
an array \tcode{x} of $n$ elements
is considered to be equivalent to
a pointer to a hypothetical array element $n$ of \tcode{x}, and
an object of type \tcode{T} that is not an array element
is considered to belong to an array with one element of type \tcode{T}.
The value representation of
pointer types is \impldef{value representation of pointer types}. Pointers to
layout-compatible types shall
have the same value representation and alignment
requirements\iref{basic.align}.
\begin{note}
Pointers to over-aligned types\iref{basic.align} have no special
representation, but their range of valid values is restricted by the extended
alignment requirement.
\end{note}

\pnum
A pointer value $P$ is
\indextext{value!valid in the context of an evaluation}%
\defn{valid in the context of} an evaluation $E$
if $P$ is a pointer to function or a null pointer value, or
if it is a pointer to or past the end of an object $O$ and
$E$ happens before the end of the duration of the region of storage for $O$.
If a pointer value $P$ is used in an evaluation $E$ and
$P$ is not valid in the context of $E$,
then the behavior is undefined if $E$ is
an indirection\iref{expr.unary.op} or
an invocation of a deallocation function\iref{basic.stc.dynamic.deallocation},
and \impldef{invalid pointer value in the context of an evaluation} otherwise.
\begin{footnote}
Some implementations might define that
copying such a pointer value causes a system-generated runtime fault.
\end{footnote}
\begin{note}
$P$ can be valid in the context of $E$ even
if it points to a type unrelated to that of $O$ or
if $O$ is not within its lifetime,
although further restrictions apply
to such pointer values\iref{basic.life, basic.lval, expr.add}.
\end{note}

\pnum
Two objects \placeholder{a} and \placeholder{b} are \defn{pointer-interconvertible} if
\begin{itemize}
\item
they are the same object, or
\item
one is a union object and
the other is a non-static data member of that object\iref{class.union}, or
\item
one is a standard-layout class object and
the other is the first non-static data member of that object or
any base class subobject of that object\iref{class.mem}, or
\item
there exists an object \placeholder{c} such that
\placeholder{a} and \placeholder{c} are pointer-interconvertible, and
\placeholder{c} and \placeholder{b} are pointer-interconvertible.
\end{itemize}
If two objects are pointer-interconvertible,
then they have the same address,
and it is possible to obtain a pointer to one
from a pointer to the other
via a \keyword{reinterpret_cast}\iref{expr.reinterpret.cast}.
\begin{note}
An array object and its first element are not pointer-interconvertible,
even though they have the same address.
\end{note}

\pnum
A byte of storage \placeholder{b}
is \defnx{reachable through}{storage!reachable through a pointer value}
a pointer value that points to an object \placeholder{x}
if there is an object \placeholder{y},
pointer-interconvertible with \placeholder{x},
such that \placeholder{b} is within the storage occupied by
\placeholder{y}, or the immediately-enclosing array object
if \placeholder{y} is an array element.

\pnum
\indextext{pointer|seealso{\tcode{void*}}}%
\indextext{\idxcode{void*}!type}%
A pointer to \cv{}~\keyword{void}
can be used to point to objects of
unknown type. Such a pointer shall be able to hold any object pointer.
An object of type ``pointer to \cv{}~\keyword{void}''
shall have the same representation and alignment
requirements as an object of type ``pointer to \cv{}~\keyword{char}''.

\rSec2[basic.type.qualifier]{CV-qualifiers}

\pnum
\indextext{cv-qualifier}%
\indextext{\idxcode{const}}%
\indextext{\idxcode{volatile}}%
Each type other than a function or reference type
is part of a group of four distinct, but related, types:
a \defn{cv-unqualified} version,
a \defn{const-qualified} version,
a \defn{volatile-qualified} version, and
a \defn{const-volatile-qualified} version.
The types in each such group shall have
the same representation and alignment requirements\iref{basic.align}.
\begin{footnote}
The same representation and alignment requirements
are meant to imply interchangeability as
arguments to functions,
return values from functions, and
non-static data members of unions.
\end{footnote}
A function or reference type is always cv-unqualified.
\begin{itemize}
\item A \defnadj{const}{object} is an object of type \tcode{\keyword{const} T} or a
  non-mutable subobject of a const object.

\item A \defnadj{volatile}{object} is an object of type
  \tcode{\keyword{volatile} T} or a subobject of a volatile object.

\item A \defnadj{const volatile}{object} is an object of type
  \tcode{\keyword{const} \keyword{volatile} T}, a non-mutable subobject of a const volatile object,
  a const subobject of a volatile object, or a non-mutable volatile
  subobject of a const object.
\end{itemize}
\begin{note}
The type of an object\iref{intro.object} includes
the \grammarterm{cv-qualifier}s specified in the
\grammarterm{decl-specifier-seq}\iref{dcl.spec},
\grammarterm{declarator}\iref{dcl.decl},
\grammarterm{type-id}\iref{dcl.name}, or
\grammarterm{new-type-id}\iref{expr.new}
when the object is created.
\end{note}

\pnum
Except for array types, a compound type\iref{basic.compound} is not cv-qualified by the
cv-qualifiers (if any) of the types from which it is compounded.

\pnum
\indextext{array!\idxcode{const}}%
An array type whose elements are cv-qualified
is also considered to have the same cv-qualifications
as its elements.
\begin{note}
Cv-qualifiers applied to an array
type attach to the underlying element type, so the notation
``\cv{}~\tcode{T}'', where \tcode{T} is an array type, refers to
an array whose elements are so-qualified\iref{dcl.array}.
\end{note}
\begin{example}
\begin{codeblock}
typedef char CA[5];
typedef const char CC;
CC arr1[5] = { 0 };
const CA arr2 = { 0 };
\end{codeblock}
The type of both \tcode{arr1} and \tcode{arr2} is ``array of 5
\tcode{\keyword{const} \keyword{char}}'', and the array type is considered to be
const-qualified.
\end{example}
\indextext{type|)}

\pnum
\begin{note}
See~\ref{dcl.fct} and~\ref{over.match.funcs} regarding function
types that have \grammarterm{cv-qualifier}{s}.
\end{note}

\pnum
There is a partial ordering on cv-qualifiers, so that a type can be
said to be \defn{more cv-qualified} than another.
\tref{basic.type.qualifier.rel} shows the relations that
constitute this ordering.

\begin{floattable}{Relations on \keyword{const} and \keyword{volatile}}{basic.type.qualifier.rel}
{ccc}
\topline
\cvqual{no cv-qualifier}    &<& \keyword{const}           \\
\cvqual{no cv-qualifier}    &<& \keyword{volatile}        \\
\cvqual{no cv-qualifier}    &<& \tcode{\keyword{const} \keyword{volatile}}  \\
\keyword{const}               &<& \tcode{\keyword{const} \keyword{volatile}}  \\
\keyword{volatile}            &<& \tcode{\keyword{const} \keyword{volatile}}  \\
\end{floattable}

\pnum
In this document, the notation \cv{} (or
\cvqual{cv1}, \cvqual{cv2}, etc.), used in the description of types,
represents an arbitrary set of cv-qualifiers, i.e., one of
\{\keyword{const}\}, \{\keyword{volatile}\}, \{\keyword{const},
\keyword{volatile}\}, or the empty set.
For a type \cv{}~\tcode{T}, the \defnx{top-level cv-qualifiers}{cv-qualifier!top-level}
of that type are those denoted by \cv.
\begin{example}
The type corresponding to the \grammarterm{type-id}
\tcode{\keyword{const} \keyword{int}\&}
has no top-level cv-qualifiers.
The type corresponding to the \grammarterm{type-id}
\tcode{\keyword{volatile} \keyword{int} * \keyword{const}}
has the top-level cv-qualifier \keyword{const}.
For a class type \tcode{C},
the type corresponding to the \grammarterm{type-id}
\tcode{\keyword{void} (C::* \keyword{volatile})(\keyword{int}) \keyword{const}}
has the top-level cv-qualifier \keyword{volatile}.
\end{example}

\rSec2[conv.rank]{Conversion ranks}%
\indextext{conversion!integer rank}

\pnum
Every integer type has an \term{integer conversion rank} defined as follows:

\begin{itemize}
\item No two signed integer types other than \keyword{char} and \tcode{\keyword{signed}
\keyword{char}} (if \keyword{char} is signed) have the same rank, even if they have
the same representation.

\item The rank of a signed integer type is greater than the rank
of any signed integer type with a smaller width.

\item The rank of \tcode{\keyword{long} \keyword{long} \keyword{int}} is greater
than the rank of \tcode{\keyword{long} \keyword{int}}, which is greater than
the rank of \keyword{int}, which is greater than the rank of
\tcode{\keyword{short} \keyword{int}}, which is greater than the rank of
\tcode{\keyword{signed} \keyword{char}}.

\item The rank of any unsigned integer type equals the rank of the
corresponding signed integer type.

\item The rank of any standard integer type is greater than the
rank of any extended integer type with the same width.

\item The rank of \keyword{char} equals the rank of \tcode{\keyword{signed} \keyword{char}}
and \tcode{\keyword{unsigned} \keyword{char}}.

\item The rank of \tcode{bool} is less than the rank of all
standard integer types.

\item
\indextext{type!\idxcode{wchar_t}}%
\indextext{type!\idxcode{char16_t}}%
\indextext{type!\idxcode{char32_t}}%
The ranks of \keyword{char8_t}, \keyword{char16_t}, \keyword{char32_t}, and
\keyword{wchar_t} equal the ranks of their underlying
types\iref{basic.fundamental}.

\item The rank of any extended signed integer type relative to another
extended signed integer type with the same width is \impldef{rank of extended signed
integer type}, but still subject to the other rules for determining the integer
conversion rank.

\item For all integer types \tcode{T1}, \tcode{T2}, and \tcode{T3}, if
\tcode{T1} has greater rank than \tcode{T2} and \tcode{T2} has greater
rank than \tcode{T3}, then \tcode{T1} has greater rank than
\tcode{T3}.
\end{itemize}

\begin{note}
The integer conversion rank is used in the definition of the integral
promotions\iref{conv.prom} and the usual arithmetic
conversions\iref{expr.arith.conv}.
\end{note}

\pnum
Every floating-point type has a \defnadj{floating-point}{conversion rank}
defined as follows:
\begin{itemize}
\item
The rank of a floating point type \tcode{T} is greater than
the rank of any floating-point type
whose set of values is a proper subset of the set of values of \tcode{T}.
\item
The rank of \tcode{\keyword{long} \keyword{double}} is greater than
the rank of \keyword{double},
which is greater than the rank of \keyword{float}.
\item
Two extended floating-point types with the same set of values have equal ranks.
\item
An extended floating-point type with the same set of values as
exactly one cv-unqualified standard floating-point type
has a rank equal to the rank of that standard floating-point type.
\item
An extended floating-point type with the same set of values as
more than one cv-unqualified standard floating-point type
has a rank equal to the rank of \keyword{double}.
\begin{tailnote}
The treatment of \tcode{std::float64_t} differs from
that of the analogous \tcode{_Float64} in C,
for example on platforms where all of
\tcode{\keyword{long} \keyword{double}},
\keyword{double}, and
\tcode{std::float64_t}
have the same set of values (see \IsoC H.4.3).
\end{tailnote}
\end{itemize}
\begin{note}
The conversion ranks of floating-point types \tcode{T1} and \tcode{T2}
are unordered if the set of values of \tcode{T1} is
neither a subset nor a superset of the set of values of \tcode{T2}.
This can happen when one type has both a larger range and a lower precision
than the other.
\end{note}

\pnum
Floating-point types that have equal floating-point conversion ranks
are ordered by floating-point conversion subrank.
The subrank forms a total order among types with equal ranks.
The types
\tcode{std::float16_t},
\tcode{std::float32_t},
\tcode{std::float64_t}, and
\tcode{std::float128_t}\iref{stdfloat.syn}
have a greater conversion subrank than any standard floating-point type
with equal conversion rank.
Otherwise, the conversion subrank order is
\impldef{floating-point conversion subrank}.

\pnum
\begin{note}
The floating-point conversion rank and subrank are used in
the definition of the usual arithmetic conversions\iref{expr.arith.conv}.
\end{note}

\rSec1[basic.exec]{Program execution}

\rSec2[intro.execution]{Sequential execution}
\indextext{program execution|(}

\pnum
An instance of each object with automatic storage
duration\iref{basic.stc.auto} is associated with each entry into its
block. Such an object exists and retains its last-stored value during
the execution of the block and while the block is suspended (by a call
of a function, suspension of a coroutine\iref{expr.await}, or receipt of a signal).

\pnum
A \defn{constituent expression} is defined as follows:
\begin{itemize}
\item
The constituent expression of an expression is that expression.
\item
The constituent expression of a conversion is
the corresponding implicit function call, if any, or
the converted expression otherwise.
\item
The constituent expressions of a \grammarterm{braced-init-list} or
of a (possibly parenthesized) \grammarterm{expression-list}
are the constituent expressions of the elements of the respective list.
\item
The constituent expressions of a \grammarterm{brace-or-equal-initializer}
of the form \tcode{=}~\grammarterm{initializer-clause}
are the constituent expressions of the \grammarterm{initializer-clause}.
\end{itemize}
\begin{example}
\begin{codeblock}
struct A { int x; };
struct B { int y; struct A a; };
B b = { 5, { 1+1 } };
\end{codeblock}
The constituent expressions of the \grammarterm{initializer}
used for the initialization of \tcode{b} are \tcode{5} and \tcode{1+1}.
\end{example}

\pnum
The \defnx{immediate subexpressions}{immediate subexpression} of an expression $E$ are
\begin{itemize}
\item
the constituent expressions of $E$'s operands\iref{expr.prop},
\item
any function call that $E$ implicitly invokes,
\item
if $E$ is a \grammarterm{lambda-expression}\iref{expr.prim.lambda},
the initialization of the entities captured by copy and
the constituent expressions of the \grammarterm{initializer} of the \grammarterm{init-capture}{s},
\item
if $E$ is a function call\iref{expr.call} or implicitly invokes a function,
the constituent expressions of each default argument\iref{dcl.fct.default}
used in the call, or
\item
if $E$ creates an aggregate object\iref{dcl.init.aggr},
the constituent expressions of each default member initializer\iref{class.mem}
used in the initialization.
\end{itemize}

\pnum
A \defn{subexpression} of an expression $E$ is
an immediate subexpression of $E$ or
a subexpression of an immediate subexpression of $E$.
\begin{note}
Expressions appearing in the \grammarterm{compound-statement} of a \grammarterm{lambda-expression}
are not subexpressions of the \grammarterm{lambda-expression}.
\end{note}
The \defnadjx{potentially-evaluated}{subexpressions}{subexpression} of
an expression, conversion, or \grammarterm{initializer} $E$ are
\begin{itemize}
\item
the constituent expressions of $E$ and
\item
the subexpressions thereof that
are not subexpressions of a nested unevaluated operand\iref{term.unevaluated.operand}.
\end{itemize}

\pnum
A \defn{full-expression} is
\begin{itemize}
\item
an unevaluated operand\iref{expr.context},
\item
a \grammarterm{constant-expression}\iref{expr.const},
\item
an immediate invocation\iref{expr.const},
\item
an \grammarterm{init-declarator}\iref{dcl.decl}
(including such introduced by a structured binding\iref{dcl.struct.bind}) or
a \grammarterm{mem-initializer}\iref{class.base.init},
including the constituent expressions of the initializer,
\item
an invocation of a destructor generated at the end of the lifetime
of an object other than a temporary object\iref{class.temporary}
whose lifetime has not been extended,
\item
the predicate of a contract assertion\iref{basic.contract}, or
\item
an expression that is not a subexpression of another expression and
that is not otherwise part of a full-expression.
\end{itemize}
If a language construct is defined to produce an implicit call of a function,
a use of the language construct is considered to be an expression
for the purposes of this definition.
Conversions applied to the result of an expression in order to satisfy the requirements
of the language construct in which the expression appears
are also considered to be part of the full-expression.
For an initializer, performing the initialization of the entity
(including evaluating default member initializers of an aggregate)
is also considered part of the full-expression.
\begin{example}
\begin{codeblock}
struct S {
  S(int i): I(i) { }            // full-expression is initialization of \tcode{I}
  int& v() { return I; }
  ~S() noexcept(false) { }
private:
  int I;
};

S s1(1);                        // full-expression comprises call of \tcode{S::S(int)}
void f() {
  S s2 = 2;                     // full-expression comprises call of \tcode{S::S(int)}
  if (S(3).v())                 // full-expression includes lvalue-to-rvalue and \tcode{int} to \tcode{bool} conversions,
                                // performed before temporary is deleted at end of full-expression
  { }
  bool b = noexcept(S(4));      // exception specification of destructor of \tcode{S} considered for \keyword{noexcept}

  // full-expression is destruction of \tcode{s2} at end of block
}
struct B {
  B(S = S(0));
};
B b[2] = { B(), B() };          // full-expression is the entire initialization
                                // including the destruction of temporaries
\end{codeblock}
\end{example}

\pnum
\begin{note}
The evaluation of a full-expression can include the
evaluation of subexpressions that are not lexically part of the
full-expression. For example, subexpressions involved in evaluating
default arguments\iref{dcl.fct.default} are considered to
be created in the expression that calls the function, not the expression
that defines the default argument.
\end{note}

\pnum
\indextext{value computation|(}%
Reading an object designated by a \keyword{volatile} glvalue\iref{basic.lval},
modifying an object,
producing an injected declaration\iref{expr.const},
calling a library I/O function, or
calling a function that does any of those operations
are all \defn{side effects},
which are changes in the state of the execution or translation environment.
\defnx{Evaluation}{evaluation} of an expression (or a
subexpression) in general includes both value computations (including
determining the identity of an object for glvalue evaluation and fetching
a value previously assigned to an object for prvalue evaluation) and
initiation of side effects. When a call to a library I/O function
returns or an access through a volatile glvalue is evaluated, the side
effect is considered complete, even though some external actions implied
by the call (such as the I/O itself) or by the \keyword{volatile} access
may not have completed yet.

\pnum
\defnx{Sequenced before}{sequenced before} is an asymmetric, transitive, pair-wise relation between
evaluations executed by a single thread\iref{intro.multithread}, which induces
a partial order among those evaluations. Given any two evaluations \placeholder{A} and
\placeholder{B}, if \placeholder{A} is sequenced before \placeholder{B}
(or, equivalently, \placeholder{B} is \defn{sequenced after} \placeholder{A}),
then the execution of
\placeholder{A} shall precede the execution of \placeholder{B}. If \placeholder{A} is not sequenced
before \placeholder{B} and \placeholder{B} is not sequenced before \placeholder{A}, then \placeholder{A} and
\placeholder{B} are \defn{unsequenced}.
\begin{note}
The execution of unsequenced
evaluations can overlap.
\end{note}
Evaluations \placeholder{A} and \placeholder{B} are
\defn{indeterminately sequenced} when either \placeholder{A} is sequenced before
\placeholder{B} or \placeholder{B} is sequenced before \placeholder{A}, but it is unspecified which.
\begin{note}
Indeterminately sequenced evaluations cannot overlap, but either
can be executed first.
\end{note}
An expression \placeholder{X}
is said to be sequenced before
an expression \placeholder{Y} if
every value computation and every side effect
associated with the expression \placeholder{X}
is sequenced before
every value computation and every side effect
associated with the expression \placeholder{Y}.

\pnum
Every
\indextext{value computation}%
value computation and
\indextext{side effects}%
side effect associated with a full-expression is
sequenced before every value computation and side effect associated with the
next full-expression to be evaluated.
\begin{footnote}
As specified
in~\ref{class.temporary}, after a full-expression is evaluated, a sequence of
zero or more invocations of destructor functions for temporary objects takes
place, usually in reverse order of the construction of each temporary object.
\end{footnote}

\pnum
\indextext{evaluation!unspecified order of}%
Except where noted, evaluations of operands of individual operators and
of subexpressions of individual expressions are unsequenced.
\begin{note}
In an expression that is evaluated more than once during the execution
of a program, unsequenced and indeterminately sequenced evaluations of
its subexpressions need not be performed consistently in different
evaluations.
\end{note}
The value computations of the operands of an
operator are sequenced before the value computation of the result of the
operator.
The behavior is undefined if
\begin{itemize}
\item
\indextext{side effects}%
a side effect on a memory location\iref{intro.memory} or
\item
starting or ending the lifetime of an object in a memory location
\end{itemize}
is unsequenced relative to
\begin{itemize}
\item
another side effect on the same memory location,
\item
starting or ending the lifetime of an object occupying storage that
overlaps with the memory location, or
\item
a value computation using the value of any object in the same memory location,
\end{itemize}
and the two evaluations are not potentially concurrent\iref{intro.multithread}.
\begin{note}
Starting the lifetime of an object in a memory location can end the lifetime of
objects in other memory locations\iref{basic.life}.
\end{note}
\begin{note}
The next subclause imposes similar, but more complex restrictions on
potentially concurrent computations.
\end{note}

\begin{example}
\begin{codeblock}
void g(int i) {
  i = 7, i++, i++;              // \tcode{i} becomes \tcode{9}

  i = i++ + 1;                  // the value of \tcode{i} is incremented
  i = i++ + i;                  // undefined behavior
  i = i + 1;                    // the value of \tcode{i} is incremented

  union U { int x, y; } u;
  (u.x = 1, 0) + (u.y = 2, 0);  // undefined behavior
}
\end{codeblock}
\end{example}

\pnum
When invoking a function \placeholder{f} (whether or not the function is inline),
every argument expression and
the postfix expression designating \placeholder{f}
are sequenced before
every precondition assertion of \placeholder{f}\iref{dcl.contract.func},
which in turn are sequenced before
every expression or statement
in the body of \placeholder{f},
which in turn are sequenced before
every postcondition assertion of \placeholder{f}.

\pnum
For each
\begin{itemize}
\item function invocation,
\item evaluation of an \grammarterm{await-expression}\iref{expr.await}, or
\item evaluation of a \grammarterm{throw-expression}\iref{expr.throw}
\end{itemize}
\placeholder{F},
each evaluation that does not occur within \placeholder{F}
but is evaluated on the same thread and as part of the same signal handler (if any)
is either sequenced before all evaluations that occur within \placeholder{F}
or sequenced after all evaluations that occur within \placeholder{F};
\begin{footnote}
In other words,
function executions do not interleave with each other.
\end{footnote}
if \placeholder{F} invokes or resumes a coroutine\iref{expr.await},
only evaluations
subsequent to the previous suspension (if any) and
prior to the next suspension (if any)
are considered to occur within \placeholder{F}.

\pnum
Several contexts in \Cpp{} cause evaluation of a function call, even
though no corresponding function call syntax appears in the translation
unit.
\begin{example}
Evaluation of a \grammarterm{new-expression} invokes one or more allocation
and constructor functions; see~\ref{expr.new}. For another example,
invocation of a conversion function\iref{class.conv.fct} can arise in
contexts in which no function call syntax appears.
\end{example}

\pnum
The sequencing constraints on the execution of the called function (as
described above) are features of the function calls as evaluated,
regardless of the syntax of the expression that calls the function.%
\indextext{value computation|)}%

\indextext{behavior!on receipt of signal}%
\indextext{signal}%
\pnum
If a signal handler is executed as a result of a call to the \tcode{std::raise}
function, then the execution of the handler is sequenced after the invocation
of the \tcode{std::raise} function and before its return.
\begin{note}
When a signal is received for another reason, the execution of the
signal handler is usually unsequenced with respect to the rest of the program.
\end{note}

\pnum
During the evaluation of an expression
as a core constant expression\iref{expr.const},
evaluations of operands of individual operators and
of subexpressions of individual expressions
that are otherwise either unsequenced or indeterminately sequenced
are evaluated in lexical order.

\rSec2[intro.multithread]{Multi-threaded executions and data races}

\rSec3[intro.multithread.general]{General}

\pnum
\indextext{threads!multiple|(}%
\indextext{atomic!operation|(}%
A \defn{thread of execution} (also known as a \defn{thread}) is a single flow of
control within a program, including the initial invocation of a specific
top-level function, and recursively including every function invocation
subsequently executed by the thread.
\begin{note}
When one thread creates another,
the initial call to the top-level function of the new thread is executed by the
new thread, not by the creating thread.
\end{note}
Every thread in a program can
potentially use every object and function in a program.
\begin{footnote}
An object
with automatic or thread storage duration\iref{basic.stc} is associated with
one specific thread, and can be accessed by a different thread only indirectly
through a pointer or reference\iref{basic.compound}.
\end{footnote}
Under a hosted
implementation, a \Cpp{} program can have more than one thread running
concurrently. The execution of each thread proceeds as defined by the remainder
of this document. The execution of the entire program consists of an execution
of all of its threads.
\begin{note}
Usually the execution can be viewed as an
interleaving of all its threads. However, some kinds of atomic operations, for
example, allow executions inconsistent with a simple interleaving, as described
below.
\end{note}
\indextext{implementation!freestanding}%
Under a freestanding implementation, it is \impldef{number of
threads in a program under a freestanding implementation} whether a program can
have more than one thread of execution.

\pnum
For a signal handler that is not executed as a result of a call to the
\tcode{std::raise} function, it is unspecified which thread of execution
contains the signal handler invocation.

\rSec3[intro.races]{Data races}

\pnum
The value of an object visible to a thread $T$ at a particular point is the
initial value of the object, a value assigned to the object by $T$, or a
value assigned to the object by another thread, according to the rules below.
\begin{note}
In some cases, there might instead be undefined behavior. Much of this
subclause is motivated by the desire to support atomic operations with explicit
and detailed visibility constraints. However, it also implicitly supports a
simpler view for more restricted programs.
\end{note}

\pnum
Two expression evaluations \defn{conflict} if one of them
\begin{itemize}
\item
modifies\iref{defns.access} a memory location\iref{intro.memory} or
\item
starts or ends the lifetime of an object in a memory location
\end{itemize}
and the other one
\begin{itemize}
\item
reads or modifies the same memory location or
\item
starts or ends the lifetime of an object occupying storage that
overlaps with the memory location.
\end{itemize}
\begin{note}
A modification can still conflict
even if it does not alter the value of any bits.
\end{note}

\pnum
The library defines a number of atomic operations\iref{atomics} and
operations on mutexes\iref{thread} that are specially identified as
synchronization operations. These operations play a special role in making
assignments in one thread visible to another. A synchronization operation on one
or more memory locations is either an acquire operation, a
release operation, or both an acquire and release operation. A synchronization
operation without an associated memory location is a fence and can be either an
acquire fence, a release fence, or both an acquire and release fence. In
addition, there are relaxed atomic operations, which are not synchronization
operations, and atomic read-modify-write operations, which have special
characteristics.
\begin{note}
For example, a call that acquires a mutex will
perform an acquire operation on the locations comprising the mutex.
Correspondingly, a call that releases the same mutex will perform a release
operation on those same locations. Informally, performing a release operation on
$A$ forces prior
\indextext{side effects}%
side effects on other memory locations to become visible
to other threads that later perform a consume or an acquire operation on
$A$. ``Relaxed'' atomic operations are not synchronization operations even
though, like synchronization operations, they cannot contribute to data races.
\end{note}

\pnum
All modifications to a particular atomic object $M$ occur in some
particular total order, called the \defn{modification order} of $M$.
\begin{note}
There is a separate order for each
atomic object. There is no requirement that these can be combined into a single
total order for all objects. In general this will be impossible since different
threads can observe modifications to different objects in inconsistent orders.
\end{note}

\pnum
A \defn{release sequence} headed
by a release operation $A$ on an atomic object $M$
is a maximal contiguous sub-sequence of
\indextext{side effects}%
side effects in the modification order of $M$,
where the first operation is $A$, and
every subsequent operation is an atomic read-modify-write operation.

\pnum
Certain library calls \defn{synchronize with} other library calls performed by
another thread. For example, an atomic store-release synchronizes with a
load-acquire that takes its value from the store\iref{atomics.order}.
\begin{note}
Except in the specified cases, reading a later value does not
necessarily ensure visibility as described below. Such a requirement would
sometimes interfere with efficient implementation.
\end{note}
\begin{note}
The
specifications of the synchronization operations define when one reads the value
written by another. For atomic objects, the definition is clear. All operations
on a given mutex occur in a single total order. Each mutex acquisition ``reads
the value written'' by the last mutex release.
\end{note}

\pnum
An evaluation $A$ \defn{happens before} an evaluation $B$
(or, equivalently, $B$ happens after $A$)
if either
\begin{itemize}
\item $A$ is sequenced before $B$, or
\item $A$ synchronizes with $B$, or
\item $A$ happens before $X$ and $X$ happens before $B$.
\end{itemize}
\begin{note}
An evaluation does not happen before itself.
\end{note}

\pnum
An evaluation $A$ \defn{strongly happens before}
an evaluation $D$ if, either
\begin{itemize}
\item $A$ is sequenced before $D$, or
\item $A$ synchronizes with $D$, and
both $A$ and $D$ are
sequentially consistent atomic operations\iref{atomics.order}, or
\item there are evaluations $B$ and $C$
such that $A$ is sequenced before $B$,
$B$ happens before $C$, and
$C$ is sequenced before $D$, or
\item there is an evaluation $B$ such that
$A$ strongly happens before $B$, and
$B$ strongly happens before $D$.
\end{itemize}
\begin{note}
Informally, if $A$ strongly happens before $B$,
then $A$ appears to be evaluated before $B$
in all contexts.
\end{note}

\pnum
A \defnadjx{visible}{side effect}{side effects} $A$ on a scalar object or bit-field $M$
with respect to a value computation $B$ of $M$ satisfies the
conditions:
\begin{itemize}
\item $A$ happens before $B$ and
\item there is no other
\indextext{side effects}%
side effect $X$ to $M$ such that $A$
happens before $X$ and $X$ happens before $B$.
\end{itemize}

The value of a non-atomic scalar object or bit-field $M$, as determined by
evaluation $B$, is the value stored by the
\indextext{side effects!visible}%
visible side effect $A$.
\begin{note}
If there is ambiguity about which side effect to a
non-atomic object or bit-field is visible, then the behavior is either
unspecified or undefined.
\end{note}
\begin{note}
This states that operations on
ordinary objects are not visibly reordered. This is not actually detectable
without data races, but is needed to ensure that data races, as defined
below, and with suitable restrictions on the use of atomics, correspond to data
races in a simple interleaved (sequentially consistent) execution.
\end{note}

\pnum
The value of an
atomic object $M$, as determined by evaluation $B$, is the value
stored by some unspecified
side effect $A$ that modifies $M$, where $B$ does not happen
before $A$.
\begin{note}
The set of such side effects is also restricted by the rest of the rules
described here, and in particular, by the coherence requirements below.
\end{note}

\pnum
\indextext{coherence!write-write}%
If an operation $A$ that modifies an atomic object $M$ happens before
an operation $B$ that modifies $M$, then $A$ is earlier
than $B$ in the modification order of $M$.
\begin{note}
This requirement is known as write-write coherence.
\end{note}

\pnum
\indextext{coherence!read-read}%
If a
\indextext{value computation}%
value computation $A$ of an atomic object $M$ happens before a
value computation $B$ of $M$, and $A$ takes its value from a side
effect $X$ on $M$, then the value computed by $B$ is either
the value stored by $X$ or the value stored by a
\indextext{side effects}%
side effect $Y$ on $M$,
where $Y$ follows $X$ in the modification order of $M$.
\begin{note}
This requirement is known as read-read coherence.
\end{note}

\pnum
\indextext{coherence!read-write}%
If a
\indextext{value computation}%
value computation $A$ of an atomic object $M$ happens before an
operation $B$ that modifies $M$, then $A$ takes its value from a side
effect $X$ on $M$, where $X$ precedes $B$ in the
modification order of $M$.
\begin{note}
This requirement is known as
read-write coherence.
\end{note}

\pnum
\indextext{coherence!write-read}%
If a
\indextext{side effects}%
side effect $X$ on an atomic object $M$ happens before a value
computation $B$ of $M$, then the evaluation $B$ takes its
value from $X$ or from a
\indextext{side effects}%
side effect $Y$ that follows $X$ in the modification order of $M$.
\begin{note}
This requirement is known as write-read coherence.
\end{note}

\pnum
\begin{note}
The four preceding coherence requirements effectively disallow
compiler reordering of atomic operations to a single object, even if both
operations are relaxed loads. This effectively makes the cache coherence
guarantee provided by most hardware available to \Cpp{} atomic operations.
\end{note}

\pnum
\begin{note}
The value observed by a load of an atomic depends on the ``happens
before'' relation, which depends on the values observed by loads of atomics.
The intended reading is that there must exist an
association of atomic loads with modifications they observe that, together with
suitably chosen modification orders and the ``happens before'' relation derived
as described above, satisfy the resulting constraints as imposed here.
\end{note}

\pnum
Two actions are \defn{potentially concurrent} if
\begin{itemize}
\item they are performed by different threads, or
\item they are unsequenced, at least one is performed by a signal handler, and
they are not both performed by the same signal handler invocation.
\end{itemize}
The execution of a program contains a \defn{data race} if it contains two
potentially concurrent conflicting actions, at least one of which is not atomic,
and neither happens before the other,
except for the special case for signal handlers described below.
Any such data race results in undefined
behavior.
\begin{note}
It can be shown that programs that correctly use mutexes
and \tcode{memory_order::seq_cst} operations to prevent all data races and use no
other synchronization operations behave as if the operations executed by their
constituent threads were simply interleaved, with each
\indextext{value computation}%
value computation of an
object being taken from the last
\indextext{side effects}%
side effect on that object in that
interleaving. This is normally referred to as ``sequential consistency''.
However, this applies only to data-race-free programs, and data-race-free
programs cannot observe most program transformations that do not change
single-threaded program semantics. In fact, most single-threaded program
transformations remain possible, since any program that behaves
differently as a result has undefined behavior.
\end{note}

\pnum
Two accesses to the same non-bit-field object
of type \tcode{\keyword{volatile} std::sig_atomic_t} do not
result in a data race if both occur in the same thread, even if one or more
occurs in a signal handler. For each signal handler invocation, evaluations
performed by the thread invoking a signal handler can be divided into two
groups $A$ and $B$, such that no evaluations in
$B$ happen before evaluations in $A$, and the
evaluations of such \tcode{\keyword{volatile} std::sig_atomic_t} objects take values as though
all evaluations in $A$ happened before the execution of the signal
handler and the execution of the signal handler happened before all evaluations
in $B$.

\pnum
\begin{note}
Compiler transformations that introduce assignments to a potentially
shared memory location that would not be modified by the abstract machine are
generally precluded by this document, since such an assignment might overwrite
another assignment by a different thread in cases in which an abstract machine
execution would not have encountered a data race. This includes implementations
of data member assignment that overwrite adjacent members in separate memory
locations. Reordering of atomic loads in cases in which the atomics in question
might alias is also generally precluded, since this could violate the coherence
rules.
\end{note}

\pnum
\begin{note}
It is possible that transformations that introduce a speculative read of a potentially
shared memory location do not preserve the semantics of the \Cpp{} program as
defined in this document, since they potentially introduce a data race. However,
they are typically valid in the context of an optimizing compiler that targets a
specific machine with well-defined semantics for data races. They would be
invalid for a hypothetical machine that is not tolerant of races or provides
hardware race detection.
\end{note}

\rSec3[intro.progress]{Forward progress}

\pnum
The implementation may assume that any thread will eventually do one of the
following:
\begin{itemize}
\item terminate,
\item invoke the function \tcode{std::this_thread::yield}\iref{thread.thread.this},
\item make a call to a library I/O function,
\item perform an access through a volatile glvalue,
\item perform an atomic or synchronization operation
other than an atomic modify-write operation\iref{atomics.order}, or
\item continue execution of a trivial infinite loop\iref{stmt.iter.general}.
\end{itemize}
\begin{note}
This is intended to allow compiler transformations
such as removal, merging, and reordering of empty loops,
even when termination cannot be proven.
An affordance is made for trivial infinite loops,
which cannot be removed nor reordered.
\end{note}

\pnum
Executions of atomic functions
that are either defined to be lock-free\iref{atomics.flag}
or indicated as lock-free\iref{atomics.lockfree}
are \defnx{lock-free executions}{lock-free execution}.
\begin{itemize}
\item
  If there is only one thread that is not blocked\iref{defns.block}
  in a standard library function,
  a lock-free execution in that thread shall complete.
  \begin{note}
    Concurrently executing threads
    might prevent progress of a lock-free execution.
    For example,
    this situation can occur
    with load-locked store-conditional implementations.
    This property is sometimes termed obstruction-free.
  \end{note}
\item
  When one or more lock-free executions run concurrently,
  at least one should complete.
  \begin{note}
    It is difficult for some implementations
    to provide absolute guarantees to this effect,
    since repeated and particularly inopportune interference
    from other threads
    could prevent forward progress,
    e.g.,
    by repeatedly stealing a cache line
    for unrelated purposes
    between load-locked and store-conditional instructions.
    For implementations that follow this recommendation and
    ensure that such effects cannot indefinitely delay progress
    under expected operating conditions,
    such anomalies
    can therefore safely be ignored by programmers.
    Outside this document,
    this property is sometimes termed lock-free.
  \end{note}
\end{itemize}

\pnum
During the execution of a thread of execution, each of the following is termed
an \defn{execution step}:
\begin{itemize}
\item termination of the thread of execution,
\item performing an access through a volatile glvalue,
\item completion of a call to a library I/O function, or
\item completion of an atomic or synchronization operation
other than an atomic modify-write operation\iref{atomics.order}.
\end{itemize}

\pnum
An invocation of a standard library function that blocks\iref{defns.block}
is considered to continuously execute execution steps while waiting for the
condition that it blocks on to be satisfied.
\begin{example}
A library I/O function that blocks until the I/O operation is complete can
be considered to continuously check whether the operation is complete. Each
such check consists of one or more execution steps, for example using
observable behavior of the abstract machine.
\end{example}

\pnum
\begin{note}
Because of this and the preceding requirement regarding what threads of execution
have to perform eventually, it follows that no thread of execution can execute
forever without an execution step occurring.
\end{note}

\pnum
A thread of execution \defnx{makes progress}{make progress!thread}
when an execution step occurs or a
lock-free execution does not complete because there are other concurrent threads
that are not blocked in a standard library function (see above).

\pnum
\indextext{forward progress guarantees!concurrent}%
For a thread of execution providing \defn{concurrent forward progress guarantees},
the implementation ensures that the thread will eventually make progress for as
long as it has not terminated.
\begin{note}
This applies regardless of whether or not other threads of execution (if any)
have been or are making progress. To eventually fulfill this requirement means that
this will happen in an unspecified but finite amount of time.
\end{note}

\pnum
It is \impldef{whether the thread that executes \tcode{main} and the threads created
by \tcode{std::thread} or \tcode{std::jthread} provide concurrent forward progress guarantees} whether the
implementation-created thread of execution that executes
\tcode{main}\iref{basic.start.main} and the threads of execution created by
\tcode{std::thread}\iref{thread.thread.class}
or \tcode{std::jthread}\iref{thread.jthread.class}
provide concurrent forward progress guarantees.
General-purpose implementations should provide these guarantees.

\pnum
\indextext{forward progress guarantees!parallel}%
For a thread of execution providing \defn{parallel forward progress guarantees},
the implementation is not required to ensure that the thread will eventually make
progress if it has not yet executed any execution step; once this thread has
executed a step, it provides concurrent forward progress guarantees.

\pnum
\begin{note}
This does not specify a requirement for when to start this thread of execution,
which will typically be specified by the entity that creates this thread of
execution. For example, a thread of execution that provides concurrent forward
progress guarantees and executes tasks from a set of tasks in an arbitrary order,
one after the other, satisfies the requirements of parallel forward progress for
these tasks.
\end{note}

\pnum
\indextext{forward progress guarantees!weakly parallel}%
For a thread of execution providing \defn{weakly parallel forward progress
guarantees}, the implementation does not ensure that the thread will eventually
make progress.

\pnum
\begin{note}
Threads of execution providing weakly parallel forward progress guarantees cannot
be expected to make progress regardless of whether other threads make progress or
not; however, blocking with forward progress guarantee delegation, as defined below,
can be used to ensure that such threads of execution make progress eventually.
\end{note}

\pnum
Concurrent forward progress guarantees are stronger than parallel forward progress
guarantees, which in turn are stronger than weakly parallel forward progress
guarantees.
\begin{note}
For example, some kinds of synchronization between threads of execution might only
make progress if the respective threads of execution provide parallel forward progress
guarantees, but will fail to make progress under weakly parallel guarantees.
\end{note}

\pnum
\indextext{forward progress guarantees!delegation of}%
When a thread of execution $P$ is specified to
\defnx{block with forward progress guarantee delegation}
{block (execution)!with forward progress guarantee delegation}
on the completion of a set $S$ of threads of execution,
then throughout the whole time of $P$ being blocked on $S$,
the implementation shall ensure that the forward progress guarantees
provided by at least one thread of execution in $S$
is at least as strong as $P$'s forward progress guarantees.
\begin{note}
It is unspecified which thread or threads of execution in $S$ are chosen
and for which number of execution steps. The strengthening is not permanent and
not necessarily in place for the rest of the lifetime of the affected thread of
execution. As long as $P$ is blocked, the implementation has to eventually
select and potentially strengthen a thread of execution in $S$.
\end{note}
Once a thread of execution in $S$ terminates, it is removed from $S$.
Once $S$ is empty, $P$ is unblocked.

\pnum
\begin{note}
A thread of execution $B$ thus can temporarily provide an effectively
stronger forward progress guarantee for a certain amount of time, due to a
second thread of execution $A$ being blocked on it with forward
progress guarantee delegation. In turn, if $B$ then blocks with
forward progress guarantee delegation on $C$, this can also temporarily
provide a stronger forward progress guarantee to $C$.
\end{note}

\pnum
\begin{note}
If all threads of execution in $S$ finish executing (e.g., they terminate
and do not use blocking synchronization incorrectly), then $P$'s execution
of the operation that blocks with forward progress guarantee delegation will not
result in $P$'s progress guarantee being effectively weakened.
\end{note}

\pnum
\begin{note}
This does not remove any constraints regarding blocking synchronization for
threads of execution providing parallel or weakly parallel forward progress
guarantees because the implementation is not required to strengthen a particular
thread of execution whose too-weak progress guarantee is preventing overall progress.
\end{note}

\pnum
An implementation should ensure that the last value (in modification order)
assigned by an atomic or synchronization operation will become visible to all
other threads in a finite period of time.%
\indextext{atomic!operation|)}%
\indextext{threads!multiple|)}

\rSec2[basic.start]{Start and termination}

\rSec3[basic.start.main]{\tcode{main} function}
\indextext{\idxcode{main} function|(}

\pnum
\indextext{program!startup|(}%
A program shall contain exactly one function called \tcode{main}
that belongs to the global scope.
Executing a program starts a main thread of execution\iref{intro.multithread,thread.threads}
in which the \tcode{main} function is invoked.
\indextext{implementation!freestanding}%
It is \impldef{defining \tcode{main} in freestanding environment}
whether a program in a freestanding environment is required to define a \tcode{main}
function.
\begin{note}
In a freestanding environment, startup and termination is
\impldef{startup and termination in freestanding environment}; startup contains the
execution of constructors for non-local objects with static storage duration;
termination contains the execution of destructors for objects with static storage
duration.
\end{note}

\pnum
An implementation shall not predefine the \tcode{main} function.
Its type shall have \Cpp{} language linkage
and it shall have a declared return type of type
\keyword{int}, but otherwise its type is \impldef{parameters to \tcode{main}}.
\indextext{\idxcode{main} function!implementation-defined parameters to}%
An implementation shall allow both
\begin{itemize}
\item a function of \tcode{()} returning \keyword{int} and
\item a function of \tcode{(\keyword{int}}, pointer to pointer to \tcode{\keyword{char})} returning \keyword{int}
\end{itemize}
\indextext{\idxcode{argc}}%
\indextext{\idxcode{argv}}%
as the type of \tcode{main}\iref{dcl.fct}.
\indextext{\idxcode{main} function!parameters to}%
\indextext{environment!program}%
In the latter form, for purposes of exposition, the first function
parameter is called \tcode{argc} and the second function parameter is
called \tcode{argv}, where \tcode{argc} shall be the number of
arguments passed to the program from the environment in which the
program is run. If
\tcode{argc} is nonzero these arguments shall be supplied in
\tcode{argv[0]} through \tcode{argv[argc - 1]} as pointers to the initial
characters of null-terminated multibyte strings (\ntmbs{}s)\iref{multibyte.strings}
and \tcode{argv[0]} shall be the pointer to
the initial character of an \ntmbs{} that represents the name used to
invoke the program or \tcode{""}. The value of \tcode{argc} shall be
non-negative. The value of \tcode{argv[argc]} shall be 0.

\recommended
Any further (optional) parameters should be added after \tcode{argv}.

\pnum
The function \tcode{main} shall not be named by an expression.
\indextext{\idxcode{main} function!implementation-defined linkage of}%
The linkage\iref{basic.link} of \tcode{main} is
\impldef{linkage of \tcode{main}}. A program that defines \tcode{main} as
deleted or that declares \tcode{main} to be
\keyword{inline}, \keyword{static}, \keyword{constexpr}, or \keyword{consteval} is ill-formed.
The function \tcode{main} shall not be a coroutine\iref{dcl.fct.def.coroutine}.
The \tcode{main} function shall not be declared with a
\grammarterm{linkage-specification}\iref{dcl.link} other than \tcode{"C++"}.
A program that declares
\begin{itemize}
\item
a variable \tcode{main} that belongs to the global scope, or
\item
a function \tcode{main} that belongs to the global scope and
is attached to a named module, or
\item
a function template \tcode{main} that belongs to the global scope, or
\item
an entity named \tcode{main}
with C language linkage (in any namespace)
\end{itemize}
is ill-formed.
The name \tcode{main} is
not otherwise reserved.
\begin{example}
Member functions, classes, and
enumerations can be called \tcode{main}, as can entities in other
namespaces.
\end{example}

\pnum
\indextext{\idxcode{exit}}%
\indexlibraryglobal{exit}%
\indextext{termination!program}%
Terminating the program
without leaving the current block (e.g., by calling the function
\tcode{std::exit(int)}\iref{support.start.term}) does not destroy any
objects with automatic storage duration\iref{class.dtor}. If
\tcode{std::exit} is invoked during the destruction of
an object with static or thread storage duration, the program has undefined
behavior.

\pnum
\indextext{termination!program}%
\indextext{\idxcode{main} function!return from}%
A \keyword{return} statement\iref{stmt.return} in \tcode{main} has the effect of leaving the \tcode{main}
function (destroying any objects with automatic storage duration
and evaluating any postcondition assertions of \tcode{main})
and calling \tcode{std::exit} with the return value as the argument.
If control flows off the end of
the \grammarterm{compound-statement} of \tcode{main},
the effect is equivalent to a \keyword{return} with operand \tcode{0}
(see also \ref{except.handle}).
\indextext{\idxcode{main} function|)}

\rSec3[basic.start.static]{Static initialization}

\pnum
\indextext{initialization}%
\indextext{initialization!static and thread}%
Variables with static storage duration
are initialized as a consequence of program initiation. Variables with
thread storage duration are initialized as a consequence of thread execution.
Within each of these phases of initiation, initialization occurs as follows.

\pnum
\indextext{initialization!constant}%
\defnx{Constant initialization}{constant initialization} is performed
if a variable with static or thread storage duration
is constant-initialized\iref{expr.const}.
\indextext{initialization!zero-initialization}%
If constant initialization is not performed, a variable with static
storage duration\iref{basic.stc.static} or thread storage
duration\iref{basic.stc.thread} is zero-initialized\iref{dcl.init}.
Together, zero-initialization and constant initialization are called
\defnadj{static}{initialization};
all other initialization is \defnadj{dynamic}{initialization}.
All static initialization strongly happens before\iref{intro.races}
any dynamic initialization.
\begin{note}
The dynamic initialization of non-block variables is described
in~\ref{basic.start.dynamic}; that of static block variables is described
in~\ref{stmt.dcl}.
\end{note}

\pnum
An implementation is permitted to perform the initialization of a
variable with static or thread storage duration as a static
initialization even if such initialization is not required to be done
statically, provided that
\begin{itemize}
\item
the dynamic version of the initialization does not change the
value of any other object of static or thread storage duration
prior to its initialization, and

\item
the static version of the initialization produces the same value
in the initialized variable as would be produced by the dynamic
initialization if all variables not required to be initialized statically
were initialized dynamically.
\end{itemize}
\begin{note}
As a consequence, if the initialization of an object \tcode{obj1} refers to an
object \tcode{obj2} potentially requiring dynamic initialization and defined
later in the same translation unit, it is unspecified whether the value of \tcode{obj2} used
will be the value of the fully initialized \tcode{obj2} (because \tcode{obj2} was statically
initialized) or will be the value of \tcode{obj2} merely zero-initialized. For example,
\begin{codeblock}
inline double fd() { return 1.0; }
extern double d1;
double d2 = d1;     // unspecified:
                    // either statically initialized to \tcode{0.0} or
                    // dynamically initialized to \tcode{0.0} if \tcode{d1} is
                    // dynamically initialized, or \tcode{1.0} otherwise
double d1 = fd();   // either initialized statically or dynamically to \tcode{1.0}
\end{codeblock}
\end{note}

\rSec3[basic.start.dynamic]{Dynamic initialization of non-block variables}

\pnum
\indextext{initialization!dynamic non-block}%
\indextext{start!program}%
\indextext{initialization!order of}%
Dynamic initialization of a non-block variable with static storage duration is
unordered if the variable is an implicitly or explicitly instantiated
specialization, is partially-ordered if the variable
is an inline variable that is not an implicitly or explicitly instantiated
specialization, and otherwise is ordered.
\begin{note}
A non-inline explicit specialization of a templated variable
has ordered initialization.
\end{note}

\pnum
A declaration \tcode{D} is
\defn{appearance-ordered} before a declaration \tcode{E} if
\begin{itemize}
\item \tcode{D} appears in the same translation unit as \tcode{E}, or
\item the translation unit containing \tcode{E}
has an interface dependency on the translation unit containing \tcode{D},
\end{itemize}
in either case prior to \tcode{E}.

\pnum
Dynamic initialization of non-block variables \tcode{V} and \tcode{W}
with static storage duration are ordered as follows:
\begin{itemize}
\item
If \tcode{V} and \tcode{W} have ordered initialization and
the definition of \tcode{V}
is appearance-ordered before the definition of \tcode{W}, or
if \tcode{V} has partially-ordered initialization,
\tcode{W} does not have unordered initialization, and
for every definition \tcode{E} of \tcode{W}
there exists a definition \tcode{D} of \tcode{V}
such that \tcode{D} is appearance-ordered before \tcode{E}, then
\begin{itemize}
\item
if the program does not start a thread\iref{intro.multithread}
other than the main thread\iref{basic.start.main}
or \tcode{V} and \tcode{W} have ordered initialization and
they are defined in the same translation unit,
the initialization of \tcode{V}
is sequenced before
the initialization of \tcode{W};
\item
otherwise,
the initialization of \tcode{V}
strongly happens before
the initialization of \tcode{W}.
\end{itemize}

\item
Otherwise, if the program starts a thread
other than the main thread
before either \tcode{V} or \tcode{W} is initialized,
it is unspecified in which threads
the initializations of \tcode{V} and \tcode{W} occur;
the initializations are unsequenced if they occur in the same thread.

\item
Otherwise, the initializations of \tcode{V} and \tcode{W} are indeterminately sequenced.
\end{itemize}
\begin{note}
This definition permits initialization of a sequence of
ordered variables concurrently with another sequence.
\end{note}

\pnum
\indextext{non-initialization odr-use|see{odr-use, non-initialization}}%
A \defnx{non-initialization odr-use}{odr-use!non-initialization}
is an odr-use\iref{term.odr.use} not caused directly or indirectly by
the initialization of a non-block static or thread storage duration variable.

\pnum
\indextext{evaluation!unspecified order of}%
It is \impldef{dynamic initialization of static variables before \tcode{main}}
whether the dynamic initialization of a
non-block non-inline variable with static storage duration
is sequenced before the first statement of \tcode{main} or is deferred.
If it is deferred, it strongly happens before
any non-initialization odr-use
of any non-inline function or non-inline variable
defined in the same translation unit as the variable to be initialized.
\begin{footnote}
A non-block variable with static storage duration
having initialization
with side effects is initialized in this case,
even if it is not itself odr-used\iref{term.odr.use,basic.stc.static}.
\end{footnote}
It is \impldef{threads and program points at which deferred dynamic initialization is performed}
in which threads and at which points in the program such deferred dynamic initialization occurs.

\recommended
An implementation should choose such points in a way
that allows the programmer to avoid deadlocks.
\begin{example}
\begin{codeblock}
// - File 1 -
#include "a.h"
#include "b.h"
B b;
A::A() {
  b.Use();
}

// - File 2 -
#include "a.h"
A a;

// - File 3 -
#include "a.h"
#include "b.h"
extern A a;
extern B b;

int main() {
  a.Use();
  b.Use();
}
\end{codeblock}

It is \impldef{dynamic initialization of static variables before \tcode{main}}
whether either \tcode{a} or \tcode{b} is
initialized before \tcode{main} is entered or whether the
initializations are delayed until \tcode{a} is first odr-used in
\tcode{main}. In particular, if \tcode{a} is initialized before
\tcode{main} is entered, it is not guaranteed that \tcode{b} will be
initialized before it is odr-used by the initialization of \tcode{a}, that
is, before \tcode{A::A} is called. If, however, \tcode{a} is initialized
at some point after the first statement of \tcode{main}, \tcode{b} will
be initialized prior to its use in \tcode{A::A}.
\end{example}

\pnum
It is \impldef{dynamic initialization of static inline variables before \tcode{main}}
whether the dynamic initialization of a
non-block inline variable with static storage duration
is sequenced before the first statement of \tcode{main} or is deferred.
If it is deferred, it strongly happens before
any non-initialization odr-use
of that variable.
It is \impldef{threads and program points at which deferred dynamic initialization is performed}
in which threads and at which points in the program such deferred dynamic initialization occurs.

\pnum
It is \impldef{dynamic initialization of thread-local variables before entry}
whether the dynamic initialization of a
non-block non-inline variable with thread storage duration
is sequenced before the first statement of the initial function of a thread or is deferred.
If it is deferred,
the initialization associated with the entity for thread \placeholder{t}
is sequenced before the first non-initialization odr-use by \placeholder{t}
of any non-inline variable with thread storage duration
defined in the same translation unit as the variable to be initialized.
It is \impldef{threads and program points at which deferred dynamic initialization is performed}
in which threads and at which points in the program such deferred dynamic initialization occurs.

\pnum
If the initialization of
a non-block variable with static or thread storage duration
exits via an exception,
the function \tcode{std::terminate} is called\iref{except.terminate}.%
\indextext{program!startup|)}

\rSec3[basic.start.term]{Termination}

\pnum
\indextext{program!termination|(}%
\indextext{object!destructor static}%
\indextext{\idxcode{main} function!return from}%
Constructed objects\iref{dcl.init}
with static storage duration are destroyed
and functions registered with \tcode{std::atexit}
are called as part of a call to
\indextext{\idxcode{exit}}%
\indexlibraryglobal{exit}%
\tcode{std::exit}\iref{support.start.term}.
The call to \tcode{std::exit} is sequenced before
the destructions and the registered functions.
\begin{note}
Returning from \tcode{main} invokes \tcode{std::exit}\iref{basic.start.main}.
\end{note}

\pnum
Constructed objects with thread storage duration within a given thread
are destroyed as a result of returning from the initial function of that thread and as a
result of that thread calling \tcode{std::exit}.
The destruction of all constructed objects with thread storage
duration within that thread strongly happens before destroying
any object with static storage duration.

\pnum
If the completion of the constructor or dynamic initialization of an object with static
storage duration strongly happens before that of another, the completion of the destructor
of the second is sequenced before the initiation of the destructor of the first.
If the completion of the constructor or dynamic initialization of an object with thread
storage duration is sequenced before that of another, the completion of the destructor
of the second is sequenced before the initiation of the destructor of the first.
If an object is
initialized statically, the object is destroyed in the same order as if
the object was dynamically initialized. For an object of array or class
type, all subobjects of that object are destroyed before any block
variable with static storage duration initialized during the construction
of the subobjects is destroyed.
If the destruction of an object with static or thread storage duration
exits via an exception,
the function \tcode{std::terminate} is called\iref{except.terminate}.

\pnum
If a function contains a block variable of static or thread storage duration that has been
destroyed and the function is called during the destruction of an object with static or
thread storage duration, the program has undefined behavior if the flow of control
passes through the definition of the previously destroyed block variable.
\begin{note}
Likewise, the behavior is undefined
if the block variable is used indirectly (e.g., through a pointer)
after its destruction.
\end{note}

\pnum
\indextext{\idxcode{atexit}}%
\indexlibraryglobal{atexit}%
If the completion of the initialization of an object with static storage
duration strongly happens before a call to \tcode{std::atexit}~(see
\libheader{cstdlib}, \ref{support.start.term}), the call to the function passed to
\tcode{std::atexit} is sequenced before the call to the destructor for the object. If a
call to \tcode{std::atexit} strongly happens before the completion of the initialization of
an object with static storage duration, the call to the destructor for the
object is sequenced before the call to the function passed to \tcode{std::atexit}. If a
call to \tcode{std::atexit} strongly happens before another call to \tcode{std::atexit}, the
call to the function passed to the second \tcode{std::atexit} call is sequenced before
the call to the function passed to the first \tcode{std::atexit} call.

\pnum
If there is a use of a standard library object or function not permitted within signal
handlers\iref{support.runtime} that does not happen before\iref{intro.multithread}
completion of destruction of objects with static storage duration and execution of
\tcode{std::atexit} registered functions\iref{support.start.term}, the program has
undefined behavior.
\begin{note}
If there is a use of an object with static storage
duration that does not happen before the object's destruction, the program has undefined
behavior. Terminating every thread before a call to \tcode{std::exit} or the exit from
\tcode{main} is sufficient, but not necessary, to satisfy these requirements. These
requirements permit thread managers as static-storage-duration objects.
\end{note}

\pnum
\indextext{\idxcode{abort}}%
\indexlibraryglobal{abort}%
\indextext{termination!program}%
Calling the function \tcode{std::abort()} declared in
\libheaderref{cstdlib} terminates the program without executing any destructors
and without calling
the functions passed to \tcode{std::atexit()} or \tcode{std::at_quick_exit()}.%
\indextext{program!termination|)}
\indextext{program execution|)}

\rSec1[basic.contract]{Contract assertions}%
\indextext{contract assertion|(}%

\rSec2[basic.contract.general]{General}%

\pnum
\defnx{Contract assertions}{contract assertion}
allow the programmer to specify
properties of the state of the program
that are expected to hold at
certain points during execution.
Contract assertions are introduced by
\grammarterm{precondition-specifier}s,
\grammarterm{postcondition-specifier}s\iref{dcl.contract.func}, and
\grammarterm{assertion-statement}s\iref{stmt.contract.assert}.

\pnum
Each contract assertion has a \defnadjx{contract-assertion}{predicate}{predicate},
which is an expression of type \tcode{bool}.

\begin{note}
The value of the predicate is used to identify program states that are
expected.
\end{note}

\pnum
An invocation of the macro \tcode{va_start}\iref{cstdarg.syn}
shall not be a subexpression
of the predicate of a contract assertion,
no diagnostic required.

\pnum
\begin{note}
Within the predicate of a contract assertion,
\grammarterm{id-expression}s referring to
variables declared outside the contract assertion
are \keyword{const}\iref{expr.prim.id.unqual},
\tcode{this} is a pointer to \keyword{const}\iref{expr.prim.this},
and the result object can be named
if a \grammarterm{result-name-introducer}\iref{dcl.contract.res} has been specified.
\end{note}

\rSec2[basic.contract.eval]{Evaluation}

\pnum
\indexdefn{evaluation semantics|see{contract evaluation semantics}}%
\indexdefn{checking semantics|see{contract evaluation semantics!checking}}%
\indexdefn{terminating semantics|see{contract evaluation semantics!terminating}}%
An evaluation of a contract assertion
uses one of the following four \defn{evaluation semantics}:
\defnx{ignore}{contract evaluation semantics!ignore},
\defnx{observe}{contract evaluation semantics!observe},
\defnx{enforce}{contract evaluation semantics!enforce}, or
\defnx{quick-enforce}{contract evaluation semantics!quick-enforce}.
Observe, enforce, and quick-enforce are \defnx{checking semantics}{contract evaluation semantics!checking};
enforce and quick-enforce are \defnx{terminating semantics}{contract evaluation semantics!terminating}.

\pnum
It is
\impldef{evaluation semantic used for the evaluation of a contract assertion}
which evaluation semantic is used
for any given evaluation of a contract assertion.
\begin{note}
The range and flexibility of available choices of
evaluation semantics depends on the implementation
and need not allow all four evaluation semantics as possibilities.
The evaluation semantics can differ
for different evaluations of the same contract assertion,
including evaluations during constant evaluation.
\end{note}

\pnum
\recommended
An implementation should provide
the option to translate a program
such that all evaluations of contract assertions use the ignore semantic
as well as
the option to translate a program
such that all evaluations of contract assertions use the enforce semantic.
By default,
evaluations of contract assertions should use the enforce semantic.

\pnum
The evaluation of a contract assertion using the ignore semantic has no effect.
\begin{note}
The predicate is potentially evaluated\iref{basic.def.odr},
but not evaluated.
\end{note}

\pnum
The evaluation $A$ of a contract assertion
using a checking semantic
determines the value of the predicate.
It is unspecified
whether the predicate is evaluated.
Let $B$ be the value that would result from evaluating the predicate.
\begin{note}
To determine whether a predicate would evaluate
to \keyword{true} or \keyword{false},
an alternative evaluation
that produces the same value as the predicate
but has no side effects
can occur.
\begin{example}
\begin{codeblock}
struct S {
  mutable int g = 5;
} s;
void f()
  pre(( s.g++, false ));    // \#1
void g()
{
  f();  // Increment of \tcode{s.g} might not occur, even if \#1 uses a checking semantic.
}
\end{codeblock}
\end{example}
\end{note}

\pnum
There is an observable checkpoint\iref{intro.abstract} $C$
that happens before $A$
such that any other operation $O$
that happens before $A$
also happens before $C$.

\pnum
A \defn{contract violation} occurs when
\begin{itemize}
\item
$B$ is \keyword{false},
\item
the evaluation of the predicate
exits via an exception, or
\item
the evaluation of the predicate
is performed in a context that is
manifestly constant-evaluated\iref{expr.const}
and the predicate
is not a core constant expression.
\end{itemize}

\begin{note}
If $B$ is \keyword{true},
no contract violation occurs and
control flow continues normally
after the point of evaluation of the contract assertion.
The evaluation of the predicate
can fail to produce a value
without causing a contract violation,
for example,
by calling \tcode{longjmp}\iref{csetjmp.syn}
or terminating the program.
\end{note}

\pnum
\indexdefn{contract evaluation semantics!terminating}%
If a contract violation occurs
in a context that is manifestly constant-evaluated\iref{expr.const},
and the evaluation semantic is
a terminating semantic,
the program is ill-formed.

\begin{note}
A diagnostic is produced
if the evaluation semantic is observe\iref{intro.compliance}.
\end{note}

\begin{note}
Different evaluation semantics
chosen for the same contract assertion
in different translation units
can result in
violations of the one-definition rule\iref{basic.def.odr}
when a contract assertion has side effects
that alter the value produced by a constant expression.
\begin{example}
\begin{codeblock}
constexpr int f(int i)
{
  contract_assert((++const_cast<int&>(i), true));
  return i;
}
inline void g()
{
  int a[f(1)];  // size dependent on the evaluation semantic of \tcode{contract_assert} above
}
\end{codeblock}
\end{example}
\end{note}

\pnum
When the program is \defn{contract-terminated},
it is
\impldef{method by which contract termination occurs}
(depending on context) whether
\begin{itemize}
\item
\tcode{std::terminate} is called,
\item
\tcode{std::abort} is called, or
\item
execution is terminated.

\begin{note}
No further execution steps occur\iref{intro.progress}.
\end{note}
\end{itemize}

\begin{note}
Performing the actions of
\tcode{std::terminate} or \tcode{std::abort}
without actually making a library call
is a conforming implementation of
contract-termination\iref{intro.abstract}.
\end{note}

\pnum
\indextext{contract evaluation semantics!enforce}%
\indextext{contract evaluation semantics!quick-enforce}%
If a contract violation occurs
in a context that is not manifestly constant-evaluated
and the evaluation semantic is quick-enforce,
the program is contract-terminated.

\pnum
\indextext{\idxcode{contract_violation}}%
\indextext{contract evaluation semantics!enforce}%
\indextext{contract evaluation semantics!observe}%
\indexlibraryglobal{contract_violation}%
If a contract violation occurs
in a context that is not manifestly constant-evaluated
and the evaluation semantic is enforce or observe,
the contract-violation handler\iref{basic.contract.handler}
is invoked with an lvalue referring to
an object \tcode{v}
of type \tcode{const std::contracts::contract_violation}\iref{support.contract.violation}
containing information about the contract violation.
Storage for \tcode{v}
is allocated in an unspecified manner
except as noted in \ref{basic.stc.dynamic.allocation}.
The lifetime of \tcode{v}
persists for the duration
of the invocation of the contract-violation handler.

\pnum
If the contract violation occurred
because the evaluation of the predicate
exited via an exception,
the contract-violation handler is invoked
from within an active implicit handler
for that exception\iref{except.handle}.
If the contract-violation handler
returns normally
and the evaluation semantic is observe,
that implicit handler
is no longer considered active.

\begin{note}
The exception can be inspected or rethrown within the contract-violation handler.
\end{note}

\pnum
\indextext{contract evaluation semantics!enforce}%
If the contract-violation handler
returns normally
and the evaluation semantic is enforce,
the program is contract-terminated;
if violation occurred
as the result of an uncaught exception
from the evaluation of the predicate,
the implicit handler
remains active when contract termination occurs.

\pnum
\indextext{contract evaluation semantics!observe}%
\begin{note}
If the contract-violation handler
returns normally
and the evaluation semantic is observe,
control flow continues normally
after the point of evaluation of the contract assertion.
\end{note}

\pnum
There is an observable checkpoint\iref{intro.abstract} $C$
that happens after the contract-violation handler returns normally
such that any other operation $O$
that happens after the contract-violation handler returns
also happens after $C$.

\pnum
\begin{note}
The terminating semantics terminate the program
if execution would otherwise continue normally
past a contract violation:
the enforce semantic provides the opportunity to
log information about the contract violation
before terminating the program
or to throw an exception to avoid termination,
and the quick-enforce semantic is intended
to terminate the program as soon as possible
as well as
to minimize the impact of contract checks
on the generated code size.
Conversely,
the observe semantic
provides the opportunity to
log information about the contract violation
without having to terminate the program.
\end{note}

\pnum
If a contract-violation handler
invoked from the evaluation of a function contract assertion\iref{dcl.contract.func}
exits via an exception,
the behavior is as if
the function body exits via that same exception.
\begin{note}
A \grammarterm{function-try-block}\iref{except.pre}
is the function body when present
and thus does not
have an opportunity to catch the exception.
If the function
has a non-throwing exception specification,
the function \tcode{std::terminate} is invoked\iref{except.terminate}.
\end{note}

\begin{note}
If a contract-violation handler
invoked from an \grammarterm{assertion-statement}\iref{stmt.contract.assert}
exits via an exception,
the search for a handler
continues from the execution of that statement.
\end{note}

\pnum
To \defn{evaluate in sequence} a list $R$ of contract assertions:
\begin{itemize}
\item
Construct a list of contract assertions $S$ such that
\begin{itemize}
\item
all elements of $R$ are in $S$,
\item
each element of $R$
may be repeated an
\impldef{maximum number of repeated evaluations of a contract assertion}
number of times
within $S$, and
\item
if a contract assertion $A$
precedes another contract assertion $B$
in $R$,
then the
first occurrence of $A$
precedes the first occurrence of $B$
in $S$.
\end{itemize}
\item
Evaluate each element of $S$ such that,
if a contract assertion $A$
precedes a contract assertion $B$
in $S$,
then the evaluation of $A$
is sequenced before
the evaluation of $B$.
\end{itemize}

\begin{example}
\begin{codeblock}
void f(int i)
{
  contract_assert(i > 0);   // \#1
  contract_assert(i < 10);  // \#2
    // valid sequence of evaluations: \#1 \#2
    // valid sequence of evaluations: \#1 \#1 \#2 \#2
    // valid sequence of evaluations: \#1 \#2 \#1 \#2
    // valid sequence of evaluations: \#1 \#2 \#2 \#1
    // invalid sequence of evaluations: \#2 \#1
}
\end{codeblock}
\end{example}

\pnum
\recommended
An implementation should
provide an option to perform
a specified number of repeated evaluations
for contract assertions.
By default,
no repeated evaluations should be performed.

\rSec2[basic.contract.handler]{Contract-violation handler}

\pnum
\indextext{\idxcode{contract_violation}}%
\indexlibraryglobal{contract_violation}%
The \defn{contract-violation handler}
of a program is a function named
\tcode{::handle_contract_violation}.
The contract-violation handler
shall have a single parameter
of type
``lvalue reference to \keyword{const} \tcode{std::\-contracts::\-contract_violation}''
and shall return \tcode{void}.
The contract-violation handler
may have a non-throwing exception specification.
The implementation
shall provide a definition of the contract-violation handler,
called the \defnadj{default}{contract-violation handler}.
\begin{note}
No declaration
for the default contract-violation handler
is provided by
any standard library header.
\end{note}

\pnum
\recommended
The default contract-violation handler
should produce diagnostic output
that suitably formats the most relevant contents
of the \tcode{std::contracts::contract_violation} object,
rate-limited for potentially repeated violations
of observed contract assertions,
and then return normally.

\pnum
It is
\impldef{replaceability of the contract-violation handler}
whether the contract-violation handler
is replaceable\iref{term.replaceable.function}.
If the contract-violation handler
is not replaceable,
a declaration of a replacement function for the contract-violation handler
is ill-formed, no diagnostic required.

\indextext{contract assertion|)}
