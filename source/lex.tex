%!TEX root = std.tex
\rSec0[lex]{Program Construction}

\gramSec[gram.lex]{Lexical conventions}

\indextext{lexical conventions|see{conventions, lexical}}
\indextext{translation!separate|see{compilation, separate}}
\indextext{separate translation|see{compilation, separate}}
\indextext{separate compilation|see{compilation, separate}}
\indextext{phases of translation|see{translation, phases}}
\indextext{source file character|see{character, source file}}
\indextext{alternative token|see{token, alternative}}
\indextext{digraph|see{token, alternative}}
\indextext{integer literal|see{literal, integer}}
\indextext{character literal|see{literal, character}}
\indextext{floating-point literal|see{literal, floating-point}}
\indextext{string literal|see{literal, string}}
\indextext{boolean literal|see{literal, boolean}}
\indextext{pointer literal|see{literal, pointer}}
\indextext{user-defined literal|see{literal, user-defined}}
\indextext{file, source|see{source file}}
\indextext{null character|see{character, null}}
\indextext{null wide character|see{wide-character, null}}

\rSec1[lex.separate]{Separate translation}

\pnum
\indextext{program}%
A \defn{program} consists of one or more translation units\iref{lex.phases,module.unit}
linked together.

\pnum
\indextext{conventions!lexical|(}%
\indextext{compilation!separate|(}%
The text of the program is kept in units called
\defnx{source files}{source file} in this document.
A source file together with all the headers\iref{headers}
and source files included\iref{cpp.include} via the preprocessing
directive \tcode{\#include}, less any source lines skipped by any of the
conditional inclusion\iref{cpp.cond} preprocessing directives,
as modified by the implementation-defined behavior of any
conditionally-supported-directives\iref{cpp.pre} and pragmas\iref{cpp.pragma},
if any, is
called a \defnadj{preprocessing}{translation unit}.
\begin{note}
A \Cpp{} program need not all be translated at the same time.
\end{note}

\pnum
\begin{note}
Previously translated translation units and instantiation
units can be preserved individually or in libraries. The separate
translation units of a program communicate\iref{basic.link} by (for
example)
calls to functions whose identifiers have external or module linkage,
manipulation of objects whose identifiers have external or module linkage, or
manipulation of data files. Translation units can be separately
translated and then later linked to produce an executable
program\iref{basic.link}.
\end{note}
\indextext{compilation!separate|)}

\rSec1[lex.phases]{Phases of translation}%

\pnum
\indextext{translation!phases|(}%
The precedence among the syntax rules of translation is specified by the
following phases.
\begin{footnote}
Implementations behave as if these separate phases
occur, although in practice different phases can be folded together.
\end{footnote}

\begin{enumerate}
\item
\indextext{character!source file}%
An implementation shall support input files
that are a sequence of UTF-8 code units (UTF-8 files).
It may also support
an \impldef{supported input files} set of other kinds of input files, and,
if so, the kind of an input file is determined in
an \impldef{determination of kind of input file} manner
that includes a means of designating input files as UTF-8 files,
independent of their content.
\begin{note}
In other words,
recognizing the \unicode{feff}{byte order mark} is not sufficient.
\end{note}
If an input file is determined to be a UTF-8 file,
then it shall be a well-formed UTF-8 code unit sequence and
it is decoded to produce a sequence of Unicode
\begin{footnote}
Unicode\textregistered\ is a registered trademark of Unicode, Inc.
This information is given for the convenience of users of this document and
does not constitute an endorsement by ISO or IEC of this product.
\end{footnote}
scalar values.
A sequence of translation character set elements is then formed
by mapping each Unicode scalar value
to the corresponding translation character set element.
In the resulting sequence,
each pair of characters in the input sequence consisting of
\unicode{000d}{carriage return} followed by \unicode{000a}{line feed},
as well as each
\unicode{000d}{carriage return} not immediately followed by a \unicode{000a}{line feed},
is replaced by a single new-line character.

For any other kind of input file supported by the implementation,
characters are mapped, in an
\impldef{mapping physical source file characters to translation character set} manner,
to a sequence of translation character set elements\iref{lex.charset},
representing end-of-line indicators as new-line characters.

\item
\indextext{line splicing}%
If the first translation character is \unicode{feff}{byte order mark},
it is deleted.
Each sequence of a backslash character (\textbackslash)
immediately followed by
zero or more whitespace characters other than new-line followed by
a new-line character is deleted, splicing
physical source lines to form \defnx{logical source lines}{source line!logical}. Only the last
backslash on any physical source line shall be eligible for being part
of such a splice.
\begin{note}
Line splicing can form
a \grammarterm{universal-character-name}\iref{lex.charset}.
\end{note}
A source file that is not empty and that (after splicing)
does not end in a new-line character
shall be processed as if an additional new-line character were appended
to the file.

\item The source file is decomposed into preprocessing
tokens\iref{lex.pptoken} and sequences of whitespace characters
(including comments). A source file shall not end in a partial
preprocessing token or in a partial comment.
\begin{footnote}
A partial preprocessing
token would arise from a source file
ending in the first portion of a multi-character token that requires a
terminating sequence of characters, such as a \grammarterm{header-name}
that is missing the closing \tcode{"}
or \tcode{>}. A partial comment
would arise from a source file ending with an unclosed \tcode{/*}
comment.
\end{footnote}
Each comment is replaced by one space character. New-line characters are
retained. Whether each nonempty sequence of whitespace characters other
than new-line is retained or replaced by one space character is
unspecified.
As characters from the source file are consumed
to form the next preprocessing token
(i.e., not being consumed as part of a comment or other forms of whitespace),
except when matching a
\grammarterm{c-char-sequence},
\grammarterm{s-char-sequence},
\grammarterm{r-char-sequence},
\grammarterm{h-char-sequence}, or
\grammarterm{q-char-sequence},
\grammarterm{universal-character-name}s are recognized and
replaced by the designated element of the translation character set.
The process of dividing a source file's
characters into preprocessing tokens is context-dependent.
\begin{example}
See the handling of \tcode{<} within a \tcode{\#include} preprocessing
directive\iref{cpp.include}.
\end{example}

\item Preprocessing directives are executed, macro invocations are
expanded, and \tcode{_Pragma} unary operator expressions are executed.
A \tcode{\#include} preprocessing directive causes the named header or
source file to be processed from phase 1 through phase 4, recursively.
All preprocessing directives are then deleted.

\item
For a sequence of two or more adjacent \grammarterm{string-literal} tokens,
a common \grammarterm{encoding-prefix} is determined
as specified in \ref{lex.string}.
Each such \grammarterm{string-literal} token is then considered to have
that common \grammarterm{encoding-prefix}.

\item
Adjacent \grammarterm{string-literal} tokens are concatenated\iref{lex.string}.

\item Whitespace characters separating tokens are no longer
significant. Each preprocessing token is converted into a
token\iref{lex.token}. The resulting tokens
constitute a \defn{translation unit} and
are syntactically and
semantically analyzed and translated.
\begin{note}
The process of analyzing and translating the tokens can occasionally
result in one token being replaced by a sequence of other
tokens\iref{temp.names}.
\end{note}
It is
\impldef{whether the sources for
module units and header units
on which the current translation unit has an interface
dependency are required to be available during translation}
whether the sources for
module units and header units
on which the current translation unit has an interface
dependency\iref{module.unit,module.import}
are required to be available.
\begin{note}
Source files, translation
units and translated translation units need not necessarily be stored as
files, nor need there be any one-to-one correspondence between these
entities and any external representation. The description is conceptual
only, and does not specify any particular implementation.
\end{note}

\item Translated translation units and instantiation units are combined
as follows:
\begin{note}
Some or all of these can be supplied from a
library.
\end{note}
Each translated translation unit is examined to
produce a list of required instantiations.
\begin{note}
This can include
instantiations which have been explicitly
requested\iref{temp.explicit}.
\end{note}
The definitions of the
required templates are located. It is \impldef{whether source of translation units must
be available to locate template definitions} whether the
source of the translation units containing these definitions is required
to be available.
\begin{note}
An implementation can choose to encode sufficient
information into the translated translation unit so as to ensure the
source is not required here.
\end{note}
All the required instantiations
are performed to produce
\defn{instantiation units}.
\begin{note}
These are similar
to translated translation units, but contain no references to
uninstantiated templates and no template definitions.
\end{note}
The
program is ill-formed if any instantiation fails.

\item All external entity references are resolved. Library
components are linked to satisfy external references to
entities not defined in the current translation. All such translator
output is collected into a program image which contains information
needed for execution in its execution environment.%
\indextext{translation!phases|)}
\end{enumerate}

\rSec1[lex.char]{Characters}%
\rSec2[lex.charset]{Character sets}

\pnum
\indextext{character set|(}%
The \defnadj{translation}{character set} consists of the following elements:
\begin{itemize}
\item
each abstract character assigned a code point in the Unicode codespace
as specified in the Unicode Standard, and
\item
a distinct character for each Unicode scalar value
not assigned to an abstract character.
\end{itemize}
\begin{note}
Unicode code points are integers
in the range $[0, \mathrm{10FFFF}]$ (hexadecimal).
A surrogate code point is a value
in the range $[\mathrm{D800}, \mathrm{DFFF}]$ (hexadecimal).
A Unicode scalar value is any code point that is not a surrogate code point.
\end{note}

\pnum
The \defnadj{basic}{character set} is a subset of the translation character set,
consisting of 99 characters as specified in \tref{lex.charset.basic}.
\begin{note}
Unicode short names are given only as a means to identifying the character;
the numerical value has no other meaning in this context.
\end{note}

\begin{floattable}{Basic character set}{lex.charset.basic}{lll}
\topline
\lhdrx{2}{character} & \rhdr{glyph} \\ \capsep
\ucode{0009} & \uname{character tabulation} & \\
\ucode{000b} & \uname{line tabulation} & \\
\ucode{000c} & \uname{form feed} & \\
\ucode{0020} & \uname{space} & \\
\ucode{000a} & \uname{line feed} & new-line \\
\ucode{0021} & \uname{exclamation mark} & \tcode{!} \\
\ucode{0022} & \uname{quotation mark} & \tcode{"} \\
\ucode{0023} & \uname{number sign} & \tcode{\#} \\
\ucode{0024} & \uname{dollar sign} & \tcode{\$} \\
\ucode{0025} & \uname{percent sign} & \tcode{\%} \\
\ucode{0026} & \uname{ampersand}  & \tcode{\&} \\
\ucode{0027} & \uname{apostrophe} & \tcode{'} \\
\ucode{0028} & \uname{left parenthesis} & \tcode{(} \\
\ucode{0029} & \uname{right parenthesis} & \tcode{)} \\
\ucode{002a} & \uname{asterisk} & \tcode{*} \\
\ucode{002b} & \uname{plus sign} & \tcode{+} \\
\ucode{002c} & \uname{comma} & \tcode{,} \\
\ucode{002d} & \uname{hyphen-minus} & \tcode{-} \\
\ucode{002e} & \uname{full stop} & \tcode{.} \\
\ucode{002f} & \uname{solidus} & \tcode{/} \\
\ucode{0030} .. \ucode{0039} & \uname{digit zero .. nine} & \tcode{0 1 2 3 4 5 6 7 8 9} \\
\ucode{003a} & \uname{colon} & \tcode{:} \\
\ucode{003b} & \uname{semicolon} & \tcode{;} \\
\ucode{003c} & \uname{less-than sign} & \tcode{<} \\
\ucode{003d} & \uname{equals sign} & \tcode{=} \\
\ucode{003e} & \uname{greater-than sign} & \tcode{>} \\
\ucode{003f} & \uname{question mark} & \tcode{?} \\
\ucode{0040} & \uname{commercial at} & \tcode{@} \\
\ucode{0041} .. \ucode{005a} & \uname{latin capital letter a .. z} & \tcode{A B C D E F G H I J K L M} \\
 & & \tcode{N O P Q R S T U V W X Y Z} \\
\ucode{005b} & \uname{left square bracket} & \tcode{[} \\
\ucode{005c} & \uname{reverse solidus} & \tcode{\textbackslash} \\
\ucode{005d} & \uname{right square bracket} & \tcode{]} \\
\ucode{005e} & \uname{circumflex accent} & \tcode{\caret} \\
\ucode{005f} & \uname{low line} & \tcode{_} \\
\ucode{0060} & \uname{grave accent} & \tcode{\`} \\
\ucode{0061} .. \ucode{007a} & \uname{latin small letter a .. z} & \tcode{a b c d e f g h i j k l m} \\
 & & \tcode{n o p q r s t u v w x y z} \\
\ucode{007b} & \uname{left curly bracket} & \tcode{\{} \\
\ucode{007c} & \uname{vertical line} & \tcode{|} \\
\ucode{007d} & \uname{right curly bracket} & \tcode{\}} \\
\ucode{007e} & \uname{tilde} & \tcode{\textasciitilde} \\
\end{floattable}

\pnum
The \defnadj{basic literal}{character set} consists of
all characters of the basic character set,
plus the control characters specified in \tref{lex.charset.literal}.

\begin{floattable}{Additional control characters in the basic literal character set}{lex.charset.literal}{ll}
\topline
\ohdrx{2}{character} \\ \capsep
\ucode{0000} & \uname{null} \\
\ucode{0007} & \uname{alert} \\
\ucode{0008} & \uname{backspace} \\
\ucode{000d} & \uname{carriage return} \\
\end{floattable}

\pnum
A \defn{code unit} is an integer value
of character type\iref{basic.fundamental}.
Characters in a \grammarterm{character-literal}
other than a multicharacter or non-encodable character literal or
in a \grammarterm{string-literal} are encoded as
a sequence of one or more code units, as determined
by the \grammarterm{encoding-prefix}\iref{lex.ccon,lex.string};
this is termed the respective \defnadj{literal}{encoding}.
The \defnadj{ordinary literal}{encoding} is
the encoding applied to an ordinary character or string literal.
The \defnadj{wide literal}{encoding} is the encoding applied
to a wide character or string literal.

\pnum
A literal encoding or a locale-specific encoding of one of
the execution character sets\iref{character.seq}
encodes each element of the basic literal character set as
a single code unit with non-negative value,
distinct from the code unit for any other such element.
\begin{note}
A character not in the basic literal character set
can be encoded with more than one code unit;
the value of such a code unit can be the same as
that of a code unit for an element of the basic literal character set.
\end{note}
\indextext{character!null}%
\indextext{wide-character!null}%
The \unicode{0000}{null} character is encoded as the value \tcode{0}.
No other element of the translation character set
is encoded with a code unit of value \tcode{0}.
The code unit value of each decimal digit character after the digit \tcode{0} (\ucode{0030})
shall be one greater than the value of the previous.
The ordinary and wide literal encodings are otherwise
\impldef{ordinary and wide literal encodings}.
\indextext{UTF-8}%
\indextext{UTF-16}%
\indextext{UTF-32}%
For a UTF-8, UTF-16, or UTF-32 literal,
the implementation shall encode
the Unicode scalar value
corresponding to each character of the translation character set
as specified in the Unicode Standard
for the respective Unicode encoding form.
\indextext{character set|)}

\rSec2[lex.universal.char]{Universal character names}

\begin{bnf}
\nontermdef{n-char}\br
     \textnormal{any member of the translation character set except the \unicode{007d}{right curly bracket} or new-line character}
\end{bnf}

\begin{bnf}
\nontermdef{n-char-sequence}\br
    n-char\br
    n-char-sequence n-char
\end{bnf}

\begin{bnf}
\nontermdef{named-universal-character}\br
    \terminal{\textbackslash N\{} n-char-sequence \terminal{\}}
\end{bnf}

\begin{bnf}
\nontermdef{hex-quad}\br
    hexadecimal-digit hexadecimal-digit hexadecimal-digit hexadecimal-digit
\end{bnf}

\begin{bnf}
\nontermdef{simple-hexadecimal-digit-sequence}\br
    hexadecimal-digit\br
    simple-hexadecimal-digit-sequence hexadecimal-digit
\end{bnf}

\begin{bnf}
\nontermdef{universal-character-name}\br
    \terminal{\textbackslash u} hex-quad\br
    \terminal{\textbackslash U} hex-quad hex-quad\br
    \terminal{\textbackslash u\{} simple-hexadecimal-digit-sequence \terminal{\}}\br
    named-universal-character
\end{bnf}

\pnum
The \grammarterm{universal-character-name} construct provides a way to name any
element in the translation character set using just the basic character set.
If a \grammarterm{universal-character-name} outside
the \grammarterm{c-char-sequence}, \grammarterm{s-char-sequence}, or
\grammarterm{r-char-sequence} of a \grammarterm{character-literal} or
\grammarterm{string-literal}
(in either case, including within a \grammarterm{user-defined-literal})
corresponds to a control character or to a character in the basic character set,
the program is ill-formed.
\begin{note}
A sequence of characters resembling a \grammarterm{universal-character-name} in an
\grammarterm{r-char-sequence}\iref{lex.string} does not form a
\grammarterm{universal-character-name}.
\end{note}

\pnum
A \grammarterm{universal-character-name}
of the form \tcode{\textbackslash u} \grammarterm{hex-quad},
\tcode{\textbackslash U} \grammarterm{hex-quad} \grammarterm{hex-quad}, or
\tcode{\textbackslash u\{\grammarterm{simple-hexadecimal-digit-sequence}\}}
designates the character in the translation character set
whose Unicode scalar value is the hexadecimal number represented by
the sequence of \grammarterm{hexadecimal-digit}s
in the \grammarterm{universal-character-name}.
The program is ill-formed if that number is not a Unicode scalar value.

\pnum
A \grammarterm{universal-character-name}
that is a \grammarterm{named-universal-character}
designates the corresponding character
in the Unicode Standard (chapter 4.8 Name)
if the \grammarterm{n-char-sequence} is equal
to its character name or
to one of its character name aliases of
type ``control'', ``correction'', or ``alternate'';
otherwise, the program is ill-formed.
\begin{note}
These aliases are listed in
the Unicode Character Database's \tcode{NameAliases.txt}.
None of these names or aliases have leading or trailing spaces.
\end{note}

\rSec1[lex.comment]{Comments}

\pnum
\indextext{comment|(}%
\indextext{comment!\tcode{/*} \tcode{*/}}%
\indextext{comment!\tcode{//}}%
The characters \tcode{/*} start a comment, which terminates with the
characters \tcode{*/}. These comments do not nest.
\indextext{comment!\tcode{//}}%
The characters \tcode{//} start a comment, which terminates immediately before the
next new-line character. If there is a form-feed or a vertical-tab
character in such a comment, only whitespace characters shall appear
between it and the new-line that terminates the comment; no diagnostic
is required.
\begin{note}
The comment characters \tcode{//}, \tcode{/*},
and \tcode{*/} have no special meaning within a \tcode{//} comment and
are treated just like other characters. Similarly, the comment
characters \tcode{//} and \tcode{/*} have no special meaning within a
\tcode{/*} comment.
\end{note}
\indextext{comment|)}

\rSec1[lex.pptokenize]{Preprocessor tokenization}
\rSec2[lex.pptoken]{Preprocessing tokens}

\indextext{token!preprocessing|(}%
\begin{bnf}
\nontermdef{preprocessing-token}\br
    import-keyword\br
    module-keyword\br
    export-keyword\br
    header-name\br
    pp-number\br
    preprocessing-op-or-punc\br
    identifier\br
    character-literal\br
    user-defined-character-literal\br
    string-literal\br
    user-defined-string-literal\br
    \textnormal{each non-whitespace character that cannot be one of the above}
\end{bnf}

\pnum
A preprocessing token is the minimal lexical element of the language in translation
phases 3 through 6.
In this document,
glyphs are used to identify
elements of the basic character set\iref{lex.charset}.
The categories of preprocessing token are: header names,
placeholder tokens produced by preprocessing \tcode{import} and \tcode{module} directives
(\grammarterm{import-keyword}, \grammarterm{module-keyword}, and \grammarterm{export-keyword}),
identifiers, preprocessing numbers, character literals (including user-defined character
literals), string literals (including user-defined string literals), preprocessing
operators and punctuators, and single non-whitespace characters that do not lexically
match the other preprocessing token categories.
If a \unicode{0027}{apostrophe} or a \unicode{0022}{quotation mark} character
matches the last category, the program is ill-formed.
If any character not in the basic character set matches the last category,
the program is ill-formed.
Preprocessing tokens can be separated by
\indextext{whitespace}%
whitespace;
\indextext{comment}%
this consists of comments\iref{lex.comment}, or whitespace characters
(\unicode{0020}{space},
\unicode{0009}{character tabulation},
new-line,
\unicode{000b}{line tabulation}, and
\unicode{000c}{form feed}), or both.
As described in \ref{cpp}, in certain
circumstances during translation phase 4, whitespace (or the absence
thereof) serves as more than preprocessing token separation. Whitespace
can appear within a preprocessing token only as part of a header name or
between the quotation characters in a character literal or
string literal.

\pnum
Each preprocessing token that is converted to a token\iref{lex.token}
shall have the lexical form of a keyword, an identifier, a literal,
or an operator or punctuator.

\pnum
The \grammarterm{import-keyword} is produced
by processing an \keyword{import} directive\iref{cpp.import},
the \grammarterm{module-keyword} is produced
by preprocessing a \keyword{module} directive\iref{cpp.module}, and
the \grammarterm{export-keyword} is produced
by preprocessing either of the previous two directives.
\begin{note}
None has any observable spelling.
\end{note}

\pnum
If the input stream has been parsed into preprocessing tokens up to a
given character:
\begin{itemize}
\item
\indextext{literal!string!raw}%
If the next character begins a sequence of characters that could be the prefix
and initial double quote of a raw string literal, such as \tcode{R"}, the next preprocessing
token shall be a raw string literal. Between the initial and final
double quote characters of the raw string, any transformations performed in phase
2 (line splicing) are reverted; this reversion
shall apply before any \grammarterm{d-char}, \grammarterm{r-char}, or delimiting
parenthesis is identified. The raw string literal is defined as the shortest sequence
of characters that matches the raw-string pattern
\begin{ncbnf}
\opt{encoding-prefix} \terminal{R} raw-string
\end{ncbnf}

\item Otherwise, if the next three characters are \tcode{<::} and the subsequent character
is neither \tcode{:} nor \tcode{>}, the \tcode{<} is treated as a preprocessing token by
itself and not as the first character of the alternative token \tcode{<:}.

\item Otherwise,
the next preprocessing token is the longest sequence of
characters that could constitute a preprocessing token, even if that
would cause further lexical analysis to fail,
except that a \grammarterm{header-name}\iref{lex.header} is only formed
\begin{itemize}
\item
after the \tcode{include} or \tcode{import} preprocessing token in an
\tcode{\#include}\iref{cpp.include} or
\tcode{import}\iref{cpp.import} directive, or

\item
within a \grammarterm{has-include-expression}.

\end{itemize}
\end{itemize}

\pnum
\begin{example}
\begin{codeblock}
#define R "x"
const char* s = R"y";           // ill-formed raw string, not \tcode{"x" "y"}
\end{codeblock}
\end{example}

\pnum
\begin{example}
The program fragment \tcode{0xe+foo} is parsed as a
preprocessing number token (one that is not a valid
\grammarterm{integer-literal} or \grammarterm{floating-point-literal} token),
even though a parse as three preprocessing tokens
\tcode{0xe}, \tcode{+}, and \tcode{foo} can produce a valid expression (for example,
if \tcode{foo} is a macro defined as \tcode{1}). Similarly, the
program fragment \tcode{1E1} is parsed as a preprocessing number (one
that is a valid \grammarterm{floating-point-literal} token),
whether or not \tcode{E} is a macro name.
\end{example}

\pnum
\begin{example}
The program fragment \tcode{x+++++y} is parsed as \tcode{x
++ ++ + y}, which, if \tcode{x} and \tcode{y} have integral types,
violates a constraint on increment operators, even though the parse
\tcode{x ++ + ++ y} can yield a correct expression.
\end{example}
\indextext{token!preprocessing|)}

\rSec2[lex.header]{Header names}

\indextext{header!name|(}%
\begin{bnf}
\microtypesetup{protrusion=false}\obeyspaces
\nontermdef{header-name}\br
    \terminal{<} h-char-sequence \terminal{>}\br
    \terminal{"} q-char-sequence \terminal{"}
\end{bnf}

\begin{bnf}
\nontermdef{h-char-sequence}\br
    h-char\br
    h-char-sequence h-char
\end{bnf}

\begin{bnf}
\nontermdef{h-char}\br
    \textnormal{any member of the translation character set except new-line and \unicode{003e}{greater-than sign}}
\end{bnf}

\begin{bnf}
\nontermdef{q-char-sequence}\br
    q-char\br
    q-char-sequence q-char
\end{bnf}

\begin{bnf}
\nontermdef{q-char}\br
    \textnormal{any member of the translation character set except new-line and \unicode{0022}{quotation mark}}
\end{bnf}

\pnum
\begin{note}
Header name preprocessing tokens only appear within
a \tcode{\#include} preprocessing directive,
a \tcode{__has_include} preprocessing expression, or
after certain occurrences of an \tcode{import} token
(see~\ref{lex.pptoken}).
\end{note}
The sequences in both forms of \grammarterm{header-name}{s} are mapped in an
\impldef{mapping header name to header or external source file} manner to headers or to
external source file names as specified in~\ref{cpp.include}.

\pnum
The appearance of either of the characters \tcode{'} or \tcode{\textbackslash} or of
either of the character sequences \tcode{/*} or \tcode{//} in a
\grammarterm{q-char-sequence} or an \grammarterm{h-char-sequence}
is conditionally-supported with \impldef{meaning of \tcode{'}, \tcode{\textbackslash},
\tcode{/*}, or \tcode{//} in a \grammarterm{q-char-sequence} or an
\grammarterm{h-char-sequence}} semantics, as is the appearance of the character
\tcode{"} in an \grammarterm{h-char-sequence}.
\begin{footnote}
Thus, a sequence of characters
that resembles an escape sequence can result in an error, be interpreted as the
character corresponding to the escape sequence, or have a completely different meaning,
depending on the implementation.
\end{footnote}
\indextext{header!name|)}

\rSec2[lex.ppnumber]{Preprocessing numbers}

\indextext{number!preprocessing|(}%
\begin{bnf}
\nontermdef{pp-number}\br
    digit\br
    \terminal{.} digit\br
    pp-number identifier-continue\br
    pp-number \terminal{'} digit\br
    pp-number \terminal{'} nondigit\br
    pp-number \terminal{e} sign\br
    pp-number \terminal{E} sign\br
    pp-number \terminal{p} sign\br
    pp-number \terminal{P} sign\br
    pp-number \terminal{.}
\end{bnf}

\pnum
Preprocessing number tokens lexically include
all \grammarterm{integer-literal} tokens\iref{lex.icon} and
all \grammarterm{floating-point-literal} tokens\iref{lex.fcon}.

\pnum
A preprocessing number does not have a type or a value; it acquires both
after a successful conversion to
an \grammarterm{integer-literal} token or
a \grammarterm{floating-point-literal} token.%
\indextext{number!preprocessing|)}

\rSec2[lex.operators]{Operators and punctuators}

\pnum
\indextext{operator|(}%
\indextext{punctuator|(}%
The lexical representation of \Cpp{} programs includes a number of
preprocessing tokens that are used in the syntax of the preprocessor or
are converted into tokens for operators and punctuators:

\begin{bnf}
\nontermdef{preprocessing-op-or-punc}\br
    preprocessing-operator\br
    operator-or-punctuator
\end{bnf}

\begin{bnf}
%% Ed. note: character protrusion would misalign various operators.
\microtypesetup{protrusion=false}\obeyspaces
\nontermdef{preprocessing-operator} \textnormal{one of}\br
    \terminal{\#        \#\#       \%:       \%:\%:}
\end{bnf}

\begin{bnf}
\microtypesetup{protrusion=false}\obeyspaces
\nontermdef{operator-or-punctuator} \textnormal{one of}\br
    \terminal{\{        \}        [        ]        (        )}\br
    \terminal{<:       :>       <\%       \%>       ;        :        ...}\br
    \terminal{?        ::       .        .*       ->       ->*      \~}\br
    \terminal{!        +        -        *        /        \%        \caret{}        \&        |}\br
    \terminal{=        +=       -=       *=       /=       \%=       \caret{}=       \&=       |=}\br
    \terminal{==       !=       <        >        <=       >=       <=>      \&\&       ||}\br
    \terminal{<<       >>       <<=      >>=      ++       --       ,}\br
    \terminal{\keyword{and}      \keyword{or}       \keyword{xor}      \keyword{not}      \keyword{bitand}   \keyword{bitor}    \keyword{compl}}\br
    \terminal{\keyword{and_eq}   \keyword{or_eq}    \keyword{xor_eq}   \keyword{not_eq}}
\end{bnf}

\pnum
\indextext{token!alternative|(}%
Alternative token representations are provided for some operators and
punctuators.
\begin{footnote}
\indextext{digraph}%
These include ``digraphs'' and additional reserved words. The term
``digraph'' (token consisting of two characters) is not perfectly
descriptive, since one of the alternative \grammarterm{preprocessing-token}s is
\tcode{\%:\%:} and of course several primary tokens contain two
characters. Nonetheless, those alternative tokens that aren't lexical
keywords are colloquially known as ``digraphs''.
\end{footnote}
In all respects of the language, each alternative token behaves the
same, respectively, as its primary token, except for its spelling.
\begin{footnote}
Thus the ``stringized'' values\iref{cpp.stringize} of
\tcode{[} and \tcode{<:} will be different, maintaining the source
spelling, but the tokens can otherwise be freely interchanged.
\end{footnote}
The set of alternative tokens is defined in
\tref{lex.digraph}.

\begin{tokentable}{Alternative tokens}{lex.digraph}{Alternative}{Primary}
\tcode{<\%}             &   \tcode{\{}         &
\keyword{and}           &   \tcode{\&\&}       &
\keyword{and_eq}        &   \tcode{\&=}        \\ \rowsep
\tcode{\%>}             &   \tcode{\}}         &
\keyword{bitor}         &   \tcode{|}          &
\keyword{or_eq}         &   \tcode{|=}         \\ \rowsep
\tcode{<:}              &   \tcode{[}          &
\keyword{or}            &   \tcode{||}         &
\keyword{xor_eq}        &   \tcode{\caret=}    \\ \rowsep
\tcode{:>}              &   \tcode{]}          &
\keyword{xor}           &   \tcode{\caret}     &
\keyword{not}           &   \tcode{!}          \\ \rowsep
\tcode{\%:}             &   \tcode{\#}         &
\keyword{compl}         &   \tcode{\~}         &
\keyword{not_eq}        &   \tcode{!=}         \\ \rowsep
\tcode{\%:\%:}          &   \tcode{\#\#}       &
\keyword{bitand}        &   \tcode{\&}         &
                        &                      \\
\end{tokentable}%
\indextext{token!alternative|)}

\pnum
Each \grammarterm{operator-or-punctuator} is converted to a single token
in translation phase 7\iref{lex.phases}.%
\indextext{punctuator|)}%
\indextext{operator|)}

\rSec2[lex.ccon]{Character literals}

\indextext{literal!character}%
\begin{bnf}
\nontermdef{character-literal}\br
    \opt{encoding-prefix} \terminal{'} c-char-sequence \terminal{'}
\end{bnf}

\begin{bnf}
\nontermdef{encoding-prefix} \textnormal{one of}\br
    \terminal{u8}\quad\terminal{u}\quad\terminal{U}\quad\terminal{L}
\end{bnf}

\begin{bnf}
\nontermdef{c-char-sequence}\br
    c-char\br
    c-char-sequence c-char
\end{bnf}

\begin{bnf}
\nontermdef{c-char}\br
    basic-c-char\br
    escape-sequence\br
    universal-character-name
\end{bnf}

\begin{bnf}
\nontermdef{basic-c-char}\br
    \textnormal{any member of the translation character set except the \unicode{0027}{apostrophe},}\br
    \bnfindent\textnormal{\unicode{005c}{reverse solidus}, or new-line character}
\end{bnf}

\begin{bnf}
\nontermdef{escape-sequence}\br
    simple-escape-sequence\br
    numeric-escape-sequence\br
    conditional-escape-sequence
\end{bnf}

\begin{bnf}
\nontermdef{simple-escape-sequence}\br
    \terminal{\textbackslash} simple-escape-sequence-char
\end{bnf}

\begin{bnf}
\nontermdef{simple-escape-sequence-char} \textnormal{one of}\br
    \terminal{'  "  ?  \textbackslash{} a  b  f  n  r  t  v}
\end{bnf}

\begin{bnf}
\nontermdef{numeric-escape-sequence}\br
    octal-escape-sequence\br
    hexadecimal-escape-sequence
\end{bnf}

\begin{bnf}
\nontermdef{simple-octal-digit-sequence}\br
    octal-digit\br
    simple-octal-digit-sequence octal-digit
\end{bnf}

\begin{bnf}
\nontermdef{octal-escape-sequence}\br
    \terminal{\textbackslash} octal-digit\br
    \terminal{\textbackslash} octal-digit octal-digit\br
    \terminal{\textbackslash} octal-digit octal-digit octal-digit\br
    \terminal{\textbackslash o\{} simple-octal-digit-sequence \terminal{\}}\br
\end{bnf}

\begin{bnf}
\nontermdef{hexadecimal-escape-sequence}\br
    \terminal{\textbackslash x} simple-hexadecimal-digit-sequence\br
    \terminal{\textbackslash x\{} simple-hexadecimal-digit-sequence \terminal{\}}
\end{bnf}

\begin{bnf}
\nontermdef{conditional-escape-sequence}\br
    \terminal{\textbackslash} conditional-escape-sequence-char
\end{bnf}

\begin{bnf}
\nontermdef{conditional-escape-sequence-char}\br
    \textnormal{any member of the basic character set that is not an} octal-digit\textnormal{, a} simple-escape-sequence-char\textnormal{, or the characters \terminal{N}, \terminal{o}, \terminal{u}, \terminal{U}, or \terminal{x}}
\end{bnf}

\pnum
\indextext{literal!character}%
\indextext{literal!\idxcode{char8_t}}%
\indextext{literal!\idxcode{char16_t}}%
\indextext{literal!\idxcode{char32_t}}%
\indextext{literal!type of character}%
\indextext{type!\idxcode{char8_t}}%
\indextext{type!\idxcode{char16_t}}%
\indextext{type!\idxcode{char32_t}}%
\indextext{wide-character}%
\indextext{type!\idxcode{wchar_t}}%
A \defnadj{multicharacter}{literal} is a \grammarterm{character-literal}
whose \grammarterm{c-char-sequence} consists of
more than one \grammarterm{c-char}.
A multicharacter literal shall not have an \grammarterm{encoding-prefix}.
If a multicharacter literal contains a \grammarterm{c-char}
that is not encodable as a single code unit in the ordinary literal encoding,
the program is ill-formed.
Multicharacter literals are conditionally-supported.

\pnum
The kind of a \grammarterm{character-literal},
its type, and its associated character encoding\iref{lex.charset}
are determined by
its \grammarterm{encoding-prefix} and its \grammarterm{c-char-sequence}
as defined by \tref{lex.ccon.literal}.

\begin{floattable}{Character literals}{lex.ccon.literal}
{l|l|l|l|l}
\topline
\lhdr{Encoding} & \chdr{Kind} & \chdr{Type} & \chdr{Associated char-} & \rhdr{Example} \\
\lhdr{prefix} & \chdr{} & \chdr{} & \chdr{acter encoding} & \\
\capsep
none &
\defnx{ordinary character literal}{literal!character!ordinary} &
\keyword{char} &
ordinary literal &
\tcode{'v'} \\ \cline{2-3}\cline{5-5}
 &
multicharacter literal &
\keyword{int} &
encoding &
\tcode{'abcd'} \\ \hline
\tcode{L} &
\defnx{wide character literal}{literal!character!wide} &
\keyword{wchar_t} &
wide literal &
\tcode{L'w'} \\
 & & & encoding & \\ \hline
\tcode{u8} &
\defnx{UTF-8 character literal}{literal!character!UTF-8} &
\keyword{char8_t} &
UTF-8 &
\tcode{u8'x'} \\ \hline
\tcode{u} &
\defnx{UTF-16 character literal}{literal!character!UTF-16} &
\keyword{char16_t} &
UTF-16 &
\tcode{u'y'} \\ \hline
\tcode{U} &
\defnx{UTF-32 character literal}{literal!character!UTF-32} &
\keyword{char32_t} &
UTF-32 &
\tcode{U'z'} \\
\end{floattable}

\pnum
In translation phase 4,
the value of a \grammarterm{character-literal} is determined
using the range of representable values
of the \grammarterm{character-literal}'s type in translation phase 7.
A multicharacter literal has an
\impldef{value of non-encodable character literal or multicharacter literal}
value.
The value of any other kind of \grammarterm{character-literal}
is determined as follows:
\begin{itemize}
\item
A \grammarterm{character-literal} with
a \grammarterm{c-char-sequence} consisting of a single
\grammarterm{basic-c-char},
\grammarterm{simple-escape-sequence}, or
\grammarterm{universal-character-name}
is the code unit value of the specified character
as encoded in the literal's associated character encoding.
If the specified character lacks
representation in the literal's associated character encoding or
if it cannot be encoded as a single code unit,
then the program is ill-formed.
\item
A \grammarterm{character-literal} with
a \grammarterm{c-char-sequence} consisting of
a single \grammarterm{numeric-escape-sequence}
has a value as follows:
\begin{itemize}
\item
Let $v$ be the integer value represented by
the octal number comprising
the sequence of \grammarterm{octal-digit}{s} in
an \grammarterm{octal-escape-sequence} or by
the hexadecimal number comprising
the sequence of \grammarterm{hexadecimal-digit}{s} in
a \grammarterm{hexadecimal-escape-sequence}.
\item
If $v$ does not exceed
the range of representable values of the \grammarterm{character-literal}'s type,
then the value is $v$.
\item
Otherwise,
if the \grammarterm{character-literal}'s \grammarterm{encoding-prefix}
is absent or \tcode{L}, and
$v$ does not exceed the range of representable values of the corresponding unsigned type for the underlying type of the \grammarterm{character-literal}'s type,
then the value is the unique value of the \grammarterm{character-literal}'s type \tcode{T} that is congruent to $v$ modulo $2^N$, where $N$ is the width of \tcode{T}.
\item
Otherwise, the program is ill-formed.
\end{itemize}
\item
A \grammarterm{character-literal} with
a \grammarterm{c-char-sequence} consisting of
a single \grammarterm{conditional-escape-sequence}
is conditionally-supported and
has an \impldef{value of \grammarterm{conditional-escape-sequence}} value.
\end{itemize}

\pnum
\indextext{backslash character}%
\indextext{\idxcode{\textbackslash}|see{backslash character}}%
\indextext{escape character|see{backslash character}}%
The character specified by a \grammarterm{simple-escape-sequence}
is specified in \tref{lex.ccon.esc}.
\begin{note}
Using an escape sequence for a question mark
is supported for compatibility with \CppXIV{} and C.
\end{note}

\begin{floattable}{Simple escape sequences}{lex.ccon.esc}
{lll}
\topline
\lhdrx{2}{character} &  \rhdr{\grammarterm{simple-escape-sequence}} \\ \capsep
\ucode{000a} & \uname{line feed}            & \tcode{\textbackslash n} \\
\ucode{0009} & \uname{character tabulation} & \tcode{\textbackslash t} \\
\ucode{000b} & \uname{line tabulation}      & \tcode{\textbackslash v} \\
\ucode{0008} & \uname{backspace}            & \tcode{\textbackslash b} \\
\ucode{000d} & \uname{carriage return}      & \tcode{\textbackslash r} \\
\ucode{000c} & \uname{form feed}            & \tcode{\textbackslash f} \\
\ucode{0007} & \uname{alert}                & \tcode{\textbackslash a} \\
\ucode{005c} & \uname{reverse solidus}      & \tcode{\textbackslash\textbackslash} \\
\ucode{003f} & \uname{question mark}        & \tcode{\textbackslash ?} \\
\ucode{0027} & \uname{apostrophe}           & \tcode{\textbackslash '} \\
\ucode{0022} & \uname{quotation mark}       & \tcode{\textbackslash "} \\
\end{floattable}

\rSec2[lex.string]{String literals}

\indextext{literal!string}%
\begin{bnf}
\nontermdef{string-literal}\br
    \opt{encoding-prefix} \terminal{"} \opt{s-char-sequence} \terminal{"}\br
    \opt{encoding-prefix} \terminal{R} raw-string
\end{bnf}

\begin{bnf}
\nontermdef{s-char-sequence}\br
    s-char\br
    s-char-sequence s-char
\end{bnf}

\begin{bnf}
\nontermdef{s-char}\br
    basic-s-char\br
    escape-sequence\br
    universal-character-name
\end{bnf}

\begin{bnf}
\nontermdef{basic-s-char}\br
    \textnormal{any member of the translation character set except the \unicode{0022}{quotation mark},}\br
    \bnfindent\textnormal{\unicode{005c}{reverse solidus}, or new-line character}
\end{bnf}

\begin{bnf}
\nontermdef{raw-string}\br
    \terminal{"} \opt{d-char-sequence} \terminal{(} \opt{r-char-sequence} \terminal{)} \opt{d-char-sequence} \terminal{"}
\end{bnf}

\begin{bnf}
\nontermdef{r-char-sequence}\br
    r-char\br
    r-char-sequence r-char
\end{bnf}

\begin{bnf}
\nontermdef{r-char}\br
    \textnormal{any member of the translation character set, except a \unicode{0029}{right parenthesis} followed by}\br
    \bnfindent\textnormal{the initial \grammarterm{d-char-sequence} (which may be empty) followed by a \unicode{0022}{quotation mark}}
\end{bnf}

\begin{bnf}
\nontermdef{d-char-sequence}\br
    d-char\br
    d-char-sequence d-char
\end{bnf}

\begin{bnf}
\nontermdef{d-char}\br
    \textnormal{any member of the basic character set except:}\br
    \bnfindent\textnormal{\unicode{0020}{space}, \unicode{0028}{left parenthesis}, \unicode{0029}{right parenthesis}, \unicode{005c}{reverse solidus},}\br
    \bnfindent\textnormal{\unicode{0009}{character tabulation}, \unicode{000b}{line tabulation}, \unicode{000c}{form feed}, and new-line}
\end{bnf}

\pnum
\indextext{literal!string}%
\indextext{character string}%
\indextext{string!type of}%
\indextext{type!\idxcode{wchar_t}}%
\indextext{prefix!\idxcode{L}}%
\indextext{literal!string!\idxcode{char16_t}}%
\indextext{type!\idxcode{char16_t}}%
\indextext{literal!string!\idxcode{char32_t}}%
\indextext{type!\idxcode{char32_t}}%
The kind of a \grammarterm{string-literal},
its type, and
its associated character encoding\iref{lex.charset}
are determined by its encoding prefix and sequence of
\grammarterm{s-char}s or \grammarterm{r-char}s
as defined by \tref{lex.string.literal}
where $n$ is the number of encoded code units as described below.

\begin{floattable}{String literals}{lex.string.literal}
{llp{2.6cm}p{2.3cm}p{4.7cm}}
\topline
\lhdr{Enco-} & \chdr{Kind} & \chdr{Type} & \chdr{Associated} & \rhdr{Examples} \\
\lhdr{ding}   & \chdr{} & \chdr{} & \chdr{character}  & \rhdr{} \\
\lhdr{prefix}         & \chdr{} & \chdr{} & \chdr{encoding}   & \rhdr{} \\
\capsep
none &
\defnx{ordinary string literal}{literal!string!ordinary} &
array of $n$\newline \tcode{\keyword{const} \keyword{char}} &
ordinary literal encoding &
\tcode{"ordinary string"}\newline
\tcode{R"(ordinary raw string)"} \\
\tcode{L} &
\defnx{wide string literal}{literal!string!wide} &
array of $n$\newline \tcode{\keyword{const} \keyword{wchar_t}} &
wide literal\newline encoding &
\tcode{L"wide string"}\newline
\tcode{LR"w(wide raw string)w"} \\
\tcode{u8} &
\defnx{UTF-8 string literal}{literal!string!UTF-8} &
array of $n$\newline \tcode{\keyword{const} \keyword{char8_t}} &
UTF-8 &
\tcode{u8"UTF-8 string"}\newline
\tcode{u8R"x(UTF-8 raw string)x"} \\
\tcode{u} &
\defnx{UTF-16 string literal}{literal!string!UTF-16} &
array of $n$\newline \tcode{\keyword{const} \keyword{char16_t}} &
UTF-16 &
\tcode{u"UTF-16 string"}\newline
\tcode{uR"y(UTF-16 raw string)y"} \\
\tcode{U} &
\defnx{UTF-32 string literal}{literal!string!UTF-32} &
array of $n$\newline \tcode{\keyword{const} \keyword{char32_t}} &
UTF-32 &
\tcode{U"UTF-32 string"}\newline
\tcode{UR"z(UTF-32 raw string)z"} \\
\end{floattable}

\pnum
\indextext{literal!string!raw}%
A \grammarterm{string-literal} that has an \tcode{R}
\indextext{prefix!\idxcode{R}}%
in the prefix is a \defn{raw string literal}. The
\grammarterm{d-char-sequence} serves as a delimiter. The terminating
\grammarterm{d-char-sequence} of a \grammarterm{raw-string} is the same sequence of
characters as the initial \grammarterm{d-char-sequence}. A \grammarterm{d-char-sequence}
shall consist of at most 16 characters.

\pnum
\begin{note}
The characters \tcode{'('} and \tcode{')'} can appear in a
\grammarterm{raw-string}. Thus, \tcode{R"delimiter((a|b))delimiter"} is equivalent to
\tcode{"(a|b)"}.
\end{note}

\pnum
\begin{note}
A source-file new-line in a raw string literal results in a new-line in the
resulting execution string literal. Assuming no
whitespace at the beginning of lines in the following example, the assert will succeed:
\begin{codeblock}
const char* p = R"(a\
b
c)";
assert(std::strcmp(p, "a\\\nb\nc") == 0);
\end{codeblock}
\end{note}

\pnum
\begin{example}
The raw string
\begin{codeblock}
R"a(
)\
a"
)a"
\end{codeblock}
is equivalent to \tcode{"\textbackslash n)\textbackslash \textbackslash \textbackslash na\textbackslash"\textbackslash n"}. The raw string
\begin{codeblock}
R"(x = "\"y\"")"
\end{codeblock}
is equivalent to \tcode{"x = \textbackslash "\textbackslash\textbackslash\textbackslash "y\textbackslash\textbackslash\textbackslash "\textbackslash ""}.
\end{example}

\pnum
\indextext{literal!narrow-character}%
Ordinary string literals and UTF-8 string literals are
also referred to as \defnx{narrow string literals}{literal!string!narrow}.

\pnum
\indextext{concatenation!string}%
The \grammarterm{string-literal}{s} in
any sequence of adjacent \grammarterm{string-literal}{s}
shall have at most one unique \grammarterm{encoding-prefix} among them.
The common \grammarterm{encoding-prefix} of the sequence is
that \grammarterm{encoding-prefix}, if any.
\begin{note}
A \grammarterm{string-literal}'s rawness has
no effect on the determination of the common \grammarterm{encoding-prefix}.
\end{note}

\pnum
In translation phase 6\iref{lex.phases},
adjacent \grammarterm{string-literal}s are concatenated.
The lexical structure and grouping of
the contents of the individual \grammarterm{string-literal}s is retained.
\begin{example}
\begin{codeblock}
"\xA" "B"
\end{codeblock}
represents
the code unit \tcode{'\textbackslash xA'} and the character \tcode{'B'}
after concatenation
(and not the single code unit \tcode{'\textbackslash xAB'}).
Similarly,
\begin{codeblock}
R"(\u00)" "41"
\end{codeblock}
represents six characters,
starting with a backslash and ending with the digit \tcode{1}
(and not the single character \tcode{'A'}
specified by a \grammarterm{universal-character-name}).

\tref{lex.string.concat} has some examples of valid concatenations.
\end{example}

\begin{floattable}{String literal concatenations}{lex.string.concat}
{lll|lll|lll}
\topline
\multicolumn{2}{|c}{Source} &
Means &
\multicolumn{2}{c}{Source} &
Means &
\multicolumn{2}{c}{Source} &
Means \\
\tcode{u"a"} & \tcode{u"b"} & \tcode{u"ab"} &
\tcode{U"a"} & \tcode{U"b"} & \tcode{U"ab"} &
\tcode{L"a"} & \tcode{L"b"} & \tcode{L"ab"} \\
\tcode{u"a"} & \tcode{"b"}  & \tcode{u"ab"} &
\tcode{U"a"} & \tcode{"b"}  & \tcode{U"ab"} &
\tcode{L"a"} & \tcode{"b"}  & \tcode{L"ab"} \\
\tcode{"a"}  & \tcode{u"b"} & \tcode{u"ab"} &
\tcode{"a"}  & \tcode{U"b"} & \tcode{U"ab"} &
\tcode{"a"}  & \tcode{L"b"} & \tcode{L"ab"} \\
\end{floattable}

\pnum
Evaluating a \grammarterm{string-literal} results in a string literal object
with static storage duration\iref{basic.stc}.
\begin{note}
String literal objects are potentially non-unique\iref{intro.object}.
Whether successive evaluations of a
\grammarterm{string-literal} yield the same or a different object is
unspecified.
\end{note}
\begin{note}
\indextext{literal!string!undefined change to}%
The effect of attempting to modify a string literal object is undefined.
\end{note}

\pnum
\indextext{\idxcode{0}!string terminator}%
\indextext{\idxcode{0}!null character|see {character, null}}%
String literal objects are initialized with
the sequence of code unit values
corresponding to the \grammarterm{string-literal}'s sequence of
\grammarterm{s-char}s (originally from non-raw string literals) and
\grammarterm{r-char}s (originally from raw string literals),
plus a terminating \unicode{0000}{null} character,
in order as follows:
\begin{itemize}
\item
The sequence of characters denoted by each contiguous sequence of
\grammarterm{basic-s-char}s,
\grammarterm{r-char}s,
\grammarterm{simple-escape-sequence}s\iref{lex.ccon}, and
\grammarterm{universal-character-name}s\iref{lex.charset}
is encoded to a code unit sequence
using the \grammarterm{string-literal}'s associated character encoding.
If a character lacks representation in the associated character encoding,
then the program is ill-formed.
\begin{note}
No character lacks representation in any Unicode encoding form.
\end{note}
When encoding a stateful character encoding,
implementations should encode the first such sequence
beginning with the initial encoding state and
encode subsequent sequences
beginning with the final encoding state of the prior sequence.
\begin{note}
The encoded code unit sequence can differ from
the sequence of code units that would be obtained by
encoding each character independently.
\end{note}
\item
Each \grammarterm{numeric-escape-sequence}\iref{lex.ccon}
contributes a single code unit with a value as follows:
\begin{itemize}
\item
Let $v$ be the integer value represented by
the octal number comprising
the sequence of \grammarterm{octal-digit}{s} in
an \grammarterm{octal-escape-sequence} or by
the hexadecimal number comprising
the sequence of \grammarterm{hexadecimal-digit}{s} in
a \grammarterm{hexadecimal-escape-sequence}.
\item
If $v$ does not exceed the range of representable values of
the \grammarterm{string-literal}'s array element type,
then the value is $v$.
\item
Otherwise,
if the \grammarterm{string-literal}'s \grammarterm{encoding-prefix}
is absent or \tcode{L}, and
$v$ does not exceed the range of representable values of
the corresponding unsigned type for the underlying type of
the \grammarterm{string-literal}'s array element type,
then the value is the unique value of
the \grammarterm{string-literal}'s array element type \tcode{T}
that is congruent to $v$ modulo $2^N$, where $N$ is the width of \tcode{T}.
\item
Otherwise, the program is ill-formed.
\end{itemize}
When encoding a stateful character encoding,
these sequences should have no effect on encoding state.
\item
Each \grammarterm{conditional-escape-sequence}\iref{lex.ccon}
contributes an
\impldef{code unit sequence for \grammarterm{conditional-escape-sequence}}
code unit sequence.
When encoding a stateful character encoding,
it is
\impldef{effect of \grammarterm{conditional-escape-sequence} on encoding state}
what effect these sequences have on encoding state.
\end{itemize}

\rSec2[lex.name]{Identifiers}

\indextext{identifier|(}%
\begin{bnf}
\nontermdef{identifier}\br
    identifier-start\br
    identifier identifier-continue
\end{bnf}

\begin{bnf}
\nontermdef{identifier-start}\br
    nondigit\br
    \textnormal{an element of the translation character set with the Unicode property XID_Start}
\end{bnf}

\begin{bnf}
\nontermdef{identifier-continue}\br
    digit\br
    nondigit\br
    \textnormal{an element of the translation character set with the Unicode property XID_Continue}
\end{bnf}

\begin{bnf}
\nontermdef{nondigit} \textnormal{one of}\br
    \terminal{a b c d e f g h i j k l m}\br
    \terminal{n o p q r s t u v w x y z}\br
    \terminal{A B C D E F G H I J K L M}\br
    \terminal{N O P Q R S T U V W X Y Z _}
\end{bnf}

\begin{bnf}
\nontermdef{digit} \textnormal{one of}\br
    \terminal{0 1 2 3 4 5 6 7 8 9}
\end{bnf}

\pnum
\indextext{name!length of}%
\indextext{name}%
\begin{note}
The character properties XID_Start and XID_Continue are Derived Core Properties
as described by \UAX{44} of the Unicode Standard.
\begin{footnote}
On systems in which linkers cannot accept extended
characters, an encoding of the \grammarterm{universal-character-name} can be used in
forming valid external identifiers. For example, some otherwise unused
character or sequence of characters can be used to encode the
\tcode{\textbackslash u} in a \grammarterm{universal-character-name}. Extended
characters can produce a long external identifier, but \Cpp{} does not
place a translation limit on significant characters for external
identifiers.
\end{footnote}
\end{note}
The program is ill-formed
if an \grammarterm{identifier} does not conform to
Normalization Form C as specified in the Unicode Standard.
\begin{note}
Identifiers are case-sensitive.
\end{note}
\begin{note}
\ref{uaxid} compares the requirements of \UAX{31} of the Unicode Standard
with the \Cpp{} rules for identifiers.
\end{note}
\begin{note}
In translation phase 4,
\grammarterm{identifier} also includes
those \grammarterm{preprocessing-token}s\iref{lex.pptoken}
differentiated as keywords\iref{lex.key}
in the later translation phase 7\iref{lex.token}.
\end{note}

\pnum
\indextext{\idxcode{_}|see{character, underscore}}%
\indextext{character!underscore!in identifier}%
\indextext{reserved identifier}%
In addition, some identifiers
appearing as a \grammarterm{token} or \grammarterm{preprocessing-token}
are reserved for use by \Cpp{}
implementations and shall
not be used otherwise; no diagnostic is required.
\begin{itemize}
\item
Each identifier that contains a double underscore
\tcode{\unun}
\indextext{character!underscore}%
or begins with an underscore followed by
an uppercase letter,
other than those specified in this document
(for example, \xname{cplusplus}\iref{cpp.predefined}),
\indextext{uppercase}%
is reserved to the implementation for any use.
\item
Each identifier that begins with an underscore is
\indextext{character!underscore}%
reserved to the implementation for use as a name in the global namespace.%
\indextext{namespace!global}
\end{itemize}%
\indextext{identifier|)}


\rSec1[cpp]{Preprocessing directives}%
\indextext{preprocessing directive|(}

\indextext{compiler control line|see{preprocessing directive}}%
\indextext{control line|see{preprocessing directive}}%
\indextext{directive, preprocessing|see{preprocessing directive}}

\gramSec[gram.cpp]{Preprocessing directives}

\rSec2[cpp.pre]{Preamble}

\begin{bnf}
\nontermdef{preprocessing-file}\br
    \opt{group}\br
    module-file
\end{bnf}

\begin{bnf}
\nontermdef{module-file}\br
    \opt{pp-global-module-fragment} pp-module \opt{group} \opt{pp-private-module-fragment}
\end{bnf}

\begin{bnf}
\nontermdef{pp-global-module-fragment}\br
    \keyword{module} \terminal{;} new-line \opt{group}
\end{bnf}

\begin{bnf}
\nontermdef{pp-private-module-fragment}\br
    \keyword{module} \terminal{:} \keyword{private} \terminal{;} new-line \opt{group}
\end{bnf}

\begin{bnf}
\nontermdef{group}\br
    group-part\br
    group group-part
\end{bnf}

\begin{bnf}
\nontermdef{group-part}\br
    control-line\br
    if-section\br
    text-line\br
    \terminal{\#} conditionally-supported-directive
\end{bnf}

\begin{bnf}\obeyspaces
\nontermdef{control-line}\br
    \terminal{\# include} pp-tokens new-line\br
    pp-import\br
    \terminal{\# define } identifier replacement-list new-line\br
    \terminal{\# define } identifier lparen \opt{identifier-list} \terminal{)} replacement-list new-line\br
    \terminal{\# define } identifier lparen \terminal{... )} replacement-list new-line\br
    \terminal{\# define } identifier lparen identifier-list \terminal{, ... )} replacement-list new-line\br
    \terminal{\# undef  } identifier new-line\br
    \terminal{\# line   } pp-tokens new-line\br
    \terminal{\# error  } \opt{pp-tokens} new-line\br
    \terminal{\# warning} \opt{pp-tokens} new-line\br
    \terminal{\# pragma } \opt{pp-tokens} new-line\br
    \terminal{\# }new-line
\end{bnf}

\begin{bnf}
\nontermdef{if-section}\br
    if-group \opt{elif-groups} \opt{else-group} endif-line
\end{bnf}

\begin{bnf}\obeyspaces
\nontermdef{if-group}\br
    \terminal{\# if     } constant-expression new-line \opt{group}\br
    \terminal{\# ifdef  } identifier new-line \opt{group}\br
    \terminal{\# ifndef } identifier new-line \opt{group}
\end{bnf}

\begin{bnf}
\nontermdef{elif-groups}\br
    elif-group\br
    elif-groups elif-group
\end{bnf}

\begin{bnf}\obeyspaces
\nontermdef{elif-group}\br
    \terminal{\# elif    } constant-expression new-line \opt{group}\br
    \terminal{\# elifdef } identifier new-line \opt{group}\br
    \terminal{\# elifndef} identifier new-line \opt{group}
\end{bnf}

\begin{bnf}\obeyspaces
\nontermdef{else-group}\br
    \terminal{\# else   } new-line \opt{group}
\end{bnf}

\begin{bnf}\obeyspaces
\nontermdef{endif-line}\br
    \terminal{\# endif  } new-line
\end{bnf}

\begin{bnf}
\nontermdef{text-line}\br
    \opt{pp-tokens} new-line
\end{bnf}

\begin{bnf}
\nontermdef{conditionally-supported-directive}\br
    pp-tokens new-line
\end{bnf}

\begin{bnf}
\nontermdef{lparen}\br
    \descr{a \terminal{(} character not immediately preceded by whitespace}
\end{bnf}

\begin{bnf}
\nontermdef{identifier-list}\br
    identifier\br
    identifier-list \terminal{,} identifier
\end{bnf}

\begin{bnf}
\nontermdef{replacement-list}\br
    \opt{pp-tokens}
\end{bnf}

\begin{bnf}
\nontermdef{pp-tokens}\br
    preprocessing-token\br
    pp-tokens preprocessing-token
\end{bnf}

\begin{bnf}
\nontermdef{new-line}\br
    \descr{the new-line character}
\end{bnf}

\pnum
A \defn{preprocessing directive} consists of a sequence of preprocessing tokens
that satisfies the following constraints:
At the start of translation phase 4,
the first token in the sequence,
referred to as a \defnadj{directive-introducing}{token},
begins with the first character in the source file
(optionally after whitespace containing no new-line characters) or
follows whitespace containing at least one new-line character,
and is

\begin{itemize}
\item
a \tcode{\#} preprocessing token, or

\item
an \keyword{import} preprocessing token
immediately followed on the same logical source line by a
\grammarterm{header-name},
\tcode{<},
\grammarterm{identifier},
\grammarterm{string-literal}, or
\tcode{:}
preprocessing token, or

\item
a \keyword{module} preprocessing token
immediately followed on the same logical source line by an
\grammarterm{identifier},
\tcode{:}, or
\tcode{;}
preprocessing token, or

\item
an \keyword{export} preprocessing token
immediately followed on the same logical source line by
one of the two preceding forms.
\end{itemize}

The last token in the sequence is the first token within the sequence that
is immediately followed by whitespace containing a new-line character.
\begin{footnote}
Thus,
preprocessing directives are commonly called ``lines''.
These ``lines'' have no other syntactic significance,
as all whitespace is equivalent except in certain situations
during preprocessing (see the
\tcode{\#}
character string literal creation operator in~\ref{cpp.stringize}, for example).
\end{footnote}
\begin{note}
A new-line character ends the preprocessing directive even if it occurs
within what would otherwise be an invocation of a function-like macro.
\end{note}

\begin{example}
\begin{codeblock}
#                       // preprocessing directive
module ;                // preprocessing directive
export module leftpad;  // preprocessing directive
import <string>;        // preprocessing directive
export import "squee";  // preprocessing directive
import rightpad;        // preprocessing directive
import :part;           // preprocessing directive

module                  // not a preprocessing directive
;                       // not a preprocessing directive

export                  // not a preprocessing directive
import                  // not a preprocessing directive
foo;                    // not a preprocessing directive

export                  // not a preprocessing directive
import foo;             // preprocessing directive (ill-formed at phase 7)

import ::               // not a preprocessing directive
import ->               // not a preprocessing directive
\end{codeblock}
\end{example}

\pnum
A sequence of preprocessing tokens is only a \grammarterm{text-line}
if it does not begin with a directive-introducing token.
A sequence of preprocessing tokens is only a \grammarterm{conditionally-supported-directive}
if it does not begin with any of the directive names
appearing after a \tcode{\#} in the syntax.
A \grammarterm{conditionally-supported-directive} is
conditionally-supported with
\impldef{additional supported forms of preprocessing directive}
semantics.

\pnum
At the start of phase 4 of translation,
the \grammarterm{group} of a \grammarterm{pp-global-module-fragment} shall
contain neither a \grammarterm{text-line} nor a \grammarterm{pp-import}.

\pnum
When in a group that is skipped\iref{cpp.cond}, the directive
syntax is relaxed to allow any sequence of preprocessing tokens to occur between
the directive name and the following new-line character.

\pnum
The only whitespace characters that shall appear
between preprocessing tokens
within a preprocessing directive
(from just after the directive-introducing token
through just before the terminating new-line character)
are space and horizontal-tab
(including spaces that have replaced comments
or possibly other whitespace characters
in translation phase 3).

\pnum
The implementation can
process and skip sections of source files conditionally,
include other source files,
import macros from header units,
and replace macros.
These capabilities are called
\defn{preprocessing},
because conceptually they occur
before translation of the resulting translation unit.

\pnum
The preprocessing tokens within a preprocessing directive
are not subject to macro expansion unless otherwise stated.

\begin{example}
In:
\begin{codeblock}
#define EMPTY
EMPTY   #   include <file.h>
\end{codeblock}
the sequence of preprocessing tokens on the second line is \textit{not}
a preprocessing directive, because it does not begin with a \tcode{\#} at the start of
translation phase 4, even though it will do so after the macro \tcode{EMPTY}
has been replaced.
\end{example}

\rSec2[cpp.module]{Module directive}
\indextext{preprocessing directive!module}%

\begin{bnf}
\nontermdef{pp-module}\br
    \opt{\keyword{export}} \keyword{module} \opt{pp-tokens} \terminal{;} new-line
\end{bnf}

\pnum
A \grammarterm{pp-module} shall not
appear in a context where \tcode{module}
or (if it is the first token of the \grammarterm{pp-module}) \tcode{export}
is an identifier defined as an object-like macro.

\pnum
The \grammarterm{pp-tokens}, if any, of a \grammarterm{pp-module}
shall be of the form:
\begin{ncsimplebnf}
pp-module-name \opt{pp-module-partition} \opt{pp-tokens}
\end{ncsimplebnf}
where the \grammarterm{pp-tokens} (if any) shall not begin with
a \tcode{(} preprocessing token and
the grammar non-terminals are defined as:
\begin{ncbnf}
\nontermdef{pp-module-name}\br
    \opt{pp-module-name-qualifier} identifier
\end{ncbnf}
\begin{ncbnf}
\nontermdef{pp-module-partition}\br
    \terminal{:} \opt{pp-module-name-qualifier} identifier
\end{ncbnf}
\begin{ncbnf}
\nontermdef{pp-module-name-qualifier}\br
    identifier \terminal{.}\br
    pp-module-name-qualifier identifier \terminal{.}
\end{ncbnf}
No \grammarterm{identifier} in
the \grammarterm{pp-module-name} or \grammarterm{pp-module-partition}
shall currently be defined as an object-like macro.

\pnum
Any preprocessing tokens after the \tcode{module} preprocessing token
in the \tcode{module} directive are processed just as in normal text.
\begin{note}
Each identifier currently defined as a macro name
is replaced by its replacement list of preprocessing tokens.
\end{note}

\pnum
The \tcode{module} and \tcode{export} (if it exists) preprocessing tokens
are replaced by the \grammarterm{module-keyword} and
\grammarterm{export-keyword} preprocessing tokens respectively.
\begin{note}
This makes the line no longer a directive
so it is not removed at the end of phase 4.
\end{note}

\rSec2[cpp.import]{Header unit importation}
\indextext{header unit!preprocessing}%
\indextext{preprocessing directive!import}%
\indextext{macro!import|(}%

\begin{bnf}
\nontermdef{pp-import}\br
    \opt{\keyword{export}} \keyword{import} header-name \opt{pp-tokens} \terminal{;} new-line\br
    \opt{\keyword{export}} \keyword{import} header-name-tokens \opt{pp-tokens} \terminal{;} new-line\br
    \opt{\keyword{export}} \keyword{import} pp-tokens \terminal{;} new-line
\end{bnf}

\pnum
A \grammarterm{pp-import} shall not
appear in a context where \tcode{import}
or (if it is the first token of the \grammarterm{pp-import}) \tcode{export}
is an identifier defined as an object-like macro.

\pnum
The preprocessing tokens after the \tcode{import} preprocessing token
in the \tcode{import} \grammarterm{control-line}
are processed just as in normal text
(i.e., each identifier currently defined as a macro name
is replaced by its replacement list of preprocessing tokens).
\begin{note}
An \tcode{import} directive
matching the first two forms of a \grammarterm{pp-import}
instructs the preprocessor to import macros
from the header unit\iref{module.import}
denoted by the \grammarterm{header-name},
as described below.
\end{note}
\indextext{point of!macro import|see{macro, point of import}}%
The \defnx{point of macro import}{macro!point of import} for the
first two forms of \grammarterm{pp-import} is
immediately after the \grammarterm{new-line} terminating
the \grammarterm{pp-import}.
The last form of \grammarterm{pp-import} is only considered
if the first two forms did not match, and
does not have a point of macro import.

\pnum
If a \grammarterm{pp-import} is produced by source file inclusion
(including by the rewrite produced
when a \tcode{\#include} directive names an importable header)
while processing the \grammarterm{group} of a \grammarterm{module-file},
the program is ill-formed.

\pnum
In all three forms of \grammarterm{pp-import},
the \tcode{import} and \tcode{export} (if it exists) preprocessing tokens
are replaced by the \grammarterm{import-keyword} and
\grammarterm{export-keyword} preprocessing tokens respectively.
\begin{note}
This makes the line no longer a directive
so it is not removed at the end of phase 4.
\end{note}
Additionally, in the second form of \grammarterm{pp-import},
a \grammarterm{header-name} token is formed as if
the \grammarterm{header-name-tokens}
were the \grammarterm{pp-tokens} of a \tcode{\#include} directive.
The \grammarterm{header-name-tokens} are replaced by
the \grammarterm{header-name} token.
\begin{note}
This ensures that imports are treated consistently by
the preprocessor and later phases of translation.
\end{note}

\pnum
Each \tcode{\#define} directive encountered when preprocessing
each translation unit in a program results in a distinct
\defnx{macro definition}{macro!definition}.
\begin{note}
A predefined macro name\iref{cpp.predefined}
is not introduced by a \tcode{\#define} directive.
Implementations providing mechanisms to predefine additional macros
are encouraged to not treat them
as being introduced by a \tcode{\#define} directive.
\end{note}
Each macro definition has at most one point of definition in
each translation unit and at most one point of undefinition, as follows:
\begin{itemize}
\item
\indextext{point of!macro definition|see{macro, point of definition}}%
The \defnx{point of definition}{macro!point of definition}
of a macro definition within a translation unit $T$ is
\begin{itemize}
\item
if the \tcode{\#define} directive of the macro definition occurs within $T$,
the point at which that directive occurs, or otherwise,
\item
if the macro name is not lexically identical to a keyword\iref{lex.key}
or to the \grammarterm{identifier}{s} \tcode{module} or \tcode{import},
the first point of macro import in $T$ of a header unit
containing a point of definition for the macro definition, if any.
\end{itemize}
In the latter case, the macro is said
to be \defnx{imported}{macro!import} from the header unit.

\item
\indextext{point of!macro undefinition|see{macro, point of undefinition}}%
The \defnx{point of undefinition}{macro!point of undefinition}
of a macro definition within a translation unit
is the first point at which a \tcode{\#undef} directive naming the macro occurs
after its point of definition, or the first point
of macro import of a header unit containing a point of undefinition for the
macro definition, whichever (if any) occurs first.
\end{itemize}

\pnum
\indextext{active macro directive|see{macro, active}}%
A macro directive is \defnx{active}{macro!active} at a source location
if it has a point of definition in that translation unit preceding the location,
and does not have a point of undefinition in that translation unit preceding
the location.

\pnum
If a macro would be replaced or redefined, and multiple macro definitions
are active for that macro name, the active macro definitions shall all be
valid redefinitions of the same macro\iref{cpp.replace}.
\begin{note}
The relative order of \grammarterm{pp-import}{s} has no bearing on whether a
particular macro definition is active.
\end{note}

\pnum
\begin{example}
\begin{codeblocktu}{Importable header \tcode{"a.h"}}
#define X 123   // \#1
#define Y 45    // \#2
#define Z a     // \#3
#undef X        // point of undefinition of \#1 in \tcode{"a.h"}
\end{codeblocktu}

\begin{codeblocktu}{Importable header \tcode{"b.h"}}
import "a.h";   // point of definition of \#1, \#2, and \#3, point of undefinition of \#1 in \tcode{"b.h"}
#define X 456   // OK, \#1 is not active
#define Y 6     // error: \#2 is active
\end{codeblocktu}

\begin{codeblocktu}{Importable header \tcode{"c.h"}}
#define Y 45    // \#4
#define Z c     // \#5
\end{codeblocktu}

\begin{codeblocktu}{Importable header \tcode{"d.h"}}
import "c.h";   // point of definition of \#4 and \#5 in \tcode{"d.h"}
\end{codeblocktu}

\begin{codeblocktu}{Importable header \tcode{"e.h"}}
import "a.h";   // point of definition of \#1, \#2, and \#3, point of undefinition of \#1 in \tcode{"e.h"}
import "d.h";   // point of definition of \#4 and \#5 in \tcode{"e.h"}
int a = Y;      // OK, active macro definitions \#2 and \#4 are valid redefinitions
int c = Z;      // error: active macro definitions \#3 and \#5 are not valid redefinitions of \tcode{Z}
\end{codeblocktu}

\begin{codeblocktu}{Module unit \tcode{f}}
export module f;
export import "a.h";

int a = Y;      // OK
\end{codeblocktu}

\begin{codeblocktu}{Translation unit \tcode{\#1}}
import f;
int x = Y;      // error: \tcode{Y} is neither a defined macro nor a declared name
\end{codeblocktu}
\end{example}
\indextext{macro!import|)}

\rSec2[cpp.null]{Null directive}%
\indextext{preprocessing directive!null}

\pnum
A preprocessing directive of the form
\begin{ncsimplebnf}
\terminal{\#} new-line
\end{ncsimplebnf}
has no effect.

\rSec2[cpp.error]{Diagnostic directives}%
\indextext{preprocessing directive!error}%
\indextext{preprocessing directive!diagnostic}%
\indextext{preprocessing directive!warning}%
\indextext{\idxcode{\#error}|see{preprocessing directive, error}}

\pnum
A preprocessing directive of the form
\begin{ncsimplebnf}
\terminal{\# error} \opt{pp-tokens} new-line
\end{ncsimplebnf}
renders the program ill-formed.
A preprocessing directive of the form
\begin{ncsimplebnf}
\terminal{\# warning} \opt{pp-tokens} new-line
\end{ncsimplebnf}
requires the implementation to produce at least one diagnostic message
for the preprocessing translation unit\iref{intro.compliance.general}.
\recommended
Any diagnostic message caused by either of these directives
should include the specified sequence of preprocessing tokens.

\rSec2[cpp.line]{Line control}%
\indextext{preprocessing directive!line control}%
\indextext{\idxcode{\#line}|see{preprocessing directive, line control}}

\pnum
The \grammarterm{string-literal} of a
\tcode{\#line}
directive, if present,
shall be a character string literal.

\pnum
The
\defn{line number}
of the current source line is one greater than
the number of new-line characters read or introduced
in translation phase 1\iref{lex.phases}
while processing the source file to the current token.

\pnum
A preprocessing directive of the form
\begin{ncsimplebnf}
\terminal{\# line} digit-sequence new-line
\end{ncsimplebnf}
causes the implementation to behave as if
the following sequence of source lines begins with a
source line that has a line number as specified
by the digit sequence (interpreted as a decimal integer).
If the digit sequence specifies zero
or a number greater than 2147483647,
the behavior is undefined.

\pnum
A preprocessing directive of the form
\begin{ncsimplebnf}
\terminal{\# line} digit-sequence \terminal{"} \opt{s-char-sequence} \terminal{"} new-line
\end{ncsimplebnf}
sets the presumed line number similarly and changes the
presumed name of the source file to be the contents
of the character string literal.

\pnum
A preprocessing directive of the form
\begin{ncsimplebnf}
\terminal{\# line} pp-tokens new-line
\end{ncsimplebnf}
(that does not match one of the two previous forms)
is permitted.
The preprocessing tokens after
\tcode{line}
on the directive are processed just as in normal text
(each identifier currently defined as a macro name is replaced by its
replacement list of preprocessing tokens).
If the directive resulting after all replacements does not match
one of the two previous forms, the behavior is undefined;
otherwise, the result is processed as appropriate.

\rSec2[cpp.pragma]{Pragma directive}%
\indextext{preprocessing directive!pragma}%
\indextext{\idxcode{\#pragma}|see{preprocessing directive, pragma}}

\pnum
A preprocessing directive of the form
\begin{ncsimplebnf}
\terminal{\# pragma} \opt{pp-tokens} new-line
\end{ncsimplebnf}
causes the implementation to behave
in an \impldef{\tcode{\#pragma}} manner.
The behavior may cause translation to fail or cause the translator or
the resulting program to behave in a non-conforming manner.
Any pragma that is not recognized by the implementation is ignored.

\rSec2[cpp.pragma.op]{Pragma operator}%
\indextext{macro!pragma operator}%
\indextext{operator!pragma|see{macro, pragma operator}}

\pnum
A unary operator expression of the form:
\begin{ncbnf}
\terminal{_Pragma} \terminal{(} string-literal \terminal{)}
\end{ncbnf}
is processed as follows: The \grammarterm{string-literal} is \defnx{destringized}{destringization}
by deleting the \tcode{L} prefix, if present, deleting the leading and trailing
double-quotes, replacing each escape sequence \tcode{\textbackslash"} by a double-quote, and
replacing each escape sequence \tcode{\textbackslash\textbackslash} by a single
backslash. The resulting sequence of characters is processed through translation phase 3
to produce preprocessing tokens that are executed as if they were the
\grammarterm{pp-tokens} in a pragma directive. The original four preprocessing
tokens in the unary operator expression are removed.

\pnum
\begin{example}
\begin{codeblock}
#pragma listing on "..\listing.dir"
\end{codeblock}
can also be expressed as:
\begin{codeblock}
_Pragma ( "listing on \"..\\listing.dir\"" )
\end{codeblock}
The latter form is processed in the same way whether it appears literally
as shown, or results from macro replacement, as in:
\begin{codeblock}
#define LISTING(x) PRAGMA(listing on #x)
#define PRAGMA(x) _Pragma(#x)

LISTING( ..\listing.dir )
\end{codeblock}
\end{example}
\indextext{preprocessing directive|)}

\rSec2[cpp.replace]{Macro replacement}%

\rSec3[cpp.replace.general]{General}%
\indextext{macro!replacement|(}%
\indextext{replacement!macro|see{macro, replacement}}%
\indextext{preprocessing directive!macro replacement|see{macro, replacement}}

\pnum
\indextext{macro!replacement list}%
Two replacement lists are identical if and only if
the preprocessing tokens in both have
the same number, ordering, spelling, and whitespace separation,
where all whitespace separations are considered identical.

\pnum
An identifier currently defined as an
\indextext{macro!object-like}%
object-like macro (see below) may be redefined by another
\tcode{\#define}
preprocessing directive provided that the second definition is an
object-like macro definition and the two replacement lists
are identical, otherwise the program is ill-formed.
Likewise, an identifier currently defined as a
\indextext{macro!function-like}%
function-like macro (see below) may be redefined by another
\tcode{\#define}
preprocessing directive provided that the second definition is a
function-like macro definition that has the same number and spelling
of parameters,
and the two replacement lists are identical,
otherwise the program is ill-formed.

\pnum
\begin{example}
The following sequence is valid:
\begin{codeblock}
#define OBJ_LIKE      (1-1)
#define OBJ_LIKE      @\tcode{/* whitespace */ (1-1) /* other */}@
#define FUNC_LIKE(a)   ( a )
#define FUNC_LIKE( a )(     @\tcode{/* note the whitespace */ \textbackslash}@
                a @\tcode{/* other stuff on this line}@
                  @\tcode{*/}@ )
\end{codeblock}
But the following redefinitions are invalid:
\begin{codeblock}
#define OBJ_LIKE    (0)         // different token sequence
#define OBJ_LIKE    (1 - 1)     // different whitespace
#define FUNC_LIKE(b) ( a )      // different parameter usage
#define FUNC_LIKE(b) ( b )      // different parameter spelling
\end{codeblock}
\end{example}

\pnum
\indextext{macro!replacement list}%
There shall be whitespace between the identifier and the replacement list
in the definition of an object-like macro.

\pnum
If the \grammarterm{identifier-list} in the macro definition does not end with
an ellipsis, the number of arguments (including those arguments consisting
of no preprocessing tokens)
in an invocation of a function-like macro shall
equal the number of parameters in the macro definition.
Otherwise, there shall be at least as many arguments in the invocation as there are
parameters in the macro definition (excluding the \tcode{...}). There
shall exist a
\tcode{)}
preprocessing token that terminates the invocation.

\pnum
\indextext{__va_args__@\mname{VA_ARGS}}%
\indextext{__va_opt__@\mname{VA_OPT}}%
The identifiers \mname{VA_ARGS} and \mname{VA_OPT}
shall occur only in the \grammarterm{replacement-list}
of a function-like macro that uses the ellipsis notation in the parameters.

\pnum
A parameter identifier in a function-like macro
shall be uniquely declared within its scope.

\pnum
The identifier immediately following the
\tcode{define}
is called the
\indextext{name!macro|see{macro, name}}%
\defnx{macro name}{macro!name}.
There is one name space for macro names.
Any whitespace characters preceding or following the
replacement list of preprocessing tokens are not considered
part of the replacement list for either form of macro.

\pnum
If a
\indextext{\#\#0 operator@\tcode{\#} operator}
\tcode{\#}
preprocessing token,
followed by an identifier,
occurs lexically
at the point at which a preprocessing directive can begin,
the identifier is not subject to macro replacement.

\pnum
A preprocessing directive of the form
\begin{ncsimplebnf}
\terminal{\# define} identifier replacement-list new-line
\indextext{\idxcode{\#define}}%
\end{ncsimplebnf}
defines an
\defnadj{object-like}{macro} that
causes each subsequent instance of the macro name
\begin{footnote}
Since, by macro-replacement time,
all \grammarterm{character-literal}s and \grammarterm{string-literal}s are preprocessing tokens,
not sequences possibly containing identifier-like subsequences
(see \ref{lex.phases}, translation phases),
they are never scanned for macro names or parameters.
\end{footnote}
to be replaced by the replacement list of preprocessing tokens
that constitute the remainder of the directive.
\begin{footnote}
An alternative token\iref{lex.operators} is not an identifier,
even when its spelling consists entirely of letters and underscores.
Therefore it is not possible to define a macro
whose name is the same as that of an alternative token.
\end{footnote}
The replacement list is then rescanned for more macro names as
specified below.

\pnum
\begin{example}
The simplest use of this facility is to define a ``manifest constant'',
as in
\begin{codeblock}
#define TABSIZE 100
int table[TABSIZE];
\end{codeblock}
\end{example}

\pnum
A preprocessing directive of the form
\begin{ncsimplebnf}
\terminal{\# define} identifier lparen \opt{identifier-list} \terminal{)} replacement-list new-line\br
\terminal{\# define} identifier lparen \terminal{...} \terminal{)} replacement-list new-line\br
\terminal{\# define} identifier lparen identifier-list \terminal{, ...} \terminal{)} replacement-list new-line
\end{ncsimplebnf}
defines a \defnadj{function-like}{macro}
with parameters, whose use is
similar syntactically to a function call.
The parameters
\indextext{parameter!macro}%
are specified by the optional list of identifiers.
Each subsequent instance of the function-like macro name followed by a
\tcode{(}
as the next preprocessing token
introduces the sequence of preprocessing tokens that is replaced
by the replacement list in the definition
(an invocation of the macro).
\indextext{invocation!macro}%
The replaced sequence of preprocessing tokens is terminated by the matching
\tcode{)}
preprocessing token, skipping intervening matched pairs of left and
right parenthesis preprocessing tokens.
Within the sequence of preprocessing tokens making up an invocation
of a function-like macro,
new-line is considered a normal whitespace character.

\pnum
\indextext{macro!function-like!arguments}%
The sequence of preprocessing tokens
bounded by the outside-most matching parentheses
forms the list of arguments for the function-like macro.
The individual arguments within the list
are separated by comma preprocessing tokens,
but comma preprocessing tokens between matching
inner parentheses do not separate arguments.
If there are sequences of preprocessing tokens within the list of
arguments that would otherwise act as preprocessing directives,
\begin{footnote}
A \grammarterm{conditionally-supported-directive} is a preprocessing directive regardless of whether the implementation supports it.
\end{footnote}
the behavior is undefined.

\pnum
\begin{example}
The following defines a function-like
macro whose value is the maximum of its arguments.
It has the disadvantages of evaluating one or the other of its arguments
a second time
(including
\indextext{side effects}%
side effects)
and generating more code than a function if invoked several times.
It also cannot have its address taken,
as it has none.

\begin{codeblock}
#define max(a, b) ((a) > (b) ? (a) : (b))
\end{codeblock}

The parentheses ensure that the arguments and
the resulting expression are bound properly.
\end{example}

\pnum
\indextext{macro!function-like!arguments}%
If there is a \tcode{...} immediately preceding the \tcode{)} in the
function-like macro
definition, then the trailing arguments (if any), including any separating comma preprocessing
tokens, are merged to form a single item: the \defn{variable arguments}. The number of
arguments so combined is such that, following merger, the number of arguments is
either equal to or
one more than the number of parameters in the macro definition (excluding the
\tcode{...}).

\rSec3[cpp.subst]{Argument substitution}%
\indextext{macro!argument substitution}%
\indextext{argument substitution|see{macro, argument substitution}}%

\indextext{__va_opt__@\mname{VA_OPT}}%
\begin{bnf}
\nontermdef{va-opt-replacement}\br
    \terminal{\mname{VA_OPT} (} \opt{pp-tokens} \terminal{)}
\end{bnf}

\pnum
After the arguments for the invocation of a function-like macro have
been identified, argument substitution takes place.
For each parameter in the replacement list that is neither
preceded by a \tcode{\#} or \tcode{\#\#} preprocessing token nor
followed by a \tcode{\#\#} preprocessing token, the preprocessing tokens
naming the parameter are replaced by a token sequence determined as follows:
\begin{itemize}
\item
  If the parameter is of the form \grammarterm{va-opt-replacement},
  the replacement preprocessing tokens are the
  preprocessing token sequence for the corresponding argument,
  as specified below.
\item
  Otherwise, the replacement preprocessing tokens are the
  preprocessing tokens of corresponding argument after all
  macros contained therein have been expanded. The argument's
  preprocessing tokens are completely macro replaced before
  being substituted as if they formed the rest of the preprocessing
  file with no other preprocessing tokens being available.
\end{itemize}
\begin{example}
\begin{codeblock}
#define LPAREN() (
#define G(Q) 42
#define F(R, X, ...)  __VA_OPT__(G R X) )
int x = F(LPAREN(), 0, <:-);    // replaced by \tcode{int x = 42;}
\end{codeblock}
\end{example}

\pnum
\indextext{__va_args__@\mname{VA_ARGS}}%
An identifier \mname{VA_ARGS} that occurs in the replacement list
shall be treated as if it were a parameter, and the variable arguments shall form
the preprocessing tokens used to replace it.

\pnum
\begin{example}
\begin{codeblock}
#define debug(...) fprintf(stderr, @\mname{VA_ARGS}@)
#define showlist(...) puts(#@\mname{VA_ARGS}@)
#define report(test, ...) ((test) ? puts(#test) : printf(@\mname{VA_ARGS}@))
debug("Flag");
debug("X = %d\n", x);
showlist(The first, second, and third items.);
report(x>y, "x is %d but y is %d", x, y);
\end{codeblock}
results in
\begin{codeblock}
fprintf(stderr, "Flag");
fprintf(stderr, "X = %d\n", x);
puts("The first, second, and third items.");
((x>y) ? puts("x>y") : printf("x is %d but y is %d", x, y));
\end{codeblock}
\end{example}

\pnum
\indextext{__va_opt__@\mname{VA_OPT}}%
The identifier \mname{VA_OPT}
shall always occur as part of the preprocessing token sequence
\grammarterm{va-opt-replacement};
its closing \tcode{)} is determined by skipping
intervening pairs of matching left and right parentheses
in its \grammarterm{pp-tokens}.
The \grammarterm{pp-tokens} of a \grammarterm{va-opt-replacement}
shall not contain \mname{VA_OPT}.
If the \grammarterm{pp-tokens} would be ill-formed
as the replacement list of the current function-like macro,
the program is ill-formed.
A \grammarterm{va-opt-replacement} is treated as if it were a parameter,
and the preprocessing token sequence for the corresponding
argument is defined as follows.
If the substitution of \mname{VA_ARGS} as neither an operand
of \tcode{\#} nor \tcode{\#\#} consists of no preprocessing tokens,
the argument consists of
a single placemarker preprocessing token\iref{cpp.concat,cpp.rescan}.
Otherwise, the argument consists of
the results of the expansion of the contained \grammarterm{pp-tokens}
as the replacement list of the current function-like macro
before removal of placemarker tokens, rescanning, and further replacement.
\begin{note}
The placemarker tokens are removed before stringization\iref{cpp.stringize},
and can be removed by rescanning and further replacement\iref{cpp.rescan}.
\end{note}
\begin{example}
\begin{codeblock}
#define F(...)           f(0 __VA_OPT__(,) __VA_ARGS__)
#define G(X, ...)        f(0, X __VA_OPT__(,) __VA_ARGS__)
#define SDEF(sname, ...) S sname __VA_OPT__(= { __VA_ARGS__ })
#define EMP

F(a, b, c)          // replaced by \tcode{f(0, a, b, c)}
F()                 // replaced by \tcode{f(0)}
F(EMP)              // replaced by \tcode{f(0)}

G(a, b, c)          // replaced by \tcode{f(0, a, b, c)}
G(a, )              // replaced by \tcode{f(0, a)}
G(a)                // replaced by \tcode{f(0, a)}

SDEF(foo);          // replaced by \tcode{S foo;}
SDEF(bar, 1, 2);    // replaced by \tcode{S bar = \{ 1, 2 \};}

#define H1(X, ...) X __VA_OPT__(##) __VA_ARGS__ // error: \tcode{\#\#} may not appear at
                                                // the beginning of a replacement list\iref{cpp.concat}

#define H2(X, Y, ...) __VA_OPT__(X ## Y,) __VA_ARGS__
H2(a, b, c, d)      // replaced by \tcode{ab, c, d}

#define H3(X, ...) #__VA_OPT__(X##X X##X)
H3(, 0)             // replaced by \tcode{""}

#define H4(X, ...) __VA_OPT__(a X ## X) ## b
H4(, 1)             // replaced by \tcode{a b}

#define H5A(...) __VA_OPT__()@\tcode{/**/}@__VA_OPT__()
#define H5B(X) a ## X ## b
#define H5C(X) H5B(X)
H5C(H5A())          // replaced by \tcode{ab}
\end{codeblock}
\end{example}

\rSec3[cpp.stringize]{The \tcode{\#} operator}%
\indextext{\#\#0 operator@\tcode{\#} operator}%
\indextext{stringize|see{\tcode{\#} operator}}

\pnum
Each
\tcode{\#}
preprocessing token in the replacement list for a function-like
macro shall be followed by a parameter as the next preprocessing
token in the replacement list.

\pnum
A \defn{character string literal} is a \grammarterm{string-literal} with no prefix.
If, in the replacement list, a parameter is immediately
preceded by a
\tcode{\#}
preprocessing token,
both are replaced by a single character string literal preprocessing token that
contains the spelling of the preprocessing token sequence for the
corresponding argument (excluding placemarker tokens).
Let the \defn{stringizing argument} be the preprocessing token sequence
for the corresponding argument with placemarker tokens removed.
Each occurrence of whitespace between the stringizing argument's preprocessing
tokens becomes a single space character in the character string literal.
Whitespace before the first preprocessing token and after the last
preprocessing token comprising the stringizing argument is deleted.
Otherwise, the original spelling of each preprocessing token in the
stringizing argument is retained in the character string literal,
except for special handling for producing the spelling of
\grammarterm{string-literal}s and \grammarterm{character-literal}s:
a
\tcode{\textbackslash}
character is inserted before each
\tcode{"}
and
\tcode{\textbackslash}
character of a \grammarterm{character-literal} or \grammarterm{string-literal}
(including the delimiting
\tcode{"}
characters).
If the replacement that results is not a valid character string literal,
the behavior is undefined. The character string literal corresponding to
an empty stringizing argument is \tcode{""}.
The order of evaluation of
\tcode{\#}
and
\tcode{\#\#}
operators is unspecified.

\rSec3[cpp.concat]{The \tcode{\#\#} operator}%
\indextext{\#\#1 operator@\tcode{\#\#} operator}%
\indextext{concatenation!macro argument|see{\tcode{\#\#} operator}}

\pnum
A
\tcode{\#\#}
preprocessing token shall not occur at the beginning or
at the end of a replacement list for either form
of macro definition.

\pnum
If, in the replacement list of a function-like macro, a parameter is
immediately preceded or followed by a
\tcode{\#\#}
preprocessing token, the parameter is replaced by the
corresponding argument's preprocessing token sequence; however, if an argument consists of no preprocessing tokens, the parameter is
replaced by a placemarker preprocessing token instead.
\begin{footnote}
Placemarker preprocessing tokens do not appear in the syntax
because they are temporary entities that exist only within translation phase 4.
\end{footnote}

\pnum
For both object-like and function-like macro invocations, before the
replacement list is reexamined for more macro names to replace,
each instance of a
\tcode{\#\#}
preprocessing token in the replacement list
(not from an argument) is deleted and the
preceding preprocessing token is concatenated
with the following preprocessing token.
Placemarker preprocessing tokens are handled specially: concatenation
of two placemarkers results in a single placemarker preprocessing token, and
concatenation of a placemarker with a non-placemarker preprocessing token results
in the non-placemarker preprocessing token.
\begin{note}
Concatenation can form
a \grammarterm{universal-character-name}\iref{lex.charset}.
\end{note}
If the result is not a valid preprocessing token,
the behavior is undefined.
The resulting token is available for further macro replacement.
The order of evaluation of
\tcode{\#\#}
operators is unspecified.

\pnum
\begin{example}
The sequence
\begin{codeblock}
#define str(s)      # s
#define xstr(s)     str(s)
#define debug(s, t) printf("x" # s "= %d, x" # t "= %s", @\textbackslash@
               x ## s, x ## t)
#define INCFILE(n)  vers ## n
#define glue(a, b)  a ## b
#define xglue(a, b) glue(a, b)
#define HIGHLOW     "hello"
#define LOW         LOW ", world"

debug(1, 2);
fputs(str(strncmp("abc@\textbackslash@0d", "abc", '@\textbackslash@4')        // this goes away
    == 0) str(: @\atsign\textbackslash@n), s);
#include xstr(INCFILE(2).h)
glue(HIGH, LOW);
xglue(HIGH, LOW)
\end{codeblock}
results in
\begin{codeblock}
printf("x" "1" "= %d, x" "2" "= %s", x1, x2);
fputs("strncmp(@\textbackslash@"abc@\textbackslash\textbackslash@0d@\textbackslash@", @\textbackslash@"abc@\textbackslash@", '@\textbackslash\textbackslash@4') == 0" ": @\atsign\textbackslash@n", s);
#include "vers2.h"      @\textrm{(\textit{after macro replacement, before file access})}@
"hello";
"hello" ", world"
\end{codeblock}
or, after concatenation of the character string literals,
\begin{codeblock}
printf("x1= %d, x2= %s", x1, x2);
fputs("strncmp(@\textbackslash@"abc@\textbackslash\textbackslash@0d@\textbackslash@", @\textbackslash@"abc@\textbackslash@", '@\textbackslash\textbackslash@4') == 0: @\atsign\textbackslash@n", s);
#include "vers2.h"      @\textrm{(\textit{after macro replacement, before file access})}@
"hello";
"hello, world"
\end{codeblock}

Space around the \tcode{\#} and \tcode{\#\#} tokens in the macro definition
is optional.
\end{example}

\pnum
\begin{example}
In the following fragment:

\begin{codeblock}
#define hash_hash # ## #
#define mkstr(a) # a
#define in_between(a) mkstr(a)
#define join(c, d) in_between(c hash_hash d)
char p[] = join(x, y);          // equivalent to \tcode{char p[] = "x \#\# y";}
\end{codeblock}

The expansion produces, at various stages:

\begin{codeblock}
join(x, y)
in_between(x hash_hash y)
in_between(x ## y)
mkstr(x ## y)
"x ## y"
\end{codeblock}

In other words, expanding \tcode{hash_hash} produces a new token,
consisting of two adjacent sharp signs, but this new token is not the
\tcode{\#\#} operator.
\end{example}

\pnum
\begin{example}
To illustrate the rules for placemarker preprocessing tokens, the sequence
\begin{codeblock}
#define t(x,y,z) x ## y ## z
int j[] = { t(1,2,3), t(,4,5), t(6,,7), t(8,9,),
  t(10,,), t(,11,), t(,,12), t(,,) };
\end{codeblock}
results in
\begin{codeblock}
int j[] = { 123, 45, 67, 89,
  10, 11, 12, };
\end{codeblock}
\end{example}

\rSec3[cpp.rescan]{Rescanning and further replacement}%
\indextext{macro!rescanning and replacement}%
\indextext{rescanning and replacement|see{macro, rescanning and replacement}}

\pnum
After all parameters in the replacement list have been substituted and \tcode{\#} and \tcode{\#\#} processing has taken
place, all placemarker preprocessing tokens are removed. Then
the resulting preprocessing token sequence is rescanned, along with all
subsequent preprocessing tokens of the source file, for more macro names
to replace.

\pnum
\begin{example}
The sequence
\begin{codeblock}
#define x       3
#define f(a)    f(x * (a))
#undef  x
#define x       2
#define g       f
#define z       z[0]
#define h       g(~
#define m(a)    a(w)
#define w       0,1
#define t(a)    a
#define p()     int
#define q(x)    x
#define r(x,y)  x ## y
#define str(x)  # x

f(y+1) + f(f(z)) % t(t(g)(0) + t)(1);
g(x+(3,4)-w) | h 5) & m
    (f)^m(m);
p() i[q()] = { q(1), r(2,3), r(4,), r(,5), r(,) };
char c[2][6] = { str(hello), str() };
\end{codeblock}
results in
\begin{codeblock}
f(2 * (y+1)) + f(2 * (f(2 * (z[0])))) % f(2 * (0)) + t(1);
f(2 * (2+(3,4)-0,1)) | f(2 * (~ 5)) & f(2 * (0,1))^m(0,1);
int i[] = { 1, 23, 4, 5, };
char c[2][6] = { "hello", "" };
\end{codeblock}
\end{example}

\pnum
If the name of the macro being replaced is found during this scan of
the replacement list
(not including the rest of the source file's preprocessing tokens),
it is not replaced.
Furthermore,
if any nested replacements encounter the name of the macro being replaced,
it is not replaced.
These nonreplaced macro name preprocessing tokens are no longer available
for further replacement even if they are later (re)examined in contexts
in which that macro name preprocessing token would otherwise have been
replaced.

\pnum
The resulting completely macro-replaced preprocessing token sequence
is not processed as a preprocessing directive even if it resembles one,
but all pragma unary operator expressions within it are then processed as
specified in~\ref{cpp.pragma.op} below.

\rSec3[cpp.scope]{Scope of macro definitions}%
\indextext{macro!scope of definition}%
\indextext{scope!macro definition|see{macro, scope of definition}}

\pnum
A macro definition lasts
(independent of block structure)
until a corresponding
\tcode{\#undef}
directive is encountered or
(if none is encountered)
until the end of the translation unit.
Macro definitions have no significance after translation phase 4.

\pnum
A preprocessing directive of the form
\begin{ncsimplebnf}
\terminal{\# undef} identifier new-line
\indextext{\idxcode{\#undef}}%
\end{ncsimplebnf}
causes the specified identifier no longer to be defined as a macro name.
It is ignored if the specified identifier is not currently defined as
a macro name.

\indextext{macro!replacement|)}

\rSec2[cpp.predefined]{Predefined macro names}
\indextext{macro!predefined}%
\indextext{name!predefined macro|see{macro, predefined}}

\pnum
The following macro names shall be defined by the implementation:

\begin{description}

\item
\indextext{\idxxname{cplusplus}}%
\xname{cplusplus}\\
The integer literal \tcode{\cppver}.
\begin{note}
Future revisions of this document will
replace the value of this macro with a greater value.
\end{note}

\item The names listed in \tref{cpp.predefined.ft}.\\
The macros defined in \tref{cpp.predefined.ft} shall be defined to
the corresponding integer literal.
\begin{note}
Future revisions of this document might replace
the values of these macros with greater values.
\end{note}

\item
\indextext{__date__@\mname{DATE}}%
\mname{DATE}\\
The date of translation of the source file:
a character string literal of the form
\tcode{"Mmm~dd~yyyy"},
where the names of the months are the same as those generated
by the
\tcode{asctime}
function,
and the first character of
\tcode{dd}
is a space character if the value is less than 10.
If the date of translation is not available,
an \impldef{text of \mname{DATE} when date of translation is not available} valid date
shall be supplied.

\item
\indextext{__file__@\mname{FILE}}%
\mname{FILE}\\
The presumed name of the current source file (a character string
literal).
\begin{footnote}
The presumed source file name can be changed by the \tcode{\#line} directive.
\end{footnote}

\item
\indextext{__line__@\mname{LINE}}%
\mname{LINE}\\
The presumed line number (within the current source file) of the current source line
(an integer literal).
\begin{footnote}
The presumed line number can be changed by the \tcode{\#line} directive.
\end{footnote}

\item
\indextext{__stdc_hosted__@\mname{STDC_HOSTED}}%
\indextext{implementation!hosted}%
\indextext{implementation!freestanding}%
\mname{STDC_HOSTED}\\
The integer literal \tcode{1}
if the implementation is a hosted implementation or
the integer literal \tcode{0}
if it is a freestanding implementation\iref{intro.compliance}.

\item
\indextext{__stdcpp_default_new_alignment__@\mname{STDCPP_DEFAULT_NEW_ALIGNMENT}}%
\mname{STDCPP_DEFAULT_NEW_ALIGNMENT}\\
An integer literal of type \tcode{std::size_t}
whose value is the alignment guaranteed
by a call to \tcode{operator new(std::size_t)}
or \tcode{operator new[](std::size_t)}.
\begin{note}
Larger alignments will be passed to
\tcode{operator new(std::size_t, std::align_val_t)}, etc.\iref{expr.new}.
\end{note}

\item
\indextext{__stdcpp_float16_t__@\mname{STDCPP_FLOAT16_T}}%
\mname{STDCPP_FLOAT16_T}\\
Defined as the integer literal \tcode{1}
if and only if the implementation supports
the \IsoFloatUndated{} floating-point interchange format binary16
as an extended floating-point type\iref{basic.extended.fp}.

\item
\indextext{__stdcpp_float32_t__@\mname{STDCPP_FLOAT32_T}}%
\mname{STDCPP_FLOAT32_T}\\
Defined as the integer literal \tcode{1}
if and only if the implementation supports
the \IsoFloatUndated{} floating-point interchange format binary32
as an extended floating-point type.

\item
\indextext{__stdcpp_float64_t__@\mname{STDCPP_FLOAT64_T}}%
\mname{STDCPP_FLOAT64_T}\\
Defined as the integer literal \tcode{1}
if and only if the implementation supports
the \IsoFloatUndated{} floating-point interchange format binary64
as an extended floating-point type.

\item
\indextext{__stdcpp_float128_t__@\mname{STDCPP_FLOAT128_T}}%
\mname{STDCPP_FLOAT128_T}\\
Defined as the integer literal \tcode{1}
if and only if the implementation supports
the \IsoFloatUndated{} floating-point interchange format binary128
as an extended floating-point type.

\item
\indextext{__stdcpp_bfloat16_t__@\mname{STDCPP_BFLOAT16_T}}%
\mname{STDCPP_BFLOAT16_T}\\
Defined as the integer literal \tcode{1}
if and only if the implementation supports an extended floating-point type
with the properties of the \grammarterm{typedef-name} \tcode{std::bfloat16_t}
as described in \ref{basic.extended.fp}.

\item
\indextext{__time__@\mname{TIME}}%
\mname{TIME}\\
The time of translation of the source file:
a character string literal of the form
\tcode{"hh:mm:ss"}
as in the time generated by the
\tcode{asctime}
function.
If the time of translation is not available,
an \impldef{text of \mname{TIME} when time of translation is not available} valid time shall be supplied.
\end{description}

\indextext{macro!feature-test}%
\indextext{feature-test macro|see{macro, feature-test}}%
\begin{LongTable}{Feature-test macros}{cpp.predefined.ft}{ll}
\\ \topline
\lhdr{Macro name} & \rhdr{Value} \\ \capsep
\endfirsthead
\continuedcaption \\
\hline
\lhdr{Name} & \rhdr{Value} \\ \capsep
\endhead
\defnxname{cpp_aggregate_bases}                   & \tcode{201603L} \\ \rowsep
\defnxname{cpp_aggregate_nsdmi}                   & \tcode{201304L} \\ \rowsep
\defnxname{cpp_aggregate_paren_init}              & \tcode{201902L} \\ \rowsep
\defnxname{cpp_alias_templates}                   & \tcode{200704L} \\ \rowsep
\defnxname{cpp_aligned_new}                       & \tcode{201606L} \\ \rowsep
\defnxname{cpp_attributes}                        & \tcode{200809L} \\ \rowsep
\defnxname{cpp_auto_cast}                         & \tcode{202110L} \\ \rowsep
\defnxname{cpp_binary_literals}                   & \tcode{201304L} \\ \rowsep
\defnxname{cpp_capture_star_this}                 & \tcode{201603L} \\ \rowsep
\defnxname{cpp_char8_t}                           & \tcode{202207L} \\ \rowsep
\defnxname{cpp_concepts}                          & \tcode{202002L} \\ \rowsep
\defnxname{cpp_conditional_explicit}              & \tcode{201806L} \\ \rowsep
\defnxname{cpp_constexpr}                         & \tcode{202406L} \\ \rowsep
\defnxname{cpp_constexpr_dynamic_alloc}           & \tcode{201907L} \\ \rowsep
\defnxname{cpp_constexpr_in_decltype}             & \tcode{201711L} \\ \rowsep
\defnxname{cpp_consteval}                         & \tcode{202211L} \\ \rowsep
\defnxname{cpp_constinit}                         & \tcode{201907L} \\ \rowsep
\defnxname{cpp_decltype}                          & \tcode{200707L} \\ \rowsep
\defnxname{cpp_decltype_auto}                     & \tcode{201304L} \\ \rowsep
\defnxname{cpp_deduction_guides}                  & \tcode{201907L} \\ \rowsep
\defnxname{cpp_delegating_constructors}           & \tcode{200604L} \\ \rowsep
\defnxname{cpp_deleted_function}                  & \tcode{202403L} \\ \rowsep
\defnxname{cpp_designated_initializers}           & \tcode{201707L} \\ \rowsep
\defnxname{cpp_enumerator_attributes}             & \tcode{201411L} \\ \rowsep
\defnxname{cpp_explicit_this_parameter}           & \tcode{202110L} \\ \rowsep
\defnxname{cpp_fold_expressions}                  & \tcode{201603L} \\ \rowsep
\defnxname{cpp_generic_lambdas}                   & \tcode{201707L} \\ \rowsep
\defnxname{cpp_guaranteed_copy_elision}           & \tcode{201606L} \\ \rowsep
\defnxname{cpp_hex_float}                         & \tcode{201603L} \\ \rowsep
\defnxname{cpp_if_consteval}                      & \tcode{202106L} \\ \rowsep
\defnxname{cpp_if_constexpr}                      & \tcode{201606L} \\ \rowsep
\defnxname{cpp_impl_coroutine}                    & \tcode{201902L} \\ \rowsep
\defnxname{cpp_impl_destroying_delete}            & \tcode{201806L} \\ \rowsep
\defnxname{cpp_impl_three_way_comparison}         & \tcode{201907L} \\ \rowsep
\defnxname{cpp_implicit_move}                     & \tcode{202207L} \\ \rowsep
\defnxname{cpp_inheriting_constructors}           & \tcode{201511L} \\ \rowsep
\defnxname{cpp_init_captures}                     & \tcode{201803L} \\ \rowsep
\defnxname{cpp_initializer_lists}                 & \tcode{200806L} \\ \rowsep
\defnxname{cpp_inline_variables}                  & \tcode{201606L} \\ \rowsep
\defnxname{cpp_lambdas}                           & \tcode{200907L} \\ \rowsep
\defnxname{cpp_modules}                           & \tcode{201907L} \\ \rowsep
\defnxname{cpp_multidimensional_subscript}        & \tcode{202211L} \\ \rowsep
\defnxname{cpp_named_character_escapes}           & \tcode{202207L} \\ \rowsep
\defnxname{cpp_namespace_attributes}              & \tcode{201411L} \\ \rowsep
\defnxname{cpp_noexcept_function_type}            & \tcode{201510L} \\ \rowsep
\defnxname{cpp_nontype_template_args}             & \tcode{201911L} \\ \rowsep
\defnxname{cpp_nontype_template_parameter_auto}   & \tcode{201606L} \\ \rowsep
\defnxname{cpp_nsdmi}                             & \tcode{200809L} \\ \rowsep
\defnxname{cpp_pack_indexing}                     & \tcode{202311L} \\ \rowsep
\defnxname{cpp_placeholder_variables}             & \tcode{202306L} \\ \rowsep
\defnxname{cpp_range_based_for}                   & \tcode{202211L} \\ \rowsep
\defnxname{cpp_raw_strings}                       & \tcode{200710L} \\ \rowsep
\defnxname{cpp_ref_qualifiers}                    & \tcode{200710L} \\ \rowsep
\defnxname{cpp_return_type_deduction}             & \tcode{201304L} \\ \rowsep
\defnxname{cpp_rvalue_references}                 & \tcode{200610L} \\ \rowsep
\defnxname{cpp_size_t_suffix}                     & \tcode{202011L} \\ \rowsep
\defnxname{cpp_sized_deallocation}                & \tcode{201309L} \\ \rowsep
\defnxname{cpp_static_assert}                     & \tcode{202306L} \\ \rowsep
\defnxname{cpp_static_call_operator}              & \tcode{202207L} \\ \rowsep
\defnxname{cpp_structured_bindings}               & \tcode{202403L} \\ \rowsep
\defnxname{cpp_template_template_args}            & \tcode{201611L} \\ \rowsep
\defnxname{cpp_threadsafe_static_init}            & \tcode{200806L} \\ \rowsep
\defnxname{cpp_unicode_characters}                & \tcode{200704L} \\ \rowsep
\defnxname{cpp_unicode_literals}                  & \tcode{200710L} \\ \rowsep
\defnxname{cpp_user_defined_literals}             & \tcode{200809L} \\ \rowsep
\defnxname{cpp_using_enum}                        & \tcode{201907L} \\ \rowsep
\defnxname{cpp_variable_templates}                & \tcode{201304L} \\ \rowsep
\defnxname{cpp_variadic_friend}                   & \tcode{202403L} \\ \rowsep
\defnxname{cpp_variadic_templates}                & \tcode{200704L} \\ \rowsep
\defnxname{cpp_variadic_using}                    & \tcode{201611L} \\
\end{LongTable}

\pnum
The following macro names are conditionally defined by the implementation:

\begin{description}
\item
\indextext{__stdc__@\mname{STDC}}%
\mname{STDC}\\
Whether \mname{STDC} is predefined and if so, what its value is,
are \impldef{definition and meaning of \mname{STDC}}.

\item
\indextext{__stdc_mb_might_neq_wc__@\mname{STDC_MB_MIGHT_NEQ_WC}}%
\mname{STDC_MB_MIGHT_NEQ_WC}\\
The integer literal \tcode{1}, intended to indicate that, in the encoding for
\keyword{wchar_t}, a member of the basic character set need not have a code value equal to
its value when used as the lone character in an ordinary character literal.

\item
\indextext{__stdc_version__@\mname{STDC_VERSION}}%
\mname{STDC_VERSION}\\
Whether \mname{STDC_VERSION} is predefined and if so, what its value is,
are \impldef{definition and meaning of \mname{STDC_VERSION}}.

\item
\indextext{__stdc_iso_10646__@\mname{STDC_ISO_10646}}%
\mname{STDC_ISO_10646}\\
An integer literal of the form \tcode{yyyymmL}
(for example, \tcode{199712L}).
Whether \mname{STDC_ISO_10646} is predefined and
if so, what its value is,
are \impldef{presence and value of \mname{STDC_ISO_10646}}.

\item
\indextext{__stdcpp_threads__@\mname{STDCPP_THREADS}}%
\mname{STDCPP_THREADS}\\
Defined, and has the value integer literal 1, if and only if a program
can have more than one thread of execution\iref{intro.multithread}.

\end{description}

\pnum
The values of the predefined macros
(except for
\mname{FILE}
and
\mname{LINE})
remain constant throughout the translation unit.

\pnum
If any of the pre-defined macro names in this subclause,
or the identifier
\tcode{defined},
is the subject of a
\tcode{\#define}
or a
\tcode{\#undef}
preprocessing directive,
the behavior is undefined.
Any other predefined macro names shall begin with a
leading underscore followed by an uppercase letter or a second
underscore.

\rSec2[cpp.cond]{Conditional inclusion}%
\indextext{preprocessing directive!conditional inclusion}%
\indextext{inclusion!conditional|see{preprocessing directive, conditional inclusion}}

\indextext{\idxcode{defined}}%
\begin{bnf}
\nontermdef{defined-macro-expression}\br
    \terminal{defined} identifier\br
    \terminal{defined (} identifier \terminal{)}
\end{bnf}

\begin{bnf}
\nontermdef{h-preprocessing-token}\br
    \textnormal{any \grammarterm{preprocessing-token} other than \terminal{>}}
\end{bnf}

\begin{bnf}
\nontermdef{h-pp-tokens}\br
    h-preprocessing-token\br
    h-pp-tokens h-preprocessing-token
\end{bnf}

\begin{bnf}
\nontermdef{header-name-tokens}\br
    string-literal\br
    \terminal{<} h-pp-tokens \terminal{>}
\end{bnf}

\indextext{\idxxname{has_include}}%
\begin{bnf}
\nontermdef{has-include-expression}\br
    \terminal{\xname{has_include}} \terminal{(} header-name \terminal{)}\br
    \terminal{\xname{has_include}} \terminal{(} header-name-tokens \terminal{)}
\end{bnf}

\indextext{\idxxname{has_cpp_attribute}}%
\begin{bnf}
\nontermdef{has-attribute-expression}\br
    \terminal{\xname{has_cpp_attribute} (} pp-tokens \terminal{)}
\end{bnf}

\pnum
The expression that controls conditional inclusion
shall be an integral constant expression except that
identifiers
(including those lexically identical to keywords)
are interpreted as described below
\begin{footnote}
Because the controlling constant expression is evaluated
during translation phase 4,
all identifiers either are or are not macro names ---
there simply are no keywords, enumeration constants, etc.
\end{footnote}
and it may contain zero or more \grammarterm{defined-macro-expression}{s} and/or
\grammarterm{has-include-expression}{s} and/or
\grammarterm{has-attribute-expression}{s} as unary operator expressions.

\pnum
A \grammarterm{defined-macro-expression} evaluates to \tcode{1}
if the identifier is currently defined
as a macro name
(that is, if it is predefined
or if it has one or more active macro definitions\iref{cpp.import},
for example because
it has been the subject of a
\tcode{\#define}
preprocessing directive
without an intervening
\tcode{\#undef}
directive with the same subject identifier), \tcode{0} if it is not.

\pnum
The second form of \grammarterm{has-include-expression}
is considered only if the first form does not match,
in which case the preprocessing tokens are processed just as in normal text.

\pnum
The header or source file identified by
the parenthesized preprocessing token sequence
in each contained \grammarterm{has-include-expression}
is searched for as if that preprocessing token sequence
were the \grammarterm{pp-tokens} in a \tcode{\#include} directive,
except that no further macro expansion is performed.
If such a directive would not satisfy the syntactic requirements
of a \tcode{\#include} directive, the program is ill-formed.
The \grammarterm{has-include-expression} evaluates
to \tcode{1} if the search for the source file succeeds, and
to \tcode{0} if the search fails.

\pnum
Each \grammarterm{has-attribute-expression} is replaced by
a non-zero \grammarterm{pp-number}
matching the form of an \grammarterm{integer-literal}
if the implementation supports an attribute
with the name specified by interpreting
the \grammarterm{pp-tokens}, after macro expansion,
as an \grammarterm{attribute-token},
and by \tcode{0} otherwise.
The program is ill-formed if the \grammarterm{pp-tokens}
do not match the form of an \grammarterm{attribute-token}.

\pnum
For an attribute specified in this document,
it is \impldef{value of \grammarterm{has-attribute-expression}
for standard attributes}
whether the value of the \grammarterm{has-attribute-expression}
is \tcode{0} or is given by \tref{cpp.cond.ha}.
For other attributes recognized by the implementation,
the value is
\impldef{value of \grammarterm{has-attribute-expression}
for non-standard attributes}.
\begin{note}
It is expected
that the availability of an attribute can be detected by any non-zero result.
\end{note}

\begin{floattable}{\xname{has_cpp_attribute} values}{cpp.cond.ha}
{ll}
\topline
\lhdr{Attribute} & \rhdr{Value} \\ \rowsep
\tcode{assume}                & \tcode{202207L} \\
\tcode{carries_dependency}    & \tcode{200809L} \\
\tcode{deprecated}            & \tcode{201309L} \\
\tcode{fallthrough}           & \tcode{201603L} \\
\tcode{likely}                & \tcode{201803L} \\
\tcode{maybe_unused}          & \tcode{201603L} \\
\tcode{no_unique_address}     & \tcode{201803L} \\
\tcode{nodiscard}             & \tcode{201907L} \\
\tcode{noreturn}              & \tcode{200809L} \\
\tcode{unlikely}              & \tcode{201803L} \\
\end{floattable}

\pnum
The
\tcode{\#ifdef}, \tcode{\#ifndef}, \tcode{\#elifdef}, and \tcode{\#elifndef}
directives, and
the \tcode{defined} conditional inclusion operator,
shall treat \xname{has_include} and \xname{has_cpp_attribute}
as if they were the names of defined macros.
The identifiers \xname{has_include} and \xname{has_cpp_attribute}
shall not appear in any context not mentioned in this subclause.

\pnum
Each preprocessing token that remains (in the list of preprocessing tokens that
will become the controlling expression)
after all macro replacements have occurred
shall be in the lexical form of a token\iref{lex.token}.

\pnum
Preprocessing directives of the forms
\begin{ncsimplebnf}\obeyspaces
\indextext{\idxcode{\#if}}%
\terminal{\# if     } constant-expression new-line \opt{group}\br
\indextext{\idxcode{\#elif}}%
\terminal{\# elif   } constant-expression new-line \opt{group}
\end{ncsimplebnf}
check whether the controlling constant expression evaluates to nonzero.

\pnum
Prior to evaluation,
macro invocations in the list of preprocessing tokens
that will become the controlling constant expression
are replaced
(except for those macro names modified by the
\tcode{defined}
unary operator),
just as in normal text.
If the token
\tcode{defined}
is generated as a result of this replacement process
or use of the
\tcode{defined}
unary operator does not match one of the two specified forms
prior to macro replacement,
the behavior is undefined.

\pnum
After all replacements due to macro expansion and
evaluations of
\grammarterm{defined-macro-expression}s,
\grammarterm{has-include-expression}s, and
\grammarterm{has-attribute-expression}s
have been performed,
all remaining identifiers and keywords,
except for
\tcode{true}
and
\tcode{false},
are replaced with the \grammarterm{pp-number}
\tcode{0},
and then each preprocessing token is converted into a token.
\begin{note}
An alternative
token\iref{lex.operators} is not an identifier,
even when its spelling consists entirely of letters and underscores.
Therefore it is not subject to this replacement.
\end{note}

\pnum
The resulting tokens comprise the controlling constant expression
which is evaluated according to the rules of~\ref{expr.const}
using arithmetic that has at least the ranges specified
in~\ref{support.limits}. For the purposes of this token conversion and evaluation
all signed and unsigned integer types
act as if they have the same representation as, respectively,
\tcode{intmax_t} or \tcode{uintmax_t}\iref{cstdint.syn}.
\begin{note}
Thus on an
implementation where \tcode{std::numeric_limits<int>::max()} is \tcode{0x7FFF}
and \tcode{std::numeric_limits<unsigned int>::max()} is \tcode{0xFFFF},
the integer literal \tcode{0x8000} is signed and positive within a \tcode{\#if}
expression even though it is unsigned in translation phase
7\iref{lex.phases}.
\end{note}
This includes interpreting \grammarterm{character-literal}s
according to the rules in \ref{lex.ccon}.
\begin{note}
The associated character encodings of literals are the same
in \tcode{\#if} and \tcode{\#elif} directives and in any expression.
\end{note}
Each subexpression with type
\tcode{bool}
is subjected to integral promotion before processing continues.

\pnum
Preprocessing directives of the forms
\begin{ncsimplebnf}\obeyspaces
\terminal{\# ifdef   } identifier new-line \opt{group}\br
\indextext{\idxcode{\#ifdef}}%
\terminal{\# ifndef  } identifier new-line \opt{group}\br
\indextext{\idxcode{\#ifndef}}%
\terminal{\# elifdef } identifier new-line \opt{group}\br
\indextext{\idxcode{\#elifdef}}%
\terminal{\# elifndef} identifier new-line \opt{group}
\indextext{\idxcode{\#elifndef}}%
\end{ncsimplebnf}
check whether the identifier is or is not currently defined as a macro name.
Their conditions are equivalent to
\tcode{\#if} \tcode{defined} \grammarterm{identifier},
\tcode{\#if} \tcode{!defined} \grammarterm{identifier},
\tcode{\#elif} \tcode{defined} \grammarterm{identifier}, and
\tcode{\#elif} \tcode{!defined} \grammarterm{identifier},
respectively.

\pnum
Each directive's condition is checked in order.
If it evaluates to false (zero),
the group that it controls is skipped:
directives are processed only through the name that determines
the directive in order to keep track of the level
of nested conditionals;
the rest of the directives' preprocessing tokens are ignored,
as are the other preprocessing tokens in the group.
Only the first group
whose control condition evaluates to true (nonzero) is processed;
any following groups are skipped and their controlling directives
are processed as if they were in a group that is skipped.
If none of the conditions evaluates to true,
and there is a
\tcode{\#else}
\indextext{\idxcode{\#else}}%
directive,
the group controlled by the
\tcode{\#else}
is processed; lacking a
\tcode{\#else}
directive, all the groups until the
\tcode{\#endif}
\indextext{\idxcode{\#endif}}%
are skipped.%
\begin{footnote}
As indicated by the syntax,
a preprocessing token cannot follow a
\tcode{\#else}
or
\tcode{\#endif}
directive before the terminating new-line character.
However,
comments can appear anywhere in a source file,
including within a preprocessing directive.
\end{footnote}

\pnum
\begin{example}
This demonstrates a way to include a library \tcode{optional} facility
only if it is available:

\begin{codeblock}
#if __has_include(<optional>)
#  include <optional>
#  if __cpp_lib_optional >= 201603
#    define have_optional 1
#  endif
#elif __has_include(<experimental/optional>)
#  include <experimental/optional>
#  if __cpp_lib_experimental_optional >= 201411
#    define have_optional 1
#    define experimental_optional 1
#  endif
#endif
#ifndef have_optional
#  define have_optional 0
#endif
\end{codeblock}
\end{example}

\pnum
\begin{example}
This demonstrates a way to use the attribute \tcode{[[acme::deprecated]]}
only if it is available.
\begin{codeblock}
#if __has_cpp_attribute(acme::deprecated)
#  define ATTR_DEPRECATED(msg) [[acme::deprecated(msg)]]
#else
#  define ATTR_DEPRECATED(msg) [[deprecated(msg)]]
#endif
ATTR_DEPRECATED("This function is deprecated") void anvil();
\end{codeblock}
\end{example}

\rSec2[cpp.include]{Source file inclusion}
\indextext{preprocessing directive!header inclusion}
\indextext{preprocessing directive!source-file inclusion}
\indextext{inclusion!source file|see{preprocessing directive, source-file inclusion}}%
\indextext{\idxcode{\#include}}%

\pnum
A
\tcode{\#include}
directive shall identify a header or source file
that can be processed by the implementation.

\pnum
A preprocessing directive of the form
\begin{ncsimplebnf}
\terminal{\# include <} h-char-sequence \terminal{>} new-line
\end{ncsimplebnf}
searches a sequence of
\impldef{sequence of places searched for a header}
places
for a header identified uniquely by the specified sequence
between the
\tcode{<}
and
\tcode{>}
delimiters,
and causes the replacement of that
directive by the entire contents of the header.
How the places are specified
or the header identified
is \impldef{search locations for \tcode{<>} header}.

\pnum
A preprocessing directive of the form
\begin{ncsimplebnf}
\terminal{\# include "} q-char-sequence \terminal{"} new-line
\end{ncsimplebnf}
causes the replacement of that
directive by the entire contents of the
source file identified by the specified sequence between the
\tcode{"}
delimiters.
The named source file is searched for in an
\impldef{manner of search for included source file}
manner.
If this search is not supported,
or if the search fails,
the directive is reprocessed as if it read
\begin{ncsimplebnf}
\terminal{\# include <} h-char-sequence \terminal{>} new-line
\end{ncsimplebnf}
with the identical contained sequence (including
\tcode{>}
characters, if any) from the original directive.

\pnum
A preprocessing directive of the form
\begin{ncsimplebnf}
\terminal{\# include} pp-tokens new-line
\end{ncsimplebnf}
(that does not match one of the two previous forms) is permitted.
The preprocessing tokens after
\tcode{include}
in the directive are processed just as in normal text
(i.e., each identifier currently defined as a macro name is replaced by its
replacement list of preprocessing tokens).
If the directive resulting after all replacements does not match
one of the two previous forms, the behavior is
undefined.
\begin{footnote}
Note that adjacent \grammarterm{string-literal}s are not concatenated into
a single \grammarterm{string-literal}
(see the translation phases in~\ref{lex.phases});
thus, an expansion that results in two \grammarterm{string-literal}s is an
invalid directive.
\end{footnote}
The method by which a sequence of preprocessing tokens between a
\tcode{<}
and a
\tcode{>}
preprocessing token pair or a pair of
\tcode{"}
characters is combined into a single header name
preprocessing token is \impldef{search locations for \tcode{""""} header}.

\pnum
The implementation shall provide unique mappings for
sequences consisting of one or more
\grammarterm{nondigit}{s} or \grammarterm{digit}{s}\iref{lex.name}
followed by a period
(\tcode{.})
and a single
\grammarterm{nondigit}.
The first character shall not be a \grammarterm{digit}.
The implementation may ignore distinctions of alphabetical case.

\pnum
A
\tcode{\#include}
preprocessing directive may appear
in a source file that has been read because of a
\tcode{\#include}
directive in another file,
up to an \impldef{nesting limit for \tcode{\#include} directives} nesting limit.

\pnum
If the header identified by the \grammarterm{header-name}
denotes an importable header\iref{module.import},
it is
\impldef{whether source file inclusion of importable header
is replaced with \tcode{import} directive}
whether the \tcode{\#include} preprocessing directive
is instead replaced by an \tcode{import} directive\iref{cpp.import} of the form
\begin{ncbnf}
\terminal{import} header-name \terminal{;} new-line
\end{ncbnf}

\pnum
\begin{note}
An implementation can provide a mechanism for making arbitrary
source files available to the \tcode{< >} search.
However, using the \tcode{< >} form for headers provided
with the implementation and the \tcode{" "} form for sources
outside the control of the implementation
achieves wider portability. For instance:

\begin{codeblock}
#include <stdio.h>
#include <unistd.h>
#include "usefullib.h"
#include "myprog.h"
\end{codeblock}

\end{note}

\pnum
\begin{example}
This illustrates macro-replaced
\tcode{\#include}
directives:

\begin{codeblock}
#if VERSION == 1
    #define INCFILE  "vers1.h"
#elif VERSION == 2
    #define INCFILE  "vers2.h"  // and so on
#else
    #define INCFILE  "versN.h"
#endif
#include INCFILE
\end{codeblock}
\end{example}

\rSec1[lex.tutokenize]{Translation unit tokenization}
\rSec2[lex.token]{Tokens}

\indextext{token|(}%
\begin{bnf}
\nontermdef{token}\br
    identifier\br
    keyword\br
    literal\br
    operator-or-punctuator
\end{bnf}

\pnum
\indextext{\idxgram{token}}%
There are five kinds of tokens: identifiers, keywords, literals,%
\begin{footnote}
Literals include strings and character and numeric literals.
\end{footnote}
operators, and other separators.
\indextext{whitespace}%
Blanks, horizontal and vertical tabs, newlines, formfeeds, and comments
(collectively, ``whitespace''), are ignored except
as they serve to separate tokens.
\begin{note}
Whitespace can separate otherwise adjacent identifiers, keywords, numeric
literals, and alternative tokens containing alphabetic characters.
\end{note}
\indextext{token|)}

\rSec2[lex.key]{Keywords}

\begin{bnf}
\nontermdef{keyword}\br
    \textnormal{any identifier listed in \tref{lex.key}}\br
    \grammarterm{import-keyword}\br
    \grammarterm{module-keyword}\br
    \grammarterm{export-keyword}
\end{bnf}

\pnum
\indextext{keyword|(}%
The identifiers shown in \tref{lex.key} are reserved for use
as keywords (that is, they are unconditionally treated as keywords in
phase 7) except in an \grammarterm{attribute-token}\iref{dcl.attr.grammar}.
\begin{note}
The \keyword{register} keyword is unused but
is reserved for future use.
\end{note}

\begin{multicolfloattable}{Keywords}{lex.key}
{lllll}
\keyword{alignas} \\
\keyword{alignof} \\
\keyword{asm} \\
\keyword{auto} \\
\keyword{bool} \\
\keyword{break} \\
\keyword{case} \\
\keyword{catch} \\
\keyword{char} \\
\keyword{char8_t} \\
\keyword{char16_t} \\
\keyword{char32_t} \\
\keyword{class} \\
\keyword{concept} \\
\keyword{const} \\
\keyword{consteval} \\
\keyword{constexpr} \\
\columnbreak
\keyword{constinit} \\
\keyword{const_cast} \\
\keyword{continue} \\
\keyword{co_await} \\
\keyword{co_return} \\
\keyword{co_yield} \\
\keyword{decltype} \\
\keyword{default} \\
\keyword{delete} \\
\keyword{do} \\
\keyword{double} \\
\keyword{dynamic_cast} \\
\keyword{else} \\
\keyword{enum} \\
\keyword{explicit} \\
\keyword{export} \\
\keyword{extern} \\
\columnbreak
\keyword{false} \\
\keyword{float} \\
\keyword{for} \\
\keyword{friend} \\
\keyword{goto} \\
\keyword{if} \\
\keyword{inline} \\
\keyword{int} \\
\keyword{long} \\
\keyword{mutable} \\
\keyword{namespace} \\
\keyword{new} \\
\keyword{noexcept} \\
\keyword{nullptr} \\
\keyword{operator} \\
\keyword{private} \\
\keyword{protected} \\
\columnbreak
\keyword{public} \\
\keyword{register} \\
\keyword{reinterpret_cast} \\
\keyword{requires} \\
\keyword{return} \\
\keyword{short} \\
\keyword{signed} \\
\keyword{sizeof} \\
\keyword{static} \\
\keyword{static_assert} \\
\keyword{static_cast} \\
\keyword{struct} \\
\keyword{switch} \\
\keyword{template} \\
\keyword{this} \\
\keyword{thread_local} \\
\keyword{throw} \\
\columnbreak
\keyword{true} \\
\keyword{try} \\
\keyword{typedef} \\
\keyword{typeid} \\
\keyword{typename} \\
\keyword{union} \\
\keyword{unsigned} \\
\keyword{using} \\
\keyword{virtual} \\
\keyword{void} \\
\keyword{volatile} \\
\keyword{wchar_t} \\
\keyword{while} \\
\end{multicolfloattable}

\pnum
\indextext{\idxcode{import}}%
\indextext{\idxcode{final}}%
\indextext{\idxcode{module}}%
\indextext{\idxcode{override}}%
The identifiers in \tref{lex.name.special} have a special meaning when
appearing in a certain context. When referred to in the grammar, these identifiers
are used explicitly rather than using the \grammarterm{identifier} grammar production.
Unless otherwise specified, any ambiguity as to whether a given
\grammarterm{identifier} has a special meaning is resolved to interpret the
token as a regular \grammarterm{identifier}.

\begin{multicolfloattable}{Identifiers with special meaning}{lex.name.special}
{llll}
\keyword{final}           \\
\columnbreak
\keyword{import}          \\
\columnbreak
\keyword{module}          \\
\columnbreak
\keyword{override}        \\
\end{multicolfloattable}

\pnum
\indextext{unit!translation}%
A translation unit shall not \tcode{\#define} or \tcode{\#undef}
names lexically identical
to keywords,
to the identifiers listed in \tref{lex.name.special}, or
to the \grammarterm{attribute-token}{s} described in~\ref{dcl.attr},
except that the names \tcode{likely} and \tcode{unlikely} may be
defined as function-like macros~\iref{cpp.replace}.

\pnum
Furthermore, the alternative representations shown in
\tref{lex.key.digraph} for certain operators and
punctuators\iref{lex.operators} are reserved and shall not be used
otherwise.

\begin{floattable}{Alternative representations}{lex.key.digraph}
{llllll}
\topline
\keyword{and}     &   \keyword{and_eq}  &   \keyword{bitand}  &   \keyword{bitor}   &   \keyword{compl}   &   \keyword{not} \\
\keyword{not_eq}  &   \keyword{or}      &   \keyword{or_eq}   &   \keyword{xor}     &   \keyword{xor_eq}  &       \\
\end{floattable}%
\indextext{keyword|)}%

\rSec2[lex.literal]{Literals}%
\indextext{literal|(}

\rSec3[lex.literal.kinds]{Kinds of literals}

\pnum
\indextext{constant}%
\indextext{literal!constant}%
There are several kinds of literals.
\begin{footnote}
The term ``literal'' generally designates, in this
document, those tokens that are called ``constants'' in C.
\end{footnote}

\begin{bnf}
\nontermdef{literal}\br
    integer-literal\br
    character-literal\br
    floating-point-literal\br
    string-literal\br
    boolean-literal\br
    pointer-literal\br
    user-defined-literal
\end{bnf}
\begin{note}
When appearing as an \grammarterm{expression},
a literal has a type and a value category\iref{expr.prim.literal}.
\end{note}

\rSec3[lex.icon]{Integer literals}

\indextext{literal!integer}%
\begin{bnf}
\nontermdef{integer-literal}\br
    binary-literal \opt{integer-suffix}\br
    octal-literal \opt{integer-suffix}\br
    decimal-literal \opt{integer-suffix}\br
    hexadecimal-literal \opt{integer-suffix}
\end{bnf}

\begin{bnf}
\nontermdef{binary-literal}\br
    \terminal{0b} binary-digit\br
    \terminal{0B} binary-digit\br
    binary-literal \opt{\terminal{'}} binary-digit
\end{bnf}

\begin{bnf}
\nontermdef{octal-literal}\br
    \terminal{0}\br
    octal-literal \opt{\terminal{'}} octal-digit
\end{bnf}

\begin{bnf}
\nontermdef{decimal-literal}\br
    nonzero-digit\br
    decimal-literal \opt{\terminal{'}} digit
\end{bnf}

\begin{bnf}
\nontermdef{hexadecimal-literal}\br
    hexadecimal-prefix hexadecimal-digit-sequence
\end{bnf}

\begin{bnf}
\nontermdef{binary-digit} \textnormal{one of}\br
    \terminal{0  1}
\end{bnf}

\begin{bnf}
\nontermdef{octal-digit} \textnormal{one of}\br
    \terminal{0  1  2  3  4  5  6  7}
\end{bnf}

\begin{bnf}
\nontermdef{nonzero-digit} \textnormal{one of}\br
    \terminal{1  2  3  4  5  6  7  8  9}
\end{bnf}

\begin{bnf}
\nontermdef{hexadecimal-prefix} \textnormal{one of}\br
    \terminal{0x  0X}
\end{bnf}

\begin{bnf}
\nontermdef{hexadecimal-digit-sequence}\br
    hexadecimal-digit\br
    hexadecimal-digit-sequence \opt{\terminal{'}} hexadecimal-digit
\end{bnf}

\begin{bnf}
\nontermdef{hexadecimal-digit} \textnormal{one of}\br
    \terminal{0  1  2  3  4  5  6  7  8  9}\br
    \terminal{a  b  c  d  e  f}\br
    \terminal{A  B  C  D  E  F}
\end{bnf}

\begin{bnf}
\nontermdef{integer-suffix}\br
    unsigned-suffix \opt{long-suffix} \br
    unsigned-suffix \opt{long-long-suffix} \br
    unsigned-suffix \opt{size-suffix} \br
    long-suffix \opt{unsigned-suffix} \br
    long-long-suffix \opt{unsigned-suffix} \br
    size-suffix \opt{unsigned-suffix}
\end{bnf}

\begin{bnf}
\nontermdef{unsigned-suffix} \textnormal{one of}\br
    \terminal{u  U}
\end{bnf}

\begin{bnf}
\nontermdef{long-suffix} \textnormal{one of}\br
    \terminal{l  L}
\end{bnf}

\begin{bnf}
\nontermdef{long-long-suffix} \textnormal{one of}\br
    \terminal{ll  LL}
\end{bnf}

\begin{bnf}
\nontermdef{size-suffix} \textnormal{one of}\br
   \terminal{z  Z}
\end{bnf}

\pnum
\indextext{literal!\idxcode{unsigned}}%
\indextext{literal!\idxcode{long}}%
\indextext{literal!base of integer}%
In an \grammarterm{integer-literal},
the sequence of
\grammarterm{binary-digit}s,
\grammarterm{octal-digit}s,
\grammarterm{digit}s, or
\grammarterm{hexadecimal-digit}s
is interpreted as a base $N$ integer as shown in table \tref{lex.icon.base};
the lexically first digit of the sequence of digits is the most significant.
\begin{note}
The prefix and any optional separating single quotes are ignored
when determining the value.
\end{note}

\begin{simpletypetable}
{Base of \grammarterm{integer-literal}{s}}
{lex.icon.base}
{lr}
\topline
\lhdr{Kind of \grammarterm{integer-literal}} & \rhdr{base $N$} \\ \capsep
\grammarterm{binary-literal} & 2 \\
\grammarterm{octal-literal} & 8 \\
\grammarterm{decimal-literal} & 10 \\
\grammarterm{hexadecimal-literal} & 16 \\
\end{simpletypetable}

\pnum
The \grammarterm{hexadecimal-digit}s
\tcode{a} through \tcode{f} and \tcode{A} through \tcode{F}
have decimal values ten through fifteen.
\begin{example}
The number twelve can be written \tcode{12}, \tcode{014},
\tcode{0XC}, or \tcode{0b1100}. The \grammarterm{integer-literal}s \tcode{1048576},
\tcode{1'048'576}, \tcode{0X100000}, \tcode{0x10'0000}, and
\tcode{0'004'000'000} all have the same value.
\end{example}

\pnum
\indextext{literal!\idxcode{long}}%
\indextext{literal!\idxcode{unsigned}}%
\indextext{literal!integer}%
\indextext{literal!type of integer}%
\indextext{suffix!\idxcode{L}}%
\indextext{suffix!\idxcode{U}}%
\indextext{suffix!\idxcode{l}}%
\indextext{suffix!\idxcode{u}}%
The type of an \grammarterm{integer-literal} is
the first type in the list in \tref{lex.icon.type}
corresponding to its optional \grammarterm{integer-suffix}
in which its value can be represented.

\begin{floattable}{Types of \grammarterm{integer-literal}s}{lex.icon.type}{l|l|l}
\topline
\lhdr{\grammarterm{integer-suffix}} & \chdr{\grammarterm{decimal-literal}}  & \rhdr{\grammarterm{integer-literal} other than \grammarterm{decimal-literal}}   \\  \capsep
none    &
  \tcode{int} &
  \tcode{int}\\
        &
  \tcode{long int} &
  \tcode{unsigned int}\\
        &
  \tcode{long long int} &
  \tcode{long int}\\
        &
        &
  \tcode{unsigned long int}\\
        &
        &
  \tcode{long long int}\\
        &
        &
  \tcode{unsigned long long int}\\\hline
\tcode{u} or \tcode{U}  &
  \tcode{unsigned int}  &
  \tcode{unsigned int}\\
                              &
  \tcode{unsigned long int}   &
  \tcode{unsigned long int}\\
                              &
  \tcode{unsigned long long int}   &
  \tcode{unsigned long long int}\\\hline
\tcode{l} or \tcode{L}  &
  \tcode{long int}  &
  \tcode{long int}\\
                              &
  \tcode{long long int}       &
  \tcode{unsigned long int}\\
                              &
                              &
  \tcode{long long int}\\
                              &
                              &
  \tcode{unsigned long long int}\\\hline
Both \tcode{u} or \tcode{U}   &
  \tcode{unsigned long int}  &
  \tcode{unsigned long int}\\
and \tcode{l} or \tcode{L}  &
  \tcode{unsigned long long int}  &
  \tcode{unsigned long long int}\\\hline
\tcode{ll} or \tcode{LL}  &
  \tcode{long long int}       &
  \tcode{long long int}\\
                              &
                              &
  \tcode{unsigned long long int}\\\hline
Both \tcode{u} or \tcode{U}   &
  \tcode{unsigned long long int}  &
  \tcode{unsigned long long int}\\
and \tcode{ll} or \tcode{LL}  &
                              &
                              \\\hline
\tcode{z} or \tcode{Z}                  &
  the signed integer type corresponding &
  the signed integer type \\
                                        &
  \qquad to \tcode{std::size_t}\iref{support.types.layout} &
  \qquad corresponding to \tcode{std::size_t} \\
                                        &
                                        &
  \tcode{std::size_t}\\\hline
Both \tcode{u} or \tcode{U}   &
  \tcode{std::size_t}         &
  \tcode{std::size_t}         \\
and \tcode{z} or \tcode{Z}  &
                              &
                              \\
\end{floattable}

\pnum
Except for \grammarterm{integer-literal}{s} containing
a \grammarterm{size-suffix},
if the value of an \grammarterm{integer-literal}
cannot be represented by any type in its list and
an extended integer type\iref{basic.fundamental} can represent its value,
it may have that extended integer type.
If all of the types in the list for the \grammarterm{integer-literal}
are signed,
the extended integer type is signed.
If all of the types in the list for the \grammarterm{integer-literal}
are unsigned,
the extended integer type is unsigned.
If the list contains both signed and unsigned types,
the extended integer type may be signed or unsigned.
If an \grammarterm{integer-literal}
cannot be represented by any of the allowed types,
the program is ill-formed.
\begin{note}
An \grammarterm{integer-literal} with a \tcode{z} or \tcode{Z} suffix
is ill-formed if it cannot be represented by \tcode{std::size_t}.
\end{note}

\rSec3[lex.fcon]{Floating-point literals}

\indextext{literal!floating-point}%
\begin{bnf}
\nontermdef{floating-point-literal}\br
    decimal-floating-point-literal\br
    hexadecimal-floating-point-literal
\end{bnf}

\begin{bnf}
\nontermdef{decimal-floating-point-literal}\br
    fractional-constant \opt{exponent-part} \opt{floating-point-suffix}\br
    digit-sequence exponent-part \opt{floating-point-suffix}
\end{bnf}

\begin{bnf}
\nontermdef{hexadecimal-floating-point-literal}\br
    hexadecimal-prefix hexadecimal-fractional-constant binary-exponent-part \opt{floating-point-suffix}\br
    hexadecimal-prefix hexadecimal-digit-sequence binary-exponent-part \opt{floating-point-suffix}
\end{bnf}

\begin{bnf}
\nontermdef{fractional-constant}\br
    \opt{digit-sequence} \terminal{.} digit-sequence\br
    digit-sequence \terminal{.}
\end{bnf}

\begin{bnf}
\nontermdef{hexadecimal-fractional-constant}\br
    \opt{hexadecimal-digit-sequence} \terminal{.} hexadecimal-digit-sequence\br
    hexadecimal-digit-sequence \terminal{.}
\end{bnf}

\begin{bnf}
\nontermdef{exponent-part}\br
    \terminal{e} \opt{sign} digit-sequence\br
    \terminal{E} \opt{sign} digit-sequence
\end{bnf}

\begin{bnf}
\nontermdef{binary-exponent-part}\br
    \terminal{p} \opt{sign} digit-sequence\br
    \terminal{P} \opt{sign} digit-sequence
\end{bnf}

\begin{bnf}
\nontermdef{sign} \textnormal{one of}\br
    \terminal{+  -}
\end{bnf}

\begin{bnf}
\nontermdef{digit-sequence}\br
    digit\br
    digit-sequence \opt{\terminal{'}} digit
\end{bnf}

\begin{bnf}
\nontermdef{floating-point-suffix} \textnormal{one of}\br
    \terminal{f  l  f16  f32  f64  f128  bf16  F  L  F16  F32  F64  F128  BF16}
\end{bnf}

\pnum
\indextext{literal!type of floating-point}%
\indextext{literal!\idxcode{float}}%
\indextext{suffix!\idxcode{F}}%
\indextext{suffix!\idxcode{f}}%
\indextext{suffix!\idxcode{L}}%
\indextext{suffix!\idxcode{l}}%
\indextext{literal!\idxcode{long double}}%
The type of
a \grammarterm{floating-point-literal}\iref{basic.fundamental,basic.extended.fp}
is determined by
its \grammarterm{floating-point-suffix} as specified in \tref{lex.fcon.type}.
\begin{note}
The floating-point suffixes
\tcode{f16}, \tcode{f32}, \tcode{f64}, \tcode{f128}, \tcode{bf16},
\tcode{F16}, \tcode{F32}, \tcode{F64}, \tcode{F128}, and \tcode{BF16}
are conditionally-supported. See \ref{basic.extended.fp}.
\end{note}
\begin{simpletypetable}
{Types of \grammarterm{floating-point-literal}{s}}
{lex.fcon.type}
{ll}
\topline
\lhdr{\grammarterm{floating-point-suffix}} & \rhdr{type} \\ \capsep
none & \keyword{double} \\
\tcode{f} or \tcode{F} & \keyword {float} \\
\tcode{l} or \tcode{L} & \keyword{long} \keyword{double} \\
\tcode{f16} or \tcode{F16} & \tcode{std::float16_t} \\
\tcode{f32} or \tcode{F32} & \tcode{std::float32_t} \\
\tcode{f64} or \tcode{F64} & \tcode{std::float64_t} \\
\tcode{f128} or \tcode{F128} & \tcode{std::float128_t} \\
\tcode{bf16} or \tcode{BF16} & \tcode{std::bfloat16_t} \\
\end{simpletypetable}

\pnum
\indextext{literal!floating-point}%
The \defn{significand} of a \grammarterm{floating-point-literal}
is the \grammarterm{fractional-constant} or \grammarterm{digit-sequence}
of a \grammarterm{decimal-floating-point-literal}
or the \grammarterm{hexadecimal-fractional-constant}
or \grammarterm{hexadecimal-digit-sequence}
of a \grammarterm{hexadecimal-floating-point-literal}.
In the significand,
the sequence of \grammarterm{digit}s or \grammarterm{hexadecimal-digit}s
and optional period are interpreted as a base $N$ real number $s$,
where $N$ is 10 for a \grammarterm{decimal-floating-point-literal} and
16 for a \grammarterm{hexadecimal-floating-point-literal}.
\begin{note}
Any optional separating single quotes are ignored when determining the value.
\end{note}
If an \grammarterm{exponent-part} or \grammarterm{binary-exponent-part}
is present,
the exponent $e$ of the \grammarterm{floating-point-literal}
is the result of interpreting
the sequence of an optional \grammarterm{sign} and the \grammarterm{digit}s
as a base 10 integer.
Otherwise, the exponent $e$ is 0.
The scaled value of the literal is
$s \times 10^e$ for a \grammarterm{decimal-floating-point-literal} and
$s \times 2^e$ for a \grammarterm{hexadecimal-floating-point-literal}.
\begin{example}
The \grammarterm{floating-point-literal}{s}
\tcode{49.625} and \tcode{0xC.68p+2} have the same value.
The \grammarterm{floating-point-literal}{s}
\tcode{1.602'176'565e-19} and \tcode{1.602176565e-19}
have the same value.
\end{example}

\pnum
If the scaled value is not in the range of representable
values for its type, the program is ill-formed.
Otherwise, the value of a \grammarterm{floating-point-literal}
is the scaled value if representable,
else the larger or smaller representable value nearest the scaled value,
chosen in an \impldef{choice of larger or smaller value of
\grammarterm{floating-point-literal}} manner.

\rSec3[lex.string.uneval]{Unevaluated strings}

\begin{bnf}
\nontermdef{unevaluated-string}\br
    string-literal
\end{bnf}

\pnum
An \grammarterm{unevaluated-string} shall have no \grammarterm{encoding-prefix}.

\pnum
Each \grammarterm{universal-character-name} and each \grammarterm{simple-escape-sequence} in an \grammarterm{unevaluated-string} is
replaced by the member of the translation character set it denotes.
An \grammarterm{unevaluated-string} that contains
a \grammarterm{numeric-escape-sequence} or
a \grammarterm{conditional-escape-sequence}
is ill-formed.

\pnum
An \grammarterm{unevaluated-string} is never evaluated and
its interpretation depends on the context in which it appears.

\rSec3[lex.bool]{Boolean literals}

\indextext{literal!boolean}%
\begin{bnf}
\nontermdef{boolean-literal}\br
    \terminal{false}\br
    \terminal{true}
\end{bnf}

\pnum
\indextext{Boolean literal}%
The Boolean literals are the keywords \tcode{false} and \tcode{true}.
Such literals have type \tcode{bool}.

\rSec3[lex.nullptr]{Pointer literals}

\indextext{literal!pointer}%
\begin{bnf}
\nontermdef{pointer-literal}\br
    \terminal{nullptr}
\end{bnf}

\pnum
The pointer literal is the keyword \keyword{nullptr}. It has type
\tcode{std::nullptr_t}.
\begin{note}
\tcode{std::nullptr_t} is a distinct type that is neither a pointer type nor a pointer-to-member type;
rather, a prvalue of this type is a null pointer constant and can be
converted to a null pointer value or null member pointer value. See~\ref{conv.ptr}
and~\ref{conv.mem}.
\end{note}

\rSec3[lex.ext]{User-defined literals}

\indextext{literal!user-defined}%
\begin{bnf}
\nontermdef{user-defined-literal}\br
    user-defined-integer-literal\br
    user-defined-floating-point-literal\br
    user-defined-string-literal\br
    user-defined-character-literal
\end{bnf}

\begin{bnf}
\nontermdef{user-defined-integer-literal}\br
    decimal-literal ud-suffix\br
    octal-literal ud-suffix\br
    hexadecimal-literal ud-suffix\br
    binary-literal ud-suffix
\end{bnf}

\begin{bnf}
\nontermdef{user-defined-floating-point-literal}\br
    fractional-constant \opt{exponent-part} ud-suffix\br
    digit-sequence exponent-part ud-suffix\br
    hexadecimal-prefix hexadecimal-fractional-constant binary-exponent-part ud-suffix\br
    hexadecimal-prefix hexadecimal-digit-sequence binary-exponent-part ud-suffix
\end{bnf}

\begin{bnf}
\nontermdef{user-defined-string-literal}\br
    string-literal ud-suffix
\end{bnf}

\begin{bnf}
\nontermdef{user-defined-character-literal}\br
    character-literal ud-suffix
\end{bnf}

\begin{bnf}
\nontermdef{ud-suffix}\br
    identifier
\end{bnf}

\pnum
If a token matches both \grammarterm{user-defined-literal} and another \grammarterm{literal} kind, it
is treated as the latter.
\begin{example}
\tcode{123_km}
is a \grammarterm{user-defined-literal}, but \tcode{12LL} is an
\grammarterm{integer-literal}.
\end{example}
The syntactic non-terminal preceding the \grammarterm{ud-suffix} in a
\grammarterm{user-defined-literal} is taken to be the longest sequence of
characters that could match that non-terminal.

\pnum
A \grammarterm{user-defined-literal} is treated as a call to a literal operator or
literal operator template\iref{over.literal}.
To determine the form of this call for
a given \grammarterm{user-defined-literal} \placeholder{L}
with \grammarterm{ud-suffix} \placeholder{X},
first let \placeholder{S} be the set of declarations
found by unqualified lookup for the \grammarterm{literal-operator-id}
whose literal suffix identifier is \placeholder{X}\iref{basic.lookup.unqual}.
\placeholder{S} shall not be empty.

\pnum
If \placeholder{L} is a \grammarterm{user-defined-integer-literal}, let \placeholder{n} be the literal
without its \grammarterm{ud-suffix}. If \placeholder{S} contains a literal operator with
parameter type \tcode{unsigned long long}, the literal \placeholder{L} is treated as a call of
the form
\begin{codeblock}
operator ""@\placeholder{X}@(@\placeholder{n}@ULL)
\end{codeblock}
Otherwise, \placeholder{S} shall contain a raw literal operator
or a numeric literal operator template\iref{over.literal} but not both.
If \placeholder{S} contains a raw literal operator,
the literal \placeholder{L} is treated as a call of the form
\begin{codeblock}
operator ""@\placeholder{X}@("@\placeholder{n}@")
\end{codeblock}
Otherwise (\placeholder{S} contains a numeric literal operator template),
\placeholder{L} is treated as a call of the form
\begin{codeblock}
operator ""@\placeholder{X}@<'@$c_1$@', '@$c_2$@', ... '@$c_k$@'>()
\end{codeblock}
where \placeholder{n} is the source character sequence $c_1c_2...c_k$.
\begin{note}
The sequence
$c_1c_2...c_k$ can only contain characters from the basic character set.
\end{note}

\pnum
If \placeholder{L} is a \grammarterm{user-defined-floating-point-literal}, let \placeholder{f} be the
literal without its \grammarterm{ud-suffix}. If \placeholder{S} contains a literal operator
with parameter type \tcode{long double}, the literal \placeholder{L} is treated as a call of
the form
\begin{codeblock}
operator ""@\placeholder{X}@(@\placeholder{f}@L)
\end{codeblock}
Otherwise, \placeholder{S} shall contain a raw literal operator
or a numeric literal operator template\iref{over.literal} but not both.
If \placeholder{S} contains a raw literal operator,
the \grammarterm{literal} \placeholder{L} is treated as a call of the form
\begin{codeblock}
operator ""@\placeholder{X}@("@\placeholder{f}@")
\end{codeblock}
Otherwise (\placeholder{S} contains a numeric literal operator template),
\placeholder{L} is treated as a call of the form
\begin{codeblock}
operator ""@\placeholder{X}@<'@$c_1$@', '@$c_2$@', ... '@$c_k$@'>()
\end{codeblock}
where \placeholder{f} is the source character sequence $c_1c_2...c_k$.
\begin{note}
The sequence
$c_1c_2...c_k$ can only contain characters from the basic character set.
\end{note}

\pnum
If \placeholder{L} is a \grammarterm{user-defined-string-literal},
let \placeholder{str} be the literal without its \grammarterm{ud-suffix}
and let \placeholder{len} be the number of code units in \placeholder{str}
(i.e., its length excluding the terminating null character).
If \placeholder{S} contains a literal operator template with
a non-type template parameter for which \placeholder{str} is
a well-formed \grammarterm{template-argument},
the literal \placeholder{L} is treated as a call of the form
\begin{codeblock}
operator ""@\placeholder{X}@<@\placeholder{str}{}@>()
\end{codeblock}
Otherwise, the literal \placeholder{L} is treated as a call of the form
\begin{codeblock}
operator ""@\placeholder{X}@(@\placeholder{str}{}@, @\placeholder{len}{}@)
\end{codeblock}

\pnum
If \placeholder{L} is a \grammarterm{user-defined-character-literal}, let \placeholder{ch} be the
literal without its \grammarterm{ud-suffix}.
\placeholder{S} shall contain a literal operator\iref{over.literal} whose only parameter has
the type of \placeholder{ch} and the
literal \placeholder{L} is treated as a call
of the form
\begin{codeblock}
operator ""@\placeholder{X}@(@\placeholder{ch}{}@)
\end{codeblock}

\pnum
\begin{example}
\begin{codeblock}
long double operator ""_w(long double);
std::string operator ""_w(const char16_t*, std::size_t);
unsigned operator ""_w(const char*);
int main() {
  1.2_w;            // calls \tcode{operator ""_w(1.2L)}
  u"one"_w;         // calls \tcode{operator ""_w(u"one", 3)}
  12_w;             // calls \tcode{operator ""_w("12")}
  "two"_w;          // error: no applicable literal operator
}
\end{codeblock}
\end{example}

\pnum
In translation phase 6\iref{lex.phases}, adjacent \grammarterm{string-literal}s are concatenated and
\grammarterm{user-defined-string-literal}{s} are considered \grammarterm{string-literal}s for that
purpose. During concatenation, \grammarterm{ud-suffix}{es} are removed and ignored and
the concatenation process occurs as described in~\ref{lex.string}. At the end of phase
6, if a \grammarterm{string-literal} is the result of a concatenation involving at least one
\grammarterm{user-defined-string-literal}, all the participating
\grammarterm{user-defined-string-literal}{s} shall have the same \grammarterm{ud-suffix}
and that suffix is applied to the result of the concatenation.

\pnum
\begin{example}
\begin{codeblock}
int main() {
  L"A" "B" "C"_x;   // OK, same as \tcode{L"ABC"_x}
  "P"_x "Q" "R"_y;  // error: two different \grammarterm{ud-suffix}{es}
}
\end{codeblock}
\end{example}
\indextext{literal|)}%

\rSec1[basic.link]{Names and linkage}%
\indextext{linkage|(}

\pnum
\indextext{type}%
\indextext{object}%
\indextext{storage class}%
\indextext{scope}%
\indextext{linkage}%
A \defn{name} is an \grammarterm{identifier}\iref{lex.name},
\grammarterm{operator-function-id}\iref{over.oper},
\grammarterm{literal-operator-id}\iref{over.literal}, or
\grammarterm{conversion-function-id}\iref{class.conv.fct}.

\pnum
Two names are \defnx{the same}{name!same} if
\begin{itemize}
\item they are \grammarterm{identifier}{s} composed of the same character sequence, or
\item they are \grammarterm{operator-function-id}{s} formed with the same operator, or
\item they are \grammarterm{literal-operator-id}{s} formed with the same
literal suffix identifier, or
\item they are \grammarterm{conversion-function-id}{s} formed with
equivalent\iref{temp.over.link} types.
\end{itemize}

\pnum
Every name is introduced by a \defn{declaration}, which is a
\begin{itemize}
\item
\grammarterm{name-declaration},
\grammarterm{block-declaration}, or
\grammarterm{member-declaration}\iref{dcl.pre,class.mem},
\item
\grammarterm{init-declarator}\iref{dcl.decl},
\item
\grammarterm{identifier}
in a structured binding declaration\iref{dcl.struct.bind},
\item
\grammarterm{init-capture}\iref{expr.prim.lambda.capture},
\item
\grammarterm{condition} with a \grammarterm{declarator}\iref{stmt.pre},
\item
\grammarterm{member-declarator}\iref{class.mem},
\item
\grammarterm{using-declarator}\iref{namespace.udecl},
\item
\grammarterm{parameter-declaration}\iref{dcl.fct},
\item
\grammarterm{type-parameter}\iref{temp.param},
\item
\grammarterm{elaborated-type-specifier}
that introduces a name\iref{dcl.type.elab},
\item
\grammarterm{class-specifier}\iref{class.pre},
\item
\grammarterm{enum-specifier} or
\grammarterm{enumerator-definition}\iref{dcl.enum},
\item
\grammarterm{exception-declaration}\iref{except.pre}, or
\item
implicit declaration of an injected-class-name\iref{class.pre}.
\end{itemize}
\begin{note}
The interpretation of a \grammarterm{for-range-declaration} produces
one or more of the above\iref{stmt.ranged}.
\end{note}

\pnum
An \defn{entity} is a value, object, reference, structured binding, function,
enumerator, type, class member, bit-field, template, template specialization,
namespace, or pack. An entity $E$ is denoted by the name (if any) that is
introduced by a declaration of $E$ or by a \grammarterm{typedef-name}
introduced by a declaration specifying $E$.

\pnum
Some names denote types or templates. In general, whenever a name is
encountered it is necessary to determine whether that name denotes one of these
entities before continuing to parse the program that contains it. The process
that determines this is called
\defnx{name lookup}{lookup!name}\iref{basic.lookup}.

\pnum
A \defn{variable} is introduced by the
declaration of
a reference other than a non-static data member or of
an object. The variable's name, if any, denotes the reference or object.

\pnum
\indextext{translation unit!name and}%
\indextext{linkage}%
A name used in more than one translation unit can potentially
refer to the same entity in these translation units depending on the
linkage\iref{basic.link} of the name specified in each
translation unit.

\pnum
A name can have
\defnadj{external}{linkage},
\defnadj{module}{linkage},
\defnadj{internal}{linkage}, or
\defnadj{no}{linkage},
as determined by the rules below.
\begin{note}
All declarations of an entity with a name with internal linkage
appear in the same translation unit.
All declarations of an entity with module linkage
are attached to the same module.
\end{note}

\pnum
\indextext{linkage!\idxcode{static} and}%
\indextext{\idxcode{static}!linkage of}%
\indextext{linkage!\idxcode{const} and}%
\indextext{\idxcode{const}!linkage of}%
\indextext{linkage!\idxcode{inline} and}%
\indextext{\idxcode{inline}!linkage of}%
The name of an entity
that belongs to a namespace scope\iref{basic.scope.namespace}
has internal linkage if it is the name of
\begin{itemize}
\item
  a variable, variable template, function, or function template that is
  explicitly declared \keyword{static}; or
\item
  a non-template variable of non-volatile const-qualified type, unless
  \begin{itemize}
  \item it is declared in the purview of a module interface unit
    (outside the \grammarterm{private-module-fragment}, if any) or
    module partition, or
  \item it is explicitly declared \keyword{extern}, or
  \item it is inline, or
  \item it was previously declared and the prior declaration did
  not have internal linkage; or
  \end{itemize}
\item
  a data member of an anonymous union.
\end{itemize}
\begin{note}
An instantiated variable template that has const-qualified type
can have external or module linkage, even if not declared \keyword{extern}.
\end{note}

\pnum
An unnamed namespace or a namespace declared directly or indirectly within an
unnamed namespace has internal linkage. All other namespaces have external linkage.
The name of an entity that belongs to a namespace scope
that has not been given internal linkage above
and that is the name of
\begin{itemize}
\item a variable; or
\item a function; or
\item
\indextext{class!linkage of}%
a named class\iref{class.pre}, or an unnamed class defined in a
typedef declaration in which the class has the typedef name for linkage
purposes\iref{dcl.typedef}; or
\item
\indextext{enumeration!linkage of}%
a named enumeration\iref{dcl.enum}, or an unnamed enumeration defined
in a typedef declaration in which the enumeration has the typedef name
for linkage purposes\iref{dcl.typedef}; or
\item an unnamed enumeration
that has an enumerator as a name for linkage purposes\iref{dcl.enum}; or
\item a template
\end{itemize}
has its linkage determined as follows:
\begin{itemize}
\item
\indextext{friend function!linkage of}%
if the entity is a function or function template
first declared in a friend declaration and
that declaration is a definition and
the enclosing class is defined within an \grammarterm{export-declaration},
the name has the same linkage, if any,
as the name of the enclosing class\iref{class.friend};
\item
otherwise,
\indextext{friend function!linkage of}%
if the entity is a function or function template
declared in a friend declaration and
a corresponding non-friend declaration is reachable,
%FIXME: Which declaration is "that prior declaration"?
%FIXME: "prior" with respect to what? And what about dependent lookup?
the name has the linkage determined from that prior declaration,
\item
otherwise,
if the enclosing namespace has internal linkage,
the name has internal linkage;
\item
otherwise,
if the declaration of the name is
attached to a named module\iref{module.unit}
and is not exported\iref{module.interface},
the name has module linkage;
\item
otherwise,
the name has external linkage.
\end{itemize}

\pnum
In addition,
a member function,
a static data member,
a named class or enumeration that inhabits a class scope, or
an unnamed class or enumeration defined in a typedef declaration
that inhabits a class scope
such that the class or enumeration
has the typedef name for linkage purposes\iref{dcl.typedef},
has the same linkage, if any, as the name of the class of which it is a member.

\pnum
\begin{example}
\begin{codeblock}
static void f();
extern "C" void h();
static int i = 0;               // \#1
void q() {
  extern void f();              // internal linkage
  extern void g();              // \tcode{::g}, external linkage
  extern void h();              // C language linkage
  int i;                        // \#2: \tcode{i} has no linkage
  {
    extern void f();            // internal linkage
    extern int i;               // \#3: internal linkage
  }
}
\end{codeblock}
Even though the declaration at line \#2 hides the declaration at line \#1,
the declaration at line \#3 still redeclares \#1 and receives internal linkage.
\end{example}

\pnum
\indextext{linkage!no}%
Names not covered by these rules have no linkage. Moreover, except as
noted, a name declared at block scope\iref{basic.scope.block} has no
linkage.

\pnum
Two declarations of entities declare the same entity
if, considering declarations of unnamed types to introduce their names
for linkage purposes, if any\iref{dcl.typedef,dcl.enum},
they correspond\iref{basic.scope.scope},
have the same target scope that is not a function or template parameter scope,
neither is a name-independent declaration,
and either
\begin{itemize}
\item
they appear in the same translation unit, or
\item
they both declare names with module linkage and are attached to the same module, or
\item
they both declare names with external linkage.
\end{itemize}
\begin{note}
There are other circumstances in which declarations declare the same entity%
\iref{dcl.link,temp.type,temp.spec.partial}.
\end{note}

\pnum
If a declaration $H$ that declares a name with internal linkage
precedes a declaration $D$ in another translation unit $U$ and
would declare the same entity as $D$ if it appeared in $U$,
the program is ill-formed.
\begin{note}
Such an $H$ can appear only in a header unit.
\end{note}

\pnum
If two declarations of an entity are
attached to different modules, the program is ill-formed;
no diagnostic is required if neither is reachable from the other.
\begin{example}
\begin{codeblocktu}{\tcode{"decls.h"}}
int f();            // \#1, attached to the global module
int g();            // \#2, attached to the global module
\end{codeblocktu}

\begin{codeblocktu}{Module interface of \tcode{M}}
module;
#include "decls.h"
export module M;
export using ::f;   // OK, does not declare an entity, exports \#1
int g();            // error: matches \#2, but attached to \tcode{M}
export int h();     // \#3
export int k();     // \#4
\end{codeblocktu}

\begin{codeblocktu}{Other translation unit}
import M;
static int h();     // error: matches \#3
int k();            // error: matches \#4
\end{codeblocktu}
\end{example}
As a consequence of these rules,
all declarations of an entity are attached to the same module;
the entity is said to be \defnx{attached}{attached!entity} to that module.

\pnum
\indextext{consistency!type declaration}%
\indextext{declaration!multiple}%
For any two declarations of an entity $E$:
\begin{itemize}
\item
If one declares $E$ to be a variable or function,
the other shall declare $E$ as one of the same type.
\item
If one declares $E$ to be an enumerator, the other shall do so.
\item
If one declares $E$ to be a namespace, the other shall do so.
\item
If one declares $E$ to be a type,
the other shall declare $E$ to be a type of the same kind\iref{dcl.type.elab}.
\item
If one declares $E$ to be a class template,
the other shall do so with the same kind and
an equivalent \grammarterm{template-head}\iref{temp.over.link}.
\begin{note}
The declarations can supply different default template arguments.
\end{note}
\item
If one declares $E$ to be a function template or
a (partial specialization of a) variable template,
the other shall declare $E$ to be one
with an equivalent \grammarterm{template-head} and type.
\item
If one declares $E$ to be an alias template,
the other shall declare $E$ to be one with
an equivalent \grammarterm{template-head} and \grammarterm{defining-type-id}.
\item
If one declares $E$ to be a concept, the other shall do so.
\end{itemize}
Types are compared after all adjustments of types (during which
typedefs\iref{dcl.typedef} are replaced by their definitions);
declarations for an array
object can specify array types that differ by the presence or absence of
a major array bound\iref{dcl.array}.
No diagnostic is required if neither declaration is reachable from the other.
\begin{example}
\begin{codeblock}
int f(int x, int x);    // error: different entities for \tcode{x}
void g();               // \#1
void g(int);            // OK, different entity from \#1
int g();                // error: same entity as \#1 with different type
void h();               // \#2
namespace h {}          // error: same entity as \#2, but not a function
\end{codeblock}
\end{example}

\pnum
\begin{note}
Linkage to non-\Cpp{} declarations can be achieved using a
\grammarterm{linkage-specification}\iref{dcl.link}.
\end{note}
\indextext{linkage|)}

\pnum
A declaration $D$ \defnx{names}{name} an entity $E$ if
\begin{itemize}
\item
$D$ contains a \grammarterm{lambda-expression} whose closure type is $E$,
\item
$E$ is not a function or function template and $D$ contains an
\grammarterm{id-expression},
\grammarterm{type-specifier},
\grammarterm{nested-name-specifier},
\grammarterm{template-name}, or
\grammarterm{concept-name}
denoting $E$, or
\item
$E$ is a function or function template and
$D$ contains an expression that names $E$\iref{basic.def.odr} or
an \grammarterm{id-expression}
that refers to a set of overloads that contains $E$.
\begin{note}
Non-dependent names in an instantiated declaration
do not refer to a set of overloads\iref{temp.res}.
\end{note}
\end{itemize}

\pnum
A declaration is an \defn{exposure}
if it either names a TU-local entity (defined below), ignoring
\begin{itemize}
\item
the \grammarterm{function-body}
for a non-inline function or function template
(but not the deduced return type
for a (possibly instantiated) definition of a function
with a declared return type that uses a placeholder type\iref{dcl.spec.auto}),
\item
the \grammarterm{initializer}
for a variable or variable template (but not the variable's type),
\item
friend declarations in a class definition, and
\item
any reference to a non-volatile const object or reference
with internal or no linkage initialized with a constant expression
that is not an odr-use\iref{term.odr.use},
\end{itemize}
or defines a constexpr variable initialized to a TU-local value (defined below).
\begin{note}
An inline function template can be an exposure even though
certain explicit specializations of it would be usable in other translation units.
\end{note}

\pnum
An entity is \defnx{TU-local}{TU-local!entity} if it is
\begin{itemize}
\item
a type, function, variable, or template that
\begin{itemize}
\item
has a name with internal linkage, or
\item
does not have a name with linkage and is declared,
or introduced by a \grammarterm{lambda-expression},
within the definition of a TU-local entity,
\end{itemize}
\item
a type with no name that is defined outside a
\grammarterm{class-specifier},
function body, or
\grammarterm{initializer}
or is introduced by a \grammarterm{defining-type-specifier}
that is used to declare only TU-local entities,
\item
a specialization of a TU-local template,
\item
a specialization of a template with any TU-local template argument, or
\item
a specialization of a template
whose (possibly instantiated) declaration is an exposure.
\begin{note}
A specialization can be produced by implicit or explicit instantiation.
\end{note}
\end{itemize}

\pnum
A value or object is \defnx{TU-local}{TU-local!value or object} if either
\begin{itemize}
\item
it is, or is a pointer to,
a TU-local function or the object associated with a TU-local variable, or
\item
it is an object of class or array type and
any of its subobjects or
any of the objects or functions
to which its non-static data members of reference type refer
is TU-local and is usable in constant expressions.
\end{itemize}

\pnum
If a (possibly instantiated) declaration of, or a deduction guide for,
a non-TU-local entity in a module interface unit
(outside the \grammarterm{private-module-fragment}, if any) or
module partition\iref{module.unit} is an exposure,
the program is ill-formed.
Such a declaration in any other context is deprecated\iref{depr.local}.

\pnum
If a declaration that appears in one translation unit
names a TU-local entity declared
in another translation unit that is not a header unit,
the program is ill-formed.
A declaration instantiated for a template specialization\iref{temp.spec}
appears at the point of instantiation of the specialization\iref{temp.point}.

\pnum
\begin{example}
\begin{codeblocktu}{Translation unit \#1}
export module A;
static void f() {}
inline void it() { f(); }           // error: is an exposure of \tcode{f}
static inline void its() { f(); }   // OK
template<int> void g() { its(); }   // OK
template void g<0>();

decltype(f) *fp;                    // error: \tcode{f} (though not its type) is TU-local
auto &fr = f;                       // OK
constexpr auto &fr2 = fr;           // error: is an exposure of \tcode{f}
constexpr static auto fp2 = fr;     // OK

struct S { void (&ref)(); } s{f};               // OK, value is TU-local
constexpr extern struct W { S &s; } wrap{s};    // OK, value is not TU-local

static auto x = []{f();};           // OK
auto x2 = x;                        // error: the closure type is TU-local
int y = ([]{f();}(),0);             // error: the closure type is not TU-local
int y2 = (x,0);                     // OK

namespace N {
  struct A {};
  void adl(A);
  static void adl(int);
}
void adl(double);

inline void h(auto x) { adl(x); }   // OK, but certain specializations are exposures
\end{codeblocktu}
\begin{codeblocktu}{Translation unit \#2}
module A;
void other() {
  g<0>();                           // OK, specialization is explicitly instantiated
  g<1>();                           // error: instantiation uses TU-local \tcode{its}
  h(N::A{});                        // error: overload set contains TU-local \tcode{N::adl(int)}
  h(0);                             // OK, calls \tcode{adl(double)}
  adl(N::A{});                      // OK; \tcode{N::adl(int)} not found, calls \tcode{N::adl(N::A)}
  fr();                             // OK, calls \tcode{f}
  constexpr auto ptr = fr;          // error: \tcode{fr} is not usable in constant expressions here
}
\end{codeblocktu}
\end{example}

\rSec1[basic.exec]{Program execution}

\rSec2[intro.execution]{Sequential execution}
\indextext{program execution|(}

\pnum
An instance of each object with automatic storage
duration\iref{basic.stc.auto} is associated with each entry into its
block. Such an object exists and retains its last-stored value during
the execution of the block and while the block is suspended (by a call
of a function, suspension of a coroutine\iref{expr.await}, or receipt of a signal).

\pnum
A \defn{constituent expression} is defined as follows:
\begin{itemize}
\item
The constituent expression of an expression is that expression.
\item
The constituent expression of a conversion is
the corresponding implicit function call, if any, or
the converted expression otherwise.
\item
The constituent expressions of a \grammarterm{braced-init-list} or
of a (possibly parenthesized) \grammarterm{expression-list}
are the constituent expressions of the elements of the respective list.
\item
The constituent expressions of a \grammarterm{brace-or-equal-initializer}
of the form \tcode{=}~\grammarterm{initializer-clause}
are the constituent expressions of the \grammarterm{initializer-clause}.
\end{itemize}
\begin{example}
\begin{codeblock}
struct A { int x; };
struct B { int y; struct A a; };
B b = { 5, { 1+1 } };
\end{codeblock}
The constituent expressions of the \grammarterm{initializer}
used for the initialization of \tcode{b} are \tcode{5} and \tcode{1+1}.
\end{example}

\pnum
The \defnx{immediate subexpressions}{immediate subexpression} of an expression $E$ are
\begin{itemize}
\item
the constituent expressions of $E$'s operands\iref{expr.prop},
\item
any function call that $E$ implicitly invokes,
\item
if $E$ is a \grammarterm{lambda-expression}\iref{expr.prim.lambda},
the initialization of the entities captured by copy and
the constituent expressions of the \grammarterm{initializer} of the \grammarterm{init-capture}{s},
\item
if $E$ is a function call\iref{expr.call} or implicitly invokes a function,
the constituent expressions of each default argument\iref{dcl.fct.default}
used in the call, or
\item
if $E$ creates an aggregate object\iref{dcl.init.aggr},
the constituent expressions of each default member initializer\iref{class.mem}
used in the initialization.
\end{itemize}

\pnum
A \defn{subexpression} of an expression $E$ is
an immediate subexpression of $E$ or
a subexpression of an immediate subexpression of $E$.
\begin{note}
Expressions appearing in the \grammarterm{compound-statement} of a \grammarterm{lambda-expression}
are not subexpressions of the \grammarterm{lambda-expression}.
\end{note}
The \defnadjx{potentially-evaluated}{subexpressions}{subexpression} of
an expression, conversion, or \grammarterm{initializer} $E$ are
\begin{itemize}
\item
the constituent expressions of $E$ and
\item
the subexpressions thereof that
are not subexpressions of a nested unevaluated operand\iref{term.unevaluated.operand}.
\end{itemize}

\pnum
A \defn{full-expression} is
\begin{itemize}
\item
an unevaluated operand\iref{expr.context},
\item
a \grammarterm{constant-expression}\iref{expr.const},
\item
an immediate invocation\iref{expr.const},
\item
an \grammarterm{init-declarator}\iref{dcl.decl}
(including such introduced by a structured binding\iref{dcl.struct.bind}) or
a \grammarterm{mem-initializer}\iref{class.base.init},
including the constituent expressions of the initializer,
\item
an invocation of a destructor generated at the end of the lifetime
of an object other than a temporary object\iref{class.temporary}
whose lifetime has not been extended, or
\item
an expression that is not a subexpression of another expression and
that is not otherwise part of a full-expression.
\end{itemize}
If a language construct is defined to produce an implicit call of a function,
a use of the language construct is considered to be an expression
for the purposes of this definition.
Conversions applied to the result of an expression in order to satisfy the requirements
of the language construct in which the expression appears
are also considered to be part of the full-expression.
For an initializer, performing the initialization of the entity
(including evaluating default member initializers of an aggregate)
is also considered part of the full-expression.
\begin{example}
\begin{codeblock}
struct S {
  S(int i): I(i) { }            // full-expression is initialization of \tcode{I}
  int& v() { return I; }
  ~S() noexcept(false) { }
private:
  int I;
};

S s1(1);                        // full-expression comprises call of \tcode{S::S(int)}
void f() {
  S s2 = 2;                     // full-expression comprises call of \tcode{S::S(int)}
  if (S(3).v())                 // full-expression includes lvalue-to-rvalue and \tcode{int} to \tcode{bool} conversions,
                                // performed before temporary is deleted at end of full-expression
  { }
  bool b = noexcept(S(4));      // exception specification of destructor of \tcode{S} considered for \keyword{noexcept}

  // full-expression is destruction of \tcode{s2} at end of block
}
struct B {
  B(S = S(0));
};
B b[2] = { B(), B() };          // full-expression is the entire initialization
                                // including the destruction of temporaries
\end{codeblock}
\end{example}

\pnum
\begin{note}
The evaluation of a full-expression can include the
evaluation of subexpressions that are not lexically part of the
full-expression. For example, subexpressions involved in evaluating
default arguments\iref{dcl.fct.default} are considered to
be created in the expression that calls the function, not the expression
that defines the default argument.
\end{note}

\pnum
\indextext{value computation|(}%
Reading an object designated by a \keyword{volatile}
glvalue\iref{basic.lval}, modifying an object, calling a library I/O
function, or calling a function that does any of those operations are
all
\defn{side effects}, which are changes in the state of the execution
environment. \defnx{Evaluation}{evaluation} of an expression (or a
subexpression) in general includes both value computations (including
determining the identity of an object for glvalue evaluation and fetching
a value previously assigned to an object for prvalue evaluation) and
initiation of side effects. When a call to a library I/O function
returns or an access through a volatile glvalue is evaluated, the side
effect is considered complete, even though some external actions implied
by the call (such as the I/O itself) or by the \keyword{volatile} access
may not have completed yet.

\pnum
\defnx{Sequenced before}{sequenced before} is an asymmetric, transitive, pair-wise relation between
evaluations executed by a single thread\iref{intro.multithread}, which induces
a partial order among those evaluations. Given any two evaluations \placeholder{A} and
\placeholder{B}, if \placeholder{A} is sequenced before \placeholder{B}
(or, equivalently, \placeholder{B} is \defn{sequenced after} \placeholder{A}),
then the execution of
\placeholder{A} shall precede the execution of \placeholder{B}. If \placeholder{A} is not sequenced
before \placeholder{B} and \placeholder{B} is not sequenced before \placeholder{A}, then \placeholder{A} and
\placeholder{B} are \defn{unsequenced}.
\begin{note}
The execution of unsequenced
evaluations can overlap.
\end{note}
Evaluations \placeholder{A} and \placeholder{B} are
\defn{indeterminately sequenced} when either \placeholder{A} is sequenced before
\placeholder{B} or \placeholder{B} is sequenced before \placeholder{A}, but it is unspecified which.
\begin{note}
Indeterminately sequenced evaluations cannot overlap, but either
can be executed first.
\end{note}
An expression \placeholder{X}
is said to be sequenced before
an expression \placeholder{Y} if
every value computation and every side effect
associated with the expression \placeholder{X}
is sequenced before
every value computation and every side effect
associated with the expression \placeholder{Y}.

\pnum
Every
\indextext{value computation}%
value computation and
\indextext{side effects}%
side effect associated with a full-expression is
sequenced before every value computation and side effect associated with the
next full-expression to be evaluated.
\begin{footnote}
As specified
in~\ref{class.temporary}, after a full-expression is evaluated, a sequence of
zero or more invocations of destructor functions for temporary objects takes
place, usually in reverse order of the construction of each temporary object.
\end{footnote}

\pnum
\indextext{evaluation!unspecified order of}%
Except where noted, evaluations of operands of individual operators and
of subexpressions of individual expressions are unsequenced.
\begin{note}
In an expression that is evaluated more than once during the execution
of a program, unsequenced and indeterminately sequenced evaluations of
its subexpressions need not be performed consistently in different
evaluations.
\end{note}
The value computations of the operands of an
operator are sequenced before the value computation of the result of the
operator. If a
\indextext{side effects}%
side effect on a memory location\iref{intro.memory} is unsequenced
relative to either another side effect on the same memory location or
a value computation using the value of any object in the same memory location,
and they are not potentially concurrent\iref{intro.multithread},
the behavior is undefined.
\begin{note}
The next subclause imposes similar, but more complex restrictions on
potentially concurrent computations.
\end{note}

\begin{example}
\begin{codeblock}
void g(int i) {
  i = 7, i++, i++;              // \tcode{i} becomes \tcode{9}

  i = i++ + 1;                  // the value of \tcode{i} is incremented
  i = i++ + i;                  // undefined behavior
  i = i + 1;                    // the value of \tcode{i} is incremented
}
\end{codeblock}
\end{example}

\pnum
When invoking a function (whether or not the function is inline),
every argument expression and
the postfix expression designating the called function
are sequenced before every expression or statement
in the body of the called function.
For each function invocation or
evaluation of an \grammarterm{await-expression} \placeholder{F},
each evaluation that does not occur within \placeholder{F} but
is evaluated on the same thread and as part of the same signal handler (if any)
is either sequenced before all evaluations that occur within \placeholder{F}
or sequenced after all evaluations that occur within \placeholder{F};
\begin{footnote}
In other words,
function executions do not interleave with each other.
\end{footnote}
if \placeholder{F} invokes or resumes a coroutine\iref{expr.await},
only evaluations
subsequent to the previous suspension (if any) and
prior to the next suspension (if any)
are considered to occur within \placeholder{F}.

Several contexts in \Cpp{} cause evaluation of a function call, even
though no corresponding function call syntax appears in the translation
unit.
\begin{example}
Evaluation of a \grammarterm{new-expression} invokes one or more allocation
and constructor functions; see~\ref{expr.new}. For another example,
invocation of a conversion function\iref{class.conv.fct} can arise in
contexts in which no function call syntax appears.
\end{example}
The sequencing constraints on the execution of the called function (as
described above) are features of the function calls as evaluated,
regardless of the syntax of the expression that calls the function.%
\indextext{value computation|)}%

\indextext{behavior!on receipt of signal}%
\indextext{signal}%
\pnum
If a signal handler is executed as a result of a call to the \tcode{std::raise}
function, then the execution of the handler is sequenced after the invocation
of the \tcode{std::raise} function and before its return.
\begin{note}
When a signal is received for another reason, the execution of the
signal handler is usually unsequenced with respect to the rest of the program.
\end{note}

\rSec2[intro.multithread]{Multi-threaded executions and data races}

\rSec3[intro.multithread.general]{General}

\pnum
\indextext{threads!multiple|(}%
\indextext{atomic!operation|(}%
A \defn{thread of execution} (also known as a \defn{thread}) is a single flow of
control within a program, including the initial invocation of a specific
top-level function, and recursively including every function invocation
subsequently executed by the thread.
\begin{note}
When one thread creates another,
the initial call to the top-level function of the new thread is executed by the
new thread, not by the creating thread.
\end{note}
Every thread in a program can
potentially access every object and function in a program.
\begin{footnote}
An object
with automatic or thread storage duration\iref{basic.stc} is associated with
one specific thread, and can be accessed by a different thread only indirectly
through a pointer or reference\iref{basic.compound}.
\end{footnote}
Under a hosted
implementation, a \Cpp{} program can have more than one thread running
concurrently. The execution of each thread proceeds as defined by the remainder
of this document. The execution of the entire program consists of an execution
of all of its threads.
\begin{note}
Usually the execution can be viewed as an
interleaving of all its threads. However, some kinds of atomic operations, for
example, allow executions inconsistent with a simple interleaving, as described
below.
\end{note}
\indextext{implementation!freestanding}%
Under a freestanding implementation, it is \impldef{number of
threads in a program under a freestanding implementation} whether a program can
have more than one thread of execution.

\pnum
For a signal handler that is not executed as a result of a call to the
\tcode{std::raise} function, it is unspecified which thread of execution
contains the signal handler invocation.

\rSec3[intro.races]{Data races}

\pnum
The value of an object visible to a thread $T$ at a particular point is the
initial value of the object, a value assigned to the object by $T$, or a
value assigned to the object by another thread, according to the rules below.
\begin{note}
In some cases, there might instead be undefined behavior. Much of this
subclause is motivated by the desire to support atomic operations with explicit
and detailed visibility constraints. However, it also implicitly supports a
simpler view for more restricted programs.
\end{note}

\pnum
Two expression evaluations \defn{conflict} if one of them modifies a memory
location\iref{intro.memory} and the other one reads or modifies the same
memory location.

\pnum
The library defines a number of atomic operations\iref{atomics} and
operations on mutexes\iref{thread} that are specially identified as
synchronization operations. These operations play a special role in making
assignments in one thread visible to another. A synchronization operation on one
or more memory locations is either a consume operation, an acquire operation, a
release operation, or both an acquire and release operation. A synchronization
operation without an associated memory location is a fence and can be either an
acquire fence, a release fence, or both an acquire and release fence. In
addition, there are relaxed atomic operations, which are not synchronization
operations, and atomic read-modify-write operations, which have special
characteristics.
\begin{note}
For example, a call that acquires a mutex will
perform an acquire operation on the locations comprising the mutex.
Correspondingly, a call that releases the same mutex will perform a release
operation on those same locations. Informally, performing a release operation on
$A$ forces prior
\indextext{side effects}%
side effects on other memory locations to become visible
to other threads that later perform a consume or an acquire operation on
$A$. ``Relaxed'' atomic operations are not synchronization operations even
though, like synchronization operations, they cannot contribute to data races.
\end{note}

\pnum
All modifications to a particular atomic object $M$ occur in some
particular total order, called the \defn{modification order} of $M$.
\begin{note}
There is a separate order for each
atomic object. There is no requirement that these can be combined into a single
total order for all objects. In general this will be impossible since different
threads can observe modifications to different objects in inconsistent orders.
\end{note}

\pnum
A \defn{release sequence} headed
by a release operation $A$ on an atomic object $M$
is a maximal contiguous sub-sequence of
\indextext{side effects}%
side effects in the modification order of $M$,
where the first operation is $A$, and
every subsequent operation is an atomic read-modify-write operation.

\pnum
Certain library calls \defn{synchronize with} other library calls performed by
another thread. For example, an atomic store-release synchronizes with a
load-acquire that takes its value from the store\iref{atomics.order}.
\begin{note}
Except in the specified cases, reading a later value does not
necessarily ensure visibility as described below. Such a requirement would
sometimes interfere with efficient implementation.
\end{note}
\begin{note}
The
specifications of the synchronization operations define when one reads the value
written by another. For atomic objects, the definition is clear. All operations
on a given mutex occur in a single total order. Each mutex acquisition ``reads
the value written'' by the last mutex release.
\end{note}

\pnum
An evaluation $A$ \defn{carries a dependency} to an evaluation $B$ if
\begin{itemize}
\item
the value of $A$ is used as an operand of $B$, unless:
\begin{itemize}
\item
$B$ is an invocation of any specialization of
\tcode{std::kill_dependency}\iref{atomics.order}, or
\item
$A$ is the left operand of a built-in logical \logop{and} (\tcode{\&\&},
see~\ref{expr.log.and}) or logical \logop{or} (\tcode{||}, see~\ref{expr.log.or})
operator, or
\item
$A$ is the left operand of a conditional (\tcode{?:}, see~\ref{expr.cond})
operator, or
\item
$A$ is the left operand of the built-in comma (\tcode{,})
operator\iref{expr.comma}; \end{itemize} or
\item
$A$ writes a scalar object or bit-field $M$, $B$ reads the value
written by $A$ from $M$, and $A$ is sequenced before $B$, or
\item
for some evaluation $X$, $A$ carries a dependency to $X$, and
$X$ carries a dependency to $B$.
\end{itemize}
\begin{note}
``Carries a dependency to'' is a subset of ``is sequenced before'',
and is similarly strictly intra-thread.
\end{note}

\pnum
An evaluation $A$ is \defn{dependency-ordered before} an evaluation
$B$ if
\begin{itemize}
\item
$A$ performs a release operation on an atomic object $M$, and, in
another thread, $B$ performs a consume operation on $M$ and reads
the value written by $A$, or

\item
for some evaluation $X$, $A$ is dependency-ordered before $X$ and
$X$ carries a dependency to $B$.

\end{itemize}
\begin{note}
The relation ``is dependency-ordered before'' is analogous to
``synchronizes with'', but uses release/consume in place of release/acquire.
\end{note}

\pnum
An evaluation $A$ \defn{inter-thread happens before} an evaluation $B$
if
\begin{itemize}
\item
  $A$ synchronizes with $B$, or
\item
  $A$ is dependency-ordered before $B$, or
\item
  for some evaluation $X$
  \begin{itemize}
  \item
    $A$ synchronizes with $X$ and $X$
    is sequenced before $B$, or
  \item
    $A$ is sequenced before $X$ and $X$
    inter-thread happens before $B$, or
  \item
    $A$ inter-thread happens before $X$ and $X$
    inter-thread happens before $B$.
  \end{itemize}
\end{itemize}
\begin{note}
The ``inter-thread happens before'' relation describes arbitrary
concatenations of ``sequenced before'', ``synchronizes with'' and
``dependency-ordered before'' relationships, with two exceptions. The first
exception is that a concatenation never ends with
``dependency-ordered before'' followed by ``sequenced before''. The reason for
this limitation is that a consume operation participating in a
``dependency-ordered before'' relationship provides ordering only with respect
to operations to which this consume operation actually carries a dependency. The
reason that this limitation applies only to the end of such a concatenation is
that any subsequent release operation will provide the required ordering for a
prior consume operation. The second exception is that a concatenation never
consist entirely of ``sequenced before''. The reasons for this
limitation are (1) to permit ``inter-thread happens before'' to be transitively
closed and (2) the ``happens before'' relation, defined below, provides for
relationships consisting entirely of ``sequenced before''.
\end{note}

\pnum
An evaluation $A$ \defn{happens before} an evaluation $B$
(or, equivalently, $B$ \defn{happens after} $A$) if
\begin{itemize}
\item $A$ is sequenced before $B$, or
\item $A$ inter-thread happens before $B$.
\end{itemize}
The implementation shall ensure that no program execution demonstrates a cycle
in the ``happens before'' relation.
\begin{note}
This cycle would otherwise be
possible only through the use of consume operations.
\end{note}

\pnum
An evaluation $A$ \defn{simply happens before} an evaluation $B$
if either
\begin{itemize}
\item $A$ is sequenced before $B$, or
\item $A$ synchronizes with $B$, or
\item $A$ simply happens before $X$ and
$X$ simply happens before $B$.
\end{itemize}
\begin{note}
In the absence of consume operations,
the happens before and simply happens before relations are identical.
\end{note}

\pnum
An evaluation $A$ \defn{strongly happens before}
an evaluation $D$ if, either
\begin{itemize}
\item $A$ is sequenced before $D$, or
\item $A$ synchronizes with $D$, and
both $A$ and $D$ are
sequentially consistent atomic operations\iref{atomics.order}, or
\item there are evaluations $B$ and $C$
such that $A$ is sequenced before $B$,
$B$ simply happens before $C$, and
$C$ is sequenced before $D$, or
\item there is an evaluation $B$ such that
$A$ strongly happens before $B$, and
$B$ strongly happens before $D$.
\end{itemize}
\begin{note}
Informally, if $A$ strongly happens before $B$,
then $A$ appears to be evaluated before $B$
in all contexts. Strongly happens before excludes consume operations.
\end{note}

\pnum
A \defnadjx{visible}{side effect}{side effects} $A$ on a scalar object or bit-field $M$
with respect to a value computation $B$ of $M$ satisfies the
conditions:
\begin{itemize}
\item $A$ happens before $B$ and
\item there is no other
\indextext{side effects}%
side effect $X$ to $M$ such that $A$
happens before $X$ and $X$ happens before $B$.
\end{itemize}

The value of a non-atomic scalar object or bit-field $M$, as determined by
evaluation $B$, is the value stored by the
\indextext{side effects!visible}%
visible side effect $A$.
\begin{note}
If there is ambiguity about which side effect to a
non-atomic object or bit-field is visible, then the behavior is either
unspecified or undefined.
\end{note}
\begin{note}
This states that operations on
ordinary objects are not visibly reordered. This is not actually detectable
without data races, but is needed to ensure that data races, as defined
below, and with suitable restrictions on the use of atomics, correspond to data
races in a simple interleaved (sequentially consistent) execution.
\end{note}

\pnum
The value of an
atomic object $M$, as determined by evaluation $B$, is the value
stored by some unspecified
side effect $A$ that modifies $M$, where $B$ does not happen
before $A$.
\begin{note}
The set of such side effects is also restricted by the rest of the rules
described here, and in particular, by the coherence requirements below.
\end{note}

\pnum
\indextext{coherence!write-write}%
If an operation $A$ that modifies an atomic object $M$ happens before
an operation $B$ that modifies $M$, then $A$ is earlier
than $B$ in the modification order of $M$.
\begin{note}
This requirement is known as write-write coherence.
\end{note}

\pnum
\indextext{coherence!read-read}%
If a
\indextext{value computation}%
value computation $A$ of an atomic object $M$ happens before a
value computation $B$ of $M$, and $A$ takes its value from a side
effect $X$ on $M$, then the value computed by $B$ is either
the value stored by $X$ or the value stored by a
\indextext{side effects}%
side effect $Y$ on $M$,
where $Y$ follows $X$ in the modification order of $M$.
\begin{note}
This requirement is known as read-read coherence.
\end{note}

\pnum
\indextext{coherence!read-write}%
If a
\indextext{value computation}%
value computation $A$ of an atomic object $M$ happens before an
operation $B$ that modifies $M$, then $A$ takes its value from a side
effect $X$ on $M$, where $X$ precedes $B$ in the
modification order of $M$.
\begin{note}
This requirement is known as
read-write coherence.
\end{note}

\pnum
\indextext{coherence!write-read}%
If a
\indextext{side effects}%
side effect $X$ on an atomic object $M$ happens before a value
computation $B$ of $M$, then the evaluation $B$ takes its
value from $X$ or from a
\indextext{side effects}%
side effect $Y$ that follows $X$ in the modification order of $M$.
\begin{note}
This requirement is known as write-read coherence.
\end{note}

\pnum
\begin{note}
The four preceding coherence requirements effectively disallow
compiler reordering of atomic operations to a single object, even if both
operations are relaxed loads. This effectively makes the cache coherence
guarantee provided by most hardware available to \Cpp{} atomic operations.
\end{note}

\pnum
\begin{note}
The value observed by a load of an atomic depends on the ``happens
before'' relation, which depends on the values observed by loads of atomics.
The intended reading is that there must exist an
association of atomic loads with modifications they observe that, together with
suitably chosen modification orders and the ``happens before'' relation derived
as described above, satisfy the resulting constraints as imposed here.
\end{note}

\pnum
Two actions are \defn{potentially concurrent} if
\begin{itemize}
\item they are performed by different threads, or
\item they are unsequenced, at least one is performed by a signal handler, and
they are not both performed by the same signal handler invocation.
\end{itemize}
The execution of a program contains a \defn{data race} if it contains two
potentially concurrent conflicting actions, at least one of which is not atomic,
and neither happens before the other,
except for the special case for signal handlers described below.
Any such data race results in undefined
behavior.
\begin{note}
It can be shown that programs that correctly use mutexes
and \tcode{memory_order::seq_cst} operations to prevent all data races and use no
other synchronization operations behave as if the operations executed by their
constituent threads were simply interleaved, with each
\indextext{value computation}%
value computation of an
object being taken from the last
\indextext{side effects}%
side effect on that object in that
interleaving. This is normally referred to as ``sequential consistency''.
However, this applies only to data-race-free programs, and data-race-free
programs cannot observe most program transformations that do not change
single-threaded program semantics. In fact, most single-threaded program
transformations remain possible, since any program that behaves
differently as a result has undefined behavior.
\end{note}

\pnum
Two accesses to the same object of type \tcode{\keyword{volatile} std::sig_atomic_t} do not
result in a data race if both occur in the same thread, even if one or more
occurs in a signal handler. For each signal handler invocation, evaluations
performed by the thread invoking a signal handler can be divided into two
groups $A$ and $B$, such that no evaluations in
$B$ happen before evaluations in $A$, and the
evaluations of such \tcode{\keyword{volatile} std::sig_atomic_t} objects take values as though
all evaluations in $A$ happened before the execution of the signal
handler and the execution of the signal handler happened before all evaluations
in $B$.

\pnum
\begin{note}
Compiler transformations that introduce assignments to a potentially
shared memory location that would not be modified by the abstract machine are
generally precluded by this document, since such an assignment might overwrite
another assignment by a different thread in cases in which an abstract machine
execution would not have encountered a data race. This includes implementations
of data member assignment that overwrite adjacent members in separate memory
locations. Reordering of atomic loads in cases in which the atomics in question
might alias is also generally precluded, since this could violate the coherence
rules.
\end{note}

\pnum
\begin{note}
It is possible that transformations that introduce a speculative read of a potentially
shared memory location do not preserve the semantics of the \Cpp{} program as
defined in this document, since they potentially introduce a data race. However,
they are typically valid in the context of an optimizing compiler that targets a
specific machine with well-defined semantics for data races. They would be
invalid for a hypothetical machine that is not tolerant of races or provides
hardware race detection.
\end{note}

\rSec3[intro.progress]{Forward progress}

\pnum
The implementation may assume that any thread will eventually do one of the
following:
\begin{itemize}
\item terminate,
\item invoke the function \tcode{std::this_thread::yield}\iref{thread.thread.this},
\item make a call to a library I/O function,
\item perform an access through a volatile glvalue,
\item perform a synchronization operation or an atomic operation, or
\item continue execution of a trivial infinite loop\iref{stmt.iter.general}.
\end{itemize}
\begin{note}
This is intended to allow compiler transformations
such as removal, merging, and reordering of empty loops,
even when termination cannot be proven.
An affordance is made for trivial infinite loops,
which cannot be removed nor reordered.
\end{note}

\pnum
Executions of atomic functions
that are either defined to be lock-free\iref{atomics.flag}
or indicated as lock-free\iref{atomics.lockfree}
are \defnx{lock-free executions}{lock-free execution}.
\begin{itemize}
\item
  If there is only one thread that is not blocked\iref{defns.block}
  in a standard library function,
  a lock-free execution in that thread shall complete.
  \begin{note}
    Concurrently executing threads
    might prevent progress of a lock-free execution.
    For example,
    this situation can occur
    with load-locked store-conditional implementations.
    This property is sometimes termed obstruction-free.
  \end{note}
\item
  When one or more lock-free executions run concurrently,
  at least one should complete.
  \begin{note}
    It is difficult for some implementations
    to provide absolute guarantees to this effect,
    since repeated and particularly inopportune interference
    from other threads
    could prevent forward progress,
    e.g.,
    by repeatedly stealing a cache line
    for unrelated purposes
    between load-locked and store-conditional instructions.
    For implementations that follow this recommendation and
    ensure that such effects cannot indefinitely delay progress
    under expected operating conditions,
    such anomalies
    can therefore safely be ignored by programmers.
    Outside this document,
    this property is sometimes termed lock-free.
  \end{note}
\end{itemize}

\pnum
During the execution of a thread of execution, each of the following is termed
an \defn{execution step}:
\begin{itemize}
\item termination of the thread of execution,
\item performing an access through a volatile glvalue, or
\item completion of a call to a library I/O function, a
      synchronization operation, or an atomic operation.
\end{itemize}

\pnum
An invocation of a standard library function that blocks\iref{defns.block}
is considered to continuously execute execution steps while waiting for the
condition that it blocks on to be satisfied.
\begin{example}
A library I/O function that blocks until the I/O operation is complete can
be considered to continuously check whether the operation is complete. Each
such check consists of one or more execution steps, for example using
observable behavior of the abstract machine.
\end{example}

\pnum
\begin{note}
Because of this and the preceding requirement regarding what threads of execution
have to perform eventually, it follows that no thread of execution can execute
forever without an execution step occurring.
\end{note}

\pnum
A thread of execution \defnx{makes progress}{make progress!thread}
when an execution step occurs or a
lock-free execution does not complete because there are other concurrent threads
that are not blocked in a standard library function (see above).

\pnum
\indextext{forward progress guarantees!concurrent}%
For a thread of execution providing \defn{concurrent forward progress guarantees},
the implementation ensures that the thread will eventually make progress for as
long as it has not terminated.
\begin{note}
This applies regardless of whether or not other threads of execution (if any)
have been or are making progress. To eventually fulfill this requirement means that
this will happen in an unspecified but finite amount of time.
\end{note}

\pnum
It is \impldef{whether the thread that executes \tcode{main} and the threads created
by \tcode{std::thread} or \tcode{std::jthread} provide concurrent forward progress guarantees} whether the
implementation-created thread of execution that executes
\tcode{main}\iref{basic.start.main} and the threads of execution created by
\tcode{std::thread}\iref{thread.thread.class}
or \tcode{std::jthread}\iref{thread.jthread.class}
provide concurrent forward progress guarantees.
General-purpose implementations should provide these guarantees.

\pnum
\indextext{forward progress guarantees!parallel}%
For a thread of execution providing \defn{parallel forward progress guarantees},
the implementation is not required to ensure that the thread will eventually make
progress if it has not yet executed any execution step; once this thread has
executed a step, it provides concurrent forward progress guarantees.

\pnum
\begin{note}
This does not specify a requirement for when to start this thread of execution,
which will typically be specified by the entity that creates this thread of
execution. For example, a thread of execution that provides concurrent forward
progress guarantees and executes tasks from a set of tasks in an arbitrary order,
one after the other, satisfies the requirements of parallel forward progress for
these tasks.
\end{note}

\pnum
\indextext{forward progress guarantees!weakly parallel}%
For a thread of execution providing \defn{weakly parallel forward progress
guarantees}, the implementation does not ensure that the thread will eventually
make progress.

\pnum
\begin{note}
Threads of execution providing weakly parallel forward progress guarantees cannot
be expected to make progress regardless of whether other threads make progress or
not; however, blocking with forward progress guarantee delegation, as defined below,
can be used to ensure that such threads of execution make progress eventually.
\end{note}

\pnum
Concurrent forward progress guarantees are stronger than parallel forward progress
guarantees, which in turn are stronger than weakly parallel forward progress
guarantees.
\begin{note}
For example, some kinds of synchronization between threads of execution might only
make progress if the respective threads of execution provide parallel forward progress
guarantees, but will fail to make progress under weakly parallel guarantees.
\end{note}

\pnum
\indextext{forward progress guarantees!delegation of}%
When a thread of execution $P$ is specified to
\defnx{block with forward progress guarantee delegation}
{block (execution)!with forward progress guarantee delegation}
on the completion of a set $S$ of threads of execution,
then throughout the whole time of $P$ being blocked on $S$,
the implementation shall ensure that the forward progress guarantees
provided by at least one thread of execution in $S$
is at least as strong as $P$'s forward progress guarantees.
\begin{note}
It is unspecified which thread or threads of execution in $S$ are chosen
and for which number of execution steps. The strengthening is not permanent and
not necessarily in place for the rest of the lifetime of the affected thread of
execution. As long as $P$ is blocked, the implementation has to eventually
select and potentially strengthen a thread of execution in $S$.
\end{note}
Once a thread of execution in $S$ terminates, it is removed from $S$.
Once $S$ is empty, $P$ is unblocked.

\pnum
\begin{note}
A thread of execution $B$ thus can temporarily provide an effectively
stronger forward progress guarantee for a certain amount of time, due to a
second thread of execution $A$ being blocked on it with forward
progress guarantee delegation. In turn, if $B$ then blocks with
forward progress guarantee delegation on $C$, this can also temporarily
provide a stronger forward progress guarantee to $C$.
\end{note}

\pnum
\begin{note}
If all threads of execution in $S$ finish executing (e.g., they terminate
and do not use blocking synchronization incorrectly), then $P$'s execution
of the operation that blocks with forward progress guarantee delegation will not
result in $P$'s progress guarantee being effectively weakened.
\end{note}

\pnum
\begin{note}
This does not remove any constraints regarding blocking synchronization for
threads of execution providing parallel or weakly parallel forward progress
guarantees because the implementation is not required to strengthen a particular
thread of execution whose too-weak progress guarantee is preventing overall progress.
\end{note}

\pnum
An implementation should ensure that the last value (in modification order)
assigned by an atomic or synchronization operation will become visible to all
other threads in a finite period of time.%
\indextext{atomic!operation|)}%
\indextext{threads!multiple|)}

\rSec2[basic.start]{Start and termination}

\rSec3[basic.start.main]{\tcode{main} function}
\indextext{\idxcode{main} function|(}

\pnum
\indextext{program!startup|(}%
A program shall contain exactly one function called \tcode{main}
that belongs to the global scope.
Executing a program starts a main thread of execution\iref{intro.multithread,thread.threads}
in which the \tcode{main} function is invoked.
\indextext{implementation!freestanding}%
It is \impldef{defining \tcode{main} in freestanding environment}
whether a program in a freestanding environment is required to define a \tcode{main}
function.
\begin{note}
In a freestanding environment, startup and termination is
\impldef{startup and termination in freestanding environment}; startup contains the
execution of constructors for non-local objects with static storage duration;
termination contains the execution of destructors for objects with static storage
duration.
\end{note}

\pnum
An implementation shall not predefine the \tcode{main} function.
Its type shall have \Cpp{} language linkage
and it shall have a declared return type of type
\keyword{int}, but otherwise its type is \impldef{parameters to \tcode{main}}.
\indextext{\idxcode{main} function!implementation-defined parameters to}%
An implementation shall allow both
\begin{itemize}
\item a function of \tcode{()} returning \keyword{int} and
\item a function of \tcode{(\keyword{int}}, pointer to pointer to \tcode{\keyword{char})} returning \keyword{int}
\end{itemize}
\indextext{\idxcode{argc}}%
\indextext{\idxcode{argv}}%
as the type of \tcode{main}\iref{dcl.fct}.
\indextext{\idxcode{main} function!parameters to}%
\indextext{environment!program}%
In the latter form, for purposes of exposition, the first function
parameter is called \tcode{argc} and the second function parameter is
called \tcode{argv}, where \tcode{argc} shall be the number of
arguments passed to the program from the environment in which the
program is run. If
\tcode{argc} is nonzero these arguments shall be supplied in
\tcode{argv[0]} through \tcode{argv[argc-1]} as pointers to the initial
characters of null-terminated multibyte strings (\ntmbs{}s)\iref{multibyte.strings}
and \tcode{argv[0]} shall be the pointer to
the initial character of an \ntmbs{} that represents the name used to
invoke the program or \tcode{""}. The value of \tcode{argc} shall be
non-negative. The value of \tcode{argv[argc]} shall be 0.

\recommended
Any further (optional) parameters should be added after \tcode{argv}.

\pnum
The function \tcode{main} shall not be named by an expression.
\indextext{\idxcode{main} function!implementation-defined linkage of}%
The linkage\iref{basic.link} of \tcode{main} is
\impldef{linkage of \tcode{main}}. A program that defines \tcode{main} as
deleted or that declares \tcode{main} to be
\keyword{inline}, \keyword{static}, \keyword{constexpr}, or \keyword{consteval} is ill-formed.
The function \tcode{main} shall not be a coroutine\iref{dcl.fct.def.coroutine}.
The \tcode{main} function shall not be declared with a
\grammarterm{linkage-specification}\iref{dcl.link}.
A program that declares
\begin{itemize}
\item
a variable \tcode{main} that belongs to the global scope, or
\item
a function \tcode{main} that belongs to the global scope and
is attached to a named module, or
\item
a function template \tcode{main} that belongs to the global scope, or
\item
an entity named \tcode{main}
with C language linkage (in any namespace)
\end{itemize}
is ill-formed.
The name \tcode{main} is
not otherwise reserved.
\begin{example}
Member functions, classes, and
enumerations can be called \tcode{main}, as can entities in other
namespaces.
\end{example}

\pnum
\indextext{\idxcode{exit}}%
\indexlibraryglobal{exit}%
\indextext{termination!program}%
Terminating the program
without leaving the current block (e.g., by calling the function
\tcode{std::exit(int)}\iref{support.start.term}) does not destroy any
objects with automatic storage duration\iref{class.dtor}. If
\tcode{std::exit} is invoked during the destruction of
an object with static or thread storage duration, the program has undefined
behavior.

\pnum
\indextext{termination!program}%
\indextext{\idxcode{main} function!return from}%
A \keyword{return} statement\iref{stmt.return} in \tcode{main} has the effect of leaving the main
function (destroying any objects with automatic storage duration) and
calling \tcode{std::exit} with the return value as the argument.
If control flows off the end of
the \grammarterm{compound-statement} of \tcode{main},
the effect is equivalent to a \keyword{return} with operand \tcode{0}
(see also \ref{except.handle}).
\indextext{\idxcode{main} function|)}

\rSec3[basic.start.static]{Static initialization}

\pnum
\indextext{initialization}%
\indextext{initialization!static and thread}%
Variables with static storage duration
are initialized as a consequence of program initiation. Variables with
thread storage duration are initialized as a consequence of thread execution.
Within each of these phases of initiation, initialization occurs as follows.

\pnum
\indextext{initialization!constant}%
\defnx{Constant initialization}{constant initialization} is performed
if a variable or temporary object with static or thread storage duration
is constant-initialized\iref{expr.const}.
\indextext{initialization!zero-initialization}%
If constant initialization is not performed, a variable with static
storage duration\iref{basic.stc.static} or thread storage
duration\iref{basic.stc.thread} is zero-initialized\iref{dcl.init}.
Together, zero-initialization and constant initialization are called
\defnadj{static}{initialization};
all other initialization is \defnadj{dynamic}{initialization}.
All static initialization strongly happens before\iref{intro.races}
any dynamic initialization.
\begin{note}
The dynamic initialization of non-block variables is described
in~\ref{basic.start.dynamic}; that of static block variables is described
in~\ref{stmt.dcl}.
\end{note}

\pnum
An implementation is permitted to perform the initialization of a
variable with static or thread storage duration as a static
initialization even if such initialization is not required to be done
statically, provided that
\begin{itemize}
\item
the dynamic version of the initialization does not change the
value of any other object of static or thread storage duration
prior to its initialization, and

\item
the static version of the initialization produces the same value
in the initialized variable as would be produced by the dynamic
initialization if all variables not required to be initialized statically
were initialized dynamically.
\end{itemize}
\begin{note}
As a consequence, if the initialization of an object \tcode{obj1} refers to an
object \tcode{obj2} potentially requiring dynamic initialization and defined
later in the same translation unit, it is unspecified whether the value of \tcode{obj2} used
will be the value of the fully initialized \tcode{obj2} (because \tcode{obj2} was statically
initialized) or will be the value of \tcode{obj2} merely zero-initialized. For example,
\begin{codeblock}
inline double fd() { return 1.0; }
extern double d1;
double d2 = d1;     // unspecified:
                    // either statically initialized to \tcode{0.0} or
                    // dynamically initialized to \tcode{0.0} if \tcode{d1} is
                    // dynamically initialized, or \tcode{1.0} otherwise
double d1 = fd();   // either initialized statically or dynamically to \tcode{1.0}
\end{codeblock}
\end{note}

\rSec3[basic.start.dynamic]{Dynamic initialization of non-block variables}

\pnum
\indextext{initialization!dynamic non-block}%
\indextext{start!program}%
\indextext{initialization!order of}%
Dynamic initialization of a non-block variable with static storage duration is
unordered if the variable is an implicitly or explicitly instantiated
specialization, is partially-ordered if the variable
is an inline variable that is not an implicitly or explicitly instantiated
specialization, and otherwise is ordered.
\begin{note}
A non-inline explicit specialization of a templated variable
has ordered initialization.
\end{note}

\pnum
A declaration \tcode{D} is
\defn{appearance-ordered} before a declaration \tcode{E} if
\begin{itemize}
\item \tcode{D} appears in the same translation unit as \tcode{E}, or
\item the translation unit containing \tcode{E}
has an interface dependency on the translation unit containing \tcode{D},
\end{itemize}
in either case prior to \tcode{E}.

\pnum
Dynamic initialization of non-block variables \tcode{V} and \tcode{W}
with static storage duration are ordered as follows:
\begin{itemize}
\item
If \tcode{V} and \tcode{W} have ordered initialization and
the definition of \tcode{V}
is appearance-ordered before the definition of \tcode{W}, or
if \tcode{V} has partially-ordered initialization,
\tcode{W} does not have unordered initialization, and
for every definition \tcode{E} of \tcode{W}
there exists a definition \tcode{D} of \tcode{V}
such that \tcode{D} is appearance-ordered before \tcode{E}, then
\begin{itemize}
\item
if the program does not start a thread\iref{intro.multithread}
other than the main thread\iref{basic.start.main}
or \tcode{V} and \tcode{W} have ordered initialization and
they are defined in the same translation unit,
the initialization of \tcode{V}
is sequenced before
the initialization of \tcode{W};
\item
otherwise,
the initialization of \tcode{V}
strongly happens before
the initialization of \tcode{W}.
\end{itemize}

\item
Otherwise, if the program starts a thread
other than the main thread
before either \tcode{V} or \tcode{W} is initialized,
it is unspecified in which threads
the initializations of \tcode{V} and \tcode{W} occur;
the initializations are unsequenced if they occur in the same thread.

\item
Otherwise, the initializations of \tcode{V} and \tcode{W} are indeterminately sequenced.
\end{itemize}
\begin{note}
This definition permits initialization of a sequence of
ordered variables concurrently with another sequence.
\end{note}

\pnum
\indextext{non-initialization odr-use|see{odr-use, non-initialization}}%
A \defnx{non-initialization odr-use}{odr-use!non-initialization}
is an odr-use\iref{term.odr.use} not caused directly or indirectly by
the initialization of a non-block static or thread storage duration variable.

\pnum
\indextext{evaluation!unspecified order of}%
It is \impldef{dynamic initialization of static variables before \tcode{main}}
whether the dynamic initialization of a
non-block non-inline variable with static storage duration
is sequenced before the first statement of \tcode{main} or is deferred.
If it is deferred, it strongly happens before
any non-initialization odr-use
of any non-inline function or non-inline variable
defined in the same translation unit as the variable to be initialized.
\begin{footnote}
A non-block variable with static storage duration
having initialization
with side effects is initialized in this case,
even if it is not itself odr-used\iref{term.odr.use,basic.stc.static}.
\end{footnote}
It is \impldef{threads and program points at which deferred dynamic initialization is performed}
in which threads and at which points in the program such deferred dynamic initialization occurs.

\recommended
An implementation should choose such points in a way
that allows the programmer to avoid deadlocks.
\begin{example}
\begin{codeblock}
// - File 1 -
#include "a.h"
#include "b.h"
B b;
A::A() {
  b.Use();
}

// - File 2 -
#include "a.h"
A a;

// - File 3 -
#include "a.h"
#include "b.h"
extern A a;
extern B b;

int main() {
  a.Use();
  b.Use();
}
\end{codeblock}

It is \impldef{dynamic initialization of static variables before \tcode{main}}
whether either \tcode{a} or \tcode{b} is
initialized before \tcode{main} is entered or whether the
initializations are delayed until \tcode{a} is first odr-used in
\tcode{main}. In particular, if \tcode{a} is initialized before
\tcode{main} is entered, it is not guaranteed that \tcode{b} will be
initialized before it is odr-used by the initialization of \tcode{a}, that
is, before \tcode{A::A} is called. If, however, \tcode{a} is initialized
at some point after the first statement of \tcode{main}, \tcode{b} will
be initialized prior to its use in \tcode{A::A}.
\end{example}

\pnum
It is \impldef{dynamic initialization of static inline variables before \tcode{main}}
whether the dynamic initialization of a
non-block inline variable with static storage duration
is sequenced before the first statement of \tcode{main} or is deferred.
If it is deferred, it strongly happens before
any non-initialization odr-use
of that variable.
It is \impldef{threads and program points at which deferred dynamic initialization is performed}
in which threads and at which points in the program such deferred dynamic initialization occurs.

\pnum
It is \impldef{dynamic initialization of thread-local variables before entry}
whether the dynamic initialization of a
non-block non-inline variable with thread storage duration
is sequenced before the first statement of the initial function of a thread or is deferred.
If it is deferred,
the initialization associated with the entity for thread \placeholder{t}
is sequenced before the first non-initialization odr-use by \placeholder{t}
of any non-inline variable with thread storage duration
defined in the same translation unit as the variable to be initialized.
It is \impldef{threads and program points at which deferred dynamic initialization is performed}
in which threads and at which points in the program such deferred dynamic initialization occurs.

\pnum
If the initialization of
a non-block variable with static or thread storage duration
exits via an exception,
the function \tcode{std::terminate} is called\iref{except.terminate}.%
\indextext{program!startup|)}

\rSec3[basic.start.term]{Termination}

\pnum
\indextext{program!termination|(}%
\indextext{object!destructor static}%
\indextext{\idxcode{main} function!return from}%
Constructed objects\iref{dcl.init}
with static storage duration are destroyed
and functions registered with \tcode{std::atexit}
are called as part of a call to
\indextext{\idxcode{exit}}%
\indexlibraryglobal{exit}%
\tcode{std::exit}\iref{support.start.term}.
The call to \tcode{std::exit} is sequenced before
the destructions and the registered functions.
\begin{note}
Returning from \tcode{main} invokes \tcode{std::exit}\iref{basic.start.main}.
\end{note}

\pnum
Constructed objects with thread storage duration within a given thread
are destroyed as a result of returning from the initial function of that thread and as a
result of that thread calling \tcode{std::exit}.
The destruction of all constructed objects with thread storage
duration within that thread strongly happens before destroying
any object with static storage duration.

\pnum
If the completion of the constructor or dynamic initialization of an object with static
storage duration strongly happens before that of another, the completion of the destructor
of the second is sequenced before the initiation of the destructor of the first.
If the completion of the constructor or dynamic initialization of an object with thread
storage duration is sequenced before that of another, the completion of the destructor
of the second is sequenced before the initiation of the destructor of the first.
If an object is
initialized statically, the object is destroyed in the same order as if
the object was dynamically initialized. For an object of array or class
type, all subobjects of that object are destroyed before any block
variable with static storage duration initialized during the construction
of the subobjects is destroyed.
If the destruction of an object with static or thread storage duration
exits via an exception,
the function \tcode{std::terminate} is called\iref{except.terminate}.

\pnum
If a function contains a block variable of static or thread storage duration that has been
destroyed and the function is called during the destruction of an object with static or
thread storage duration, the program has undefined behavior if the flow of control
passes through the definition of the previously destroyed block variable.
\begin{note}
Likewise, the behavior is undefined
if the block variable is used indirectly (e.g., through a pointer)
after its destruction.
\end{note}

\pnum
\indextext{\idxcode{atexit}}%
\indexlibraryglobal{atexit}%
If the completion of the initialization of an object with static storage
duration strongly happens before a call to \tcode{std::atexit}~(see
\libheader{cstdlib}, \ref{support.start.term}), the call to the function passed to
\tcode{std::atexit} is sequenced before the call to the destructor for the object. If a
call to \tcode{std::atexit} strongly happens before the completion of the initialization of
an object with static storage duration, the call to the destructor for the
object is sequenced before the call to the function passed to \tcode{std::atexit}. If a
call to \tcode{std::atexit} strongly happens before another call to \tcode{std::atexit}, the
call to the function passed to the second \tcode{std::atexit} call is sequenced before
the call to the function passed to the first \tcode{std::atexit} call.

\pnum
If there is a use of a standard library object or function not permitted within signal
handlers\iref{support.runtime} that does not happen before\iref{intro.multithread}
completion of destruction of objects with static storage duration and execution of
\tcode{std::atexit} registered functions\iref{support.start.term}, the program has
undefined behavior.
\begin{note}
If there is a use of an object with static storage
duration that does not happen before the object's destruction, the program has undefined
behavior. Terminating every thread before a call to \tcode{std::exit} or the exit from
\tcode{main} is sufficient, but not necessary, to satisfy these requirements. These
requirements permit thread managers as static-storage-duration objects.
\end{note}

\pnum
\indextext{\idxcode{abort}}%
\indexlibraryglobal{abort}%
\indextext{termination!program}%
Calling the function \tcode{std::abort()} declared in
\libheaderref{cstdlib} terminates the program without executing any destructors
and without calling
the functions passed to \tcode{std::atexit()} or \tcode{std::at_quick_exit()}.%
\indextext{program!termination|)}
\indextext{program execution|)}
\indextext{conventions!lexical|)}
