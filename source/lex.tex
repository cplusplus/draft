%!TEX root = std.tex
\rSec0[lex]{Lexical conventions}

\gramSec[gram.lex]{Lexical conventions}

\indextext{lexical conventions|see{conventions, lexical}}
\indextext{translation!separate|see{compilation, separate}}
\indextext{separate translation|see{compilation, separate}}
\indextext{separate compilation|see{compilation, separate}}
\indextext{phases of translation|see{translation, phases}}
\indextext{source file character|see{character, source file}}
\indextext{alternative token|see{token, alternative}}
\indextext{digraph|see{token, alternative}}
\indextext{integer literal|see{literal, integer}}
\indextext{character literal|see{literal, character}}
\indextext{floating-point literal|see{literal, floating-point}}
\indextext{string literal|see{literal, string}}
\indextext{boolean literal|see{literal, boolean}}
\indextext{pointer literal|see{literal, pointer}}
\indextext{user-defined literal|see{literal, user-defined}}
\indextext{file, source|see{source file}}
\indextext{null character|see{character, null}}
\indextext{null wide character|see{wide-character, null}}

\rSec1[lex.separate]{Separate translation}

\pnum
\indextext{conventions!lexical|(}%
\indextext{compilation!separate|(}%
The text of the program is kept in units called
\defnx{source files}{source file} in this document.
A source file together with all the headers\iref{headers}
and source files included\iref{cpp.include} via the preprocessing
directive \tcode{\#include}, less any source lines skipped by any of the
conditional inclusion\iref{cpp.cond} preprocessing directives,
as modified by the implementation-defined behavior of any
conditionally-supported-directives\iref{cpp.pre} and pragmas\iref{cpp.pragma},
if any, is called a \defnadj{preprocessing}{translation unit}.

\pnum
\begin{note}
A \Cpp{} program need not all be translated at the same time.
\end{note}

\pnum
\begin{note}
Previously translated translation units and instantiation
units can be preserved individually or in libraries. The separate
translation units of a program communicate\iref{basic.link} by (for example)
calls to functions whose identifiers have external or module linkage,
manipulation of objects whose identifiers have external or module linkage, or
manipulation of data files. Translation units can be separately
translated and then later linked to produce an executable
program\iref{basic.link}.
\end{note}
\indextext{compilation!separate|)}

\pnum
\indextext{translation!phases|(}%
The precedence among the syntax rules of translation is specified by the
following phases of tranlation\iref{lex.phases}.

\pnum
\begin{note}
Implementations behave as if these separate phases
occur, although in practice different phases can be folded together.
\end{note}

\rSec1[lex.phases]{Phases of translation}%

\rSec2[lex.phase.1]{Mapping to translation characters}%

\pnum
\indextext{character!source file}%
An implementation shall support source files
that are a sequence of UTF-8 code units (UTF-8 files).
It may also support
an \impldef{supported source files} set of other kinds of source files, and,
if so, the kind of an source file is determined in
an \impldef{determination of kind of source file} manner
that includes a means of designating source files as UTF-8 files,
independent of their content.

\pnum
\begin{note}
In other words,
recognizing the \unicode{feff}{byte order mark} is not sufficient.
\end{note}

\pnum
If a source file is determined to be a UTF-8 file,
then it shall be a well-formed UTF-8 code unit sequence and
it is decoded to produce a sequence of Unicode
\begin{footnote}
Unicode\textregistered\ is a registered trademark of Unicode, Inc.
This information is given for the convenience of users of this document and
does not constitute an endorsement by ISO or IEC of this product.
\end{footnote}
scalar values.
A sequence of translation character set elements is then formed
by mapping each Unicode scalar value
to the corresponding translation character set element.
In the resulting sequence,
each pair of characters in the input sequence consisting of
\unicode{000d}{carriage return} followed by \unicode{000a}{line feed},
as well as each
\unicode{000d}{carriage return} not immediately followed by a \unicode{000a}{line feed},
is replaced by a single new-line character.

\pnum
For any other kind of source file supported by the implementation,
characters are mapped, in an
\impldef{mapping physical source file characters to translation character set} manner,
to a sequence of translation character set elements\iref{lex.charset},
representing end-of-line indicators as new-line characters.

\rSec2[lex.phase.2]{Line splicing}%
\pnum
\indextext{line splicing}%
If the first translation character is \unicode{feff}{byte order mark},
it is deleted.
Each sequence of a backslash character (\textbackslash)
immediately followed by
zero or more whitespace characters other than new-line followed by
a new-line character is deleted, splicing
physical source lines to form logical source lines. Only the last
backslash on any physical source line shall be eligible for being part
of such a splice.

\pnum
\begin{note}
Line splicing can form
a \grammarterm{universal-character-name}\iref{lex.charset}.
\end{note}

\pnum
A source file that is not empty and that (after splicing)
does not end in a new-line character
shall be processed as if an additional new-line character were appended
to the file.

\rSec2[lex.phase.3]{Preprocessor tokenization}%
\pnum
The source file is decomposed into preprocessing
tokens\iref{lex.pptoken} and sequences of whitespace characters
(including comments). A source file shall not end in a partial
preprocessing token or in a partial comment.
\begin{footnote}
A partial preprocessing
token would arise from a source file
ending in the first portion of a multi-character token that requires a
terminating sequence of characters, such as a \grammarterm{header-name}
that is missing the closing \tcode{"}
or \tcode{>}. A partial comment
would arise from a source file ending with an unclosed \tcode{/*}
comment.
\end{footnote}

\pnum
\indextext{comment|(}%
\indextext{comment!\tcode{/*} \tcode{*/}}%
\indextext{comment!\tcode{//}}%
The characters \tcode{/*} start a comment, which terminates with the
characters \tcode{*/}. These comments do not nest.
\indextext{comment!\tcode{//}}%
The characters \tcode{//} start a comment, which terminates immediately before the
next new-line character. If there is a form-feed or a vertical-tab
character in such a comment, only whitespace characters shall appear
between it and the new-line that terminates the comment; no diagnostic
is required.

\pnum
\begin{note}
The comment characters \tcode{//}, \tcode{/*},
and \tcode{*/} have no special meaning within a \tcode{//} comment and
are treated just like other characters. Similarly, the comment
characters \tcode{//} and \tcode{/*} have no special meaning within a
\tcode{/*} comment.
\end{note}
\indextext{comment|)}

\pnum
Each comment is replaced by one space character. New-line characters are
retained. Whether each nonempty sequence of whitespace characters other
than new-line is retained or replaced by one space character is
unspecified.
As characters from the source file are consumed
to form the next preprocessing token
(i.e., not being consumed as part of a comment or other forms of whitespace),
except when matching a
\grammarterm{c-char-sequence},
\grammarterm{s-char-sequence},
\grammarterm{r-char-sequence},
\grammarterm{h-char-sequence}, or
\grammarterm{q-char-sequence},
\grammarterm{universal-character-name}s are recognized and
replaced by the designated element of the translation character set.
The process of dividing a source file's
characters into preprocessing tokens is context-dependent.
\begin{example}
See the handling of \tcode{<} within a \tcode{\#include} preprocessing
directive\iref{cpp.include}.
\end{example}

\rSec2[lex.phase.4]{Preprocessing directives}%

\pnum
Preprocessing directives are executed, macro invocations are
expanded, and \tcode{_Pragma} unary operator expressions are executed.
A \tcode{\#include} preprocessing directive causes the named header or
source file to be processed from phase 1 through phase 4, recursively.
All preprocessing directives are then deleted.

\rSec2[lex.phase.5]{String literal encoding}%

\pnum
For a sequence of two or more adjacent \grammarterm{string-literal} tokens,
a common \grammarterm{encoding-prefix} is determined
as specified in \ref{lex.string}.
Each such \grammarterm{string-literal} token is then considered to have
that common \grammarterm{encoding-prefix}.

\rSec2[lex.phase.6]{String literal concatenation}%

\pnum
Adjacent \grammarterm{string-literal} tokens are concatenated\iref{lex.string}.

\rSec2[lex.phase.7]{Syntactic and semantic analysis}%

\pnum
Whitespace characters separating tokens are no longer
significant. Each preprocessing token is converted into a
token\iref{lex.token}. The resulting tokens
constitute a \defn{translation unit} and
are syntactically and semantically analyzed and translated.
\begin{note}
The process of analyzing and translating the tokens can occasionally
result in one token being replaced by a sequence of other
tokens\iref{temp.names}.
\end{note}

\pnum
It is \impldef{whether the sources for module units and header units
on which the current translation unit has an interface
dependency are required to be available during translation}
whether the sources for module units and header units
on which the current translation unit has an interface
dependency\iref{module.unit,module.import}
are required to be available.
\begin{note}
Source files, translation
units and translated translation units need not necessarily be stored as
files, nor need there be any one-to-one correspondence between these
entities and any external representation. The description is conceptual
only, and does not specify any particular implementation.
\end{note}

\rSec2[lex.phase.8]{Template instantiation}%

\pnum
Translated translation units and instantiation units are combined
as follows:
\begin{note}
Some or all of these can be supplied from a
library.
\end{note}

\pnum
Each translated translation unit is examined to
produce a list of required instantiations.
\begin{note}
This can include
instantiations which have been explicitly
requested\iref{temp.explicit}.
\end{note}

\pnum
The definitions of the required templates are located.
It is \impldef{whether source of translation units must
be available to locate template definitions} whether the
source of the translation units containing these definitions
is required to be available.
\begin{note}
An implementation can choose to encode sufficient
information into the translated translation unit so as to ensure the
source is not required here.
\end{note}

\pnum
All the required instantiations are performed
to produce \defn{instantiation units}.
\begin{note}
These are similar to translated translation units, but contain no references to
uninstantiated templates and no template definitions.
\end{note}

\pnum
The program is ill-formed if any instantiation fails.

\rSec2[lex.phase.9]{Linking}%

\pnum
All external entity references are resolved. Library
components are linked to satisfy external references to
entities not defined in the current translation. All such translator
output is collected into a program image which contains information
needed for execution in its execution environment.%
\indextext{translation!phases|)}

\rSec1[lex.char]{Characters}%

\rSec2[lex.charset]{Character sets}

\pnum
\indextext{character set|(}%
The \defnadj{translation}{character set} consists of the following elements:
\begin{itemize}
\item
each abstract character assigned a code point in the Unicode codespace
as specified in the Unicode Standard, and
\item
a distinct character for each Unicode scalar value
not assigned to an abstract character.
\end{itemize}
\begin{note}
Unicode code points are integers
in the range $[0, \mathrm{10FFFF}]$ (hexadecimal).
A surrogate code point is a value
in the range $[\mathrm{D800}, \mathrm{DFFF}]$ (hexadecimal).
A Unicode scalar value is any code point that is not a surrogate code point.
\end{note}

\pnum
The \defnadj{basic}{character set} is a subset of the translation character set,
consisting of 99 characters as specified in \tref{lex.charset.basic}.
\begin{note}
Unicode short names are given only as a means to identifying the character;
the numerical value has no other meaning in this context.
\end{note}

\begin{floattable}{Basic character set}{lex.charset.basic}{lll}
\topline
\lhdrx{2}{character} & \rhdr{glyph} \\ \capsep
\ucode{0009} & \uname{character tabulation} & \\
\ucode{000b} & \uname{line tabulation} & \\
\ucode{000c} & \uname{form feed} & \\
\ucode{0020} & \uname{space} & \\
\ucode{000a} & \uname{line feed} & new-line \\
\ucode{0021} & \uname{exclamation mark} & \tcode{!} \\
\ucode{0022} & \uname{quotation mark} & \tcode{"} \\
\ucode{0023} & \uname{number sign} & \tcode{\#} \\
\ucode{0024} & \uname{dollar sign} & \tcode{\$} \\
\ucode{0025} & \uname{percent sign} & \tcode{\%} \\
\ucode{0026} & \uname{ampersand}  & \tcode{\&} \\
\ucode{0027} & \uname{apostrophe} & \tcode{'} \\
\ucode{0028} & \uname{left parenthesis} & \tcode{(} \\
\ucode{0029} & \uname{right parenthesis} & \tcode{)} \\
\ucode{002a} & \uname{asterisk} & \tcode{*} \\
\ucode{002b} & \uname{plus sign} & \tcode{+} \\
\ucode{002c} & \uname{comma} & \tcode{,} \\
\ucode{002d} & \uname{hyphen-minus} & \tcode{-} \\
\ucode{002e} & \uname{full stop} & \tcode{.} \\
\ucode{002f} & \uname{solidus} & \tcode{/} \\
\ucode{0030} .. \ucode{0039} & \uname{digit zero .. nine} & \tcode{0 1 2 3 4 5 6 7 8 9} \\
\ucode{003a} & \uname{colon} & \tcode{:} \\
\ucode{003b} & \uname{semicolon} & \tcode{;} \\
\ucode{003c} & \uname{less-than sign} & \tcode{<} \\
\ucode{003d} & \uname{equals sign} & \tcode{=} \\
\ucode{003e} & \uname{greater-than sign} & \tcode{>} \\
\ucode{003f} & \uname{question mark} & \tcode{?} \\
\ucode{0040} & \uname{commercial at} & \tcode{@} \\
\ucode{0041} .. \ucode{005a} & \uname{latin capital letter a .. z} & \tcode{A B C D E F G H I J K L M} \\
 & & \tcode{N O P Q R S T U V W X Y Z} \\
\ucode{005b} & \uname{left square bracket} & \tcode{[} \\
\ucode{005c} & \uname{reverse solidus} & \tcode{\textbackslash} \\
\ucode{005d} & \uname{right square bracket} & \tcode{]} \\
\ucode{005e} & \uname{circumflex accent} & \tcode{\caret} \\
\ucode{005f} & \uname{low line} & \tcode{_} \\
\ucode{0060} & \uname{grave accent} & \tcode{\`} \\
\ucode{0061} .. \ucode{007a} & \uname{latin small letter a .. z} & \tcode{a b c d e f g h i j k l m} \\
 & & \tcode{n o p q r s t u v w x y z} \\
\ucode{007b} & \uname{left curly bracket} & \tcode{\{} \\
\ucode{007c} & \uname{vertical line} & \tcode{|} \\
\ucode{007d} & \uname{right curly bracket} & \tcode{\}} \\
\ucode{007e} & \uname{tilde} & \tcode{\textasciitilde} \\
\end{floattable}

\pnum
The \defnadj{basic literal}{character set} consists of
all characters of the basic character set,
plus the control characters specified in \tref{lex.charset.literal}.

\begin{floattable}{Additional control characters in the basic literal character set}{lex.charset.literal}{ll}
\topline
\ohdrx{2}{character} \\ \capsep
\ucode{0000} & \uname{null} \\
\ucode{0007} & \uname{alert} \\
\ucode{0008} & \uname{backspace} \\
\ucode{000d} & \uname{carriage return} \\
\end{floattable}

\pnum
A \defn{code unit} is an integer value
of character type\iref{basic.fundamental}.
Characters in a \grammarterm{character-literal}
other than a multicharacter or non-encodable character literal or
in a \grammarterm{string-literal} are encoded as
a sequence of one or more code units, as determined
by the \grammarterm{encoding-prefix}\iref{lex.ccon,lex.string};
this is termed the respective \defnadj{literal}{encoding}.
The \defnadj{ordinary literal}{encoding} is
the encoding applied to an ordinary character or string literal.
The \defnadj{wide literal}{encoding} is the encoding applied
to a wide character or string literal.

\pnum
A literal encoding or a locale-specific encoding of one of
the execution character sets\iref{character.seq}
encodes each element of the basic literal character set as
a single code unit with non-negative value,
distinct from the code unit for any other such element.
\begin{note}
A character not in the basic literal character set
can be encoded with more than one code unit;
the value of such a code unit can be the same as
that of a code unit for an element of the basic literal character set.
\end{note}
\indextext{character!null}%
\indextext{wide-character!null}%
The \unicode{0000}{null} character is encoded as the value \tcode{0}.
No other element of the translation character set
is encoded with a code unit of value \tcode{0}.
The code unit value of each decimal digit character after the digit \tcode{0} (\ucode{0030})
shall be one greater than the value of the previous.
The ordinary and wide literal encodings are otherwise
\impldef{ordinary and wide literal encodings}.
\indextext{UTF-8}%
\indextext{UTF-16}%
\indextext{UTF-32}%
For a UTF-8, UTF-16, or UTF-32 literal,
the implementation shall encode
the Unicode scalar value
corresponding to each character of the translation character set
as specified in the Unicode Standard
for the respective Unicode encoding form.
\indextext{character set|)}

\rSec2[lex.universal.char]{Universal character names}

\begin{bnf}
\nontermdef{n-char}\br
     \textnormal{any member of the translation character set except the \unicode{007d}{right curly bracket} or new-line character}
\end{bnf}

\begin{bnf}
\nontermdef{n-char-sequence}\br
    n-char\br
    n-char-sequence n-char
\end{bnf}

\begin{bnf}
\nontermdef{named-universal-character}\br
    \terminal{\textbackslash N\{} n-char-sequence \terminal{\}}
\end{bnf}

\begin{bnf}
\nontermdef{hex-quad}\br
    hexadecimal-digit hexadecimal-digit hexadecimal-digit hexadecimal-digit
\end{bnf}

\begin{bnf}
\nontermdef{simple-hexadecimal-digit-sequence}\br
    hexadecimal-digit\br
    simple-hexadecimal-digit-sequence hexadecimal-digit
\end{bnf}

\begin{bnf}
\nontermdef{universal-character-name}\br
    \terminal{\textbackslash u} hex-quad\br
    \terminal{\textbackslash U} hex-quad hex-quad\br
    \terminal{\textbackslash u\{} simple-hexadecimal-digit-sequence \terminal{\}}\br
    named-universal-character
\end{bnf}

\pnum
The \grammarterm{universal-character-name} construct provides a way to name any
element in the translation character set using just the basic character set.
If a \grammarterm{universal-character-name} outside
the \grammarterm{c-char-sequence}, \grammarterm{s-char-sequence}, or
\grammarterm{r-char-sequence} of
a \grammarterm{character-literal} or \grammarterm{string-literal}
(in either case, including within a \grammarterm{user-defined-literal})
corresponds to a control character or
to a character in the basic character set, the program is ill-formed.
\begin{note}
A sequence of characters resembling a \grammarterm{universal-character-name} in an
\grammarterm{r-char-sequence}\iref{lex.string} does not form a
\grammarterm{universal-character-name}.
\end{note}

\pnum
A \grammarterm{universal-character-name}
of the form \tcode{\textbackslash u} \grammarterm{hex-quad},
\tcode{\textbackslash U} \grammarterm{hex-quad} \grammarterm{hex-quad}, or
\tcode{\textbackslash u\{\grammarterm{simple-hexadecimal-digit-sequence}\}}
designates the character in the translation character set
whose Unicode scalar value is the hexadecimal number represented by
the sequence of \grammarterm{hexadecimal-digit}s
in the \grammarterm{universal-character-name}.
The program is ill-formed if that number is not a Unicode scalar value.

\pnum
A \grammarterm{universal-character-name}
that is a \grammarterm{named-universal-character}
designates the corresponding character
in the Unicode Standard (chapter 4.8 Name)
if the \grammarterm{n-char-sequence} is equal
to its character name or
to one of its character name aliases of
type ``control'', ``correction'', or ``alternate'';
otherwise, the program is ill-formed.
\begin{note}
These aliases are listed in
the Unicode Character Database's \tcode{NameAliases.txt}.
None of these names or aliases have leading or trailing spaces.
\end{note}

\rSec1[lex.pptoken]{Preprocessing tokens}

\rSec2[lex.ppbasic]{Basic tokens}

\indextext{token!preprocessing|(}%
\begin{bnf}
\nontermdef{preprocessing-token}\br
    header-name\br
    pp-number\br
    character-literal\br
    user-defined-character-literal\br
    string-literal\br
    user-defined-string-literal\br
    preprocessing-op-or-punc\br
    identifier\br
    import-keyword\br
    module-keyword\br
    export-keyword\br
    \textnormal{each non-whitespace character that cannot be one of the above}
\end{bnf}

\pnum
A preprocessing token is the minimal lexical element of the language in translation
phases 3 through 6.
In this document,
glyphs are used to identify
elements of the basic character set\iref{lex.charset}.
The categories of preprocessing token are: header names, preprocessing numbers,
character literals (including user-defined character literals), string literals
(including user-defined string literals), preprocessing operators and punctuators,
identifiers, placeholder tokens produced by preprocessing \tcode{import} and
\tcode{module} directives (\grammarterm{import-keyword},
\grammarterm{module-keyword}, and \grammarterm{export-keyword}),
and single non-whitespace characters that do not lexically match the other
preprocessing token categories.
If a \unicode{0027}{apostrophe} or a \unicode{0022}{quotation mark} character
matches the last category, the program is ill-formed.
If any character not in the basic character set matches the last category,
the program is ill-formed.

\pnum
Preprocessing tokens can be separated by
\indextext{whitespace}%
whitespace;
\indextext{comment}%
this consists of comments\iref{lex.phase.3}, or whitespace characters
(\unicode{0020}{space},
\unicode{0009}{character tabulation},
new-line,
\unicode{000b}{line tabulation}, and
\unicode{000c}{form feed}), or both.

\pnum
As described in \ref{cpp}, in certain
circumstances during translation phase 4, whitespace (or the absence
thereof) serves as more than preprocessing token separation. Whitespace
can appear within a preprocessing token only as part of a header name or
between the quotation characters in a character literal or
string literal.

\pnum
If the input stream has been parsed into preprocessing tokens up to a
given character:
\begin{itemize}
\item
\indextext{literal!string!raw}%
If the next character begins a sequence of characters that could be the prefix
and initial double quote of a raw string literal, such as \tcode{R"}, the next preprocessing
token shall be a raw string literal. Between the initial and final
double quote characters of the raw string, any transformations performed in phase
2 (line splicing) are reverted; this reversion
shall apply before any \grammarterm{d-char}, \grammarterm{r-char}, or delimiting
parenthesis is identified. The raw string literal is defined as the shortest sequence
of characters that matches the raw-string pattern
\begin{ncbnf}
\opt{encoding-prefix} \terminal{R} raw-string
\end{ncbnf}

\item Otherwise, if the next three characters are \tcode{<::} and the subsequent character
is neither \tcode{:} nor \tcode{>}, the \tcode{<} is treated as a preprocessing token by
itself and not as the first character of the alternative token \tcode{<:}.

\item Otherwise,
the next preprocessing token is the longest sequence of
characters that could constitute a preprocessing token, even if that
would cause further lexical analysis to fail,
except that a \grammarterm{header-name}\iref{lex.header} is only formed
\begin{itemize}
\item
after the \tcode{include} or \tcode{import} preprocessing token in an
\tcode{\#include}\iref{cpp.include} or
\tcode{import}\iref{cpp.import} directive, or

\item
within a \grammarterm{has-include-expression}.

\end{itemize}
\end{itemize}

\begin{example}
\begin{codeblock}
#define R "x"
const char* s = R"y";           // ill-formed raw string, not \tcode{"x" "y"}
\end{codeblock}
\end{example}

\pnum
\begin{example}
The program fragment \tcode{0xe+foo} is parsed as a
preprocessing number token (one that is not a valid
\grammarterm{integer-literal} or \grammarterm{floating-point-literal} token),
even though a parse as three preprocessing tokens
\tcode{0xe}, \tcode{+}, and \tcode{foo} can produce a valid expression (for example,
if \tcode{foo} is a macro defined as \tcode{1}). Similarly, the
program fragment \tcode{1E1} is parsed as a preprocessing number (one
that is a valid \grammarterm{floating-point-literal} token),
whether or not \tcode{E} is a macro name.
\end{example}

\pnum
\begin{example}
The program fragment \tcode{x+++++y} is parsed as \tcode{x
++ ++ + y}, which, if \tcode{x} and \tcode{y} have integral types,
violates a constraint on increment operators, even though the parse
\tcode{x ++ + ++ y} can yield a correct expression.
\end{example}
\indextext{token!preprocessing|)}

\rSec2[lex.digraph]{Alternative tokens}

\pnum
\indextext{token!alternative|(}%
Alternative token representations are provided for some operators and
punctuators.
\begin{footnote}
\indextext{digraph}%
These include ``digraphs'' and additional reserved words. The term
``digraph'' (token consisting of two characters) is not perfectly
descriptive, since one of the alternative \grammarterm{preprocessing-token}s is
\tcode{\%:\%:} and of course several primary tokens contain two
characters. Nonetheless, those alternative tokens that aren't lexical
keywords are colloquially known as ``digraphs''.
\end{footnote}

\pnum
In all respects of the language, each alternative token behaves the
same, respectively, as its primary token, except for its spelling.
\begin{footnote}
Thus the ``stringized'' values\iref{cpp.stringize} of
\tcode{[} and \tcode{<:} will be different, maintaining the source
spelling, but the tokens can otherwise be freely interchanged.
\end{footnote}
The set of alternative tokens is defined in
\tref{lex.digraph}.

\begin{tokentable}{Alternative tokens}{lex.digraph}{Alternative}{Primary}
\tcode{<\%}             &   \tcode{\{}         &
\keyword{and}           &   \tcode{\&\&}       &
\keyword{and_eq}        &   \tcode{\&=}        \\ \rowsep
\tcode{\%>}             &   \tcode{\}}         &
\keyword{bitor}         &   \tcode{|}          &
\keyword{or_eq}         &   \tcode{|=}         \\ \rowsep
\tcode{<:}              &   \tcode{[}          &
\keyword{or}            &   \tcode{||}         &
\keyword{xor_eq}        &   \tcode{\caret=}    \\ \rowsep
\tcode{:>}              &   \tcode{]}          &
\keyword{xor}           &   \tcode{\caret}     &
\keyword{not}           &   \tcode{!}          \\ \rowsep
\tcode{\%:}             &   \tcode{\#}         &
\keyword{compl}         &   \tcode{\~}         &
\keyword{not_eq}        &   \tcode{!=}         \\ \rowsep
\tcode{\%:\%:}          &   \tcode{\#\#}       &
\keyword{bitand}        &   \tcode{\&}         &
                        &                      \\
\end{tokentable}%
\indextext{token!alternative|)}

\rSec2[lex.header]{Header names}

\indextext{header!name|(}%
\begin{bnf}
\microtypesetup{protrusion=false}\obeyspaces
\nontermdef{header-name}\br
    \terminal{<} h-char-sequence \terminal{>}\br
    \terminal{"} q-char-sequence \terminal{"}
\end{bnf}

\begin{bnf}
\nontermdef{h-char-sequence}\br
    h-char\br
    h-char-sequence h-char
\end{bnf}

\begin{bnf}
\nontermdef{h-char}\br
    \textnormal{any member of the translation character set except new-line and \unicode{003e}{greater-than sign}}
\end{bnf}

\begin{bnf}
\nontermdef{q-char-sequence}\br
    q-char\br
    q-char-sequence q-char
\end{bnf}

\begin{bnf}
\nontermdef{q-char}\br
    \textnormal{any member of the translation character set except new-line and \unicode{0022}{quotation mark}}
\end{bnf}

\pnum
\begin{note}
Header name preprocessing tokens only appear within
a \tcode{\#include} preprocessing directive,
a \tcode{__has_include} preprocessing expression, or
after certain occurrences of an \tcode{import} token
(see~\ref{lex.pptoken}).
\end{note}
The sequences in both forms of \grammarterm{header-name}{s} are mapped in an
\impldef{mapping header name to header or external source file} manner to headers or to
external source file names as specified in~\ref{cpp.include}.

\pnum
The appearance of either of the characters \tcode{'} or \tcode{\textbackslash} or of
either of the character sequences \tcode{/*} or \tcode{//} in a
\grammarterm{q-char-sequence} or an \grammarterm{h-char-sequence}
is conditionally-supported with \impldef{meaning of \tcode{'}, \tcode{\textbackslash},
\tcode{/*}, or \tcode{//} in a \grammarterm{q-char-sequence} or an
\grammarterm{h-char-sequence}} semantics, as is the appearance of the character
\tcode{"} in an \grammarterm{h-char-sequence}.
\begin{footnote}
Thus, a sequence of characters
that resembles an escape sequence can result in an error, be interpreted as the
character corresponding to the escape sequence, or have a completely different meaning,
depending on the implementation.
\end{footnote}
\indextext{header!name|)}

\rSec2[lex.ppnumber]{Preprocessing numbers}

\indextext{number!preprocessing|(}%
\begin{bnf}
\nontermdef{pp-number}\br
    digit\br
    \terminal{.} digit\br
    pp-number identifier-continue\br
    pp-number \terminal{'} digit\br
    pp-number \terminal{'} nondigit\br
    pp-number \terminal{e} sign\br
    pp-number \terminal{E} sign\br
    pp-number \terminal{p} sign\br
    pp-number \terminal{P} sign\br
    pp-number \terminal{.}
\end{bnf}

\pnum
Preprocessing number tokens lexically include
all \grammarterm{integer-literal} tokens\iref{lex.icon} and
all \grammarterm{floating-point-literal} tokens\iref{lex.fcon}.

\pnum
A preprocessing number does not have a type or a value; it acquires both
after a successful conversion to
an \grammarterm{integer-literal} token or
a \grammarterm{floating-point-literal} token.%
\indextext{number!preprocessing|)}

\rSec2[lex.ppliteral]{Literals}

\rSec3[lex.ccon]{Character literals}

\indextext{literal!character}%
\begin{bnf}
\nontermdef{character-literal}\br
    \opt{encoding-prefix} \terminal{'} c-char-sequence \terminal{'}
\end{bnf}

\begin{bnf}
\nontermdef{encoding-prefix} \textnormal{one of}\br
    \terminal{u8}\quad\terminal{u}\quad\terminal{U}\quad\terminal{L}
\end{bnf}

\begin{bnf}
\nontermdef{c-char-sequence}\br
    c-char\br
    c-char-sequence c-char
\end{bnf}

\begin{bnf}
\nontermdef{c-char}\br
    basic-c-char\br
    escape-sequence\br
    universal-character-name
\end{bnf}

\begin{bnf}
\nontermdef{basic-c-char}\br
    \textnormal{any member of the translation character set except the \unicode{0027}{apostrophe},}\br
    \bnfindent\textnormal{\unicode{005c}{reverse solidus}, or new-line character}
\end{bnf}

\begin{bnf}
\nontermdef{escape-sequence}\br
    simple-escape-sequence\br
    numeric-escape-sequence\br
    conditional-escape-sequence
\end{bnf}

\begin{bnf}
\nontermdef{simple-escape-sequence}\br
    \terminal{\textbackslash} simple-escape-sequence-char
\end{bnf}

\begin{bnf}
\nontermdef{simple-escape-sequence-char} \textnormal{one of}\br
    \terminal{'  "  ?  \textbackslash{} a  b  f  n  r  t  v}
\end{bnf}

\begin{bnf}
\nontermdef{numeric-escape-sequence}\br
    octal-escape-sequence\br
    hexadecimal-escape-sequence
\end{bnf}

\begin{bnf}
\nontermdef{simple-octal-digit-sequence}\br
    octal-digit\br
    simple-octal-digit-sequence octal-digit
\end{bnf}

\begin{bnf}
\nontermdef{octal-escape-sequence}\br
    \terminal{\textbackslash} octal-digit\br
    \terminal{\textbackslash} octal-digit octal-digit\br
    \terminal{\textbackslash} octal-digit octal-digit octal-digit\br
    \terminal{\textbackslash o\{} simple-octal-digit-sequence \terminal{\}}\br
\end{bnf}

\begin{bnf}
\nontermdef{hexadecimal-escape-sequence}\br
    \terminal{\textbackslash x} simple-hexadecimal-digit-sequence\br
    \terminal{\textbackslash x\{} simple-hexadecimal-digit-sequence \terminal{\}}
\end{bnf}

\begin{bnf}
\nontermdef{conditional-escape-sequence}\br
    \terminal{\textbackslash} conditional-escape-sequence-char
\end{bnf}

\begin{bnf}
\nontermdef{conditional-escape-sequence-char}\br
    \textnormal{any member of the basic character set that is not an} octal-digit\textnormal{, a} simple-escape-sequence-char\textnormal{, or the characters \terminal{N}, \terminal{o}, \terminal{u}, \terminal{U}, or \terminal{x}}
\end{bnf}

\pnum
\indextext{literal!character}%
\indextext{literal!\idxcode{char8_t}}%
\indextext{literal!\idxcode{char16_t}}%
\indextext{literal!\idxcode{char32_t}}%
\indextext{literal!type of character}%
\indextext{type!\idxcode{char8_t}}%
\indextext{type!\idxcode{char16_t}}%
\indextext{type!\idxcode{char32_t}}%
\indextext{wide-character}%
\indextext{type!\idxcode{wchar_t}}%
A \defnadj{multicharacter}{literal} is a \grammarterm{character-literal}
whose \grammarterm{c-char-sequence} consists of
more than one \grammarterm{c-char}.
A multicharacter literal shall not have an \grammarterm{encoding-prefix}.
If a multicharacter literal contains a \grammarterm{c-char}
that is not encodable as a single code unit in the ordinary literal encoding,
the program is ill-formed.
Multicharacter literals are conditionally-supported.

\pnum
The kind of a \grammarterm{character-literal},
its type, and its associated character encoding\iref{lex.charset}
are determined by
its \grammarterm{encoding-prefix} and its \grammarterm{c-char-sequence}
as defined by \tref{lex.ccon.literal}.

\begin{floattable}{Character literals}{lex.ccon.literal}
{l|l|l|l|l}
\topline
\lhdr{Encoding} & \chdr{Kind} & \chdr{Type} & \chdr{Associated char-} & \rhdr{Example} \\
\lhdr{prefix} & \chdr{} & \chdr{} & \chdr{acter encoding} & \\
\capsep
none &
\defnx{ordinary character literal}{literal!character!ordinary} &
\keyword{char} &
ordinary literal &
\tcode{'v'} \\ \cline{2-3}\cline{5-5}
 &
multicharacter literal &
\keyword{int} &
encoding &
\tcode{'abcd'} \\ \hline
\tcode{L} &
\defnx{wide character literal}{literal!character!wide} &
\keyword{wchar_t} &
wide literal &
\tcode{L'w'} \\
 & & & encoding & \\ \hline
\tcode{u8} &
\defnx{UTF-8 character literal}{literal!character!UTF-8} &
\keyword{char8_t} &
UTF-8 &
\tcode{u8'x'} \\ \hline
\tcode{u} &
\defnx{UTF-16 character literal}{literal!character!UTF-16} &
\keyword{char16_t} &
UTF-16 &
\tcode{u'y'} \\ \hline
\tcode{U} &
\defnx{UTF-32 character literal}{literal!character!UTF-32} &
\keyword{char32_t} &
UTF-32 &
\tcode{U'z'} \\
\end{floattable}

\pnum
In translation phase 4,
the value of a \grammarterm{character-literal} is determined
using the range of representable values
of the \grammarterm{character-literal}'s type in translation phase 7.
A multicharacter literal has an
\impldef{value of non-encodable character literal or multicharacter literal}
value.
The value of any other kind of \grammarterm{character-literal}
is determined as follows:
\begin{itemize}
\item
A \grammarterm{character-literal} with
a \grammarterm{c-char-sequence} consisting of a single
\grammarterm{basic-c-char},
\grammarterm{simple-escape-sequence}, or
\grammarterm{universal-character-name}
is the code unit value of the specified character
as encoded in the literal's associated character encoding.
If the specified character lacks
representation in the literal's associated character encoding or
if it cannot be encoded as a single code unit,
then the program is ill-formed.
\item
A \grammarterm{character-literal} with
a \grammarterm{c-char-sequence} consisting of
a single \grammarterm{numeric-escape-sequence}
has a value as follows:
\begin{itemize}
\item
Let $v$ be the integer value represented by
the octal number comprising
the sequence of \grammarterm{octal-digit}{s} in
an \grammarterm{octal-escape-sequence} or by
the hexadecimal number comprising
the sequence of \grammarterm{hexadecimal-digit}{s} in
a \grammarterm{hexadecimal-escape-sequence}.
\item
If $v$ does not exceed
the range of representable values of the \grammarterm{character-literal}'s type,
then the value is $v$.
\item
Otherwise,
if the \grammarterm{character-literal}'s \grammarterm{encoding-prefix}
is absent or \tcode{L}, and
$v$ does not exceed the range of representable values of the corresponding unsigned type for the underlying type of the \grammarterm{character-literal}'s type,
then the value is the unique value of the \grammarterm{character-literal}'s type \tcode{T} that is congruent to $v$ modulo $2^N$, where $N$ is the width of \tcode{T}.
\item
Otherwise, the program is ill-formed.
\end{itemize}
\item
A \grammarterm{character-literal} with
a \grammarterm{c-char-sequence} consisting of
a single \grammarterm{conditional-escape-sequence}
is conditionally-supported and
has an \impldef{value of \grammarterm{conditional-escape-sequence}} value.
\end{itemize}

\pnum
\indextext{backslash character}%
\indextext{\idxcode{\textbackslash}|see{backslash character}}%
\indextext{escape character|see{backslash character}}%
The character specified by a \grammarterm{simple-escape-sequence}
is specified in \tref{lex.ccon.esc}.
\begin{note}
Using an escape sequence for a question mark
is supported for compatibility with \CppXIV{} and C.
\end{note}

\begin{floattable}{Simple escape sequences}{lex.ccon.esc}
{lll}
\topline
\lhdrx{2}{character} &  \rhdr{\grammarterm{simple-escape-sequence}} \\ \capsep
\ucode{000a} & \uname{line feed}            & \tcode{\textbackslash n} \\
\ucode{0009} & \uname{character tabulation} & \tcode{\textbackslash t} \\
\ucode{000b} & \uname{line tabulation}      & \tcode{\textbackslash v} \\
\ucode{0008} & \uname{backspace}            & \tcode{\textbackslash b} \\
\ucode{000d} & \uname{carriage return}      & \tcode{\textbackslash r} \\
\ucode{000c} & \uname{form feed}            & \tcode{\textbackslash f} \\
\ucode{0007} & \uname{alert}                & \tcode{\textbackslash a} \\
\ucode{005c} & \uname{reverse solidus}      & \tcode{\textbackslash\textbackslash} \\
\ucode{003f} & \uname{question mark}        & \tcode{\textbackslash ?} \\
\ucode{0027} & \uname{apostrophe}           & \tcode{\textbackslash '} \\
\ucode{0022} & \uname{quotation mark}       & \tcode{\textbackslash "} \\
\end{floattable}

\rSec3[lex.string]{String literals}

\indextext{literal!string}%
\begin{bnf}
\nontermdef{string-literal}\br
    \opt{encoding-prefix} \terminal{"} \opt{s-char-sequence} \terminal{"}\br
    \opt{encoding-prefix} \terminal{R} raw-string
\end{bnf}

\begin{bnf}
\nontermdef{s-char-sequence}\br
    s-char\br
    s-char-sequence s-char
\end{bnf}

\begin{bnf}
\nontermdef{s-char}\br
    basic-s-char\br
    escape-sequence\br
    universal-character-name
\end{bnf}

\begin{bnf}
\nontermdef{basic-s-char}\br
    \textnormal{any member of the translation character set except the \unicode{0022}{quotation mark},}\br
    \bnfindent\textnormal{\unicode{005c}{reverse solidus}, or new-line character}
\end{bnf}

\begin{bnf}
\nontermdef{raw-string}\br
    \terminal{"} \opt{d-char-sequence} \terminal{(} \opt{r-char-sequence} \terminal{)} \opt{d-char-sequence} \terminal{"}
\end{bnf}

\begin{bnf}
\nontermdef{r-char-sequence}\br
    r-char\br
    r-char-sequence r-char
\end{bnf}

\begin{bnf}
\nontermdef{r-char}\br
    \textnormal{any member of the translation character set, except a \unicode{0029}{right parenthesis} followed by}\br
    \bnfindent\textnormal{the initial \grammarterm{d-char-sequence} (which may be empty) followed by a \unicode{0022}{quotation mark}}
\end{bnf}

\begin{bnf}
\nontermdef{d-char-sequence}\br
    d-char\br
    d-char-sequence d-char
\end{bnf}

\begin{bnf}
\nontermdef{d-char}\br
    \textnormal{any member of the basic character set except:}\br
    \bnfindent\textnormal{\unicode{0020}{space}, \unicode{0028}{left parenthesis}, \unicode{0029}{right parenthesis}, \unicode{005c}{reverse solidus},}\br
    \bnfindent\textnormal{\unicode{0009}{character tabulation}, \unicode{000b}{line tabulation}, \unicode{000c}{form feed}, and new-line}
\end{bnf}

\pnum
\indextext{literal!string}%
\indextext{character string}%
\indextext{string!type of}%
\indextext{type!\idxcode{wchar_t}}%
\indextext{prefix!\idxcode{L}}%
\indextext{literal!string!\idxcode{char16_t}}%
\indextext{type!\idxcode{char16_t}}%
\indextext{literal!string!\idxcode{char32_t}}%
\indextext{type!\idxcode{char32_t}}%
The kind of a \grammarterm{string-literal},
its type, and
its associated character encoding\iref{lex.charset}
are determined by its encoding prefix and sequence of
\grammarterm{s-char}s or \grammarterm{r-char}s
as defined by \tref{lex.string.literal}
where $n$ is the number of encoded code units as described below.

\begin{floattable}{String literals}{lex.string.literal}
{llp{2.6cm}p{2.3cm}p{4.7cm}}
\topline
\lhdr{Enco-} & \chdr{Kind} & \chdr{Type} & \chdr{Associated} & \rhdr{Examples} \\
\lhdr{ding}   & \chdr{} & \chdr{} & \chdr{character}  & \rhdr{} \\
\lhdr{prefix}         & \chdr{} & \chdr{} & \chdr{encoding}   & \rhdr{} \\
\capsep
none &
\defnx{ordinary string literal}{literal!string!ordinary} &
array of $n$\newline \tcode{\keyword{const} \keyword{char}} &
ordinary literal encoding &
\tcode{"ordinary string"}\newline
\tcode{R"(ordinary raw string)"} \\
\tcode{L} &
\defnx{wide string literal}{literal!string!wide} &
array of $n$\newline \tcode{\keyword{const} \keyword{wchar_t}} &
wide literal\newline encoding &
\tcode{L"wide string"}\newline
\tcode{LR"w(wide raw string)w"} \\
\tcode{u8} &
\defnx{UTF-8 string literal}{literal!string!UTF-8} &
array of $n$\newline \tcode{\keyword{const} \keyword{char8_t}} &
UTF-8 &
\tcode{u8"UTF-8 string"}\newline
\tcode{u8R"x(UTF-8 raw string)x"} \\
\tcode{u} &
\defnx{UTF-16 string literal}{literal!string!UTF-16} &
array of $n$\newline \tcode{\keyword{const} \keyword{char16_t}} &
UTF-16 &
\tcode{u"UTF-16 string"}\newline
\tcode{uR"y(UTF-16 raw string)y"} \\
\tcode{U} &
\defnx{UTF-32 string literal}{literal!string!UTF-32} &
array of $n$\newline \tcode{\keyword{const} \keyword{char32_t}} &
UTF-32 &
\tcode{U"UTF-32 string"}\newline
\tcode{UR"z(UTF-32 raw string)z"} \\
\end{floattable}

\pnum
\indextext{literal!string!raw}%
A \grammarterm{string-literal} that has an \tcode{R}
\indextext{prefix!\idxcode{R}}%
in the prefix is a \defn{raw string literal}. The
\grammarterm{d-char-sequence} serves as a delimiter. The terminating
\grammarterm{d-char-sequence} of a \grammarterm{raw-string} is the same sequence of
characters as the initial \grammarterm{d-char-sequence}. A \grammarterm{d-char-sequence}
shall consist of at most 16 characters.

\pnum
\begin{note}
The characters \tcode{'('} and \tcode{')'} can appear in a
\grammarterm{raw-string}. Thus, \tcode{R"delimiter((a|b))delimiter"} is equivalent to
\tcode{"(a|b)"}.
\end{note}

\pnum
\begin{note}
A source-file new-line in a raw string literal results in a new-line in the
resulting execution string literal. Assuming no
whitespace at the beginning of lines in the following example, the assert will succeed:
\begin{codeblock}
const char* p = R"(a\
b
c)";
assert(std::strcmp(p, "a\\\nb\nc") == 0);
\end{codeblock}
\end{note}

\pnum
\begin{example}
The raw string
\begin{codeblock}
R"a(
)\
a"
)a"
\end{codeblock}
is equivalent to \tcode{"\textbackslash n)\textbackslash \textbackslash \textbackslash na\textbackslash"\textbackslash n"}. The raw string
\begin{codeblock}
R"(x = "\"y\"")"
\end{codeblock}
is equivalent to \tcode{"x = \textbackslash "\textbackslash\textbackslash\textbackslash "y\textbackslash\textbackslash\textbackslash "\textbackslash ""}.
\end{example}

\pnum
\indextext{literal!narrow-character}%
Ordinary string literals and UTF-8 string literals are
also referred to as \defnx{narrow string literals}{literal!string!narrow}.

\pnum
\indextext{concatenation!string}%
The common \grammarterm{encoding-prefix}
for a sequence of adjacent \grammarterm{string-literal}s
is determined pairwise as follows.
If two \grammarterm{string-literal}{s} have
the same \grammarterm{encoding-prefix},
the common \grammarterm{encoding-prefix} is that \grammarterm{encoding-prefix}.
If one \grammarterm{string-literal} has no \grammarterm{encoding-prefix},
the common \grammarterm{encoding-prefix} is that
of the other \grammarterm{string-literal}.
Any other combinations are ill-formed.
\begin{note}
A \grammarterm{string-literal}'s rawness has
no effect on the determination of the common \grammarterm{encoding-prefix}.
\end{note}

\pnum
In translation phase 6\iref{lex.phase.6},
adjacent \grammarterm{string-literal}s are concatenated.
The lexical structure and grouping of
the contents of the individual \grammarterm{string-literal}s is retained.
\begin{example}
\begin{codeblock}
"\xA" "B"
\end{codeblock}
represents
the code unit \tcode{'\textbackslash xA'} and the character \tcode{'B'}
after concatenation
(and not the single code unit \tcode{'\textbackslash xAB'}).
Similarly,
\begin{codeblock}
R"(\u00)" "41"
\end{codeblock}
represents six characters,
starting with a backslash and ending with the digit \tcode{1}
(and not the single character \tcode{'A'}
specified by a \grammarterm{universal-character-name}).

\tref{lex.string.concat} has some examples of valid concatenations.
\end{example}

\begin{floattable}{String literal concatenations}{lex.string.concat}
{lll|lll|lll}
\topline
\multicolumn{2}{|c}{Source} &
Means &
\multicolumn{2}{c}{Source} &
Means &
\multicolumn{2}{c}{Source} &
Means \\
\tcode{u"a"} & \tcode{u"b"} & \tcode{u"ab"} &
\tcode{U"a"} & \tcode{U"b"} & \tcode{U"ab"} &
\tcode{L"a"} & \tcode{L"b"} & \tcode{L"ab"} \\
\tcode{u"a"} & \tcode{"b"}  & \tcode{u"ab"} &
\tcode{U"a"} & \tcode{"b"}  & \tcode{U"ab"} &
\tcode{L"a"} & \tcode{"b"}  & \tcode{L"ab"} \\
\tcode{"a"}  & \tcode{u"b"} & \tcode{u"ab"} &
\tcode{"a"}  & \tcode{U"b"} & \tcode{U"ab"} &
\tcode{"a"}  & \tcode{L"b"} & \tcode{L"ab"} \\
\end{floattable}

\pnum
Evaluating a \grammarterm{string-literal} results in a string literal object
with static storage duration\iref{basic.stc}.
\begin{note}
String literal objects are potentially non-unique\iref{intro.object}.
Whether successive evaluations of a
\grammarterm{string-literal} yield the same or a different object is
unspecified.
\end{note}
\begin{note}
\indextext{literal!string!undefined change to}%
The effect of attempting to modify a string literal object is undefined.
\end{note}

\pnum
\indextext{\idxcode{0}!string terminator}%
\indextext{\idxcode{0}!null character|see {character, null}}%
String literal objects are initialized with
the sequence of code unit values
corresponding to the \grammarterm{string-literal}'s sequence of
\grammarterm{s-char}s (originally from non-raw string literals) and
\grammarterm{r-char}s (originally from raw string literals),
plus a terminating \unicode{0000}{null} character,
in order as follows:
\begin{itemize}
\item
The sequence of characters denoted by each contiguous sequence of
\grammarterm{basic-s-char}s,
\grammarterm{r-char}s,
\grammarterm{simple-escape-sequence}s\iref{lex.ccon}, and
\grammarterm{universal-character-name}s\iref{lex.charset}
is encoded to a code unit sequence
using the \grammarterm{string-literal}'s associated character encoding.
If a character lacks representation in the associated character encoding,
then the program is ill-formed.
\begin{note}
No character lacks representation in any Unicode encoding form.
\end{note}
When encoding a stateful character encoding,
implementations should encode the first such sequence
beginning with the initial encoding state and
encode subsequent sequences
beginning with the final encoding state of the prior sequence.
\begin{note}
The encoded code unit sequence can differ from
the sequence of code units that would be obtained by
encoding each character independently.
\end{note}
\item
Each \grammarterm{numeric-escape-sequence}\iref{lex.ccon}
contributes a single code unit with a value as follows:
\begin{itemize}
\item
Let $v$ be the integer value represented by
the octal number comprising
the sequence of \grammarterm{octal-digit}{s} in
an \grammarterm{octal-escape-sequence} or by
the hexadecimal number comprising
the sequence of \grammarterm{hexadecimal-digit}{s} in
a \grammarterm{hexadecimal-escape-sequence}.
\item
If $v$ does not exceed the range of representable values of
the \grammarterm{string-literal}'s array element type,
then the value is $v$.
\item
Otherwise,
if the \grammarterm{string-literal}'s \grammarterm{encoding-prefix}
is absent or \tcode{L}, and
$v$ does not exceed the range of representable values of
the corresponding unsigned type for the underlying type of
the \grammarterm{string-literal}'s array element type,
then the value is the unique value of
the \grammarterm{string-literal}'s array element type \tcode{T}
that is congruent to $v$ modulo $2^N$, where $N$ is the width of \tcode{T}.
\item
Otherwise, the program is ill-formed.
\end{itemize}
When encoding a stateful character encoding,
these sequences should have no effect on encoding state.
\item
Each \grammarterm{conditional-escape-sequence}\iref{lex.ccon}
contributes an
\impldef{code unit sequence for \grammarterm{conditional-escape-sequence}}
code unit sequence.
When encoding a stateful character encoding,
it is
\impldef{effect of \grammarterm{conditional-escape-sequence} on encoding state}
what effect these sequences have on encoding state.
\end{itemize}

\rSec2[lex.operators]{Operators and punctuators}

\pnum
\indextext{operator|(}%
\indextext{punctuator|(}%
The lexical representation of \Cpp{} programs includes a number of
preprocessing tokens that are used in the syntax of the preprocessor or
are converted into tokens for operators and punctuators:

\begin{bnf}
\nontermdef{preprocessing-op-or-punc}\br
    preprocessing-operator\br
    operator-or-punctuator
\end{bnf}

\begin{bnf}
%% Ed. note: character protrusion would misalign various operators.
\microtypesetup{protrusion=false}\obeyspaces
\nontermdef{preprocessing-operator} \textnormal{one of}\br
    \terminal{\#        \#\#       \%:       \%:\%:}
\end{bnf}

\begin{bnf}
\microtypesetup{protrusion=false}\obeyspaces
\nontermdef{operator-or-punctuator} \textnormal{one of}\br
    \terminal{\{        \}        [        ]        (        )}\br
    \terminal{<:       :>       <\%       \%>       ;        :        ...}\br
    \terminal{?        ::       .        .*       ->       ->*      \~}\br
    \terminal{!        +        -        *        /        \%        \caret{}        \&        |}\br
    \terminal{=        +=       -=       *=       /=       \%=       \caret{}=       \&=       |=}\br
    \terminal{==       !=       <        >        <=       >=       <=>      \&\&       ||}\br
    \terminal{<<       >>       <<=      >>=      ++       --       ,}\br
    \terminal{\keyword{and}      \keyword{or}       \keyword{xor}      \keyword{not}      \keyword{bitand}   \keyword{bitor}    \keyword{compl}}\br
    \terminal{\keyword{and_eq}   \keyword{or_eq}    \keyword{xor_eq}   \keyword{not_eq}}
\end{bnf}

Each \grammarterm{operator-or-punctuator} is converted to a single token
in translation phase 7\iref{lex.phase.7}.%
\indextext{punctuator|)}%
\indextext{operator|)}

\rSec2[lex.name]{Identifiers}

\indextext{identifier|(}%
\begin{bnf}
\nontermdef{identifier}\br
    identifier-start\br
    identifier identifier-continue
\end{bnf}

\begin{bnf}
\nontermdef{identifier-start}\br
    nondigit\br
    \textnormal{an element of the translation character set with the Unicode property XID_Start}
\end{bnf}

\begin{bnf}
\nontermdef{identifier-continue}\br
    digit\br
    nondigit\br
    \textnormal{an element of the translation character set with the Unicode property XID_Continue}
\end{bnf}

\begin{bnf}
\nontermdef{nondigit} \textnormal{one of}\br
    \terminal{a b c d e f g h i j k l m}\br
    \terminal{n o p q r s t u v w x y z}\br
    \terminal{A B C D E F G H I J K L M}\br
    \terminal{N O P Q R S T U V W X Y Z _}
\end{bnf}

\begin{bnf}
\nontermdef{digit} \textnormal{one of}\br
    \terminal{0 1 2 3 4 5 6 7 8 9}
\end{bnf}

\pnum
\indextext{name!length of}%
\indextext{name}%
\begin{note}
The character properties XID_Start and XID_Continue are Derived Core Properties
as described by \UAX{44} of the Unicode Standard.
\begin{footnote}
On systems in which linkers cannot accept extended
characters, an encoding of the \grammarterm{universal-character-name} can be used in
forming valid external identifiers. For example, some otherwise unused
character or sequence of characters can be used to encode the
\tcode{\textbackslash u} in a \grammarterm{universal-character-name}. Extended
characters can produce a long external identifier, but \Cpp{} does not
place a translation limit on significant characters for external
identifiers.
\end{footnote}
\end{note}
The program is ill-formed
if an \grammarterm{identifier} does not conform to
Normalization Form C as specified in the Unicode Standard.
\begin{note}
Identifiers are case-sensitive.
\end{note}
\begin{note}
\ref{uaxid} compares the requirements of \UAX{31} of the Unicode Standard
with the \Cpp{} rules for identifiers.
\end{note}
\begin{note}
In translation phase 4,
\grammarterm{identifier} also includes
those \grammarterm{preprocessing-token}s\iref{lex.pptoken}
differentiated as keywords\iref{lex.key}
in the later translation phase 7\iref{lex.token}.
\end{note}

\pnum
\indextext{\idxcode{import}}%
\indextext{\idxcode{final}}%
\indextext{\idxcode{module}}%
\indextext{\idxcode{override}}%
The identifiers in \tref{lex.name.special} have a special meaning when
appearing in a certain context. When referred to in the grammar, these identifiers
are used explicitly rather than using the \grammarterm{identifier} grammar production.
Unless otherwise specified, any ambiguity as to whether a given
\grammarterm{identifier} has a special meaning is resolved to interpret the
token as a regular \grammarterm{identifier}.

\begin{multicolfloattable}{Identifiers with special meaning}{lex.name.special}
{llll}
\keyword{final}           \\
\columnbreak
\keyword{import}          \\
\columnbreak
\keyword{module}          \\
\columnbreak
\keyword{override}        \\
\end{multicolfloattable}

\pnum
\indextext{\idxcode{_}|see{character, underscore}}%
\indextext{character!underscore!in identifier}%
\indextext{reserved identifier}%
In addition, some identifiers
appearing as a \grammarterm{token} or \grammarterm{preprocessing-token}
are reserved for use by \Cpp{}
implementations and shall
not be used otherwise; no diagnostic is required.
\begin{itemize}
\item
Each identifier that contains a double underscore
\tcode{\unun}
\indextext{character!underscore}%
or begins with an underscore followed by
an uppercase letter
\indextext{uppercase}%
is reserved to the implementation for any use.
\item
Each identifier that begins with an underscore is
\indextext{character!underscore}%
reserved to the implementation for use as a name in the global namespace.%
\indextext{namespace!global}
\end{itemize}%
\indextext{identifier|)}

\rSec1[cpp]{Preprocessing directives}%
\indextext{preprocessing directive|(}

\indextext{compiler control line|see{preprocessing directive}}%
\indextext{control line|see{preprocessing directive}}%
\indextext{directive, preprocessing|see{preprocessing directive}}

\gramSec[gram.cpp]{Preprocessing directives}

\rSec2[cpp.pre]{Preamble}

\begin{bnf}
\nontermdef{preprocessing-file}\br
    \opt{group}\br
    module-file
\end{bnf}

\begin{bnf}
\nontermdef{module-file}\br
    \opt{pp-global-module-fragment} pp-module \opt{group} \opt{pp-private-module-fragment}
\end{bnf}

\begin{bnf}
\nontermdef{pp-global-module-fragment}\br
    \keyword{module} \terminal{;} new-line \opt{group}
\end{bnf}

\begin{bnf}
\nontermdef{pp-private-module-fragment}\br
    \keyword{module} \terminal{:} \keyword{private} \terminal{;} new-line \opt{group}
\end{bnf}

\begin{bnf}
\nontermdef{group}\br
    group-part\br
    group group-part
\end{bnf}

\begin{bnf}
\nontermdef{group-part}\br
    control-line\br
    if-section\br
    text-line\br
    \terminal{\#} conditionally-supported-directive
\end{bnf}

\begin{bnf}\obeyspaces
\nontermdef{control-line}\br
    \terminal{\# include} pp-tokens new-line\br
    pp-import\br
    \terminal{\# define } identifier replacement-list new-line\br
    \terminal{\# define } identifier lparen \opt{identifier-list} \terminal{)} replacement-list new-line\br
    \terminal{\# define } identifier lparen \terminal{... )} replacement-list new-line\br
    \terminal{\# define } identifier lparen identifier-list \terminal{, ... )} replacement-list new-line\br
    \terminal{\# undef  } identifier new-line\br
    \terminal{\# line   } pp-tokens new-line\br
    \terminal{\# error  } \opt{pp-tokens} new-line\br
    \terminal{\# warning} \opt{pp-tokens} new-line\br
    \terminal{\# pragma } \opt{pp-tokens} new-line\br
    \terminal{\# }new-line
\end{bnf}

\begin{bnf}
\nontermdef{if-section}\br
    if-group \opt{elif-groups} \opt{else-group} endif-line
\end{bnf}

\begin{bnf}\obeyspaces
\nontermdef{if-group}\br
    \terminal{\# if     } constant-expression new-line \opt{group}\br
    \terminal{\# ifdef  } identifier new-line \opt{group}\br
    \terminal{\# ifndef } identifier new-line \opt{group}
\end{bnf}

\begin{bnf}
\nontermdef{elif-groups}\br
    elif-group\br
    elif-groups elif-group
\end{bnf}

\begin{bnf}\obeyspaces
\nontermdef{elif-group}\br
    \terminal{\# elif    } constant-expression new-line \opt{group}\br
    \terminal{\# elifdef } identifier new-line \opt{group}\br
    \terminal{\# elifndef} identifier new-line \opt{group}
\end{bnf}

\begin{bnf}\obeyspaces
\nontermdef{else-group}\br
    \terminal{\# else   } new-line \opt{group}
\end{bnf}

\begin{bnf}\obeyspaces
\nontermdef{endif-line}\br
    \terminal{\# endif  } new-line
\end{bnf}

\begin{bnf}
\nontermdef{text-line}\br
    \opt{pp-tokens} new-line
\end{bnf}

\begin{bnf}
\nontermdef{conditionally-supported-directive}\br
    pp-tokens new-line
\end{bnf}

\begin{bnf}
\nontermdef{lparen}\br
    \descr{a \terminal{(} character not immediately preceded by whitespace}
\end{bnf}

\begin{bnf}
\nontermdef{identifier-list}\br
    identifier\br
    identifier-list \terminal{,} identifier
\end{bnf}

\begin{bnf}
\nontermdef{replacement-list}\br
    \opt{pp-tokens}
\end{bnf}

\begin{bnf}
\nontermdef{pp-tokens}\br
    preprocessing-token\br
    pp-tokens preprocessing-token
\end{bnf}

\begin{bnf}
\nontermdef{new-line}\br
    \descr{the new-line character}
\end{bnf}

\pnum
A \defn{preprocessing directive} consists of a sequence of preprocessing tokens
that satisfies the following constraints:
At the start of translation phase 4,
the first token in the sequence,
referred to as a \defnadj{directive-introducing}{token},
begins with the first character in the source file
(optionally after whitespace containing no new-line characters) or
follows whitespace containing at least one new-line character,
and is

\begin{itemize}
\item
a \tcode{\#} preprocessing token, or

\item
an \keyword{import} preprocessing token
immediately followed on the same logical line by a
\grammarterm{header-name},
\tcode{<},
\grammarterm{identifier},
\grammarterm{string-literal}, or
\tcode{:}
preprocessing token, or

\item
a \keyword{module} preprocessing token
immediately followed on the same logical line by an
\grammarterm{identifier},
\tcode{:}, or
\tcode{;}
preprocessing token, or

\item
an \keyword{export} preprocessing token
immediately followed on the same logical line by
one of the two preceding forms.
\end{itemize}

The last token in the sequence is the first token within the sequence that
is immediately followed by whitespace containing a new-line character.
\begin{footnote}
Thus,
preprocessing directives are commonly called ``lines''.
These ``lines'' have no other syntactic significance,
as all whitespace is equivalent except in certain situations
during preprocessing (see the
\tcode{\#}
character string literal creation operator in~\ref{cpp.stringize}, for example).
\end{footnote}
\begin{note}
A new-line character ends the preprocessing directive even if it occurs
within what would otherwise be an invocation of a function-like macro.
\end{note}

\begin{example}
\begin{codeblock}
#                       // preprocessing directive
module ;                // preprocessing directive
export module leftpad;  // preprocessing directive
import <string>;        // preprocessing directive
export import "squee";  // preprocessing directive
import rightpad;        // preprocessing directive
import :part;           // preprocessing directive

module                  // not a preprocessing directive
;                       // not a preprocessing directive

export                  // not a preprocessing directive
import                  // not a preprocessing directive
foo;                    // not a preprocessing directive

export                  // not a preprocessing directive
import foo;             // preprocessing directive (ill-formed at phase 7)

import ::               // not a preprocessing directive
import ->               // not a preprocessing directive
\end{codeblock}
\end{example}

\pnum
A sequence of preprocessing tokens is only a \grammarterm{text-line}
if it does not begin with a directive-introducing token.
A sequence of preprocessing tokens is only a \grammarterm{conditionally-supported-directive}
if it does not begin with any of the directive names
appearing after a \tcode{\#} in the syntax.
A \grammarterm{conditionally-supported-directive} is
conditionally-supported with
\impldef{additional supported forms of preprocessing directive}
semantics.

\pnum
At the start of phase 4 of translation,
the \grammarterm{group} of a \grammarterm{pp-global-module-fragment} shall
contain neither a \grammarterm{text-line} nor a \grammarterm{pp-import}.

\pnum
When in a group that is skipped\iref{cpp.cond}, the directive
syntax is relaxed to allow any sequence of preprocessing tokens to occur between
the directive name and the following new-line character.

\pnum
The only whitespace characters that shall appear
between preprocessing tokens
within a preprocessing directive
(from just after the directive-introducing token
through just before the terminating new-line character)
are space and horizontal-tab
(including spaces that have replaced comments
or possibly other whitespace characters
in translation phase 3).

\pnum
The implementation can
process and skip sections of source files conditionally,
include other source files,
import macros from header units,
and replace macros.
These capabilities are called
\defn{preprocessing},
because conceptually they occur
before translation of the resulting translation unit.

\pnum
The preprocessing tokens within a preprocessing directive
are not subject to macro expansion unless otherwise stated.

\begin{example}
In:
\begin{codeblock}
#define EMPTY
EMPTY   #   include <file.h>
\end{codeblock}
the sequence of preprocessing tokens on the second line is \textit{not}
a preprocessing directive, because it does not begin with a \tcode{\#} at the start of
translation phase 4, even though it will do so after the macro \tcode{EMPTY}
has been replaced.
\end{example}

\rSec2[cpp.module]{Module directive}
\indextext{preprocessing directive!module}%

\begin{bnf}
\nontermdef{pp-module}\br
    \opt{\keyword{export}} \keyword{module} \opt{pp-tokens} \terminal{;} new-line
\end{bnf}

\pnum
A \grammarterm{pp-module} shall not
appear in a context where \tcode{module}
or (if it is the first token of the \grammarterm{pp-module}) \tcode{export}
is an identifier defined as an object-like macro.

\pnum
The \grammarterm{pp-tokens}, if any, of a \grammarterm{pp-module}
shall be of the form:
\begin{ncsimplebnf}
pp-module-name \opt{pp-module-partition} \opt{pp-tokens}
\end{ncsimplebnf}
where the \grammarterm{pp-tokens} (if any) shall not begin with
a \tcode{(} preprocessing token and
the grammar non-terminals are defined as:
\begin{ncbnf}
\nontermdef{pp-module-name}\br
    \opt{pp-module-name-qualifier} identifier
\end{ncbnf}
\begin{ncbnf}
\nontermdef{pp-module-partition}\br
    \terminal{:} \opt{pp-module-name-qualifier} identifier
\end{ncbnf}
\begin{ncbnf}
\nontermdef{pp-module-name-qualifier}\br
    identifier \terminal{.}\br
    pp-module-name-qualifier identifier \terminal{.}
\end{ncbnf}
No \grammarterm{identifier} in
the \grammarterm{pp-module-name} or \grammarterm{pp-module-partition}
shall currently be defined as an object-like macro.

\pnum
Any preprocessing tokens after the \tcode{module} preprocessing token
in the \tcode{module} directive are processed just as in normal text.
\begin{note}
Each identifier currently defined as a macro name
is replaced by its replacement list of preprocessing tokens.
\end{note}

\pnum
The \tcode{module} and \tcode{export} (if it exists) preprocessing tokens
are replaced by the \grammarterm{module-keyword} and
\grammarterm{export-keyword} preprocessing tokens respectively.
\begin{note}
This makes the line no longer a directive
so it is not removed at the end of phase 4.
\end{note}

\rSec2[cpp.null]{Null directive}%
\indextext{preprocessing directive!null}

\pnum
A preprocessing directive of the form
\begin{ncsimplebnf}
\terminal{\#} new-line
\end{ncsimplebnf}
has no effect.

\rSec2[cpp.cond]{Conditional inclusion}%
\indextext{preprocessing directive!conditional inclusion}%
\indextext{inclusion!conditional|see{preprocessing directive, conditional inclusion}}

\indextext{\idxcode{defined}}%
\begin{bnf}
\nontermdef{defined-macro-expression}\br
    \terminal{defined} identifier\br
    \terminal{defined (} identifier \terminal{)}
\end{bnf}

\begin{bnf}
\nontermdef{h-preprocessing-token}\br
    \textnormal{any \grammarterm{preprocessing-token} other than \terminal{>}}
\end{bnf}

\begin{bnf}
\nontermdef{h-pp-tokens}\br
    h-preprocessing-token\br
    h-pp-tokens h-preprocessing-token
\end{bnf}

\begin{bnf}
\nontermdef{header-name-tokens}\br
    string-literal\br
    \terminal{<} h-pp-tokens \terminal{>}
\end{bnf}

\indextext{\idxxname{has_include}}%
\begin{bnf}
\nontermdef{has-include-expression}\br
    \terminal{\xname{has_include}} \terminal{(} header-name \terminal{)}\br
    \terminal{\xname{has_include}} \terminal{(} header-name-tokens \terminal{)}
\end{bnf}

\indextext{\idxxname{has_cpp_attribute}}%
\begin{bnf}
\nontermdef{has-attribute-expression}\br
    \terminal{\xname{has_cpp_attribute} (} pp-tokens \terminal{)}
\end{bnf}

\pnum
The expression that controls conditional inclusion
shall be an integral constant expression except that
identifiers
(including those lexically identical to keywords)
are interpreted as described below
\begin{footnote}
Because the controlling constant expression is evaluated
during translation phase 4,
all identifiers either are or are not macro names ---
there simply are no keywords, enumeration constants, etc.
\end{footnote}
and it may contain zero or more \grammarterm{defined-macro-expression}{s} and/or
\grammarterm{has-include-expression}{s} and/or
\grammarterm{has-attribute-expression}{s} as unary operator expressions.

\pnum
A \grammarterm{defined-macro-expression} evaluates to \tcode{1}
if the identifier is currently defined
as a macro name
(that is, if it is predefined
or if it has one or more active macro definitions\iref{cpp.import},
for example because
it has been the subject of a
\tcode{\#define}
preprocessing directive
without an intervening
\tcode{\#undef}
directive with the same subject identifier), \tcode{0} if it is not.

\pnum
The second form of \grammarterm{has-include-expression}
is considered only if the first form does not match,
in which case the preprocessing tokens are processed just as in normal text.

\pnum
The header or source file identified by
the parenthesized preprocessing token sequence
in each contained \grammarterm{has-include-expression}
is searched for as if that preprocessing token sequence
were the \grammarterm{pp-tokens} in a \tcode{\#include} directive,
except that no further macro expansion is performed.
If such a directive would not satisfy the syntactic requirements
of a \tcode{\#include} directive, the program is ill-formed.
The \grammarterm{has-include-expression} evaluates
to \tcode{1} if the search for the source file succeeds, and
to \tcode{0} if the search fails.

\pnum
Each \grammarterm{has-attribute-expression} is replaced by
a non-zero \grammarterm{pp-number}
matching the form of an \grammarterm{integer-literal}
if the implementation supports an attribute
with the name specified by interpreting
the \grammarterm{pp-tokens}, after macro expansion,
as an \grammarterm{attribute-token},
and by \tcode{0} otherwise.
The program is ill-formed if the \grammarterm{pp-tokens}
do not match the form of an \grammarterm{attribute-token}.

\pnum
For an attribute specified in this document,
it is \impldef{value of \grammarterm{has-attribute-expression}
for standard attributes}
whether the value of the \grammarterm{has-attribute-expression}
is \tcode{0} or is given by \tref{cpp.cond.ha}.
For other attributes recognized by the implementation,
the value is
\impldef{value of \grammarterm{has-attribute-expression}
for non-standard attributes}.
\begin{note}
It is expected
that the availability of an attribute can be detected by any non-zero result.
\end{note}

\begin{floattable}{\xname{has_cpp_attribute} values}{cpp.cond.ha}
{ll}
\topline
\lhdr{Attribute} & \rhdr{Value} \\ \rowsep
\tcode{assume}                & \tcode{202207L} \\
\tcode{carries_dependency}    & \tcode{200809L} \\
\tcode{deprecated}            & \tcode{201309L} \\
\tcode{fallthrough}           & \tcode{201603L} \\
\tcode{likely}                & \tcode{201803L} \\
\tcode{maybe_unused}          & \tcode{201603L} \\
\tcode{no_unique_address}     & \tcode{201803L} \\
\tcode{nodiscard}             & \tcode{201907L} \\
\tcode{noreturn}              & \tcode{200809L} \\
\tcode{unlikely}              & \tcode{201803L} \\
\end{floattable}

\pnum
The
\tcode{\#ifdef}, \tcode{\#ifndef}, \tcode{\#elifdef}, and \tcode{\#elifndef}
directives, and
the \tcode{defined} conditional inclusion operator,
shall treat \xname{has_include} and \xname{has_cpp_attribute}
as if they were the names of defined macros.
The identifiers \xname{has_include} and \xname{has_cpp_attribute}
shall not appear in any context not mentioned in this subclause.

\pnum
Each preprocessing token that remains (in the list of preprocessing tokens that
will become the controlling expression)
after all macro replacements have occurred
shall be in the lexical form of a token\iref{lex.token}.

\pnum
Preprocessing directives of the forms
\begin{ncsimplebnf}\obeyspaces
\indextext{\idxcode{\#if}}%
\terminal{\# if     } constant-expression new-line \opt{group}\br
\indextext{\idxcode{\#elif}}%
\terminal{\# elif   } constant-expression new-line \opt{group}
\end{ncsimplebnf}
check whether the controlling constant expression evaluates to nonzero.

\pnum
Prior to evaluation,
macro invocations in the list of preprocessing tokens
that will become the controlling constant expression
are replaced
(except for those macro names modified by the
\tcode{defined}
unary operator),
just as in normal text.
If the token
\tcode{defined}
is generated as a result of this replacement process
or use of the
\tcode{defined}
unary operator does not match one of the two specified forms
prior to macro replacement,
the behavior is undefined.

\pnum
After all replacements due to macro expansion and
evaluations of
\grammarterm{defined-macro-expression}s,
\grammarterm{has-include-expression}s, and
\grammarterm{has-attribute-expression}s
have been performed,
all remaining identifiers and keywords,
except for
\tcode{true}
and
\tcode{false},
are replaced with the \grammarterm{pp-number}
\tcode{0},
and then each preprocessing token is converted into a token.
\begin{note}
An alternative
token\iref{lex.digraph} is not an identifier,
even when its spelling consists entirely of letters and underscores.
Therefore it is not subject to this replacement.
\end{note}

\pnum
The resulting tokens comprise the controlling constant expression
which is evaluated according to the rules of~\ref{expr.const}
using arithmetic that has at least the ranges specified
in~\ref{support.limits}. For the purposes of this token conversion and evaluation
all signed and unsigned integer types
act as if they have the same representation as, respectively,
\tcode{intmax_t} or \tcode{uintmax_t}\iref{cstdint.syn}.
\begin{note}
Thus on an
implementation where \tcode{std::numeric_limits<int>::max()} is \tcode{0x7FFF}
and \tcode{std::numeric_limits<unsigned int>::max()} is \tcode{0xFFFF},
the integer literal \tcode{0x8000} is signed and positive within a \tcode{\#if}
expression even though it is unsigned in translation phase
7\iref{lex.phases}.
\end{note}
This includes interpreting \grammarterm{character-literal}s
according to the rules in \ref{lex.ccon}.
\begin{note}
The associated character encodings of literals are the same
in \tcode{\#if} and \tcode{\#elif} directives and in any expression.
\end{note}
Each subexpression with type
\tcode{bool}
is subjected to integral promotion before processing continues.

\pnum
Preprocessing directives of the forms
\begin{ncsimplebnf}\obeyspaces
\terminal{\# ifdef   } identifier new-line \opt{group}\br
\indextext{\idxcode{\#ifdef}}%
\terminal{\# ifndef  } identifier new-line \opt{group}\br
\indextext{\idxcode{\#ifndef}}%
\terminal{\# elifdef } identifier new-line \opt{group}\br
\indextext{\idxcode{\#elifdef}}%
\terminal{\# elifndef} identifier new-line \opt{group}
\indextext{\idxcode{\#elifndef}}%
\end{ncsimplebnf}
check whether the identifier is or is not currently defined as a macro name.
Their conditions are equivalent to
\tcode{\#if} \tcode{defined} \grammarterm{identifier},
\tcode{\#if} \tcode{!defined} \grammarterm{identifier},
\tcode{\#elif} \tcode{defined} \grammarterm{identifier}, and
\tcode{\#elif} \tcode{!defined} \grammarterm{identifier},
respectively.

\pnum
Each directive's condition is checked in order.
If it evaluates to false (zero),
the group that it controls is skipped:
directives are processed only through the name that determines
the directive in order to keep track of the level
of nested conditionals;
the rest of the directives' preprocessing tokens are ignored,
as are the other preprocessing tokens in the group.
Only the first group
whose control condition evaluates to true (nonzero) is processed;
any following groups are skipped and their controlling directives
are processed as if they were in a group that is skipped.
If none of the conditions evaluates to true,
and there is a
\tcode{\#else}
\indextext{\idxcode{\#else}}%
directive,
the group controlled by the
\tcode{\#else}
is processed; lacking a
\tcode{\#else}
directive, all the groups until the
\tcode{\#endif}
\indextext{\idxcode{\#endif}}%
are skipped.%
\begin{footnote}
As indicated by the syntax,
a preprocessing token cannot follow a
\tcode{\#else}
or
\tcode{\#endif}
directive before the terminating new-line character.
However,
comments can appear anywhere in a source file,
including within a preprocessing directive.
\end{footnote}

\pnum
\begin{example}
This demonstrates a way to include a library \tcode{optional} facility
only if it is available:

\begin{codeblock}
#if __has_include(<optional>)
#  include <optional>
#  if __cpp_lib_optional >= 201603
#    define have_optional 1
#  endif
#elif __has_include(<experimental/optional>)
#  include <experimental/optional>
#  if __cpp_lib_experimental_optional >= 201411
#    define have_optional 1
#    define experimental_optional 1
#  endif
#endif
#ifndef have_optional
#  define have_optional 0
#endif
\end{codeblock}
\end{example}

\pnum
\begin{example}
This demonstrates a way to use the attribute \tcode{[[acme::deprecated]]}
only if it is available.
\begin{codeblock}
#if __has_cpp_attribute(acme::deprecated)
#  define ATTR_DEPRECATED(msg) [[acme::deprecated(msg)]]
#else
#  define ATTR_DEPRECATED(msg) [[deprecated(msg)]]
#endif
ATTR_DEPRECATED("This function is deprecated") void anvil();
\end{codeblock}
\end{example}

\rSec2[cpp.include]{Source file inclusion}
\indextext{preprocessing directive!header inclusion}
\indextext{preprocessing directive!source-file inclusion}
\indextext{inclusion!source file|see{preprocessing directive, source-file inclusion}}%
\indextext{\idxcode{\#include}}%

\pnum
A
\tcode{\#include}
directive shall identify a header or source file
that can be processed by the implementation.

\pnum
A preprocessing directive of the form
\begin{ncsimplebnf}
\terminal{\# include <} h-char-sequence \terminal{>} new-line
\end{ncsimplebnf}
searches a sequence of
\impldef{sequence of places searched for a header}
places
for a header identified uniquely by the specified sequence
between the
\tcode{<}
and
\tcode{>}
delimiters,
and causes the replacement of that
directive by the entire contents of the header.
How the places are specified
or the header identified
is \impldef{search locations for \tcode{<>} header}.

\pnum
A preprocessing directive of the form
\begin{ncsimplebnf}
\terminal{\# include "} q-char-sequence \terminal{"} new-line
\end{ncsimplebnf}
causes the replacement of that
directive by the entire contents of the
source file identified by the specified sequence between the
\tcode{"}
delimiters.
The named source file is searched for in an
\impldef{manner of search for included source file}
manner.
If this search is not supported,
or if the search fails,
the directive is reprocessed as if it read
\begin{ncsimplebnf}
\terminal{\# include <} h-char-sequence \terminal{>} new-line
\end{ncsimplebnf}
with the identical contained sequence (including
\tcode{>}
characters, if any) from the original directive.

\pnum
A preprocessing directive of the form
\begin{ncsimplebnf}
\terminal{\# include} pp-tokens new-line
\end{ncsimplebnf}
(that does not match one of the two previous forms) is permitted.
The preprocessing tokens after
\tcode{include}
in the directive are processed just as in normal text
(i.e., each identifier currently defined as a macro name is replaced by its
replacement list of preprocessing tokens).
If the directive resulting after all replacements does not match
one of the two previous forms, the behavior is
undefined.
\begin{footnote}
Note that adjacent \grammarterm{string-literal}s are not concatenated into
a single \grammarterm{string-literal}
(see the translation phases in~\ref{lex.phases});
thus, an expansion that results in two \grammarterm{string-literal}s is an
invalid directive.
\end{footnote}
The method by which a sequence of preprocessing tokens between a
\tcode{<}
and a
\tcode{>}
preprocessing token pair or a pair of
\tcode{"}
characters is combined into a single header name
preprocessing token is \impldef{search locations for \tcode{""""} header}.

\pnum
The implementation shall provide unique mappings for
sequences consisting of one or more
\grammarterm{nondigit}{s} or \grammarterm{digit}{s}\iref{lex.name}
followed by a period
(\tcode{.})
and a single
\grammarterm{nondigit}.
The first character shall not be a \grammarterm{digit}.
The implementation may ignore distinctions of alphabetical case.

\pnum
A
\tcode{\#include}
preprocessing directive may appear
in a source file that has been read because of a
\tcode{\#include}
directive in another file,
up to an \impldef{nesting limit for \tcode{\#include} directives} nesting limit.

\pnum
If the header identified by the \grammarterm{header-name}
denotes an importable header\iref{module.import},
it is
\impldef{whether source file inclusion of importable header
is replaced with \tcode{import} directive}
whether the \tcode{\#include} preprocessing directive
is instead replaced by an \tcode{import} directive\iref{cpp.import} of the form
\begin{ncbnf}
\terminal{import} header-name \terminal{;} new-line
\end{ncbnf}

\pnum
\begin{note}
An implementation can provide a mechanism for making arbitrary
source files available to the \tcode{< >} search.
However, using the \tcode{< >} form for headers provided
with the implementation and the \tcode{" "} form for sources
outside the control of the implementation
achieves wider portability. For instance:

\begin{codeblock}
#include <stdio.h>
#include <unistd.h>
#include "usefullib.h"
#include "myprog.h"
\end{codeblock}

\end{note}

\pnum
\begin{example}
This illustrates macro-replaced
\tcode{\#include}
directives:

\begin{codeblock}
#if VERSION == 1
    #define INCFILE  "vers1.h"
#elif VERSION == 2
    #define INCFILE  "vers2.h"  // and so on
#else
    #define INCFILE  "versN.h"
#endif
#include INCFILE
\end{codeblock}
\end{example}

\rSec2[cpp.import]{Header unit importation}
\indextext{header unit!preprocessing}%
\indextext{preprocessing directive!import}%
\indextext{macro!import|(}%

\begin{bnf}
\nontermdef{pp-import}\br
    \opt{\keyword{export}} \keyword{import} header-name \opt{pp-tokens} \terminal{;} new-line\br
    \opt{\keyword{export}} \keyword{import} header-name-tokens \opt{pp-tokens} \terminal{;} new-line\br
    \opt{\keyword{export}} \keyword{import} pp-tokens \terminal{;} new-line
\end{bnf}

\pnum
A \grammarterm{pp-import} shall not
appear in a context where \tcode{import}
or (if it is the first token of the \grammarterm{pp-import}) \tcode{export}
is an identifier defined as an object-like macro.

\pnum
The preprocessing tokens after the \tcode{import} preprocessing token
in the \tcode{import} \grammarterm{control-line}
are processed just as in normal text
(i.e., each identifier currently defined as a macro name
is replaced by its replacement list of preprocessing tokens).
\begin{note}
An \tcode{import} directive
matching the first two forms of a \grammarterm{pp-import}
instructs the preprocessor to import macros
from the header unit\iref{module.import}
denoted by the \grammarterm{header-name},
as described below.
\end{note}
\indextext{point of!macro import|see{macro, point of import}}%
The \defnx{point of macro import}{macro!point of import} for the
first two forms of \grammarterm{pp-import} is
immediately after the \grammarterm{new-line} terminating
the \grammarterm{pp-import}.
The last form of \grammarterm{pp-import} is only considered
if the first two forms did not match, and
does not have a point of macro import.

\pnum
If a \grammarterm{pp-import} is produced by source file inclusion
(including by the rewrite produced
when a \tcode{\#include} directive names an importable header)
while processing the \grammarterm{group} of a \grammarterm{module-file},
the program is ill-formed.

\pnum
In all three forms of \grammarterm{pp-import},
the \tcode{import} and \tcode{export} (if it exists) preprocessing tokens
are replaced by the \grammarterm{import-keyword} and
\grammarterm{export-keyword} preprocessing tokens respectively.
\begin{note}
This makes the line no longer a directive
so it is not removed at the end of phase 4.
\end{note}
Additionally, in the second form of \grammarterm{pp-import},
a \grammarterm{header-name} token is formed as if
the \grammarterm{header-name-tokens}
were the \grammarterm{pp-tokens} of a \tcode{\#include} directive.
The \grammarterm{header-name-tokens} are replaced by
the \grammarterm{header-name} token.
\begin{note}
This ensures that imports are treated consistently by
the preprocessor and later phases of translation.
\end{note}

\pnum
Each \tcode{\#define} directive encountered when preprocessing
each translation unit in a program results in a distinct
\defnx{macro definition}{macro!definition}.
\begin{note}
A predefined macro name\iref{cpp.predefined}
is not introduced by a \tcode{\#define} directive.
Implementations providing mechanisms to predefine additional macros
are encouraged to not treat them
as being introduced by a \tcode{\#define} directive.
\end{note}
Each macro definition has at most one point of definition in
each translation unit and at most one point of undefinition, as follows:
\begin{itemize}
\item
\indextext{point of!macro definition|see{macro, point of definition}}%
The \defnx{point of definition}{macro!point of definition}
of a macro definition within a translation unit $T$ is
\begin{itemize}
\item
if the \tcode{\#define} directive of the macro definition occurs within $T$,
the point at which that directive occurs, or otherwise,
\item
if the macro name is not lexically identical to a keyword\iref{lex.key}
or to the \grammarterm{identifier}{s} \tcode{module} or \tcode{import},
the first point of macro import in $T$ of a header unit
containing a point of definition for the macro definition, if any.
\end{itemize}
In the latter case, the macro is said
to be \defnx{imported}{macro!import} from the header unit.

\item
\indextext{point of!macro undefinition|see{macro, point of undefinition}}%
The \defnx{point of undefinition}{macro!point of undefinition}
of a macro definition within a translation unit
is the first point at which a \tcode{\#undef} directive naming the macro occurs
after its point of definition, or the first point
of macro import of a header unit containing a point of undefinition for the
macro definition, whichever (if any) occurs first.
\end{itemize}

\pnum
\indextext{active macro directive|see{macro, active}}%
A macro directive is \defnx{active}{macro!active} at a source location
if it has a point of definition in that translation unit preceding the location,
and does not have a point of undefinition in that translation unit preceding
the location.

\pnum
If a macro would be replaced or redefined, and multiple macro definitions
are active for that macro name, the active macro definitions shall all be
valid redefinitions of the same macro\iref{cpp.replace}.
\begin{note}
The relative order of \grammarterm{pp-import}{s} has no bearing on whether a
particular macro definition is active.
\end{note}

\pnum
\begin{example}
\begin{codeblocktu}{Importable header \tcode{"a.h"}}
#define X 123   // \#1
#define Y 45    // \#2
#define Z a     // \#3
#undef X        // point of undefinition of \#1 in \tcode{"a.h"}
\end{codeblocktu}

\begin{codeblocktu}{Importable header \tcode{"b.h"}}
import "a.h";   // point of definition of \#1, \#2, and \#3, point of undefinition of \#1 in \tcode{"b.h"}
#define X 456   // OK, \#1 is not active
#define Y 6     // error: \#2 is active
\end{codeblocktu}

\begin{codeblocktu}{Importable header \tcode{"c.h"}}
#define Y 45    // \#4
#define Z c     // \#5
\end{codeblocktu}

\begin{codeblocktu}{Importable header \tcode{"d.h"}}
import "c.h";   // point of definition of \#4 and \#5 in \tcode{"d.h"}
\end{codeblocktu}

\begin{codeblocktu}{Importable header \tcode{"e.h"}}
import "a.h";   // point of definition of \#1, \#2, and \#3, point of undefinition of \#1 in \tcode{"e.h"}
import "d.h";   // point of definition of \#4 and \#5 in \tcode{"e.h"}
int a = Y;      // OK, active macro definitions \#2 and \#4 are valid redefinitions
int c = Z;      // error: active macro definitions \#3 and \#5 are not valid redefinitions of \tcode{Z}
\end{codeblocktu}

\begin{codeblocktu}{Module unit \tcode{f}}
export module f;
export import "a.h";

int a = Y;      // OK
\end{codeblocktu}

\begin{codeblocktu}{Translation unit \tcode{\#1}}
import f;
int x = Y;      // error: \tcode{Y} is neither a defined macro nor a declared name
\end{codeblocktu}
\end{example}
\indextext{macro!import|)}

\rSec2[cpp.replace]{Macro replacement}%

\rSec3[cpp.replace.general]{General}%
\indextext{macro!replacement|(}%
\indextext{replacement!macro|see{macro, replacement}}%
\indextext{preprocessing directive!macro replacement|see{macro, replacement}}

\pnum
\indextext{macro!replacement list}%
Two replacement lists are identical if and only if
the preprocessing tokens in both have
the same number, ordering, spelling, and whitespace separation,
where all whitespace separations are considered identical.

\pnum
An identifier currently defined as an
\indextext{macro!object-like}%
object-like macro (see below) may be redefined by another
\tcode{\#define}
preprocessing directive provided that the second definition is an
object-like macro definition and the two replacement lists
are identical, otherwise the program is ill-formed.
Likewise, an identifier currently defined as a
\indextext{macro!function-like}%
function-like macro (see below) may be redefined by another
\tcode{\#define}
preprocessing directive provided that the second definition is a
function-like macro definition that has the same number and spelling
of parameters,
and the two replacement lists are identical,
otherwise the program is ill-formed.

\pnum
\begin{example}
The following sequence is valid:
\begin{codeblock}
#define OBJ_LIKE      (1-1)
#define OBJ_LIKE      @\tcode{/* whitespace */ (1-1) /* other */}@
#define FUNC_LIKE(a)   ( a )
#define FUNC_LIKE( a )(     @\tcode{/* note the whitespace */ \textbackslash}@
                a @\tcode{/* other stuff on this line}@
                  @\tcode{*/}@ )
\end{codeblock}
But the following redefinitions are invalid:
\begin{codeblock}
#define OBJ_LIKE    (0)         // different token sequence
#define OBJ_LIKE    (1 - 1)     // different whitespace
#define FUNC_LIKE(b) ( a )      // different parameter usage
#define FUNC_LIKE(b) ( b )      // different parameter spelling
\end{codeblock}
\end{example}

\pnum
\indextext{macro!replacement list}%
There shall be whitespace between the identifier and the replacement list
in the definition of an object-like macro.

\pnum
If the \grammarterm{identifier-list} in the macro definition does not end with
an ellipsis, the number of arguments (including those arguments consisting
of no preprocessing tokens)
in an invocation of a function-like macro shall
equal the number of parameters in the macro definition.
Otherwise, there shall be at least as many arguments in the invocation as there are
parameters in the macro definition (excluding the \tcode{...}). There
shall exist a
\tcode{)}
preprocessing token that terminates the invocation.

\pnum
\indextext{__va_args__@\mname{VA_ARGS}}%
\indextext{__va_opt__@\mname{VA_OPT}}%
The identifiers \mname{VA_ARGS} and \mname{VA_OPT}
shall occur only in the \grammarterm{replacement-list}
of a function-like macro that uses the ellipsis notation in the parameters.

\pnum
A parameter identifier in a function-like macro
shall be uniquely declared within its scope.

\pnum
The identifier immediately following the
\tcode{define}
is called the
\indextext{name!macro|see{macro, name}}%
\defnx{macro name}{macro!name}.
There is one name space for macro names.
Any whitespace characters preceding or following the
replacement list of preprocessing tokens are not considered
part of the replacement list for either form of macro.

\pnum
If a
\indextext{\#\#0 operator@\tcode{\#} operator}
\tcode{\#}
preprocessing token,
followed by an identifier,
occurs lexically
at the point at which a preprocessing directive can begin,
the identifier is not subject to macro replacement.

\pnum
A preprocessing directive of the form
\begin{ncsimplebnf}
\terminal{\# define} identifier replacement-list new-line
\indextext{\idxcode{\#define}}%
\end{ncsimplebnf}
defines an
\defnadj{object-like}{macro} that
causes each subsequent instance of the macro name
\begin{footnote}
Since, by macro-replacement time,
all \grammarterm{character-literal}s and \grammarterm{string-literal}s are preprocessing tokens,
not sequences possibly containing identifier-like subsequences
(see \ref{lex.phases}, translation phases),
they are never scanned for macro names or parameters.
\end{footnote}
to be replaced by the replacement list of preprocessing tokens
that constitute the remainder of the directive.
\begin{footnote}
An alternative token\iref{lex.digraph} is not an identifier,
even when its spelling consists entirely of letters and underscores.
Therefore it is not possible to define a macro
whose name is the same as that of an alternative token.
\end{footnote}
The replacement list is then rescanned for more macro names as
specified below.

\pnum
\begin{example}
The simplest use of this facility is to define a ``manifest constant'',
as in
\begin{codeblock}
#define TABSIZE 100
int table[TABSIZE];
\end{codeblock}
\end{example}

\pnum
A preprocessing directive of the form
\begin{ncsimplebnf}
\terminal{\# define} identifier lparen \opt{identifier-list} \terminal{)} replacement-list new-line\br
\terminal{\# define} identifier lparen \terminal{...} \terminal{)} replacement-list new-line\br
\terminal{\# define} identifier lparen identifier-list \terminal{, ...} \terminal{)} replacement-list new-line
\end{ncsimplebnf}
defines a \defnadj{function-like}{macro}
with parameters, whose use is
similar syntactically to a function call.
The parameters
\indextext{parameter!macro}%
are specified by the optional list of identifiers.
Each subsequent instance of the function-like macro name followed by a
\tcode{(}
as the next preprocessing token
introduces the sequence of preprocessing tokens that is replaced
by the replacement list in the definition
(an invocation of the macro).
\indextext{invocation!macro}%
The replaced sequence of preprocessing tokens is terminated by the matching
\tcode{)}
preprocessing token, skipping intervening matched pairs of left and
right parenthesis preprocessing tokens.
Within the sequence of preprocessing tokens making up an invocation
of a function-like macro,
new-line is considered a normal whitespace character.

\pnum
\indextext{macro!function-like!arguments}%
The sequence of preprocessing tokens
bounded by the outside-most matching parentheses
forms the list of arguments for the function-like macro.
The individual arguments within the list
are separated by comma preprocessing tokens,
but comma preprocessing tokens between matching
inner parentheses do not separate arguments.
If there are sequences of preprocessing tokens within the list of
arguments that would otherwise act as preprocessing directives,
\begin{footnote}
A \grammarterm{conditionally-supported-directive} is a preprocessing directive regardless of whether the implementation supports it.
\end{footnote}
the behavior is undefined.

\pnum
\begin{example}
The following defines a function-like
macro whose value is the maximum of its arguments.
It has the disadvantages of evaluating one or the other of its arguments
a second time
(including
\indextext{side effects}%
side effects)
and generating more code than a function if invoked several times.
It also cannot have its address taken,
as it has none.

\begin{codeblock}
#define max(a, b) ((a) > (b) ? (a) : (b))
\end{codeblock}

The parentheses ensure that the arguments and
the resulting expression are bound properly.
\end{example}

\pnum
\indextext{macro!function-like!arguments}%
If there is a \tcode{...} immediately preceding the \tcode{)} in the
function-like macro
definition, then the trailing arguments (if any), including any separating comma preprocessing
tokens, are merged to form a single item: the \defn{variable arguments}. The number of
arguments so combined is such that, following merger, the number of arguments is
either equal to or
one more than the number of parameters in the macro definition (excluding the
\tcode{...}).

\rSec3[cpp.subst]{Argument substitution}%
\indextext{macro!argument substitution}%
\indextext{argument substitution|see{macro, argument substitution}}%

\indextext{__va_opt__@\mname{VA_OPT}}%
\begin{bnf}
\nontermdef{va-opt-replacement}\br
    \terminal{\mname{VA_OPT} (} \opt{pp-tokens} \terminal{)}
\end{bnf}

\pnum
After the arguments for the invocation of a function-like macro have
been identified, argument substitution takes place.
For each parameter in the replacement list that is neither
preceded by a \tcode{\#} or \tcode{\#\#} preprocessing token nor
followed by a \tcode{\#\#} preprocessing token, the preprocessing tokens
naming the parameter are replaced by a token sequence determined as follows:
\begin{itemize}
\item
  If the parameter is of the form \grammarterm{va-opt-replacement},
  the replacement preprocessing tokens are the
  preprocessing token sequence for the corresponding argument,
  as specified below.
\item
  Otherwise, the replacement preprocessing tokens are the
  preprocessing tokens of corresponding argument after all
  macros contained therein have been expanded. The argument's
  preprocessing tokens are completely macro replaced before
  being substituted as if they formed the rest of the preprocessing
  file with no other preprocessing tokens being available.
\end{itemize}
\begin{example}
\begin{codeblock}
#define LPAREN() (
#define G(Q) 42
#define F(R, X, ...)  __VA_OPT__(G R X) )
int x = F(LPAREN(), 0, <:-);    // replaced by \tcode{int x = 42;}
\end{codeblock}
\end{example}

\pnum
\indextext{__va_args__@\mname{VA_ARGS}}%
An identifier \mname{VA_ARGS} that occurs in the replacement list
shall be treated as if it were a parameter, and the variable arguments shall form
the preprocessing tokens used to replace it.

\pnum
\begin{example}
\begin{codeblock}
#define debug(...) fprintf(stderr, @\mname{VA_ARGS}@)
#define showlist(...) puts(#@\mname{VA_ARGS}@)
#define report(test, ...) ((test) ? puts(#test) : printf(@\mname{VA_ARGS}@))
debug("Flag");
debug("X = %d\n", x);
showlist(The first, second, and third items.);
report(x>y, "x is %d but y is %d", x, y);
\end{codeblock}
results in
\begin{codeblock}
fprintf(stderr, "Flag");
fprintf(stderr, "X = %d\n", x);
puts("The first, second, and third items.");
((x>y) ? puts("x>y") : printf("x is %d but y is %d", x, y));
\end{codeblock}
\end{example}

\pnum
\indextext{__va_opt__@\mname{VA_OPT}}%
The identifier \mname{VA_OPT}
shall always occur as part of the preprocessing token sequence
\grammarterm{va-opt-replacement};
its closing \tcode{)} is determined by skipping
intervening pairs of matching left and right parentheses
in its \grammarterm{pp-tokens}.
The \grammarterm{pp-tokens} of a \grammarterm{va-opt-replacement}
shall not contain \mname{VA_OPT}.
If the \grammarterm{pp-tokens} would be ill-formed
as the replacement list of the current function-like macro,
the program is ill-formed.
A \grammarterm{va-opt-replacement} is treated as if it were a parameter,
and the preprocessing token sequence for the corresponding
argument is defined as follows.
If the substitution of \mname{VA_ARGS} as neither an operand
of \tcode{\#} nor \tcode{\#\#} consists of no preprocessing tokens,
the argument consists of
a single placemarker preprocessing token\iref{cpp.concat,cpp.rescan}.
Otherwise, the argument consists of
the results of the expansion of the contained \grammarterm{pp-tokens}
as the replacement list of the current function-like macro
before removal of placemarker tokens, rescanning, and further replacement.
\begin{note}
The placemarker tokens are removed before stringization\iref{cpp.stringize},
and can be removed by rescanning and further replacement\iref{cpp.rescan}.
\end{note}
\begin{example}
\begin{codeblock}
#define F(...)           f(0 __VA_OPT__(,) __VA_ARGS__)
#define G(X, ...)        f(0, X __VA_OPT__(,) __VA_ARGS__)
#define SDEF(sname, ...) S sname __VA_OPT__(= { __VA_ARGS__ })
#define EMP

F(a, b, c)          // replaced by \tcode{f(0, a, b, c)}
F()                 // replaced by \tcode{f(0)}
F(EMP)              // replaced by \tcode{f(0)}

G(a, b, c)          // replaced by \tcode{f(0, a, b, c)}
G(a, )              // replaced by \tcode{f(0, a)}
G(a)                // replaced by \tcode{f(0, a)}

SDEF(foo);          // replaced by \tcode{S foo;}
SDEF(bar, 1, 2);    // replaced by \tcode{S bar = \{ 1, 2 \};}

#define H1(X, ...) X __VA_OPT__(##) __VA_ARGS__ // error: \tcode{\#\#} may not appear at
                                                // the beginning of a replacement list\iref{cpp.concat}

#define H2(X, Y, ...) __VA_OPT__(X ## Y,) __VA_ARGS__
H2(a, b, c, d)      // replaced by \tcode{ab, c, d}

#define H3(X, ...) #__VA_OPT__(X##X X##X)
H3(, 0)             // replaced by \tcode{""}

#define H4(X, ...) __VA_OPT__(a X ## X) ## b
H4(, 1)             // replaced by \tcode{a b}

#define H5A(...) __VA_OPT__()@\tcode{/**/}@__VA_OPT__()
#define H5B(X) a ## X ## b
#define H5C(X) H5B(X)
H5C(H5A())          // replaced by \tcode{ab}
\end{codeblock}
\end{example}

\rSec3[cpp.stringize]{The \tcode{\#} operator}%
\indextext{\#\#0 operator@\tcode{\#} operator}%
\indextext{stringize|see{\tcode{\#} operator}}

\pnum
Each
\tcode{\#}
preprocessing token in the replacement list for a function-like
macro shall be followed by a parameter as the next preprocessing
token in the replacement list.

\pnum
A \defn{character string literal} is a \grammarterm{string-literal} with no prefix.
If, in the replacement list, a parameter is immediately
preceded by a
\tcode{\#}
preprocessing token,
both are replaced by a single character string literal preprocessing token that
contains the spelling of the preprocessing token sequence for the
corresponding argument (excluding placemarker tokens).
Let the \defn{stringizing argument} be the preprocessing token sequence
for the corresponding argument with placemarker tokens removed.
Each occurrence of whitespace between the stringizing argument's preprocessing
tokens becomes a single space character in the character string literal.
Whitespace before the first preprocessing token and after the last
preprocessing token comprising the stringizing argument is deleted.
Otherwise, the original spelling of each preprocessing token in the
stringizing argument is retained in the character string literal,
except for special handling for producing the spelling of
\grammarterm{string-literal}s and \grammarterm{character-literal}s:
a
\tcode{\textbackslash}
character is inserted before each
\tcode{"}
and
\tcode{\textbackslash}
character of a \grammarterm{character-literal} or \grammarterm{string-literal}
(including the delimiting
\tcode{"}
characters).
If the replacement that results is not a valid character string literal,
the behavior is undefined. The character string literal corresponding to
an empty stringizing argument is \tcode{""}.
The order of evaluation of
\tcode{\#}
and
\tcode{\#\#}
operators is unspecified.

\rSec3[cpp.concat]{The \tcode{\#\#} operator}%
\indextext{\#\#1 operator@\tcode{\#\#} operator}%
\indextext{concatenation!macro argument|see{\tcode{\#\#} operator}}

\pnum
A
\tcode{\#\#}
preprocessing token shall not occur at the beginning or
at the end of a replacement list for either form
of macro definition.

\pnum
If, in the replacement list of a function-like macro, a parameter is
immediately preceded or followed by a
\tcode{\#\#}
preprocessing token, the parameter is replaced by the
corresponding argument's preprocessing token sequence; however, if an argument consists of no preprocessing tokens, the parameter is
replaced by a placemarker preprocessing token instead.
\begin{footnote}
Placemarker preprocessing tokens do not appear in the syntax
because they are temporary entities that exist only within translation phase 4.
\end{footnote}

\pnum
For both object-like and function-like macro invocations, before the
replacement list is reexamined for more macro names to replace,
each instance of a
\tcode{\#\#}
preprocessing token in the replacement list
(not from an argument) is deleted and the
preceding preprocessing token is concatenated
with the following preprocessing token.
Placemarker preprocessing tokens are handled specially: concatenation
of two placemarkers results in a single placemarker preprocessing token, and
concatenation of a placemarker with a non-placemarker preprocessing token results
in the non-placemarker preprocessing token.
\begin{note}
Concatenation can form
a \grammarterm{universal-character-name}\iref{lex.charset}.
\end{note}
If the result is not a valid preprocessing token,
the behavior is undefined.
The resulting token is available for further macro replacement.
The order of evaluation of
\tcode{\#\#}
operators is unspecified.

\pnum
\begin{example}
The sequence
\begin{codeblock}
#define str(s)      # s
#define xstr(s)     str(s)
#define debug(s, t) printf("x" # s "= %d, x" # t "= %s", @\textbackslash@
               x ## s, x ## t)
#define INCFILE(n)  vers ## n
#define glue(a, b)  a ## b
#define xglue(a, b) glue(a, b)
#define HIGHLOW     "hello"
#define LOW         LOW ", world"

debug(1, 2);
fputs(str(strncmp("abc@\textbackslash@0d", "abc", '@\textbackslash@4')        // this goes away
    == 0) str(: @\atsign\textbackslash@n), s);
#include xstr(INCFILE(2).h)
glue(HIGH, LOW);
xglue(HIGH, LOW)
\end{codeblock}
results in
\begin{codeblock}
printf("x" "1" "= %d, x" "2" "= %s", x1, x2);
fputs("strncmp(@\textbackslash@"abc@\textbackslash\textbackslash@0d@\textbackslash@", @\textbackslash@"abc@\textbackslash@", '@\textbackslash\textbackslash@4') == 0" ": @\atsign\textbackslash@n", s);
#include "vers2.h"      @\textrm{(\textit{after macro replacement, before file access})}@
"hello";
"hello" ", world"
\end{codeblock}
or, after concatenation of the character string literals,
\begin{codeblock}
printf("x1= %d, x2= %s", x1, x2);
fputs("strncmp(@\textbackslash@"abc@\textbackslash\textbackslash@0d@\textbackslash@", @\textbackslash@"abc@\textbackslash@", '@\textbackslash\textbackslash@4') == 0: @\atsign\textbackslash@n", s);
#include "vers2.h"      @\textrm{(\textit{after macro replacement, before file access})}@
"hello";
"hello, world"
\end{codeblock}

Space around the \tcode{\#} and \tcode{\#\#} tokens in the macro definition
is optional.
\end{example}

\pnum
\begin{example}
In the following fragment:

\begin{codeblock}
#define hash_hash # ## #
#define mkstr(a) # a
#define in_between(a) mkstr(a)
#define join(c, d) in_between(c hash_hash d)
char p[] = join(x, y);          // equivalent to \tcode{char p[] = "x \#\# y";}
\end{codeblock}

The expansion produces, at various stages:

\begin{codeblock}
join(x, y)
in_between(x hash_hash y)
in_between(x ## y)
mkstr(x ## y)
"x ## y"
\end{codeblock}

In other words, expanding \tcode{hash_hash} produces a new token,
consisting of two adjacent sharp signs, but this new token is not the
\tcode{\#\#} operator.
\end{example}

\pnum
\begin{example}
To illustrate the rules for placemarker preprocessing tokens, the sequence
\begin{codeblock}
#define t(x,y,z) x ## y ## z
int j[] = { t(1,2,3), t(,4,5), t(6,,7), t(8,9,),
  t(10,,), t(,11,), t(,,12), t(,,) };
\end{codeblock}
results in
\begin{codeblock}
int j[] = { 123, 45, 67, 89,
  10, 11, 12, };
\end{codeblock}
\end{example}

\rSec3[cpp.rescan]{Rescanning and further replacement}%
\indextext{macro!rescanning and replacement}%
\indextext{rescanning and replacement|see{macro, rescanning and replacement}}

\pnum
After all parameters in the replacement list have been substituted and \tcode{\#} and \tcode{\#\#} processing has taken
place, all placemarker preprocessing tokens are removed. Then
the resulting preprocessing token sequence is rescanned, along with all
subsequent preprocessing tokens of the source file, for more macro names
to replace.

\pnum
\begin{example}
The sequence
\begin{codeblock}
#define x       3
#define f(a)    f(x * (a))
#undef  x
#define x       2
#define g       f
#define z       z[0]
#define h       g(~
#define m(a)    a(w)
#define w       0,1
#define t(a)    a
#define p()     int
#define q(x)    x
#define r(x,y)  x ## y
#define str(x)  # x

f(y+1) + f(f(z)) % t(t(g)(0) + t)(1);
g(x+(3,4)-w) | h 5) & m
    (f)^m(m);
p() i[q()] = { q(1), r(2,3), r(4,), r(,5), r(,) };
char c[2][6] = { str(hello), str() };
\end{codeblock}
results in
\begin{codeblock}
f(2 * (y+1)) + f(2 * (f(2 * (z[0])))) % f(2 * (0)) + t(1);
f(2 * (2+(3,4)-0,1)) | f(2 * (~ 5)) & f(2 * (0,1))^m(0,1);
int i[] = { 1, 23, 4, 5, };
char c[2][6] = { "hello", "" };
\end{codeblock}
\end{example}

\pnum
If the name of the macro being replaced is found during this scan of
the replacement list
(not including the rest of the source file's preprocessing tokens),
it is not replaced.
Furthermore,
if any nested replacements encounter the name of the macro being replaced,
it is not replaced.
These nonreplaced macro name preprocessing tokens are no longer available
for further replacement even if they are later (re)examined in contexts
in which that macro name preprocessing token would otherwise have been
replaced.

\pnum
The resulting completely macro-replaced preprocessing token sequence
is not processed as a preprocessing directive even if it resembles one,
but all pragma unary operator expressions within it are then processed as
specified in~\ref{cpp.pragma.op} below.

\rSec3[cpp.scope]{Scope of macro definitions}%
\indextext{macro!scope of definition}%
\indextext{scope!macro definition|see{macro, scope of definition}}

\pnum
A macro definition lasts
(independent of block structure)
until a corresponding
\tcode{\#undef}
directive is encountered or
(if none is encountered)
until the end of the translation unit.
Macro definitions have no significance after translation phase 4.

\pnum
A preprocessing directive of the form
\begin{ncsimplebnf}
\terminal{\# undef} identifier new-line
\indextext{\idxcode{\#undef}}%
\end{ncsimplebnf}
causes the specified identifier no longer to be defined as a macro name.
It is ignored if the specified identifier is not currently defined as
a macro name.

\indextext{macro!replacement|)}

\rSec2[cpp.line]{Line control}%
\indextext{preprocessing directive!line control}%
\indextext{\idxcode{\#line}|see{preprocessing directive, line control}}

\pnum
The \grammarterm{string-literal} of a
\tcode{\#line}
directive, if present,
shall be a character string literal.

\pnum
The
\defn{line number}
of the current source line is one greater than
the number of new-line characters read or introduced
in translation phase 1\iref{lex.phases}
while processing the source file to the current token.

\pnum
A preprocessing directive of the form
\begin{ncsimplebnf}
\terminal{\# line} digit-sequence new-line
\end{ncsimplebnf}
causes the implementation to behave as if
the following sequence of source lines begins with a
source line that has a line number as specified
by the digit sequence (interpreted as a decimal integer).
If the digit sequence specifies zero
or a number greater than 2147483647,
the behavior is undefined.

\pnum
A preprocessing directive of the form
\begin{ncsimplebnf}
\terminal{\# line} digit-sequence \terminal{"} \opt{s-char-sequence} \terminal{"} new-line
\end{ncsimplebnf}
sets the presumed line number similarly and changes the
presumed name of the source file to be the contents
of the character string literal.

\pnum
A preprocessing directive of the form
\begin{ncsimplebnf}
\terminal{\# line} pp-tokens new-line
\end{ncsimplebnf}
(that does not match one of the two previous forms)
is permitted.
The preprocessing tokens after
\tcode{line}
on the directive are processed just as in normal text
(each identifier currently defined as a macro name is replaced by its
replacement list of preprocessing tokens).
If the directive resulting after all replacements does not match
one of the two previous forms, the behavior is undefined;
otherwise, the result is processed as appropriate.

\rSec2[cpp.error]{Diagnostic directives}%
\indextext{preprocessing directive!error}%
\indextext{preprocessing directive!diagnostic}%
\indextext{preprocessing directive!warning}%
\indextext{\idxcode{\#error}|see{preprocessing directive, error}}

\pnum
A preprocessing directive of the form
\begin{ncsimplebnf}
\terminal{\# error} \opt{pp-tokens} new-line
\end{ncsimplebnf}
renders the program ill-formed.
A preprocessing directive of the form
\begin{ncsimplebnf}
\terminal{\# warning} \opt{pp-tokens} new-line
\end{ncsimplebnf}
requires the implementation to produce at least one diagnostic message
for the preprocessing translation unit\iref{intro.compliance.general}.
\recommended
Any diagnostic message caused by either of these directives
should include the specified sequence of preprocessing tokens.

\rSec2[cpp.pragmas]{Pragmas}%

\rSec3[cpp.pragma]{Pragma directive}%
\indextext{preprocessing directive!pragma}%
\indextext{\idxcode{\#pragma}|see{preprocessing directive, pragma}}

\pnum
A preprocessing directive of the form
\begin{ncsimplebnf}
\terminal{\# pragma} \opt{pp-tokens} new-line
\end{ncsimplebnf}
causes the implementation to behave
in an \impldef{\tcode{\#pragma}} manner.
The behavior may cause translation to fail or cause the translator or
the resulting program to behave in a non-conforming manner.
Any pragma that is not recognized by the implementation is ignored.

\rSec3[cpp.pragma.op]{Pragma operator}%
\indextext{macro!pragma operator}%
\indextext{operator!pragma|see{macro, pragma operator}}

\pnum
A unary operator expression of the form:
\begin{ncbnf}
\terminal{_Pragma} \terminal{(} string-literal \terminal{)}
\end{ncbnf}
is processed as follows: The \grammarterm{string-literal} is \defnx{destringized}{destringization}
by deleting the \tcode{L} prefix, if present, deleting the leading and trailing
double-quotes, replacing each escape sequence \tcode{\textbackslash"} by a double-quote, and
replacing each escape sequence \tcode{\textbackslash\textbackslash} by a single
backslash. The resulting sequence of characters is processed through translation phase 3
to produce preprocessing tokens that are executed as if they were the
\grammarterm{pp-tokens} in a pragma directive. The original four preprocessing
tokens in the unary operator expression are removed.

\pnum
\begin{example}
\begin{codeblock}
#pragma listing on "..\listing.dir"
\end{codeblock}
can also be expressed as:
\begin{codeblock}
_Pragma ( "listing on \"..\\listing.dir\"" )
\end{codeblock}
The latter form is processed in the same way whether it appears literally
as shown, or results from macro replacement, as in:
\begin{codeblock}
#define LISTING(x) PRAGMA(listing on #x)
#define PRAGMA(x) _Pragma(#x)

LISTING( ..\listing.dir )
\end{codeblock}
\end{example}

\rSec2[cpp.predefined]{Predefined macro names}
\indextext{macro!predefined}%
\indextext{name!predefined macro|see{macro, predefined}}

\pnum
The following macro names shall be defined by the implementation:

\begin{description}

\item
\indextext{\idxxname{cplusplus}}%
\xname{cplusplus}\\
The integer literal \tcode{\cppver}.
\begin{note}
Future revisions of this document will
replace the value of this macro with a greater value.
\end{note}

\item The names listed in \tref{cpp.predefined.ft}.\\
The macros defined in \tref{cpp.predefined.ft} shall be defined to
the corresponding integer literal.
\begin{note}
Future revisions of this document might replace
the values of these macros with greater values.
\end{note}

\item
\indextext{__date__@\mname{DATE}}%
\mname{DATE}\\
The date of translation of the source file:
a character string literal of the form
\tcode{"Mmm~dd~yyyy"},
where the names of the months are the same as those generated
by the
\tcode{asctime}
function,
and the first character of
\tcode{dd}
is a space character if the value is less than 10.
If the date of translation is not available,
an \impldef{text of \mname{DATE} when date of translation is not available} valid date
shall be supplied.

\item
\indextext{__file__@\mname{FILE}}%
\mname{FILE}\\
The presumed name of the current source file (a character string
literal).
\begin{footnote}
The presumed source file name can be changed by the \tcode{\#line} directive.
\end{footnote}

\item
\indextext{__line__@\mname{LINE}}%
\mname{LINE}\\
The presumed line number (within the current source file) of the current source line
(an integer literal).
\begin{footnote}
The presumed line number can be changed by the \tcode{\#line} directive.
\end{footnote}

\item
\indextext{__stdc_hosted__@\mname{STDC_HOSTED}}%
\indextext{implementation!hosted}%
\indextext{implementation!freestanding}%
\mname{STDC_HOSTED}\\
The integer literal \tcode{1}
if the implementation is a hosted implementation or
the integer literal \tcode{0}
if it is a freestanding implementation\iref{intro.compliance}.

\item
\indextext{__stdcpp_default_new_alignment__@\mname{STDCPP_DEFAULT_NEW_ALIGNMENT}}%
\mname{STDCPP_DEFAULT_NEW_ALIGNMENT}\\
An integer literal of type \tcode{std::size_t}
whose value is the alignment guaranteed
by a call to \tcode{operator new(std::size_t)}
or \tcode{operator new[](std::size_t)}.
\begin{note}
Larger alignments will be passed to
\tcode{operator new(std::size_t, std::align_val_t)}, etc.\iref{expr.new}.
\end{note}

\item
\indextext{__stdcpp_float16_t__@\mname{STDCPP_FLOAT16_T}}%
\mname{STDCPP_FLOAT16_T}\\
Defined as the integer literal \tcode{1}
if and only if the implementation supports
the ISO/IEC/IEEE 60559 floating-point interchange format binary16
as an extended floating-point type\iref{basic.extended.fp}.

\item
\indextext{__stdcpp_float32_t__@\mname{STDCPP_FLOAT32_T}}%
\mname{STDCPP_FLOAT32_T}\\
Defined as the integer literal \tcode{1}
if and only if the implementation supports
the ISO/IEC/IEEE 60559 floating-point interchange format binary32
as an extended floating-point type.

\item
\indextext{__stdcpp_float64_t__@\mname{STDCPP_FLOAT64_T}}%
\mname{STDCPP_FLOAT64_T}\\
Defined as the integer literal \tcode{1}
if and only if the implementation supports
the ISO/IEC/IEEE 60559 floating-point interchange format binary64
as an extended floating-point type.

\item
\indextext{__stdcpp_float128_t__@\mname{STDCPP_FLOAT128_T}}%
\mname{STDCPP_FLOAT128_T}\\
Defined as the integer literal \tcode{1}
if and only if the implementation supports
the ISO/IEC/IEEE 60559 floating-point interchange format binary128
as an extended floating-point type.

\item
\indextext{__stdcpp_bfloat16_t__@\mname{STDCPP_BFLOAT16_T}}%
\mname{STDCPP_BFLOAT16_T}\\
Defined as the integer literal \tcode{1}
if and only if the implementation supports an extended floating-point type
with the properties of the \grammarterm{typedef-name} \tcode{std::bfloat16_t}
as described in \ref{basic.extended.fp}.

\item
\indextext{__time__@\mname{TIME}}%
\mname{TIME}\\
The time of translation of the source file:
a character string literal of the form
\tcode{"hh:mm:ss"}
as in the time generated by the
\tcode{asctime}
function.
If the time of translation is not available,
an \impldef{text of \mname{TIME} when time of translation is not available} valid time shall be supplied.
\end{description}

\indextext{macro!feature-test}%
\indextext{feature-test macro|see{macro, feature-test}}%
\begin{LongTable}{Feature-test macros}{cpp.predefined.ft}{ll}
\\ \topline
\lhdr{Macro name} & \rhdr{Value} \\ \capsep
\endfirsthead
\continuedcaption \\
\hline
\lhdr{Name} & \rhdr{Value} \\ \capsep
\endhead
\defnxname{cpp_aggregate_bases}                   & \tcode{201603L} \\ \rowsep
\defnxname{cpp_aggregate_nsdmi}                   & \tcode{201304L} \\ \rowsep
\defnxname{cpp_aggregate_paren_init}              & \tcode{201902L} \\ \rowsep
\defnxname{cpp_alias_templates}                   & \tcode{200704L} \\ \rowsep
\defnxname{cpp_aligned_new}                       & \tcode{201606L} \\ \rowsep
\defnxname{cpp_attributes}                        & \tcode{200809L} \\ \rowsep
\defnxname{cpp_auto_cast}                         & \tcode{202110L} \\ \rowsep
\defnxname{cpp_binary_literals}                   & \tcode{201304L} \\ \rowsep
\defnxname{cpp_capture_star_this}                 & \tcode{201603L} \\ \rowsep
\defnxname{cpp_char8_t}                           & \tcode{202207L} \\ \rowsep
\defnxname{cpp_concepts}                          & \tcode{202002L} \\ \rowsep
\defnxname{cpp_conditional_explicit}              & \tcode{201806L} \\ \rowsep
\defnxname{cpp_constexpr}                         & \tcode{202306L} \\ \rowsep
\defnxname{cpp_constexpr_dynamic_alloc}           & \tcode{201907L} \\ \rowsep
\defnxname{cpp_constexpr_in_decltype}             & \tcode{201711L} \\ \rowsep
\defnxname{cpp_consteval}                         & \tcode{202211L} \\ \rowsep
\defnxname{cpp_constinit}                         & \tcode{201907L} \\ \rowsep
\defnxname{cpp_decltype}                          & \tcode{200707L} \\ \rowsep
\defnxname{cpp_decltype_auto}                     & \tcode{201304L} \\ \rowsep
\defnxname{cpp_deduction_guides}                  & \tcode{201907L} \\ \rowsep
\defnxname{cpp_delegating_constructors}           & \tcode{200604L} \\ \rowsep
\defnxname{cpp_deleted_function}                  & \tcode{202403L} \\ \rowsep
\defnxname{cpp_designated_initializers}           & \tcode{201707L} \\ \rowsep
\defnxname{cpp_enumerator_attributes}             & \tcode{201411L} \\ \rowsep
\defnxname{cpp_explicit_this_parameter}           & \tcode{202110L} \\ \rowsep
\defnxname{cpp_fold_expressions}                  & \tcode{201603L} \\ \rowsep
\defnxname{cpp_generic_lambdas}                   & \tcode{201707L} \\ \rowsep
\defnxname{cpp_guaranteed_copy_elision}           & \tcode{201606L} \\ \rowsep
\defnxname{cpp_hex_float}                         & \tcode{201603L} \\ \rowsep
\defnxname{cpp_if_consteval}                      & \tcode{202106L} \\ \rowsep
\defnxname{cpp_if_constexpr}                      & \tcode{201606L} \\ \rowsep
\defnxname{cpp_impl_coroutine}                    & \tcode{201902L} \\ \rowsep
\defnxname{cpp_impl_destroying_delete}            & \tcode{201806L} \\ \rowsep
\defnxname{cpp_impl_three_way_comparison}         & \tcode{201907L} \\ \rowsep
\defnxname{cpp_implicit_move}                     & \tcode{202207L} \\ \rowsep
\defnxname{cpp_inheriting_constructors}           & \tcode{201511L} \\ \rowsep
\defnxname{cpp_init_captures}                     & \tcode{201803L} \\ \rowsep
\defnxname{cpp_initializer_lists}                 & \tcode{200806L} \\ \rowsep
\defnxname{cpp_inline_variables}                  & \tcode{201606L} \\ \rowsep
\defnxname{cpp_lambdas}                           & \tcode{200907L} \\ \rowsep
\defnxname{cpp_modules}                           & \tcode{201907L} \\ \rowsep
\defnxname{cpp_multidimensional_subscript}        & \tcode{202211L} \\ \rowsep
\defnxname{cpp_named_character_escapes}           & \tcode{202207L} \\ \rowsep
\defnxname{cpp_namespace_attributes}              & \tcode{201411L} \\ \rowsep
\defnxname{cpp_noexcept_function_type}            & \tcode{201510L} \\ \rowsep
\defnxname{cpp_nontype_template_args}             & \tcode{201911L} \\ \rowsep
\defnxname{cpp_nontype_template_parameter_auto}   & \tcode{201606L} \\ \rowsep
\defnxname{cpp_nsdmi}                             & \tcode{200809L} \\ \rowsep
\defnxname{cpp_pack_indexing}                     & \tcode{202311L} \\ \rowsep
\defnxname{cpp_placeholder_variables}             & \tcode{202306L} \\ \rowsep
\defnxname{cpp_range_based_for}                   & \tcode{202211L} \\ \rowsep
\defnxname{cpp_raw_strings}                       & \tcode{200710L} \\ \rowsep
\defnxname{cpp_ref_qualifiers}                    & \tcode{200710L} \\ \rowsep
\defnxname{cpp_return_type_deduction}             & \tcode{201304L} \\ \rowsep
\defnxname{cpp_rvalue_references}                 & \tcode{200610L} \\ \rowsep
\defnxname{cpp_size_t_suffix}                     & \tcode{202011L} \\ \rowsep
\defnxname{cpp_sized_deallocation}                & \tcode{201309L} \\ \rowsep
\defnxname{cpp_static_assert}                     & \tcode{202306L} \\ \rowsep
\defnxname{cpp_static_call_operator}              & \tcode{202207L} \\ \rowsep
\defnxname{cpp_structured_bindings}               & \tcode{202403L} \\ \rowsep
\defnxname{cpp_template_template_args}            & \tcode{201611L} \\ \rowsep
\defnxname{cpp_threadsafe_static_init}            & \tcode{200806L} \\ \rowsep
\defnxname{cpp_unicode_characters}                & \tcode{200704L} \\ \rowsep
\defnxname{cpp_unicode_literals}                  & \tcode{200710L} \\ \rowsep
\defnxname{cpp_user_defined_literals}             & \tcode{200809L} \\ \rowsep
\defnxname{cpp_using_enum}                        & \tcode{201907L} \\ \rowsep
\defnxname{cpp_variable_templates}                & \tcode{201304L} \\ \rowsep
\defnxname{cpp_variadic_friend}                   & \tcode{202403L} \\ \rowsep
\defnxname{cpp_variadic_templates}                & \tcode{200704L} \\ \rowsep
\defnxname{cpp_variadic_using}                    & \tcode{201611L} \\
\end{LongTable}

\pnum
The following macro names are conditionally defined by the implementation:

\begin{description}
\item
\indextext{__stdc__@\mname{STDC}}%
\mname{STDC}\\
Whether \mname{STDC} is predefined and if so, what its value is,
are \impldef{definition and meaning of \mname{STDC}}.

\item
\indextext{__stdc_mb_might_neq_wc__@\mname{STDC_MB_MIGHT_NEQ_WC}}%
\mname{STDC_MB_MIGHT_NEQ_WC}\\
The integer literal \tcode{1}, intended to indicate that, in the encoding for
\keyword{wchar_t}, a member of the basic character set need not have a code value equal to
its value when used as the lone character in an ordinary character literal.

\item
\indextext{__stdc_version__@\mname{STDC_VERSION}}%
\mname{STDC_VERSION}\\
Whether \mname{STDC_VERSION} is predefined and if so, what its value is,
are \impldef{definition and meaning of \mname{STDC_VERSION}}.

\item
\indextext{__stdc_iso_10646__@\mname{STDC_ISO_10646}}%
\mname{STDC_ISO_10646}\\
An integer literal of the form \tcode{yyyymmL}
(for example, \tcode{199712L}).
Whether \mname{STDC_ISO_10646} is predefined and
if so, what its value is,
are \impldef{presence and value of \mname{STDC_ISO_10646}}.

\item
\indextext{__stdcpp_threads__@\mname{STDCPP_THREADS}}%
\mname{STDCPP_THREADS}\\
Defined, and has the value integer literal 1, if and only if a program
can have more than one thread of execution\iref{intro.multithread}.

\end{description}

\pnum
The values of the predefined macros
(except for
\mname{FILE}
and
\mname{LINE})
remain constant throughout the translation unit.

\pnum
If any of the pre-defined macro names in this subclause,
or the identifier
\tcode{defined},
is the subject of a
\tcode{\#define}
or a
\tcode{\#undef}
preprocessing directive,
the behavior is undefined.
Any other predefined macro names shall begin with a
leading underscore followed by an uppercase letter or a second
underscore.
\indextext{preprocessing directive|)}

\rSec1[lex.token]{Tokens}

\indextext{token|(}%
\begin{bnf}
\nontermdef{token}\br
    identifier\br
    keyword\br
    literal\br
    operator-or-punctuator
\end{bnf}

\pnum
Each preprocessing token that is converted to a token\iref{lex.token}
shall have the lexical form of a keyword, an identifier, a literal,
or an operator or punctuator.

\pnum
\indextext{\idxgram{token}}%
There are five kinds of tokens: identifiers, keywords, literals,%
\begin{footnote}
Literals include strings and character and numeric literals.
\end{footnote}
operators, and other separators.
\indextext{whitespace}%
Blanks, horizontal and vertical tabs, newlines, formfeeds, and comments
(collectively, ``whitespace''), as described below, are ignored except
as they serve to separate tokens.
\begin{note}
Whitespace can separate otherwise adjacent identifiers, keywords, numeric
literals, and alternative tokens containing alphabetic characters.
\end{note}
\indextext{token|)}

\rSec2[lex.key]{Keywords}

\begin{bnf}
\nontermdef{keyword}\br
    \textnormal{any identifier listed in \tref{lex.key}}\br
    \grammarterm{import-keyword}\br
    \grammarterm{module-keyword}\br
    \grammarterm{export-keyword}
\end{bnf}

\pnum
The \grammarterm{import-keyword} is produced
by processing an \keyword{import} directive\iref{cpp.import},
the \grammarterm{module-keyword} is produced
by preprocessing a \keyword{module} directive\iref{cpp.module}, and
the \grammarterm{export-keyword} is produced
by preprocessing either of the previous two directives.
\begin{note}
None has any observable spelling.
\end{note}

\pnum
\indextext{keyword|(}%
The identifiers shown in \tref{lex.key} are reserved for use
as keywords (that is, they are unconditionally treated as keywords in
phase 7) except in an \grammarterm{attribute-token}\iref{dcl.attr.grammar}.
\begin{note}
The \keyword{register} keyword is unused but
is reserved for future use.
\end{note}

\begin{multicolfloattable}{Keywords}{lex.key}
{lllll}
\keyword{alignas} \\
\keyword{alignof} \\
\keyword{asm} \\
\keyword{auto} \\
\keyword{bool} \\
\keyword{break} \\
\keyword{case} \\
\keyword{catch} \\
\keyword{char} \\
\keyword{char8_t} \\
\keyword{char16_t} \\
\keyword{char32_t} \\
\keyword{class} \\
\keyword{concept} \\
\keyword{const} \\
\keyword{consteval} \\
\keyword{constexpr} \\
\columnbreak
\keyword{constinit} \\
\keyword{const_cast} \\
\keyword{continue} \\
\keyword{co_await} \\
\keyword{co_return} \\
\keyword{co_yield} \\
\keyword{decltype} \\
\keyword{default} \\
\keyword{delete} \\
\keyword{do} \\
\keyword{double} \\
\keyword{dynamic_cast} \\
\keyword{else} \\
\keyword{enum} \\
\keyword{explicit} \\
\keyword{export} \\
\keyword{extern} \\
\columnbreak
\keyword{false} \\
\keyword{float} \\
\keyword{for} \\
\keyword{friend} \\
\keyword{goto} \\
\keyword{if} \\
\keyword{inline} \\
\keyword{int} \\
\keyword{long} \\
\keyword{mutable} \\
\keyword{namespace} \\
\keyword{new} \\
\keyword{noexcept} \\
\keyword{nullptr} \\
\keyword{operator} \\
\keyword{private} \\
\keyword{protected} \\
\columnbreak
\keyword{public} \\
\keyword{register} \\
\keyword{reinterpret_cast} \\
\keyword{requires} \\
\keyword{return} \\
\keyword{short} \\
\keyword{signed} \\
\keyword{sizeof} \\
\keyword{static} \\
\keyword{static_assert} \\
\keyword{static_cast} \\
\keyword{struct} \\
\keyword{switch} \\
\keyword{template} \\
\keyword{this} \\
\keyword{thread_local} \\
\keyword{throw} \\
\columnbreak
\keyword{true} \\
\keyword{try} \\
\keyword{typedef} \\
\keyword{typeid} \\
\keyword{typename} \\
\keyword{union} \\
\keyword{unsigned} \\
\keyword{using} \\
\keyword{virtual} \\
\keyword{void} \\
\keyword{volatile} \\
\keyword{wchar_t} \\
\keyword{while} \\
\end{multicolfloattable}

\pnum
Furthermore, the alternative representations shown in
\tref{lex.key.digraph} for certain operators and
punctuators\iref{lex.digraph} are reserved and shall not be used
otherwise.

\begin{floattable}{Alternative representations}{lex.key.digraph}
{llllll}
\topline
\keyword{and}     &   \keyword{and_eq}  &   \keyword{bitand}  &   \keyword{bitor}   &   \keyword{compl}   &   \keyword{not} \\
\keyword{not_eq}  &   \keyword{or}      &   \keyword{or_eq}   &   \keyword{xor}     &   \keyword{xor_eq}  &       \\
\end{floattable}%
\indextext{keyword|)}%

\rSec2[lex.literal]{Literals}%
\indextext{literal|(}

\rSec3[lex.literal.kinds]{Kinds of literals}

\pnum
\indextext{constant}%
\indextext{literal!constant}%
There are several kinds of literals.
\begin{footnote}
The term ``literal'' generally designates, in this
document, those tokens that are called ``constants'' in C.
\end{footnote}

\begin{bnf}
\nontermdef{literal}\br
    integer-literal\br
    character-literal\br
    floating-point-literal\br
    string-literal\br
    boolean-literal\br
    pointer-literal\br
    user-defined-literal
\end{bnf}
\begin{note}
When appearing as an \grammarterm{expression},
a literal has a type and a value category\iref{expr.prim.literal}.
\end{note}

\rSec3[lex.icon]{Integer literals}

\indextext{literal!integer}%
\begin{bnf}
\nontermdef{integer-literal}\br
    binary-literal \opt{integer-suffix}\br
    octal-literal \opt{integer-suffix}\br
    decimal-literal \opt{integer-suffix}\br
    hexadecimal-literal \opt{integer-suffix}
\end{bnf}

\begin{bnf}
\nontermdef{binary-literal}\br
    \terminal{0b} binary-digit\br
    \terminal{0B} binary-digit\br
    binary-literal \opt{\terminal{'}} binary-digit
\end{bnf}

\begin{bnf}
\nontermdef{octal-literal}\br
    \terminal{0}\br
    octal-literal \opt{\terminal{'}} octal-digit
\end{bnf}

\begin{bnf}
\nontermdef{decimal-literal}\br
    nonzero-digit\br
    decimal-literal \opt{\terminal{'}} digit
\end{bnf}

\begin{bnf}
\nontermdef{hexadecimal-literal}\br
    hexadecimal-prefix hexadecimal-digit-sequence
\end{bnf}

\begin{bnf}
\nontermdef{binary-digit} \textnormal{one of}\br
    \terminal{0  1}
\end{bnf}

\begin{bnf}
\nontermdef{octal-digit} \textnormal{one of}\br
    \terminal{0  1  2  3  4  5  6  7}
\end{bnf}

\begin{bnf}
\nontermdef{nonzero-digit} \textnormal{one of}\br
    \terminal{1  2  3  4  5  6  7  8  9}
\end{bnf}

\begin{bnf}
\nontermdef{hexadecimal-prefix} \textnormal{one of}\br
    \terminal{0x  0X}
\end{bnf}

\begin{bnf}
\nontermdef{hexadecimal-digit-sequence}\br
    hexadecimal-digit\br
    hexadecimal-digit-sequence \opt{\terminal{'}} hexadecimal-digit
\end{bnf}

\begin{bnf}
\nontermdef{hexadecimal-digit} \textnormal{one of}\br
    \terminal{0  1  2  3  4  5  6  7  8  9}\br
    \terminal{a  b  c  d  e  f}\br
    \terminal{A  B  C  D  E  F}
\end{bnf}

\begin{bnf}
\nontermdef{integer-suffix}\br
    unsigned-suffix \opt{long-suffix} \br
    unsigned-suffix \opt{long-long-suffix} \br
    unsigned-suffix \opt{size-suffix} \br
    long-suffix \opt{unsigned-suffix} \br
    long-long-suffix \opt{unsigned-suffix} \br
    size-suffix \opt{unsigned-suffix}
\end{bnf}

\begin{bnf}
\nontermdef{unsigned-suffix} \textnormal{one of}\br
    \terminal{u  U}
\end{bnf}

\begin{bnf}
\nontermdef{long-suffix} \textnormal{one of}\br
    \terminal{l  L}
\end{bnf}

\begin{bnf}
\nontermdef{long-long-suffix} \textnormal{one of}\br
    \terminal{ll  LL}
\end{bnf}

\begin{bnf}
\nontermdef{size-suffix} \textnormal{one of}\br
   \terminal{z  Z}
\end{bnf}

\pnum
\indextext{literal!\idxcode{unsigned}}%
\indextext{literal!\idxcode{long}}%
\indextext{literal!base of integer}%
In an \grammarterm{integer-literal},
the sequence of
\grammarterm{binary-digit}s,
\grammarterm{octal-digit}s,
\grammarterm{digit}s, or
\grammarterm{hexadecimal-digit}s
is interpreted as a base $N$ integer as shown in table \tref{lex.icon.base};
the lexically first digit of the sequence of digits is the most significant.
\begin{note}
The prefix and any optional separating single quotes are ignored
when determining the value.
\end{note}

\begin{simpletypetable}
{Base of \grammarterm{integer-literal}{s}}
{lex.icon.base}
{lr}
\topline
\lhdr{Kind of \grammarterm{integer-literal}} & \rhdr{base $N$} \\ \capsep
\grammarterm{binary-literal} & 2 \\
\grammarterm{octal-literal} & 8 \\
\grammarterm{decimal-literal} & 10 \\
\grammarterm{hexadecimal-literal} & 16 \\
\end{simpletypetable}

\pnum
The \grammarterm{hexadecimal-digit}s
\tcode{a} through \tcode{f} and \tcode{A} through \tcode{F}
have decimal values ten through fifteen.
\begin{example}
The number twelve can be written \tcode{12}, \tcode{014},
\tcode{0XC}, or \tcode{0b1100}. The \grammarterm{integer-literal}s \tcode{1048576},
\tcode{1'048'576}, \tcode{0X100000}, \tcode{0x10'0000}, and
\tcode{0'004'000'000} all have the same value.
\end{example}

\pnum
\indextext{literal!\idxcode{long}}%
\indextext{literal!\idxcode{unsigned}}%
\indextext{literal!integer}%
\indextext{literal!type of integer}%
\indextext{suffix!\idxcode{L}}%
\indextext{suffix!\idxcode{U}}%
\indextext{suffix!\idxcode{l}}%
\indextext{suffix!\idxcode{u}}%
The type of an \grammarterm{integer-literal} is
the first type in the list in \tref{lex.icon.type}
corresponding to its optional \grammarterm{integer-suffix}
in which its value can be represented.

\begin{floattable}{Types of \grammarterm{integer-literal}s}{lex.icon.type}{l|l|l}
\topline
\lhdr{\grammarterm{integer-suffix}} & \chdr{\grammarterm{decimal-literal}}  & \rhdr{\grammarterm{integer-literal} other than \grammarterm{decimal-literal}}   \\  \capsep
none    &
  \tcode{int} &
  \tcode{int}\\
        &
  \tcode{long int} &
  \tcode{unsigned int}\\
        &
  \tcode{long long int} &
  \tcode{long int}\\
        &
        &
  \tcode{unsigned long int}\\
        &
        &
  \tcode{long long int}\\
        &
        &
  \tcode{unsigned long long int}\\\hline
\tcode{u} or \tcode{U}  &
  \tcode{unsigned int}  &
  \tcode{unsigned int}\\
                              &
  \tcode{unsigned long int}   &
  \tcode{unsigned long int}\\
                              &
  \tcode{unsigned long long int}   &
  \tcode{unsigned long long int}\\\hline
\tcode{l} or \tcode{L}  &
  \tcode{long int}  &
  \tcode{long int}\\
                              &
  \tcode{long long int}       &
  \tcode{unsigned long int}\\
                              &
                              &
  \tcode{long long int}\\
                              &
                              &
  \tcode{unsigned long long int}\\\hline
Both \tcode{u} or \tcode{U}   &
  \tcode{unsigned long int}  &
  \tcode{unsigned long int}\\
and \tcode{l} or \tcode{L}  &
  \tcode{unsigned long long int}  &
  \tcode{unsigned long long int}\\\hline
\tcode{ll} or \tcode{LL}  &
  \tcode{long long int}       &
  \tcode{long long int}\\
                              &
                              &
  \tcode{unsigned long long int}\\\hline
Both \tcode{u} or \tcode{U}   &
  \tcode{unsigned long long int}  &
  \tcode{unsigned long long int}\\
and \tcode{ll} or \tcode{LL}  &
                              &
                              \\\hline
\tcode{z} or \tcode{Z}                  &
  the signed integer type corresponding &
  the signed integer type \\
                                        &
  \qquad to \tcode{std::size_t}\iref{support.types.layout} &
  \qquad corresponding to \tcode{std::size_t} \\
                                        &
                                        &
  \tcode{std::size_t}\\\hline
Both \tcode{u} or \tcode{U}   &
  \tcode{std::size_t}         &
  \tcode{std::size_t}         \\
and \tcode{z} or \tcode{Z}  &
                              &
                              \\
\end{floattable}

\pnum
Except for \grammarterm{integer-literal}{s} containing
a \grammarterm{size-suffix},
if the value of an \grammarterm{integer-literal}
cannot be represented by any type in its list and
an extended integer type\iref{basic.fundamental} can represent its value,
it may have that extended integer type.
If all of the types in the list for the \grammarterm{integer-literal}
are signed,
the extended integer type is signed.
If all of the types in the list for the \grammarterm{integer-literal}
are unsigned,
the extended integer type is unsigned.
If the list contains both signed and unsigned types,
the extended integer type may be signed or unsigned.
If an \grammarterm{integer-literal}
cannot be represented by any of the allowed types,
the program is ill-formed.
\begin{note}
An \grammarterm{integer-literal} with a \tcode{z} or \tcode{Z} suffix
is ill-formed if it cannot be represented by \tcode{std::size_t}.
\end{note}

\rSec3[lex.fcon]{Floating-point literals}

\indextext{literal!floating-point}%
\begin{bnf}
\nontermdef{floating-point-literal}\br
    decimal-floating-point-literal\br
    hexadecimal-floating-point-literal
\end{bnf}

\begin{bnf}
\nontermdef{decimal-floating-point-literal}\br
    fractional-constant \opt{exponent-part} \opt{floating-point-suffix}\br
    digit-sequence exponent-part \opt{floating-point-suffix}
\end{bnf}

\begin{bnf}
\nontermdef{hexadecimal-floating-point-literal}\br
    hexadecimal-prefix hexadecimal-fractional-constant binary-exponent-part \opt{floating-point-suffix}\br
    hexadecimal-prefix hexadecimal-digit-sequence binary-exponent-part \opt{floating-point-suffix}
\end{bnf}

\begin{bnf}
\nontermdef{fractional-constant}\br
    \opt{digit-sequence} \terminal{.} digit-sequence\br
    digit-sequence \terminal{.}
\end{bnf}

\begin{bnf}
\nontermdef{hexadecimal-fractional-constant}\br
    \opt{hexadecimal-digit-sequence} \terminal{.} hexadecimal-digit-sequence\br
    hexadecimal-digit-sequence \terminal{.}
\end{bnf}

\begin{bnf}
\nontermdef{exponent-part}\br
    \terminal{e} \opt{sign} digit-sequence\br
    \terminal{E} \opt{sign} digit-sequence
\end{bnf}

\begin{bnf}
\nontermdef{binary-exponent-part}\br
    \terminal{p} \opt{sign} digit-sequence\br
    \terminal{P} \opt{sign} digit-sequence
\end{bnf}

\begin{bnf}
\nontermdef{sign} \textnormal{one of}\br
    \terminal{+  -}
\end{bnf}

\begin{bnf}
\nontermdef{digit-sequence}\br
    digit\br
    digit-sequence \opt{\terminal{'}} digit
\end{bnf}

\begin{bnf}
\nontermdef{floating-point-suffix} \textnormal{one of}\br
    \terminal{f  l  f16  f32  f64  f128  bf16  F  L  F16  F32  F64  F128  BF16}
\end{bnf}

\pnum
\indextext{literal!type of floating-point}%
\indextext{literal!\idxcode{float}}%
\indextext{suffix!\idxcode{F}}%
\indextext{suffix!\idxcode{f}}%
\indextext{suffix!\idxcode{L}}%
\indextext{suffix!\idxcode{l}}%
\indextext{literal!\idxcode{long double}}%
The type of
a \grammarterm{floating-point-literal}\iref{basic.fundamental,basic.extended.fp}
is determined by
its \grammarterm{floating-point-suffix} as specified in \tref{lex.fcon.type}.
\begin{note}
The floating-point suffixes
\tcode{f16}, \tcode{f32}, \tcode{f64}, \tcode{f128}, \tcode{bf16},
\tcode{F16}, \tcode{F32}, \tcode{F64}, \tcode{F128}, and \tcode{BF16}
are conditionally-supported. See \ref{basic.extended.fp}.
\end{note}
\begin{simpletypetable}
{Types of \grammarterm{floating-point-literal}{s}}
{lex.fcon.type}
{ll}
\topline
\lhdr{\grammarterm{floating-point-suffix}} & \rhdr{type} \\ \capsep
none & \keyword{double} \\
\tcode{f} or \tcode{F} & \keyword {float} \\
\tcode{l} or \tcode{L} & \keyword{long} \keyword{double} \\
\tcode{f16} or \tcode{F16} & \tcode{std::float16_t} \\
\tcode{f32} or \tcode{F32} & \tcode{std::float32_t} \\
\tcode{f64} or \tcode{F64} & \tcode{std::float64_t} \\
\tcode{f128} or \tcode{F128} & \tcode{std::float128_t} \\
\tcode{bf16} or \tcode{BF16} & \tcode{std::bfloat16_t} \\
\end{simpletypetable}

\pnum
\indextext{literal!floating-point}%
The \defn{significand} of a \grammarterm{floating-point-literal}
is the \grammarterm{fractional-constant} or \grammarterm{digit-sequence}
of a \grammarterm{decimal-floating-point-literal}
or the \grammarterm{hexadecimal-fractional-constant}
or \grammarterm{hexadecimal-digit-sequence}
of a \grammarterm{hexadecimal-floating-point-literal}.
In the significand,
the sequence of \grammarterm{digit}s or \grammarterm{hexadecimal-digit}s
and optional period are interpreted as a base $N$ real number $s$,
where $N$ is 10 for a \grammarterm{decimal-floating-point-literal} and
16 for a \grammarterm{hexadecimal-floating-point-literal}.
\begin{note}
Any optional separating single quotes are ignored when determining the value.
\end{note}
If an \grammarterm{exponent-part} or \grammarterm{binary-exponent-part}
is present,
the exponent $e$ of the \grammarterm{floating-point-literal}
is the result of interpreting
the sequence of an optional \grammarterm{sign} and the \grammarterm{digit}s
as a base 10 integer.
Otherwise, the exponent $e$ is 0.
The scaled value of the literal is
$s \times 10^e$ for a \grammarterm{decimal-floating-point-literal} and
$s \times 2^e$ for a \grammarterm{hexadecimal-floating-point-literal}.
\begin{example}
The \grammarterm{floating-point-literal}{s}
\tcode{49.625} and \tcode{0xC.68p+2} have the same value.
The \grammarterm{floating-point-literal}{s}
\tcode{1.602'176'565e-19} and \tcode{1.602176565e-19}
have the same value.
\end{example}

\pnum
If the scaled value is not in the range of representable
values for its type, the program is ill-formed.
Otherwise, the value of a \grammarterm{floating-point-literal}
is the scaled value if representable,
else the larger or smaller representable value nearest the scaled value,
chosen in an \impldef{choice of larger or smaller value of
\grammarterm{floating-point-literal}} manner.

\rSec3[lex.string.uneval]{Unevaluated strings}

\begin{bnf}
\nontermdef{unevaluated-string}\br
    string-literal
\end{bnf}

\pnum
An \grammarterm{unevaluated-string} shall have no \grammarterm{encoding-prefix}.

\pnum
Each \grammarterm{universal-character-name} and each \grammarterm{simple-escape-sequence} in an \grammarterm{unevaluated-string} is
replaced by the member of the translation character set it denotes.
An \grammarterm{unevaluated-string} that contains
a \grammarterm{numeric-escape-sequence} or
a \grammarterm{conditional-escape-sequence}
is ill-formed.

\pnum
An \grammarterm{unevaluated-string} is never evaluated and
its interpretation depends on the context in which it appears.

\rSec3[lex.bool]{Boolean literals}

\indextext{literal!boolean}%
\begin{bnf}
\nontermdef{boolean-literal}\br
    \terminal{false}\br
    \terminal{true}
\end{bnf}

\pnum
\indextext{Boolean literal}%
The Boolean literals are the keywords \tcode{false} and \tcode{true}.
Such literals have type \tcode{bool}.

\rSec3[lex.nullptr]{Pointer literals}

\indextext{literal!pointer}%
\begin{bnf}
\nontermdef{pointer-literal}\br
    \terminal{nullptr}
\end{bnf}

\pnum
The pointer literal is the keyword \keyword{nullptr}. It has type
\tcode{std::nullptr_t}.
\begin{note}
\tcode{std::nullptr_t} is a distinct type that is neither a pointer type nor a pointer-to-member type;
rather, a prvalue of this type is a null pointer constant and can be
converted to a null pointer value or null member pointer value. See~\ref{conv.ptr}
and~\ref{conv.mem}.
\end{note}

\rSec3[lex.ext]{User-defined literals}

\indextext{literal!user-defined}%
\begin{bnf}
\nontermdef{user-defined-literal}\br
    user-defined-integer-literal\br
    user-defined-floating-point-literal\br
    user-defined-string-literal\br
    user-defined-character-literal
\end{bnf}

\begin{bnf}
\nontermdef{user-defined-integer-literal}\br
    decimal-literal ud-suffix\br
    octal-literal ud-suffix\br
    hexadecimal-literal ud-suffix\br
    binary-literal ud-suffix
\end{bnf}

\begin{bnf}
\nontermdef{user-defined-floating-point-literal}\br
    fractional-constant \opt{exponent-part} ud-suffix\br
    digit-sequence exponent-part ud-suffix\br
    hexadecimal-prefix hexadecimal-fractional-constant binary-exponent-part ud-suffix\br
    hexadecimal-prefix hexadecimal-digit-sequence binary-exponent-part ud-suffix
\end{bnf}

\begin{bnf}
\nontermdef{user-defined-string-literal}\br
    string-literal ud-suffix
\end{bnf}

\begin{bnf}
\nontermdef{user-defined-character-literal}\br
    character-literal ud-suffix
\end{bnf}

\begin{bnf}
\nontermdef{ud-suffix}\br
    identifier
\end{bnf}

\pnum
If a token matches both \grammarterm{user-defined-literal} and another \grammarterm{literal} kind, it
is treated as the latter.
\begin{example}
\tcode{123_km}
is a \grammarterm{user-defined-literal}, but \tcode{12LL} is an
\grammarterm{integer-literal}.
\end{example}
The syntactic non-terminal preceding the \grammarterm{ud-suffix} in a
\grammarterm{user-defined-literal} is taken to be the longest sequence of
characters that could match that non-terminal.

\pnum
A \grammarterm{user-defined-literal} is treated as a call to a literal operator or
literal operator template\iref{over.literal}.
To determine the form of this call for
a given \grammarterm{user-defined-literal} \placeholder{L}
with \grammarterm{ud-suffix} \placeholder{X},
first let \placeholder{S} be the set of declarations
found by unqualified lookup for the \grammarterm{literal-operator-id}
whose literal suffix identifier is \placeholder{X}\iref{basic.lookup.unqual}.
\placeholder{S} shall not be empty.

\pnum
If \placeholder{L} is a \grammarterm{user-defined-integer-literal}, let \placeholder{n} be the literal
without its \grammarterm{ud-suffix}. If \placeholder{S} contains a literal operator with
parameter type \tcode{unsigned long long}, the literal \placeholder{L} is treated as a call of
the form
\begin{codeblock}
operator ""@\placeholder{X}@(@\placeholder{n}@ULL)
\end{codeblock}
Otherwise, \placeholder{S} shall contain a raw literal operator
or a numeric literal operator template\iref{over.literal} but not both.
If \placeholder{S} contains a raw literal operator,
the literal \placeholder{L} is treated as a call of the form
\begin{codeblock}
operator ""@\placeholder{X}@("@\placeholder{n}@")
\end{codeblock}
Otherwise (\placeholder{S} contains a numeric literal operator template),
\placeholder{L} is treated as a call of the form
\begin{codeblock}
operator ""@\placeholder{X}@<'@$c_1$@', '@$c_2$@', ... '@$c_k$@'>()
\end{codeblock}
where \placeholder{n} is the source character sequence $c_1c_2...c_k$.
\begin{note}
The sequence
$c_1c_2...c_k$ can only contain characters from the basic character set.
\end{note}

\pnum
If \placeholder{L} is a \grammarterm{user-defined-floating-point-literal}, let \placeholder{f} be the
literal without its \grammarterm{ud-suffix}. If \placeholder{S} contains a literal operator
with parameter type \tcode{long double}, the literal \placeholder{L} is treated as a call of
the form
\begin{codeblock}
operator ""@\placeholder{X}@(@\placeholder{f}@L)
\end{codeblock}
Otherwise, \placeholder{S} shall contain a raw literal operator
or a numeric literal operator template\iref{over.literal} but not both.
If \placeholder{S} contains a raw literal operator,
the \grammarterm{literal} \placeholder{L} is treated as a call of the form
\begin{codeblock}
operator ""@\placeholder{X}@("@\placeholder{f}@")
\end{codeblock}
Otherwise (\placeholder{S} contains a numeric literal operator template),
\placeholder{L} is treated as a call of the form
\begin{codeblock}
operator ""@\placeholder{X}@<'@$c_1$@', '@$c_2$@', ... '@$c_k$@'>()
\end{codeblock}
where \placeholder{f} is the source character sequence $c_1c_2...c_k$.
\begin{note}
The sequence
$c_1c_2...c_k$ can only contain characters from the basic character set.
\end{note}

\pnum
If \placeholder{L} is a \grammarterm{user-defined-string-literal},
let \placeholder{str} be the literal without its \grammarterm{ud-suffix}
and let \placeholder{len} be the number of code units in \placeholder{str}
(i.e., its length excluding the terminating null character).
If \placeholder{S} contains a literal operator template with
a non-type template parameter for which \placeholder{str} is
a well-formed \grammarterm{template-argument},
the literal \placeholder{L} is treated as a call of the form
\begin{codeblock}
operator ""@\placeholder{X}@<@\placeholder{str}{}@>()
\end{codeblock}
Otherwise, the literal \placeholder{L} is treated as a call of the form
\begin{codeblock}
operator ""@\placeholder{X}@(@\placeholder{str}{}@, @\placeholder{len}{}@)
\end{codeblock}

\pnum
If \placeholder{L} is a \grammarterm{user-defined-character-literal}, let \placeholder{ch} be the
literal without its \grammarterm{ud-suffix}.
\placeholder{S} shall contain a literal operator\iref{over.literal} whose only parameter has
the type of \placeholder{ch} and the
literal \placeholder{L} is treated as a call
of the form
\begin{codeblock}
operator ""@\placeholder{X}@(@\placeholder{ch}{}@)
\end{codeblock}

\pnum
\begin{example}
\begin{codeblock}
long double operator ""_w(long double);
std::string operator ""_w(const char16_t*, std::size_t);
unsigned operator ""_w(const char*);
int main() {
  1.2_w;            // calls \tcode{operator ""_w(1.2L)}
  u"one"_w;         // calls \tcode{operator ""_w(u"one", 3)}
  12_w;             // calls \tcode{operator ""_w("12")}
  "two"_w;          // error: no applicable literal operator
}
\end{codeblock}
\end{example}

\pnum
In translation phase 6\iref{lex.phase.6}, adjacent \grammarterm{string-literal}s are concatenated and
\grammarterm{user-defined-string-literal}{s} are considered \grammarterm{string-literal}s for that
purpose. During concatenation, \grammarterm{ud-suffix}{es} are removed and ignored and
the concatenation process occurs as described in~\ref{lex.string}. At the end of phase
6, if a \grammarterm{string-literal} is the result of a concatenation involving at least one
\grammarterm{user-defined-string-literal}, all the participating
\grammarterm{user-defined-string-literal}{s} shall have the same \grammarterm{ud-suffix}
and that suffix is applied to the result of the concatenation.

\pnum
\begin{example}
\begin{codeblock}
int main() {
  L"A" "B" "C"_x;   // OK, same as \tcode{L"ABC"_x}
  "P"_x "Q" "R"_y;  // error: two different \grammarterm{ud-suffix}{es}
}
\end{codeblock}
\end{example}
\indextext{literal|)}%
\indextext{conventions!lexical|)}
