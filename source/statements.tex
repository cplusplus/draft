%!TEX root = std.tex
\rSec0[stmt.stmt]{Statements}%
\indextext{statement|(}

\gramSec[gram.stmt]{Statements}

\indextext{block statement|see{statement, compound}}

\pnum
Except as indicated, statements are executed in sequence.

\begin{bnf}
\nontermdef{statement}\br
    labeled-statement\br
    \opt{attribute-specifier-seq} expression-statement\br
    \opt{attribute-specifier-seq} compound-statement\br
    \opt{attribute-specifier-seq} selection-statement\br
    \opt{attribute-specifier-seq} iteration-statement\br
    \opt{attribute-specifier-seq} jump-statement\br
    declaration-statement\br
    \opt{attribute-specifier-seq} try-block

\nontermdef{init-statement}\br
    expression-statement\br
    simple-declaration

\nontermdef{condition}\br
    expression\br
    \opt{attribute-specifier-seq} decl-specifier-seq declarator brace-or-equal-initializer
\end{bnf}

The optional \grammarterm{attribute-specifier-seq} appertains to the respective statement.

\pnum
A \defn{substatement} of a \grammarterm{statement} is one of the following:
\begin{itemize}
\item
  for a \grammarterm{labeled-statement}, its contained \grammarterm{statement},
\item
  for a \grammarterm{compound-statement}, any \grammarterm{statement} of its \grammarterm{statement-seq},
\item
  for a \grammarterm{selection-statement}, any of its \grammarterm{statement}{s} (but not its \grammarterm{init-statement}), or
\item
  for an \grammarterm{iteration-statement}, its contained \grammarterm{statement} (but not an \grammarterm{init-statement}).
\end{itemize}
\begin{note}
The \grammarterm{compound-statement} of a \grammarterm{lambda-expression}
is not a substatement of the \grammarterm{statement} (if any)
in which the \grammarterm{lambda-expression} lexically appears.
\end{note}

\pnum
A \grammarterm{statement} \tcode{S1} \defnx{encloses}{enclosing statement}
a \grammarterm{statement} \tcode{S2} if
\begin{itemize}
\item
  \tcode{S2} is a substatement of \tcode{S1}\iref{dcl.dcl},
\item
  \tcode{S1} is a \grammarterm{selection-statement} or
  \grammarterm{iteration-statement} and
  \tcode{S2} is the \grammarterm{init-statement} of \tcode{S1},
\item
  \tcode{S1} is a \grammarterm{try-block} and \tcode{S2}
  is its \grammarterm{compound-statement} or
  any of the \grammarterm{compound-statement}{s} of
  its \grammarterm{handler}{s}, or
\item
  \tcode{S1} encloses a statement \tcode{S3} and \tcode{S3} encloses \tcode{S2}.
\end{itemize}

\pnum
\indextext{\idxgram{condition}{s}!rules for}%
The rules for \grammarterm{condition}{s} apply both to
\grammarterm{selection-statement}{s} and to the \tcode{for} and \tcode{while}
statements\iref{stmt.iter}.
A \grammarterm{condition} that is not an \grammarterm{expression} is a
declaration\iref{dcl.dcl}.
The \grammarterm{declarator} shall not
specify a function or an array. The \grammarterm{decl-specifier-seq} shall not
define a class or enumeration. If the \tcode{auto} \grammarterm{type-specifier} appears in
the \grammarterm{decl-specifier-seq},
the type of the identifier being declared is deduced from the initializer as described in~\ref{dcl.spec.auto}.

\pnum
\indextext{statement!declaration in \tcode{if}}%
\indextext{statement!declaration in \tcode{switch}}%
A name introduced by a declaration in a \grammarterm{condition}
(introduced by either the \grammarterm{decl-specifier-seq} or the
\grammarterm{declarator} of the condition) is in scope from its point of
declaration until the end of the substatements controlled by the
condition. If the name is redeclared in the outermost block of a
substatement controlled by the condition, the declaration that
redeclares the name is ill-formed.
\begin{example}

\begin{codeblock}
if (int x = f()) {
  int x;            // ill-formed, redeclaration of \tcode{x}
}
else {
  int x;            // ill-formed, redeclaration of \tcode{x}
}
\end{codeblock}
\end{example}

\pnum
The value of a \grammarterm{condition} that is an initialized declaration
in a statement other than a \tcode{switch} statement is the value of the
declared variable
contextually converted to \tcode{bool}\iref{conv}.
If that
conversion is ill-formed, the program is ill-formed. The value of a
\grammarterm{condition} that is an initialized declaration in a
\tcode{switch} statement is the value of the declared variable if it has
integral or enumeration type, or of that variable implicitly converted
to integral or enumeration type otherwise. The value of a
\grammarterm{condition} that is an expression is the value of the
expression, contextually converted to \tcode{bool}
for statements other
than \tcode{switch};
if that conversion is ill-formed, the program is
ill-formed. The value of the condition will be referred to as simply
``the condition'' where the usage is unambiguous.

\pnum
If a \grammarterm{condition} can be syntactically resolved as either an
expression or the declaration of a block-scope name, it is interpreted as a
declaration.

\pnum
In the \grammarterm{decl-specifier-seq} of a \grammarterm{condition}, each
\grammarterm{decl-specifier} shall be either a \grammarterm{type-specifier}
or \tcode{constexpr}.

\rSec1[stmt.label]{Labeled statement}%
\indextext{statement!labeled}

\pnum
\indextext{statement!labeled}%
\indextext{\idxcode{:}!label specifier}%
A statement can be labeled.

\begin{bnf}
\nontermdef{labeled-statement}\br
    \opt{attribute-specifier-seq} identifier \terminal{:} statement\br
    \opt{attribute-specifier-seq} \keyword{case} constant-expression \terminal{:} statement\br
    \opt{attribute-specifier-seq} \keyword{default} \terminal{:} statement
\end{bnf}

The optional \grammarterm{attribute-specifier-seq} appertains to the label. An
\defn{identifier label} declares the identifier. The only use of an
identifier label is as the target of a
\indextext{statement!\idxcode{goto}}%
\tcode{goto}.
\indextext{label!scope of}%
The scope of a label is the function in which it appears. Labels shall
not be redeclared within a function. A label can be used in a
\tcode{goto} statement before its declaration.
\indextext{name space!label}%
Labels have their own name space and do not interfere with other
identifiers.
\begin{note}
A label may have the same name as another declaration in the same scope or a
\grammarterm{template-parameter} from an enclosing scope. Unqualified name
lookup\iref{basic.lookup.unqual} ignores labels.
\end{note}

\pnum
\indextext{label!\idxcode{case}}%
\indextext{label!\idxcode{default}}%
Case labels and default labels shall occur only in \tcode{switch} statements.


\rSec1[stmt.expr]{Expression statement}%
\indextext{statement!expression}

\pnum
Expression statements have the form

\begin{bnf}
\nontermdef{expression-statement}\br
    \opt{expression} \terminal{;}
\end{bnf}

The expression is
a discarded-value expression\iref{expr.prop}.
All
\indextext{side effects}%
side effects from an expression statement
are completed before the next statement is executed.
\indextext{statement!null}%
\indextext{statement!empty}%
An expression statement with the expression missing is called
a \defn{null statement}.
\begin{note}
Most statements are expression statements --- usually assignments or
function calls. A null statement is useful to carry a label just before
the \tcode{\}} of a compound statement and to supply a null body to an
iteration statement such as a \tcode{while}
statement\iref{stmt.while}.
\end{note}

\rSec1[stmt.block]{Compound statement or block}%
\indextext{statement!compound}%
\indextext{\idxcode{\{\}}!block statement}%

\pnum
So that several statements can be used where one is expected, the
compound statement (also, and equivalently, called ``block'') is
provided.

\begin{bnf}
\nontermdef{compound-statement}\br
    \terminal{\{} \opt{statement-seq} \terminal{\}}
\end{bnf}

\begin{bnf}
\nontermdef{statement-seq}\br
    statement\br
    statement-seq statement
\end{bnf}

A compound statement defines a block scope\iref{basic.scope}.
\begin{note}
A declaration is a \grammarterm{statement}\iref{stmt.dcl}.
\end{note}

\rSec1[stmt.select]{Selection statements}%
\indextext{statement!selection|(}

\pnum
Selection statements choose one of several flows of control.

\indextext{statement!\idxcode{if}}%
\indextext{statement!\idxcode{switch}}%
%
\begin{bnf}
\nontermdef{selection-statement}\br
    \keyword{if} \opt{\keyword{constexpr}} \terminal{(} \opt{init-statement} condition \terminal{)} statement\br
    \keyword{if} \opt{\keyword{constexpr}} \terminal{(} \opt{init-statement} condition \terminal{)} statement \keyword{else} statement\br
    \keyword{switch} \terminal{(} \opt{init-statement} condition \terminal{)} statement
\end{bnf}

See~\ref{dcl.meaning} for the optional \grammarterm{attribute-specifier-seq} in a condition.
\begin{note}
An \grammarterm{init-statement} ends with a semicolon.
\end{note}

\pnum
\indextext{scope!\idxgram{selection-statement}}%
The substatement in a \grammarterm{selection-statement} (each substatement,
in the \tcode{else} form of the \tcode{if} statement) implicitly defines
a block scope\iref{basic.scope}. If the substatement in a
\grammarterm{selection-statement} is a single statement and not a
\grammarterm{compound-statement}, it is as if it was rewritten to be a
\grammarterm{compound-statement} containing the original substatement.
\begin{example}

\begin{codeblock}
if (x)
  int i;
\end{codeblock}

can be equivalently rewritten as

\begin{codeblock}
if (x) {
  int i;
}
\end{codeblock}

Thus after the \tcode{if} statement, \tcode{i} is no longer in scope.
\end{example}

\rSec2[stmt.if]{The \tcode{if} statement}%
\indextext{statement!\idxcode{if}}

\pnum
If the condition\iref{stmt.select} yields \tcode{true} the first
substatement is executed. If the \tcode{else} part of the selection
statement is present and the condition yields \tcode{false}, the second
substatement is executed. If the first substatement is reached via a
label, the condition is not evaluated and the second substatement is
not executed. In the second form of \tcode{if} statement
(the one including \tcode{else}), if the first substatement is also an
\tcode{if} statement then that inner \tcode{if} statement shall contain
an \tcode{else} part.\footnote{In other words, the \tcode{else} is associated with the nearest un-elsed
\tcode{if}.}

\pnum
If the \tcode{if} statement is of the form \tcode{if constexpr}, the value
of the condition shall be a contextually
converted constant expression of type \tcode{bool}\iref{expr.const}; this
form is called a \defn{constexpr if} statement. If the value of the
converted condition is \tcode{false}, the first substatement is a
\defn{discarded statement}, otherwise the second substatement, if
present, is a discarded statement. During the instantiation of an
enclosing templated entity\iref{temp}, if the condition is
not value-dependent after its instantiation, the discarded substatement
(if any) is not instantiated.
\begin{note}
Odr-uses\iref{basic.def.odr} in a discarded statement do not require
an entity to be defined.
\end{note}
A \tcode{case} or \tcode{default} label appearing within such an
\tcode{if} statement shall be associated with a \tcode{switch}
statement\iref{stmt.switch} within the same \tcode{if} statement.
A label\iref{stmt.label} declared in a substatement of a constexpr if
statement shall only be referred to by a statement\iref{stmt.goto} in
the same substatement.
\begin{example}
\begin{codeblock}
template<typename T, typename ... Rest> void g(T&& p, Rest&& ...rs) {
  // ... handle \tcode{p}

  if constexpr (sizeof...(rs) > 0)
    g(rs...);       // never instantiated with an empty argument list
}

extern int x;       // no definition of \tcode{x} required

int f() {
  if constexpr (true)
    return 0;
  else if (x)
    return x;
  else
    return -x;
}
\end{codeblock}
\end{example}

\pnum
An \tcode{if} statement of the form
\begin{ncsimplebnf}
\keyword{if} \opt{\keyword{constexpr}} \terminal{(} init-statement condition \terminal{)} statement
\end{ncsimplebnf}
is equivalent to
\begin{ncsimplebnf}
\terminal{\{}\br
\bnfindent init-statement\br
\bnfindent \keyword{if} \opt{\keyword{constexpr}} \terminal{(} condition \terminal{)} statement\br
\terminal{\}}
\end{ncsimplebnf}
and an \tcode{if} statement of the form
\begin{ncsimplebnf}
\keyword{if} \opt{\keyword{constexpr}} \terminal{(} init-statement condition \terminal{)} statement \keyword{else} statement
\end{ncsimplebnf}
is equivalent to
\begin{ncsimplebnf}
\terminal{\{}\br
\bnfindent init-statement\br
\bnfindent \keyword{if} \opt{\keyword{constexpr}} \terminal{(} condition \terminal{)} statement \keyword{else} statement\br
\terminal{\}}
\end{ncsimplebnf}
except that names declared in the \grammarterm{init-statement} are in
the same declarative region as those declared in the
\grammarterm{condition}.

\rSec2[stmt.switch]{The \tcode{switch} statement}%
\indextext{statement!\idxcode{switch}}

\pnum
The \tcode{switch} statement causes control to be transferred to one of
several statements depending on the value of a condition.

\pnum
The condition shall be of integral type, enumeration type, or class
type. If of class type, the
condition is contextually implicitly converted\iref{conv} to
an integral or enumeration type.
If the (possibly converted) type is subject to integral
promotions\iref{conv.prom}, the condition is converted
to the promoted type.
Any
statement within the \tcode{switch} statement can be labeled with one or
more case labels as follows:

\begin{ncbnf}
\indextext{label!\idxcode{case}}%
\keyword{case} constant-expression \terminal{:}
\end{ncbnf}

where the \grammarterm{constant-expression} shall be
a converted constant expression\iref{expr.const} of the
adjusted type of the switch condition. No two of the case constants in
the same switch shall have the same value after conversion.

\pnum
\indextext{label!\idxcode{default}}%
There shall be at most one label of the form

\begin{codeblock}
default :
\end{codeblock}

within a \tcode{switch} statement.

\pnum
Switch statements can be nested; a \tcode{case} or \tcode{default} label
is associated with the smallest switch enclosing it.

\pnum
When the \tcode{switch} statement is executed, its condition is
evaluated.
\indextext{label!\idxcode{case}}%
If one of the case constants has the same value as the condition,
control is passed to the statement following the matched case label. If
no case constant matches the condition, and if there is a
\indextext{label!\idxcode{default}}%
\tcode{default} label, control passes to the statement labeled by the
default label. If no case matches and if there is no \tcode{default}
then none of the statements in the switch is executed.

\pnum
\tcode{case} and \tcode{default} labels in themselves do not alter the
flow of control, which continues unimpeded across such labels. To exit
from a switch, see \tcode{break}, \ref{stmt.break}.
\begin{note}
Usually, the substatement that is the subject of a switch is compound
and \tcode{case} and \tcode{default} labels appear on the top-level
statements contained within the (compound) substatement, but this is not
required.
\indextext{statement!declaration in \tcode{switch}}%
Declarations can appear in the substatement of a
\tcode{switch} statement.
\end{note}

\pnum
A \tcode{switch} statement of the form
\begin{ncsimplebnf}
\keyword{switch} \terminal{(} init-statement condition \terminal{)} statement
\end{ncsimplebnf}
is equivalent to
\begin{ncsimplebnf}
\terminal{\{}\br
\bnfindent init-statement\br
\bnfindent \keyword{switch} \terminal{(} condition \terminal{)} statement\br
\terminal{\}}
\end{ncsimplebnf}
except that names declared in the \grammarterm{init-statement} are in
the same declarative region as those declared in the
\grammarterm{condition}.%
\indextext{statement!selection|)}

\rSec1[stmt.iter]{Iteration statements}%
\indextext{statement!iteration|(}

\pnum
Iteration statements specify looping.

\indextext{statement!\idxcode{while}}%
\indextext{statement!\idxcode{do}}%
\indextext{statement!\idxcode{for}}%
%
\begin{bnf}
\nontermdef{iteration-statement}\br
    \keyword{while} \terminal{(} condition \terminal{)} statement\br
    \keyword{do} statement \keyword{while} \terminal{(} expression \terminal{) ;}\br
    \keyword{for} \terminal{(} init-statement \opt{condition} \terminal{;} \opt{expression} \terminal{)} statement\br
    \keyword{for} \terminal{(} \opt{init-statement} for-range-declaration \terminal{:} for-range-initializer \terminal{)} statement
\end{bnf}

\begin{bnf}
\nontermdef{for-range-declaration}\br
    \opt{attribute-specifier-seq} decl-specifier-seq declarator\br
    \opt{attribute-specifier-seq} decl-specifier-seq \opt{ref-qualifier} \terminal{[} identifier-list \terminal{]}
\end{bnf}

\begin{bnf}
\nontermdef{for-range-initializer}\br
    expr-or-braced-init-list
\end{bnf}

See~\ref{dcl.meaning} for the optional \grammarterm{attribute-specifier-seq} in a
\grammarterm{for-range-declaration}.
\begin{note}
An \grammarterm{init-statement} ends with a semicolon.
\end{note}

\pnum
\indextext{scope!\idxgram{iteration-statement}}%
The substatement in an \grammarterm{iteration-statement} implicitly defines
a block scope\iref{basic.scope} which is entered and exited each time
through the loop.
If the substatement in an \grammarterm{iteration-statement} is
a single statement and not a \grammarterm{compound-statement},
it is as if it was rewritten to be
a \grammarterm{compound-statement} containing the original statement.
\begin{example}

\begin{codeblock}
while (--x >= 0)
  int i;
\end{codeblock}

can be equivalently rewritten as

\begin{codeblock}
while (--x >= 0) {
  int i;
}
\end{codeblock}

Thus after the \tcode{while} statement, \tcode{i} is no longer in scope.
\end{example}

\pnum
If a name introduced in an
\grammarterm{init-statement} or \grammarterm{for-range-declaration}
is redeclared in the outermost block of the substatement, the program is ill-formed.
\begin{example}
\begin{codeblock}
void f() {
  for (int i = 0; i < 10; ++i)
    int i = 0;          // error: redeclaration
  for (int i : { 1, 2, 3 })
    int i = 1;          // error: redeclaration
}
\end{codeblock}
\end{example}

\rSec2[stmt.while]{The \tcode{while} statement}%
\indextext{statement!\idxcode{while}}

\pnum
In the \tcode{while} statement the substatement is executed repeatedly
until the value of the condition\iref{stmt.select} becomes
\tcode{false}. The test takes place before each execution of the
substatement.

\pnum
\indextext{statement!declaration in \tcode{while}}%
When the condition of a \tcode{while} statement is a declaration, the scope of
the variable that is declared extends from its point of
declaration\iref{basic.scope.pdecl} to the end of the \tcode{while}
\grammarterm{statement}. A \tcode{while} statement is equivalent to
\begin{ncsimplebnf}
\exposid{label} \terminal{:}\br
\terminal{\{}\br
\bnfindent \keyword{if} \terminal{(} condition \terminal{) \{}\br
\bnfindent \bnfindent statement\br
\bnfindent \bnfindent \keyword{goto} \exposid{label} \terminal{;}\br
\bnfindent \terminal{\}}\br
\terminal{\}}
\end{ncsimplebnf}
\begin{note}
The variable created in the condition is destroyed and created with each
iteration of the loop.
\begin{example}
\begin{codeblock}
struct A {
  int val;
  A(int i) : val(i) { }
  ~A() { }
  operator bool() { return val != 0; }
};
int i = 1;
while (A a = i) {
  // ...
  i = 0;
}
\end{codeblock}
In the while-loop, the constructor and destructor are each called twice,
once for the condition that succeeds and once for the condition that
fails.
\end{example}
\end{note}

\rSec2[stmt.do]{The \tcode{do} statement}%
\indextext{statement!\idxcode{do}}

\pnum
The expression is contextually converted to \tcode{bool}\iref{conv};
if that conversion is ill-formed, the program is ill-formed.

\pnum
In the \tcode{do} statement the substatement is executed repeatedly
until the value of the expression becomes \tcode{false}. The test takes
place after each execution of the statement.

\rSec2[stmt.for]{The \tcode{for} statement}%
\indextext{statement!\idxcode{for}}

\pnum
The \tcode{for} statement
\begin{ncsimplebnf}
\keyword{for} \terminal{(} init-statement \opt{condition} \terminal{;} \opt{expression} \terminal{)} statement
\end{ncsimplebnf}
is equivalent to
\begin{ncsimplebnf}
\terminal{\{}\br
\bnfindent init-statement\br
\bnfindent \keyword{while} \terminal{(} condition \terminal{) \{}\br
\bnfindent\bnfindent statement\br
\bnfindent\bnfindent expression \terminal{;}\br
\bnfindent \terminal{\}}\br
\terminal{\}}
\end{ncsimplebnf}
except that names declared in the \grammarterm{init-statement} are in
the same declarative region as those declared in the
\grammarterm{condition}, and except that a
\indextext{statement!\tcode{continue} in \tcode{for}}%
\tcode{continue} in \grammarterm{statement} (not enclosed in another
iteration statement) will execute \grammarterm{expression} before
re-evaluating \grammarterm{condition}.
\begin{note}
Thus the first statement specifies initialization for the loop; the
condition\iref{stmt.select} specifies a test, sequenced before each
iteration, such that the loop is exited when the condition becomes
\tcode{false}; the expression often specifies incrementing that is
sequenced after each iteration.
\end{note}

\pnum
Either or both of the \grammarterm{condition}
and the \grammarterm{expression} can be omitted.
A missing \grammarterm{condition}
makes the implied \tcode{while} clause
equivalent to \tcode{while(true)}.

\pnum
\indextext{statement!declaration in \tcode{for}}%
\indextext{\idxcode{for}!scope of declaration in}%
If the \grammarterm{init-statement} is a declaration, the scope of the
name(s) declared extends to the end of the \tcode{for} statement.
\begin{example}

\begin{codeblock}
int i = 42;
int a[10];

for (int i = 0; i < 10; i++)
  a[i] = i;

int j = i;          // \tcode{j = 42}
\end{codeblock}
\end{example}

\rSec2[stmt.ranged]{The range-based \tcode{for} statement}%
\indextext{statement!range based for@range based \tcode{for}}

\pnum
The range-based \tcode{for} statement
\begin{ncsimplebnf}
\keyword{for} \terminal{(} \opt{init-statement} for-range-declaration \terminal{:} for-range-initializer \terminal{)} statement
\end{ncsimplebnf}
is equivalent to
\begin{ncsimplebnf}
\terminal{\{}\br
\bnfindent \opt{init-statement}\br
\bnfindent \keyword{auto} \terminal{\&\&}\exposid{range} \terminal{=} for-range-initializer \terminal{;}\br
\bnfindent \keyword{auto} \exposid{begin} \terminal{=} begin-expr \terminal{;}\br
\bnfindent \keyword{auto} \exposid{end} \terminal{=} end-expr \terminal{;}\br
\bnfindent \keyword{for} \terminal{( ;} \exposid{begin} \terminal{!=} \exposid{end}\terminal{; ++}\exposid{begin} \terminal{) \{}\br
\bnfindent\bnfindent for-range-declaration \terminal{= *} \exposid{begin} \terminal{;}\br
\bnfindent\bnfindent statement\br
\bnfindent \terminal{\}}\br
\terminal{\}}
\end{ncsimplebnf}
where
\begin{itemize}
\item
if the \grammarterm{for-range-initializer} is an \grammarterm{expression},
it is regarded as if it were surrounded by parentheses (so that a comma operator
cannot be reinterpreted as delimiting two \grammarterm{init-declarator}{s});

\item \exposid{range}, \exposid{begin}, and \exposid{end} are variables defined for
exposition only; and

\item
\placeholder{begin-expr} and \placeholder{end-expr} are determined as follows:

\begin{itemize}
\item if the \grammarterm{for-range-initializer} is an expression of
array type \tcode{R}, \placeholder{begin-expr} and \placeholder{end-expr} are
\exposid{range} and \exposid{range} \tcode{+} \tcode{N}, respectively,
where \tcode{N} is
the array bound. If \tcode{R} is an array of unknown bound or an array of
incomplete type, the program is ill-formed;

\item if the \grammarterm{for-range-initializer} is an expression of
class type \tcode{C}, the \grammarterm{unqualified-id}{s}
\tcode{begin} and \tcode{end} are looked up in the scope of \tcode{C}
as if by class member access lookup\iref{basic.lookup.classref}, and if
both find at least one declaration, \placeholder{begin-expr} and
\placeholder{end-expr} are \tcode{\exposid{range}.begin()} and \tcode{\exposid{range}.end()},
respectively;

\item otherwise, \placeholder{begin-expr} and \placeholder{end-expr} are
\tcode{begin(\exposid{range})} and \tcode{end(\exposid{range})}, respectively,
where \tcode{begin} and \tcode{end} are looked
up in the associated namespaces\iref{basic.lookup.argdep}.
\begin{note} Ordinary unqualified lookup\iref{basic.lookup.unqual} is not
performed. \end{note}
\end{itemize}
\end{itemize}

\begin{example}
\begin{codeblock}
int array[5] = { 1, 2, 3, 4, 5 };
for (int& x : array)
  x *= 2;
\end{codeblock}
\end{example}

\pnum
In the \grammarterm{decl-specifier-seq} of a \grammarterm{for-range-declaration},
each \grammarterm{decl-specifier} shall be either a \grammarterm{type-specifier}
or \tcode{constexpr}. The \grammarterm{decl-specifier-seq} shall not define a
class or enumeration.%
\indextext{statement!iteration|)}

\rSec1[stmt.jump]{Jump statements}%
\indextext{statement!jump}

\pnum
Jump statements unconditionally transfer control.
\indextext{statement!jump}%

\indextext{statement!\idxcode{break}}%
\indextext{statement!\idxcode{continue}}%
\indextext{return statement@\tcode{return} statement|see{\tcode{return}}}%
\indextext{\idxcode{return}}%
\indextext{statement!\idxcode{goto}}%
%
\begin{bnf}
\nontermdef{jump-statement}\br
    \keyword{break ;}\br
    \keyword{continue ;}\br
    \keyword{return} \opt{expr-or-braced-init-list} \terminal{;}\br
    coroutine-return-statement\br
    \keyword{goto} identifier \terminal{;}
\end{bnf}

\pnum
\indextext{local variable!destruction of}%
\indextext{scope!destructor and exit from}%
On exit from a scope (however accomplished), variables that refer to objects with
automatic storage duration\iref{basic.stc.auto} declared in that scope will be destroyed
in the reverse order of their declaration. If a variable does not refer to an object
and the type of the variable has a non-trivial destructor, the behavior is undefined. 
\begin{note} For temporaries, see~\ref{class.temporary}. \end{note} Transfer out of a loop, out of a block, or back
past an initialized variable with automatic storage duration involves the
destruction of objects with automatic storage duration that are in
scope at the point transferred from but not at the point transferred to.
(See~\ref{stmt.dcl} for transfers into blocks).
\begin{note}
However, the program can be terminated (by calling
\indextext{\idxcode{exit}}%
\indexlibrary{\idxcode{exit}}%
\tcode{std::exit()} or
\indextext{\idxcode{abort}}%
\indexlibrary{\idxcode{abort}}%
\tcode{std::abort()}\iref{support.start.term}, for example) without
destroying objects with automatic storage duration.
\end{note}
\begin{note}
A suspension of a coroutine\iref{expr.await} is not considered to be an exit from a scope.
\end{note}

\rSec2[stmt.break]{The \tcode{break} statement}%
\indextext{statement!\idxcode{break}}

\pnum
The \tcode{break} statement shall occur only in an
\indextext{\idxgram{iteration-statement}}%
\indextext{statement!\idxcode{switch}}%
\grammarterm{iteration-statement} or a \tcode{switch} statement and causes
termination of the smallest enclosing \grammarterm{iteration-statement} or
\tcode{switch} statement; control passes to the statement following the
terminated statement, if any.

\rSec2[stmt.cont]{The \tcode{continue} statement}%
\indextext{statement!\idxcode{continue}}

\pnum
The
\tcode{continue}
statement shall occur only in an
\indextext{\idxgram{iteration-statement}}%
\grammarterm{iteration-statement}
and causes control to pass to the loop-continuation portion of the
smallest enclosing \grammarterm{iteration-statement}, that is, to the end
of the loop. More precisely, in each of the statements

\begin{minipage}{.30\hsize}
\begin{codeblock}
while (foo) {
  {
    // ...
  }
@\exposid{contin}@: ;
}
\end{codeblock}
\end{minipage}
\begin{minipage}{.30\hsize}
\begin{codeblock}
do {
  {
    // ...
  }
@\exposid{contin}@: ;
} while (foo);
\end{codeblock}
\end{minipage}
\begin{minipage}{.30\hsize}
\begin{codeblock}
for (;;) {
  {
    // ...
  }
@\exposid{contin}@: ;
}
\end{codeblock}
\end{minipage}

a \tcode{continue} not contained in an enclosed iteration statement is
equivalent to \tcode{goto} \exposid{contin}.

\rSec2[stmt.return]{The \tcode{return} statement}%
\indextext{\idxcode{return}}%
\indextext{function return|see{\tcode{return}}}%

\pnum
A function returns to its caller by the \tcode{return} statement.

\pnum
The \grammarterm{expr-or-braced-init-list}
of a \tcode{return} statement is called its operand. A \tcode{return} statement with
no operand shall be used only in a function whose return type is
\cv{}~\tcode{void}, a constructor\iref{class.ctor}, or a
destructor\iref{class.dtor}.
\indextext{\idxcode{return}!constructor and}%
\indextext{\idxcode{return}!constructor and}%
A \tcode{return} statement with an operand of type \tcode{void} shall be used only
in a function whose return type is \cv{}~\tcode{void}.
A \tcode{return} statement with any other operand shall be used only
in a function whose return type is not \cv{}~\tcode{void};
\indextext{conversion!return type}%
the \tcode{return} statement initializes the
glvalue result or prvalue result object of the (explicit or implicit) function call
by copy-initialization\iref{dcl.init} from the operand.
\begin{note}
A \tcode{return} statement can involve
an invocation of a constructor to perform a copy or move of the operand
if it is not a prvalue or if its type differs from the return type of the function.
A copy operation associated with a \tcode{return} statement may be elided or
converted to a move operation if an automatic storage duration variable is returned\iref{class.copy.elision}.
\end{note}
\begin{example}
\begin{codeblock}
std::pair<std::string,int> f(const char* p, int x) {
  return {p,x};
}
\end{codeblock}
\end{example}
Flowing off the end of
a constructor,
a destructor, or
a non-coroutine function with a \cv{}~\tcode{void} return type is
equivalent to a \tcode{return} with no operand.
Otherwise, flowing off the end of a function
other than \tcode{main}\iref{basic.start.main} or a coroutine\iref{dcl.fct.def.coroutine}
results in undefined behavior.

\pnum
The copy-initialization of the result of the call is sequenced before the
destruction of temporaries at the end of the full-expression established
by the operand of the \tcode{return} statement, which, in turn, is sequenced
before the destruction of local variables\iref{stmt.jump} of the block
enclosing the \tcode{return} statement.

\rSec3[stmt.return.coroutine]{The \tcode{co_return} statement}%
\indextext{\idxcode{co_return}}%
\indextext{coroutine return|see{\tcode{co_return}}}%

\begin{bnf}
\nontermdef{coroutine-return-statement}\br
    \terminal{co_return} \opt{expr-or-braced-init-list} \terminal{;}
\end{bnf}

\pnum
A coroutine returns to its caller or resumer\iref{dcl.fct.def.coroutine}
by the \tcode{co_return} statement or when suspended\iref{expr.await}.
A coroutine shall not return to its caller or resumer
by a \tcode{return} statement\iref{stmt.return}.

\pnum
The \grammarterm{expr-or-braced-init-list} of a \tcode{co_return} statement is
called its operand.
Let \placeholder{p} be an lvalue naming the coroutine
promise object\iref{dcl.fct.def.coroutine}.
A \tcode{co_return} statement is equivalent to:
\begin{ncsimplebnf}
\terminal{\{} S\terminal{; goto} \exposid{final-suspend}\terminal{;} \terminal{\}}
\end{ncsimplebnf}

where \exposid{final-suspend} is the exposition-only label
defined in \ref{dcl.fct.def.coroutine}
and \placeholder{S} is defined as follows:

\begin{itemize}
\item
\placeholder{S} is \placeholder{p}\tcode{.return_value(}\grammarterm{expr-or-braced-init-list}{}\tcode{)},
if the operand is a \grammarterm{braced-init-list} or an expression of non-\tcode{void} type;

\item
\placeholder{S} is \tcode{\{}{ }\opt{\grammarterm{expression}} \tcode{;} \placeholder{p}\tcode{.return_void()}\tcode{;{ }\}}, otherwise;
\end{itemize}

\placeholder{S} shall be a prvalue of type \tcode{void}.

\pnum
If \placeholder{p}\tcode{.return_void()} is a valid expression,
flowing off the end of a coroutine is equivalent to a \tcode{co_return} with no operand;
otherwise flowing off the end of a coroutine results in undefined behavior.

\rSec2[stmt.goto]{The \tcode{goto} statement}%
\indextext{statement!\idxcode{goto}}

\pnum
The \tcode{goto} statement unconditionally transfers control to the
statement labeled by the identifier. The identifier shall be a
\indextext{label}%
label\iref{stmt.label} located in the current function.

\rSec1[stmt.dcl]{Declaration statement}%
\indextext{statement!declaration}

\pnum
A declaration statement introduces one or more new identifiers into a
block; it has the form

\begin{bnf}
\nontermdef{declaration-statement}\br
    block-declaration
\end{bnf}

If an identifier introduced by a declaration was previously declared in
an outer block,
\indextext{declaration hiding|see{name hiding}}%
\indextext{name hiding}%
\indextext{block structure}%
the outer declaration is hidden for the remainder of the block, after
which it resumes its force.

\pnum
\indextext{block!initialization in}%
\indextext{initialization!automatic}%
Variables with automatic storage duration\iref{basic.stc.auto} are
initialized each time their \grammarterm{declaration-statement} is executed.
\indextext{local variable!destruction of}%
Variables with automatic storage duration declared in the block are
destroyed on exit from the block\iref{stmt.jump}.

\pnum
\indextext{initialization!jump past}%
\indextext{\idxcode{goto}!initialization and}%
It is possible to transfer into a block, but not in a way that bypasses
declarations with initialization (including ones in \grammarterm{condition}s
and \grammarterm{init-statement}s).
A program that jumps\footnote{The transfer from the condition of a \tcode{switch} statement to a
\tcode{case} label is considered a jump in this respect.}
from a point where a variable with automatic storage duration is
not in scope to a point where it is in scope is ill-formed unless
the variable has vacuous initialization\iref{dcl.init}.
In such a case, the variables with vacuous initialization
are constructed in the order of their declaration.
\begin{example}

\begin{codeblock}
void f() {
  // ...
  goto lx;          // ill-formed: jump into scope of \tcode{a}
  // ...
ly:
  X a = 1;
  // ...
lx:
  goto ly;          // OK, jump implies destructor call for \tcode{a} followed by
                    // construction again immediately following label \tcode{ly}
}
\end{codeblock}
\end{example}

\pnum
\indextext{initialization!automatic}%
\indextext{initialization!dynamic block-scope}%
\indextext{initialization!local \tcode{static}}%
\indextext{initialization!local \tcode{thread_local}}%
Dynamic initialization of a block-scope variable with
static storage duration\iref{basic.stc.static} or
thread storage duration\iref{basic.stc.thread} is performed
the first time control passes through its declaration; such a variable is
considered initialized upon the completion of its initialization. If the
initialization exits by throwing an exception, the initialization is not
complete, so it will be tried again the next time control enters the
declaration.
If control enters the declaration concurrently while the variable is
being initialized, the concurrent execution shall wait for completion
of the initialization.\footnote{The implementation must not introduce
any deadlock around execution of the initializer. Deadlocks might
still be caused by the program logic; the implementation need only
avoid deadlocks due to its own synchronization operations.} If control
re-enters the declaration recursively while
the variable is being initialized, the behavior is undefined.
\begin{example}

\begin{codeblock}
int foo(int i) {
  static int s = foo(2*i);      // recursive call - undefined
  return i+1;
}
\end{codeblock}
\end{example}

\pnum
\indextext{\idxcode{static}!destruction of local}%
A block-scope object with static or thread storage duration will be
destroyed if and only if it was constructed.
\begin{note}
\ref{basic.start.term} describes the order in which block-scope objects with
static and thread storage duration are destroyed.
\end{note}

\rSec1[stmt.ambig]{Ambiguity resolution}%
\indextext{ambiguity!declaration versus expression}

\pnum
There is an ambiguity in the grammar involving
\grammarterm{expression-statement}{s} and \grammarterm{declaration}{s}: An
\grammarterm{expression-statement} with a function-style explicit type
conversion\iref{expr.type.conv} as its leftmost subexpression can be
indistinguishable from a \grammarterm{declaration} where the first
\grammarterm{declarator} starts with a \tcode{(}. In those cases the
\grammarterm{statement} is a \grammarterm{declaration}.

\pnum
\begin{note}
If the \grammarterm{statement} cannot syntactically be a
\grammarterm{declaration}, there is no ambiguity,
so this rule does not apply.
The whole \grammarterm{statement} might need to be examined
to determine whether this is the case. This resolves the meaning
of many examples.
\begin{example}
Assuming \tcode{T} is a
\grammarterm{simple-type-specifier}\iref{dcl.type},

\begin{codeblock}
T(a)->m = 7;        // expression-statement
T(a)++;             // expression-statement
T(a,5)<<c;          // expression-statement

T(*d)(int);         //  declaration
T(e)[5];            //  declaration
T(f) = { 1, 2 };    //  declaration
T(*g)(double(3));   //  declaration
\end{codeblock}

In the last example above, \tcode{g}, which is a pointer to \tcode{T},
is initialized to \tcode{double(3)}. This is of course ill-formed for
semantic reasons, but that does not affect the syntactic analysis.
\end{example}

The remaining cases are \grammarterm{declaration}{s}.
\begin{example}

\begin{codeblock}
class T {
  // ...
public:
  T();
  T(int);
  T(int, int);
};
T(a);               //  declaration
T(*b)();            //  declaration
T(c)=7;             //  declaration
T(d),e,f=3;         //  declaration
extern int h;
T(g)(h,2);          //  declaration
\end{codeblock}
\end{example}
\end{note}

\pnum
The disambiguation is purely syntactic; that is, the meaning of the
names occurring in such a statement, beyond whether they are
\grammarterm{type-name}{s} or not, is not generally used in or changed by the
disambiguation. Class templates are instantiated as necessary to
determine if a qualified name is a \grammarterm{type-name}. Disambiguation
precedes parsing, and a statement disambiguated as a declaration may be
an ill-formed declaration. If, during parsing, a name in a template
parameter is bound differently than it would be bound during a trial
parse, the program is ill-formed. No diagnostic is required.
\begin{note}
This can occur only when the name is declared earlier in the
declaration.
\end{note}
\begin{example}

\begin{codeblock}
struct T1 {
  T1 operator()(int x) { return T1(x); }
  int operator=(int x) { return x; }
  T1(int) { }
};
struct T2 { T2(int){ } };
int a, (*(*b)(T2))(int), c, d;

void f() {
  // disambiguation requires this to be parsed as a declaration:
  T1(a) = 3,
  T2(4),                        // \tcode{T2} will be declared as a variable of type \tcode{T1}, but this will not
  (*(*b)(T2(c)))(int(d));       // allow the last part of the declaration to parse properly,
                                // since it depends on \tcode{T2} being a type-name
}
\end{codeblock}
\end{example}%
\indextext{statement|)}
