%!TEX root = std.tex
\rSec0[exec]{Execution control library}

\rSec1[exec.general]{General}

\pnum
This Clause describes components
supporting execution of function objects\iref{function.objects}.

\pnum
The following subclauses describe
the requirements, concepts, and components
for execution control primitives as summarized in \tref{exec.summary}.

\begin{libsumtab}{Execution control library summary}{exec.summary}
\ref{exec.sched}     & Schedulers   & \tcode{<execution>} \\
\ref{exec.recv}      & Receivers    & \\
\ref{exec.opstate}   & Operation states & \\
\ref{exec.snd}       & Senders & \\
\end{libsumtab}

\pnum
\tref{exec.pos} shows
the types of customization point objects\iref{customization.point.object}
used in the execution control library.

\begin{floattable}{Types of customization point objects in the execution control library}{exec.pos}{lx{0.23\hsize}x{0.45\hsize}}
\topline
\lhdr{Customization point} & \chdr{Purpose} & \rhdr{Examples} \\
\lhdr{object type} & &  \\
\capsep
core &
  provide core execution functionality, and connection between core components &
  e.g., \tcode{connect}, \tcode{start} \\
completion functions &
  called by senders to announce the completion of the work (success, error, or cancellation) &
  \tcode{set_value}, \tcode{set_error}, \tcode{set_stopped} \\
senders &
  allow the specialization of the provided sender algorithms &
  \begin{itemize}
  \item sender factories (e.g., \tcode{schedule}, \tcode{just}, \tcode{read_env})
  \item sender adaptors (e.g., \tcode{continues_on}, \tcode{then}, \tcode{let_value})
  \item sender consumers (e.g., \tcode{sync_wait})
  \end{itemize}
    \\
queries &
  allow querying different properties of objects &
  \begin{itemize}
  \item general queries (e.g., \tcode{get_allocator}, \tcode{get_stop_token})
  \item environment queries (e.g., \tcode{get_scheduler}, \tcode{get_delegation_scheduler})
  \item scheduler queries (e.g., \tcode{get_forward_progress_guarantee})
  \item sender attribute queries (e.g., \tcode{get_completion_scheduler})
  \end{itemize}
    \\
\end{floattable}

\pnum
This clause makes use of the following exposition-only entities.

\pnum
For a subexpression \tcode{expr},
let \tcode{\exposid{MANDATE-NOTHROW}(expr)} be
expression-equivalent to \tcode{expr}.

\mandates
\tcode{noexcept(expr)} is \tcode{true}.

\pnum
\begin{codeblock}
namespace std {
  template<class T>
    concept @\defexposconcept{movable-value}@ =                                     // \expos
      @\libconcept{move_constructible}@<decay_t<T>> &&
      @\libconcept{constructible_from}@<decay_t<T>, T> &&
      (!is_array_v<remove_reference_t<T>>);
}
\end{codeblock}

\pnum
For function types \tcode{F1} and \tcode{F2} denoting
\tcode{R1(Args1...)} and \tcode{R2(Args2...)}, respectively,
\tcode{\exposid{MATCHING-SIG}(F1, F2)} is \tcode{true} if and only if
\tcode{\libconcept{same_as}<R1(Args1\&\&...), R2(Args2\&\&...)>}
is \tcode{true}.

\pnum
For a subexpression \tcode{err},
let \tcode{Err} be \tcode{decltype((err))} and
let \tcode{\exposid{AS-EXCEPT-PTR}(err)} be:
\begin{itemize}
\item
\tcode{err} if \tcode{decay_t<Err>} denotes the type \tcode{exception_ptr}.

\expects
\tcode{!err} is \tcode{false}.
\item
Otherwise,
\tcode{make_exception_ptr(system_error(err))}
if \tcode{decay_t<Err>} denotes the type \tcode{error_code}.
\item
Otherwise, \tcode{make_exception_ptr(err)}.
\end{itemize}

\rSec1[exec.queryable]{Queries and queryables}

\rSec2[exec.queryable.general]{General}

\pnum
A \defnadj{queryable}{object} is
a read-only collection of key/value pair
where each key is a customization point object known as a \defn{query object}.
A \defn{query} is an invocation of a query object
with a queryable object as its first argument and
a (possibly empty) set of additional arguments.
A query imposes syntactic and semantic requirements on its invocations.

\pnum
Let \tcode{q} be a query object,
let \tcode{args} be a (possibly empty) pack of subexpressions,
let \tcode{env} be a subexpression
that refers to a queryable object \tcode{o} of type \tcode{O}, and
let \tcode{cenv} be a subexpression referring to \tcode{o}
such that \tcode{decltype((cenv))} is \tcode{const O\&}.
The expression \tcode{q(env, args...)} is equal to\iref{concepts.equality}
the expression \tcode{q(cenv, args...)}.

\pnum
The type of a query expression cannot be \tcode{void}.

\pnum
The expression \tcode{q(env, args...)} is
equality-preserving\iref{concepts.equality} and
does not modify the query object or the arguments.

\pnum
If the expression \tcode{env.query(q, args...)} is well-formed,
then it is expression-equivalent to \tcode{q(env, args...)}.

\pnum
Unless otherwise specified,
the result of a query is valid as long as the queryable object is valid.

\rSec2[exec.queryable.concept]{\tcode{queryable} concept}

\begin{codeblock}
namespace std {
  template<class T>
    concept @\defexposconcept{queryable}@ = @\libconcept{destructible}@<T>;   // \expos
}
\end{codeblock}

\pnum
The exposition-only \exposconcept{queryable} concept specifies
the constraints on the types of queryable objects.

\pnum
Let \tcode{env} be an object of type \tcode{Env}.
The type \tcode{Env} models \exposconcept{queryable}
if for each callable object \tcode{q} and a pack of subexpressions \tcode{args},
if \tcode{requires \{ q(env, args...) \}} is \tcode{true} then
\tcode{q(env, args...)} meets any semantic requirements imposed by \tcode{q}.

\rSec1[exec.async.ops]{Asynchronous operations}

\pnum
An \defnadj{execution}{resource} is a program entity that manages
a (possibly dynamic) set of execution agents\iref{thread.req.lockable.general},
which it uses to execute parallel work on behalf of callers.
\begin{example}
The currently active thread,
a system-provided thread pool, and
uses of an API associated with an external hardware accelerator
are all examples of execution resources.
\end{example}
Execution resources execute asynchronous operations.
An execution resource is either valid or invalid.

\pnum
An \defnadj{asynchronous}{operation} is
a distinct unit of program execution that
\begin{itemize}
\item
is explicitly created;
\item
can be explicitly started once at most;
\item
once started, eventually completes exactly once
with a (possibly empty) set of result datums and
in exactly one of three \defnx{dispositions}{disposition}:
success, failure, or cancellation;
\begin{itemize}
\item
A successful completion, also known as a \defnadj{value}{completion},
can have an arbitrary number of result datums.
\item
A failure completion, also known as an \defnadj{error}{completion},
has a single result datum.
\item
A cancellation completion, also known as a \defnadj{stopped}{completion},
has no result datum.
\end{itemize}
An asynchronous operation's \defnadj{async}{result}
is its disposition and its (possibly empty) set of result datums.
\item
can complete on a different execution resource
than the execution resource on which it started; and
\item
can create and start other asynchronous operations
called \defnadj{child}{operations}.
A child operation is an asynchronous operation
that is created by the parent operation and,
if started, completes before the parent operation completes.
A \defnadj{parent}{operation} is the asynchronous operation
that created a particular child operation.
\end{itemize}
\begin{note}
An asynchronous operation can execute synchronously;
that is, it can complete during the execution of its start operation
on the thread of execution that started it.
\end{note}

\pnum
An asynchronous operation has associated state
known as its \defnadj{operation}{state}.

\pnum
An asynchronous operation has an associated environment.
An \defn{environment} is a queryable object\iref{exec.queryable}
representing the execution-time properties of the operation's caller.
The caller of an asynchronous operation is
its parent operation or the function that created it.

\pnum
An asynchronous operation has an associated receiver.
A \defn{receiver} is an aggregation of three handlers
for the three asynchronous completion dispositions:
\begin{itemize}
\item a value completion handler for a value completion,
\item an error completion handler for an error completion, and
\item a stopped completion handler for a stopped completion.
\end{itemize}
A receiver has an associated environment.
An asynchronous operation's operation state owns the operation's receiver.
The environment of an asynchronous operation
is equal to its receiver's environment.

\pnum
For each completion disposition, there is a \defnadj{completion}{function}.
A completion function is
a customization point object\iref{customization.point.object}
that accepts an asynchronous operation's receiver as the first argument and
the result datums of the asynchronous operation as additional arguments.
The value completion function invokes
the receiver's value completion handler with the value result datums;
likewise for the error completion function and the stopped completion function.
A completion function has
an associated type known as its \defnadj{completion}{tag}
that is the unqualified type of the completion function.
A valid invocation of a completion function is called
a \defnadj{completion}{operation}.

\pnum
The \defn{lifetime of an asynchronous operation},
also known as the operation's \defn{async lifetime},
begins when its start operation begins executing and
ends when its completion operation begins executing.
If the lifetime of an asynchronous operation's associated operation state
ends before the lifetime of the asynchronous operation,
the behavior is undefined.
After an asynchronous operation executes a completion operation,
its associated operation state is invalid.
Accessing any part of an invalid operation state is undefined behavior.

\pnum
An asynchronous operation shall not execute a completion operation
before its start operation has begun executing.
After its start operation has begun executing,
exactly one completion operation shall execute.
The lifetime of an asynchronous operation's operation state can end
during the execution of the completion operation.

\pnum
A \defn{sender} is a factory for one or more asynchronous operations.
\defnx{Connecting}{connect} a sender and a receiver creates
an asynchronous operation.
The asynchronous operation's associated receiver is equal to
the receiver used to create it, and
its associated environment is equal to
the environment associated with the receiver used to create it.
The lifetime of an asynchronous operation's associated operation state
does not depend on the lifetimes of either the sender or the receiver
from which it was created.
A sender is started when it is connected to a receiver and
the resulting asynchronous operation is started.
A sender's async result is the async result of the asynchronous operation
created by connecting it to a receiver.
A sender sends its results by way of the asynchronous operation(s) it produces,
and a receiver receives those results.
A sender is either valid or invalid;
it becomes invalid when its parent sender (see below) becomes invalid.

\pnum
A \defn{scheduler} is an abstraction of an execution resource
with a uniform, generic interface for scheduling work onto that resource.
It is a factory for senders
whose asynchronous operations execute value completion operations
on an execution agent belonging to
the scheduler's associated execution resource.
A \defn{schedule-expression} obtains such a sender from a scheduler.
A \defn{schedule sender} is the result of a schedule expression.
On success, an asynchronous operation produced by a schedule sender executes
a value completion operation with an empty set of result datums.
Multiple schedulers can refer to the same execution resource.
A scheduler can be valid or invalid.
A scheduler becomes invalid when the execution resource to which it refers
becomes invalid,
as do any schedule senders obtained from the scheduler, and
any operation states obtained from those senders.

\pnum
An asynchronous operation has one or more associated completion schedulers
for each of its possible dispositions.
A \defn{completion scheduler} is a scheduler
whose associated execution resource is used to execute
a completion operation for an asynchronous operation.
A value completion scheduler is a scheduler
on which an asynchronous operation's value completion operation can execute.
Likewise for error completion schedulers and stopped completion schedulers.

\pnum
A sender has an associated queryable object\iref{exec.queryable}
known as its \defnx{attributes}{attribute}
that describes various characteristics of the sender and
of the asynchronous operation(s) it produces.
For each disposition,
there is a query object for reading the associated completion scheduler
from a sender's attributes;
i.e., a value completion scheduler query object
for reading a sender's value completion scheduler, etc.
If a completion scheduler query is well-formed,
the returned completion scheduler is unique
for that disposition for any asynchronous operation the sender creates.
A schedule sender is required to have a value completion scheduler attribute
whose value is equal to the scheduler that produced the schedule sender.

\pnum
A \defn{completion signature} is a function type
that describes a completion operation.
An asynchronous operation has a finite set of possible completion signatures
corresponding to the completion operations
that the asynchronous operation potentially evaluates\iref{basic.def.odr}.
For a completion function \tcode{set},
receiver \tcode{rcvr}, and
pack of arguments \tcode{args},
let \tcode{c} be the completion operation \tcode{set(rcvr, args...)}, and
let \tcode{F} be
the function type \tcode{decltype(auto(set))(decltype((args))...)}.
A completion signature \tcode{Sig} is associated with \tcode{c}
if and only if
\tcode{\exposid{MATCHING-SIG}(Sig, F)} is \tcode{true}\iref{exec.general}.
Together, a sender type and an environment type \tcode{Env} determine
the set of completion signatures of an asynchronous operation
that results from connecting the sender with a receiver
that has an environment of type \tcode{Env}.
The type of the receiver does not affect
an asynchronous operation's completion signatures,
only the type of the receiver's environment.

\pnum
A sender algorithm is a function that takes and/or returns a sender.
There are three categories of sender algorithms:
\begin{itemize}
\item
A \defn{sender factory} is a function
that takes non-senders as arguments and that returns a sender.
\item
A \defn{sender adaptor} is a function
that constructs and returns a parent sender
from a set of one or more child senders and
a (possibly empty) set of additional arguments.
An asynchronous operation created by a parent sender is
a parent operation to the child operations created by the child senders.
\item
A \defn{sender consumer} is a function
that takes one or more senders and
a (possibly empty) set of additional arguments, and
whose return type is not the type of a sender.
\end{itemize}

\rSec1[execution.syn]{Header \tcode{<execution>} synopsis}

\indexheader{execution}%
\begin{codeblock}
namespace std {
  // \ref{execpol.type}, execution policy type trait
  template<class T> struct is_execution_policy;
  template<class T> constexpr bool @\libglobal{is_execution_policy_v}@ = is_execution_policy<T>::value;
}

namespace std::execution {
  // \ref{execpol.seq}, sequenced execution policy
  class sequenced_policy;

  // \ref{execpol.par}, parallel execution policy
  class parallel_policy;

  // \ref{execpol.parunseq}, parallel and unsequenced execution policy
  class parallel_unsequenced_policy;

  // \ref{execpol.unseq}, unsequenced execution policy
  class unsequenced_policy;

  // \ref{execpol.objects}, execution policy objects
  inline constexpr sequenced_policy            seq{ @\unspec@ };
  inline constexpr parallel_policy             par{ @\unspec@ };
  inline constexpr parallel_unsequenced_policy par_unseq{ @\unspec@ };
  inline constexpr unsequenced_policy          unseq{ @\unspec@ };
}

namespace std {
  // \ref{exec.general}, helper concepts
  template<class T>
    concept @\exposconceptnc{movable-value}@ = @\seebelownc@;                          // \expos

  template<class From, class To>
    concept @\defexposconceptnc{decays-to}@ = @\libconcept{same_as}@<decay_t<From>, To>;             // \expos

  template<class T>
    concept @\defexposconceptnc{class-type}@ = @\exposconceptnc{decays-to}@<T, T> && is_class_v<T>;      // \expos

  // \ref{exec.queryable}, queryable objects
  template<class T>
    concept @\exposconceptnc{queryable}@ = @\seebelownc@;                              // \expos

  // \ref{exec.queries}, queries
  struct @\libglobal{forwarding_query_t}@ { @\unspec@ };
  struct @\libglobal{get_allocator_t}@ { @\unspec@ };
  struct @\libglobal{get_stop_token_t}@ { @\unspec@ };

  inline constexpr forwarding_query_t @\libglobal{forwarding_query}@{};
  inline constexpr get_allocator_t @\libglobal{get_allocator}@{};
  inline constexpr get_stop_token_t @\libglobal{get_stop_token}@{};

  template<class T>
    using stop_token_of_t = remove_cvref_t<decltype(get_stop_token(declval<T>()))>;

  template<class T>
    concept @\defexposconceptnc{forwarding-query}@ = forwarding_query(T{});           // \expos
}

namespace std::execution {
  // \ref{exec.queries}, queries
  struct @\libglobal{get_domain_t}@ { @\unspec@ };
  struct @\libglobal{get_scheduler_t}@ { @\unspec@ };
  struct @\libglobal{get_delegation_scheduler_t}@ { @\unspec@ };
  struct @\libglobal{get_forward_progress_guarantee_t}@ { @\unspec@ };
  template<class CPO>
    struct @\libglobal{get_completion_scheduler_t}@ { @\unspec@ };

  inline constexpr get_domain_t @\libglobal{get_domain}@{};
  inline constexpr get_scheduler_t @\libglobal{get_scheduler}@{};
  inline constexpr get_delegation_scheduler_t @\libglobal{get_delegation_scheduler}@{};
  enum class forward_progress_guarantee;
  inline constexpr get_forward_progress_guarantee_t @\libglobal{get_forward_progress_guarantee}@{};
  template<class CPO>
    constexpr get_completion_scheduler_t<CPO> @\libglobal{get_completion_scheduler}@{};

  struct @\libglobal{get_env_t}@ { @\unspec@ };
  inline constexpr get_env_t @\libglobal{get_env}@{};

  template<class T>
    using @\libglobal{env_of_t}@ = decltype(get_env(declval<T>()));

  // \ref{exec.prop}, class template \tcode{prop}
  template<class QueryTag, class ValueType>
    struct prop;

  // \ref{exec.env}, class template \tcode{env}
  template<@\exposconcept{queryable}@... Envs>
    struct env;

  // \ref{exec.domain.default}, execution domains
  struct default_domain;

  // \ref{exec.sched}, schedulers
  struct @\libglobal{scheduler_t}@ {};

  template<class Sch>
    concept @\libconcept{scheduler}@ = @\seebelow@;

  // \ref{exec.recv}, receivers
  struct @\libglobal{receiver_t}@ {};

  template<class Rcvr>
    concept @\libconcept{receiver}@ = @\seebelow@;

  template<class Rcvr, class Completions>
    concept @\libconcept{receiver_of}@ = @\seebelow@;

  struct @\libglobal{set_value_t}@ { @\unspec@ };
  struct @\libglobal{set_error_t}@ { @\unspec@ };
  struct @\libglobal{set_stopped_t}@ { @\unspec@ };

  inline constexpr set_value_t @\libglobal{set_value}@{};
  inline constexpr set_error_t @\libglobal{set_error}@{};
  inline constexpr set_stopped_t @\libglobal{set_stopped}@{};

  // \ref{exec.opstate}, operation states
  struct @\libglobal{operation_state_t}@ {};

  template<class O>
    concept @\libconcept{operation_state}@ = @\seebelow@;

  struct @\libglobal{start_t}@;
  inline constexpr start_t @\libglobal{start}@{};

  // \ref{exec.snd}, senders
  struct @\libglobal{sender_t}@ {};

  template<class Sndr>
    concept @\libconcept{sender}@ = @\seebelow@;

  template<class Sndr, class Env = env<>>
    concept @\libconcept{sender_in}@ = @\seebelow@;

  template<class Sndr, class Rcvr>
    concept @\libconcept{sender_to}@ = @\seebelow@;

  template<class... Ts>
    struct @\exposidnc{type-list}@;                                           // \expos

  // \ref{exec.getcomplsigs}, completion signatures
  struct get_completion_signatures_t;
  inline constexpr get_completion_signatures_t get_completion_signatures {};

  template<class Sndr, class Env = env<>>
      requires @\libconcept{sender_in}@<Sndr, Env>
    using completion_signatures_of_t = @\exposid{call-result-t}@<get_completion_signatures_t, Sndr, Env>;

  template<class... Ts>
    using @\exposidnc{decayed-tuple}@ = tuple<decay_t<Ts>...>;                // \expos

  template<class... Ts>
    using @\exposidnc{variant-or-empty}@ = @\seebelownc@;                         // \expos

  template<class Sndr, class Env = env<>,
           template<class...> class Tuple = @\exposid{decayed-tuple}@,
           template<class...> class Variant = @\exposid{variant-or-empty}@>
      requires @\libconcept{sender_in}@<Sndr, Env>
    using value_types_of_t = @\seebelow@;

  template<class Sndr, class Env = env<>,
           template<class...> class Variant = @\exposid{variant-or-empty}@>
      requires @\libconcept{sender_in}@<Sndr, Env>
    using error_types_of_t = @\seebelow@;

  template<class Sndr, class Env = env<>>
      requires @\libconcept{sender_in}@<Sndr, Env>
    constexpr bool sends_stopped = @\seebelow@;

  template<class Sndr, class Env>
    using @\exposidnc{single-sender-value-type}@ = @\seebelownc@;                 // \expos

  template<class Sndr, class Env>
    concept @\exposconcept{single-sender}@ = @\seebelow@; // \expos

  template<@\libconcept{sender}@ Sndr>
    using tag_of_t = @\seebelow@;

  // \ref{exec.snd.transform}, sender transformations
  template<class Domain, @\libconcept{sender}@ Sndr, @\exposconcept{queryable}@... Env>
      requires (sizeof...(Env) <= 1)
    constexpr @\libconcept{sender}@ decltype(auto) transform_sender(
      Domain dom, Sndr&& sndr, const Env&... env) noexcept(@\seebelow@);

  // \ref{exec.snd.transform.env}, environment transformations
  template<class Domain, @\libconcept{sender}@ Sndr, @\exposconcept{queryable}@ Env>
    constexpr @\exposconcept{queryable}@ decltype(auto) transform_env(
      Domain dom, Sndr&& sndr, Env&& env) noexcept;

  // \ref{exec.snd.apply}, sender algorithm application
  template<class Domain, class Tag, @\libconcept{sender}@ Sndr, class... Args>
    constexpr decltype(auto) apply_sender(
      Domain dom, Tag, Sndr&& sndr, Args&&... args) noexcept(@\seebelow@);

  // \ref{exec.connect}, the connect sender algorithm
  struct @\libglobal{connect_t}@;
  inline constexpr connect_t @\libglobal{connect}@{};

  template<class Sndr, class Rcvr>
    using @\libglobal{connect_result_t}@ =
      decltype(connect(declval<Sndr>(), declval<Rcvr>()));

  // \ref{exec.factories}, sender factories
  struct @\libglobal{just_t}@ { @\unspec@ };
  struct @\libglobal{just_error_t}@ { @\unspec@ };
  struct @\libglobal{just_stopped_t}@ { @\unspec@ };
  struct @\libglobal{schedule_t}@ { @\unspec@ };

  inline constexpr just_t @\libglobal{just}@{};
  inline constexpr just_error_t @\libglobal{just_error}@{};
  inline constexpr just_stopped_t @\libglobal{just_stopped}@{};
  inline constexpr schedule_t @\libglobal{schedule}@{};
  inline constexpr @\unspec@ @\libglobal{read_env}@{};

  template<@\libconcept{scheduler}@ Sndr>
    using @\libglobal{schedule_result_t}@ = decltype(schedule(declval<Sndr>()));

  // \ref{exec.adapt}, sender adaptors
  template<@\exposconcept{class-type}@ D>
    struct @\libglobal{sender_adaptor_closure}@ { };

  struct @\libglobal{starts_on_t}@ { @\unspec@ };
  struct @\libglobal{continues_on_t}@ { @\unspec@ };
  struct @\libglobal{on_t}@ { @\unspec@ };
  struct @\libglobal{schedule_from_t}@ { @\unspec@ };
  struct @\libglobal{then_t}@ { @\unspec@ };
  struct @\libglobal{upon_error_t}@ { @\unspec@ };
  struct @\libglobal{upon_stopped_t}@ { @\unspec@ };
  struct @\libglobal{let_value_t}@ { @\unspec@ };
  struct @\libglobal{let_error_t}@ { @\unspec@ };
  struct @\libglobal{let_stopped_t}@ { @\unspec@ };
  struct @\libglobal{bulk_t}@ { @\unspec@ };
  struct @\libglobal{split_t}@ { @\unspec@ };
  struct @\libglobal{when_all_t}@ { @\unspec@ };
  struct @\libglobal{when_all_with_variant_t}@ { @\unspec@ };
  struct @\libglobal{into_variant_t}@ { @\unspec@ };
  struct @\libglobal{stopped_as_optional_t}@ { @\unspec@ };
  struct @\libglobal{stopped_as_error_t}@ { @\unspec@ };

  inline constexpr starts_on_t @\libglobal{starts_on}@{};
  inline constexpr continues_on_t @\libglobal{continues_on}@{};
  inline constexpr on_t @\libglobal{on}@{};
  inline constexpr schedule_from_t @\libglobal{schedule_from}@{};
  inline constexpr then_t @\libglobal{then}@{};
  inline constexpr upon_error_t @\libglobal{upon_error}@{};
  inline constexpr upon_stopped_t @\libglobal{upon_stopped}@{};
  inline constexpr let_value_t @\libglobal{let_value}@{};
  inline constexpr let_error_t @\libglobal{let_error}@{};
  inline constexpr let_stopped_t @\libglobal{let_stopped}@{};
  inline constexpr bulk_t @\libglobal{bulk}@{};
  inline constexpr split_t @\libglobal{split}@{};
  inline constexpr when_all_t @\libglobal{when_all}@{};
  inline constexpr when_all_with_variant_t @\libglobal{when_all_with_variant}@{};
  inline constexpr into_variant_t @\libglobal{into_variant}@{};
  inline constexpr stopped_as_optional_t @\libglobal{stopped_as_optional}@{};
  inline constexpr stopped_as_error_t @\libglobal{stopped_as_error}@{};

  // \ref{exec.util}, sender and receiver utilities
  // \ref{exec.util.cmplsig}
  template<class Fn>
    concept @\exposconceptnc{completion-signature}@ = @\seebelownc@;                   // \expos

  template<@\exposconcept{completion-signature}@... Fns>
    struct @\libglobal{completion_signatures}@ {};

  template<class Sigs>
    concept @\exposconceptnc{valid-completion-signatures}@ = @\seebelownc@;            // \expos

  // \ref{exec.util.cmplsig.trans}
  template<
    @\exposconcept{valid-completion-signatures}@ InputSignatures,
    @\exposconcept{valid-completion-signatures}@ AdditionalSignatures = completion_signatures<>,
    template<class...> class SetValue = @\seebelow@,
    template<class> class SetError = @\seebelow@,
    @\exposconcept{valid-completion-signatures}@ SetStopped = completion_signatures<set_stopped_t()>>
  using transform_completion_signatures = completion_signatures<@\seebelow@>;

  template<
    @\libconcept{sender}@ Sndr,
    class Env = env<>,
    @\exposconcept{valid-completion-signatures}@ AdditionalSignatures = completion_signatures<>,
    template<class...> class SetValue = @\seebelow@,
    template<class> class SetError = @\seebelow@,
    @\exposconcept{valid-completion-signatures}@ SetStopped = completion_signatures<set_stopped_t()>>
      requires @\libconcept{sender_in}@<Sndr, Env>
  using transform_completion_signatures_of =
    transform_completion_signatures<
      completion_signatures_of_t<Sndr, Env>,
      AdditionalSignatures, SetValue, SetError, SetStopped>;

  // \ref{exec.run.loop}, run_loop
  class run_loop;
}

namespace std::this_thread {
  // \ref{exec.consumers}, consumers
  struct @\libglobal{sync_wait_t}@ { @\unspec@ };
  struct @\libglobal{sync_wait_with_variant_t}@ { @\unspec@ };

  inline constexpr sync_wait_t @\libglobal{sync_wait}@{};
  inline constexpr sync_wait_with_variant_t @\libglobal{sync_wait_with_variant}@{};
}

namespace std::execution {
  // \ref{exec.as.awaitable}
  struct @\libglobal{as_awaitable_t}@ { @\unspec@ };
  inline constexpr as_awaitable_t @\libglobal{as_awaitable}@{};

  // \ref{exec.with.awaitable.senders}
  template<@\exposconcept{class-type}@ Promise>
    struct with_awaitable_senders;
}
\end{codeblock}

\pnum
The exposition-only type \tcode{\exposid{variant-or-empty}<Ts...>}
is defined as follows:
\begin{itemize}
\item
If \tcode{sizeof...(Ts)} is greater than zero,
\tcode{\exposid{variant-or-empty}<Ts...>} denotes \tcode{variant<Us...>}
where \tcode{Us...} is the pack \tcode{decay_t<Ts>...}
with duplicate types removed.
\item
Otherwise, \tcode{\exposid{variant-or-empty}<Ts...>} denotes
the exposition-only class type:
\begin{codeblock}
namespace std::execution {
  struct @\exposidnc{empty-variant}@ {        // \expos
    @\exposidnc{empty-variant}@() = delete;
  };
}
\end{codeblock}
\end{itemize}

\pnum
For types \tcode{Sndr} and \tcode{Env},
\tcode{\exposid{single-sender-value-type}<Sndr, Env>} is an alias for:
\begin{itemize}
\item
\tcode{value_types_of_t<Sndr, Env, decay_t, type_identity_t>}
if that type is well-formed,
\item
Otherwise, \tcode{void}
if \tcode{value_types_of_t<Sndr, Env, tuple, variant>} is
\tcode{variant<tuple<>>} or \tcode{vari\-ant<>},
\item
Otherwise, \tcode{value_types_of_t<Sndr, Env, \exposid{decayed-tuple}, type_identity_t>}
if that type is well-formed,
\item
Otherwise, \tcode{\exposid{single-sender-value-type}<Sndr, Env>} is ill-formed.
\end{itemize}

\pnum
The exposition-only concept \exposconcept{single-sender} is defined as follows:
\begin{codeblock}
namespace std::execution {
  template<class Sndr, class Env>
    concept @\defexposconcept{single-sender}@ = @\libconcept{sender_in}@<Sndr, Env> &&
      requires {
        typename @\exposid{single-sender-value-type}@<Sndr, Env>;
      };
}
\end{codeblock}

\rSec1[exec.queries]{Queries}

\rSec2[exec.fwd.env]{\tcode{forwarding_query}}

\pnum
\tcode{forwarding_query} asks a query object
whether it should be forwarded through queryable adaptors.

\pnum
The name \tcode{forwarding_query} denotes a query object.
For some query object \tcode{q} of type \tcode{Q},
\tcode{forwarding_query(q)} is expression-equivalent to:
\begin{itemize}
\item
\tcode{\exposid{MANDATE-NOTHROW}(q.query(forwarding_query))}
if that expression is well-formed.

\mandates
The expression above has type \tcode{bool} and
is a core constant expression if \tcode{q} is a core constant expression.
\item
Otherwise, \tcode{true} if \tcode{\libconcept{derived_from}<Q, forwarding_query_t>} is \tcode{true}.
\item
Otherwise, \tcode{false}.
\end{itemize}

\rSec2[exec.get.allocator]{\tcode{get_allocator}}

\pnum
\tcode{get_allocator} asks a queryable object for its associated allocator.

\pnum
The name \tcode{get_allocator} denotes a query object.
For a subexpression \tcode{env},
\tcode{get_allocator(env)} is expression-equivalent to
\tcode{\exposid{MANDATE-NOTHROW}(as_const(env).query(get_allocator))}.

\mandates
If the expression above is well-formed,
its type satisfies
\exposconcept{simple-allocator}\iref{allocator.requirements.general}.

\pnum
\tcode{forwarding_query(get_allocator)} is a core constant expression and
has value \tcode{true}.

\rSec2[exec.get.stop.token]{\tcode{get_stop_token}}

\pnum
\tcode{get_stop_token} asks a queryable object for an associated stop token.

\pnum
The name \tcode{get_stop_token} denotes a query object.
For a subexpression \tcode{env},
\tcode{get_stop_token(env)} is expression-equivalent to:
\begin{itemize}
\item
\tcode{\exposid{MANDATE-NOTHROW}(as_const(env).query(get_stop_token))}
if that expression is well-formed.

\mandates
The type of the expression above satisfies \libconcept{stoppable_token}.

\item
Otherwise, \tcode{never_stop_token\{\}}.
\end{itemize}

\pnum
\tcode{forwarding_query(get_stop_token)} is a core constant expression and
has value \tcode{true}.

\rSec2[exec.get.env]{\tcode{execution::get_env}}

\pnum
\tcode{execution::get_env} is a customization point object.
For a subexpression \tcode{o},
\tcode{execution::get_env(o)} is expression-equivalent to:
\begin{itemize}
\item
\tcode{\exposid{MANDATE-NOTHROW}(as_const(o).get_env())}
if that expression is well-formed.

\mandates
The type of the expression above satisfies
\exposconcept{queryable}\iref{exec.queryable}.
\item
Otherwise, \tcode{env<>\{\}}.
\end{itemize}

\pnum
The value of \tcode{get_env(o)} shall be valid while \tcode{o} is valid.

\pnum
\begin{note}
When passed a sender object,
\tcode{get_env} returns the sender's associated attributes.
When passed a receiver,
\tcode{get_env} returns the receiver's associated execution environment.
\end{note}

\rSec2[exec.get.domain]{\tcode{execution::get_domain}}

\pnum
\tcode{get_domain} asks a queryable object
for its associated execution domain tag.

\pnum
The name \tcode{get_domain} denotes a query object.
For a subexpression \tcode{env},
\tcode{get_domain(env)} is expression-equivalent to
\tcode{\exposid{MANDATE-NOTHROW}(as_const(env).query(get_domain))}.

\pnum
\tcode{forwarding_query(execution::get_domain)} is
a core constant expression and has value \tcode{true}.

\rSec2[exec.get.scheduler]{\tcode{execution::get_scheduler}}

\pnum
\tcode{get_scheduler} asks a queryable object for its associated scheduler.

\pnum
The name \tcode{get_scheduler} denotes a query object.
For a subexpression \tcode{env},
\tcode{get_scheduler(env)} is expression-equivalent to
\tcode{\exposid{MANDATE-NOTHROW}(as_const(env).query(get_scheduler))}.

\mandates
If the expression above is well-formed,
its type satisfies \libconcept{scheduler}.

\pnum
\tcode{forwarding_query(execution::get_scheduler)} is
a core constant expression and has value \tcode{true}.

\rSec2[exec.get.delegation.scheduler]{\tcode{execution::get_delegation_scheduler}}

\pnum
\tcode{get_delegation_scheduler} asks a queryable object for a scheduler
that can be used to delegate work to
for the purpose of forward progress delegation\iref{intro.progress}.

\pnum
The name \tcode{get_delegation_scheduler} denotes a query object.
For a subexpression \tcode{env},
\tcode{get_delegation_scheduler(env)} is expression-equivalent to
\tcode{\exposid{MANDATE-NOTHROW}(as_const(env).query(get_delegation_scheduler))}.

\mandates
If the expression above is well-formed,
its type satisfies \libconcept{scheduler}.

\pnum
\tcode{forwarding_query(execution::get_delegation_scheduler)} is
a core constant expression and has value \tcode{true}.

\rSec2[exec.get.fwd.progress]{\tcode{execution::get_forward_progress_guarantee}}

\begin{codeblock}
namespace std::execution {
  enum class @\libglobal{forward_progress_guarantee}@ {
    concurrent,
    parallel,
    weakly_parallel
  };
}
\end{codeblock}

\pnum
\tcode{get_forward_progress_guarantee} asks a scheduler about
the forward progress guarantee of execution agents
created by that scheduler's associated execution resource\iref{intro.progress}.

\pnum
The name \tcode{get_forward_progress_guarantee} denotes a query object.
For a subexpression \tcode{sch}, let \tcode{Sch} be \tcode{decltype((sch))}.
If \tcode{Sch} does not satisfy \libconcept{scheduler},
\tcode{get_forward_progress_guarantee} is ill-formed.
Otherwise,
\tcode{get_forward_progress_guarantee(sch)} is expression-equivalent to:
\begin{itemize}
\item
\tcode{\exposid{MANDATE-NOTHROW}(as_const(sch).query(get_forward_progress_guarantee))},
if that expression is well-formed.

\mandates
The type of the expression above is \tcode{forward_progress_guarantee}.
\item
Otherwise, \tcode{forward_progress_guarantee::weakly_parallel}.
\end{itemize}

\pnum
If \tcode{get_forward_progress_guarantee(sch)} for some scheduler \tcode{sch}
returns \tcode{forward_progress_guaran\-tee::concurrent},
all execution agents created by that scheduler's associated execution resource
shall provide the concurrent forward progress guarantee.
If it returns \tcode{forward_progress_guarantee::parallel},
all such execution agents
shall provide at least the parallel forward progress guarantee.

\rSec2[exec.get.compl.sched]{\tcode{execution::get_completion_scheduler}}

\pnum
\tcode{get_completion_scheduler<\exposid{completion-tag>}} obtains
the completion scheduler associated with a completion tag
from a sender's attributes.

\pnum
The name \tcode{get_completion_scheduler} denotes a query object template.
For a subexpression \tcode{q},
the expression \tcode{get_completion_scheduler<\exposid{completion-tag}>(q)}
is ill-formed if \exposid{completion-tag} is not one of
\tcode{set_value_t}, \tcode{set_error_t}, or \tcode{set_stopped_t}.
Otherwise, \tcode{get_completion_scheduler<\exposid{completion-tag}>(q)}
is expression-equivalent to
\begin{codeblock}
@\exposid{MANDATE-NOTHROW}@(as_const(q).query(get_completion_scheduler<@\exposid{completion-tag}@>))
\end{codeblock}
\mandates
If the expression above is well-formed,
its type satisfies \libconcept{scheduler}.

\pnum
Let \exposid{completion-fn} be a completion function\iref{exec.async.ops};
let \exposid{completion-tag} be
the associated completion tag of \exposid{completion-fn};
let \tcode{args} be a pack of subexpressions; and
let \tcode{sndr} be a subexpression
such that \tcode{\libconcept{sender}<decltype((sndr))>} is \tcode{true} and
\tcode{get_completion_scheduler<\exposid{completion-tag}>(get_env(sndr))}
is well-formed and denotes a scheduler \tcode{sch}.
If an asynchronous operation
created by connecting \tcode{sndr} with a receiver \tcode{rcvr}
causes the evaluation of \tcode{\exposid{completion-fn}(rcvr, args...)},
the behavior is undefined
unless the evaluation happens on an execution agent
that belongs to \tcode{sch}'s associated execution resource.

\pnum
The expression
\tcode{forwarding_query(get_completion_scheduler<\exposid{completion-tag}>)}
is a core constant expression and has value \tcode{true}.

\rSec1[exec.sched]{Schedulers}

\pnum
The \libconcept{scheduler} concept defines
the requirements of a scheduler type\iref{exec.async.ops}.
\tcode{schedule} is a customization point object
that accepts a scheduler.
A valid invocation of \tcode{schedule} is a schedule-expression.
\begin{codeblock}
namespace std::execution {
  template<class Sch>
    concept @\deflibconcept{scheduler}@ =
      @\libconcept{derived_from}@<typename remove_cvref_t<Sch>::scheduler_concept, scheduler_t> &&
      @\exposconcept{queryable}@<Sch> &&
      requires(Sch&& sch) {
        { schedule(std::forward<Sch>(sch)) } -> @\libconcept{sender}@;
        { auto(get_completion_scheduler<set_value_t>(
            get_env(schedule(std::forward<Sch>(sch))))) }
              -> @\libconcept{same_as}@<remove_cvref_t<Sch>>;
      } &&
      @\libconcept{equality_comparable}@<remove_cvref_t<Sch>> &&
      @\libconcept{copyable}@<remove_cvref_t<Sch>>;
}
\end{codeblock}

\pnum
Let \tcode{Sch} be the type of a scheduler and
let \tcode{Env} be the type of an execution environment
for which \tcode{\libconcept{sender_in}<schedule_result_t<Sch>, Env>}
is satisfied.
Then \tcode{\exposconcept{sender-in-of}<schedule_result_t<Sch>, Env>}
shall be modeled.

\pnum
No operation required by
\tcode{\libconcept{copyable}<remove_cvref_t<Sch>>} and
\tcode{\libconcept{equality_comparable}<remove_cvref_t<Sch>>}
shall exit via an exception.
None of these operations,
nor a scheduler type's \tcode{schedule} function,
shall introduce data races
as a result of potentially concurrent\iref{intro.races} invocations
of those operations from different threads.

\pnum
For any two values \tcode{sch1} and \tcode{sch2}
of some scheduler type \tcode{Sch},
\tcode{sch1 == sch2} shall return \tcode{true}
only if both \tcode{sch1} and \tcode{sch2} share
the same associated execution resource.

\pnum
For a given scheduler expression \tcode{sch},
the expression
\tcode{get_completion_scheduler<set_value_t>(get_env(schedule(sch)))}
shall compare equal to \tcode{sch}.

\pnum
For a given scheduler expression \tcode{sch},
if the expression \tcode{get_domain(sch)} is well-formed,
then the expression \tcode{get_domain(get_env(schedule(sch)))}
is also well-formed and has the same type.

\pnum
A scheduler type's destructor shall not block
pending completion of any receivers
connected to the sender objects returned from \tcode{schedule}.
\begin{note}
The ability to wait for completion of submitted function objects
can be provided by the associated execution resource of the scheduler.
\end{note}

\rSec1[exec.recv]{Receivers}

\rSec2[exec.recv.concepts]{Receiver concepts}

\pnum
A receiver represents the continuation of an asynchronous operation.
The \libconcept{receiver} concept defines
the requirements for a receiver type\iref{exec.async.ops}.
The \libconcept{receiver_of} concept defines
the requirements for a receiver type that is usable as
the first argument of a set of completion operations
corresponding to a set of completion signatures.
The \tcode{get_env} customization point object is used to access
a receiver's associated environment.
\begin{codeblock}
namespace std::execution {
  template<class Rcvr>
    concept @\deflibconcept{receiver}@ =
      @\libconcept{derived_from}@<typename remove_cvref_t<Rcvr>::receiver_concept, receiver_t> &&
      requires(const remove_cvref_t<Rcvr>& rcvr) {
        { get_env(rcvr) } -> @\exposconcept{queryable}@;
      } &&
      @\libconcept{move_constructible}@<remove_cvref_t<Rcvr>> &&       // rvalues are movable, and
      @\libconcept{constructible_from}@<remove_cvref_t<Rcvr>, Rcvr>;   // lvalues are copyable

  template<class Signature, class Rcvr>
    concept @\defexposconcept{valid-completion-for}@ =                 // \expos
      requires (Signature* sig) {
        []<class Tag, class... Args>(Tag(*)(Args...))
            requires @\exposconcept{callable}@<Tag, remove_cvref_t<Rcvr>, Args...>
        {}(sig);
      };

  template<class Rcvr, class Completions>
    concept @\defexposconcept{has-completions}@ =                      // \expos
      requires (Completions* completions) {
        []<@\exposconcept{valid-completion-for}@<Rcvr>...Sigs>(completion_signatures<Sigs...>*)
        {}(completions);
      };

  template<class Rcvr, class Completions>
    concept @\deflibconcept{receiver_of}@ =
      @\libconcept{receiver}@<Rcvr> && @\exposconcept{has-completions}@<Rcvr, Completions>;
}
\end{codeblock}

\pnum
Class types that are marked \tcode{final} do not model the \libconcept{receiver} concept.

\pnum
Let \tcode{rcvr} be a receiver and
let \tcode{op_state} be an operation state associated with
an asynchronous operation created by connecting \tcode{rcvr} with a sender.
Let \tcode{token} be a stop token equal to
\tcode{get_stop_token(get_env(rcvr))}.
\tcode{token} shall remain valid
for the duration of the asynchronous operation's lifetime\iref{exec.async.ops}.
\begin{note}
This means that, unless it knows about further guarantees
provided by the type of \tcode{rcvr},
the implementation of \tcode{op_state} cannot use \tcode{token}
after it executes a completion operation.
This also implies that any stop callbacks registered on token
must be destroyed before the invocation of the completion operation.
\end{note}

\rSec2[exec.set.value]{\tcode{execution::set_value}}

\pnum
\tcode{set_value} is a value completion function\iref{exec.async.ops}.
Its associated completion tag is \tcode{set_value_t}.
The expression \tcode{set_value(rcvr, vs...)}
for a subexpression \tcode{rcvr} and
pack of subexpressions \tcode{vs} is ill-formed
if \tcode{rcvr} is an lvalue or an rvalue of const type.
Otherwise, it is expression-equivalent to
\tcode{\exposid{MANDATE-NOTHROW}(rcvr.set_value(vs...))}.

\rSec2[exec.set.error]{\tcode{execution::set_error}}

\pnum
\tcode{set_error} is an error completion function\iref{exec.async.ops}.
Its associated completion tag is \tcode{set_error_t}.
The expression \tcode{set_error(rcvr, err)}
for some \tcode{subexpressions} \tcode{rcvr} and \tcode{err} is ill-formed
if \tcode{rcvr} is an lvalue or an rvalue of const type.
Otherwise, it is expression-equivalent to
\tcode{\exposid{MANDATE-NOTHROW}(rcvr.set_error(err))}.

\rSec2[exec.set.stopped]{\tcode{execution::set_stopped}}

\pnum
\tcode{set_stopped} is a stopped completion function\iref{exec.async.ops}.
Its associated completion tag is \tcode{set_stopped_t}.
The expression \tcode{set_stopped(rcvr)}
for a subexpression \tcode{rcvr} is ill-formed
if \tcode{rcvr} is an lvalue or an rvalue of const type.
Otherwise, it is expression-equivalent to
\tcode{\exposid{MANDATE-NOTHROW}(rcvr.set_stopped())}.

\rSec1[exec.opstate]{Operation states}

\rSec2[exec.opstate.general]{General}

\pnum
The \libconcept{operation_state} concept defines
the requirements of an operation state type\iref{exec.async.ops}.
\begin{codeblock}
namespace std::execution {
  template<class O>
    concept @\deflibconcept{operation_state}@ =
      @\libconcept{derived_from}@<typename O::operation_state_concept, operation_state_t> &&
      is_object_v<O> &&
      requires (O& o) {
        { start(o) } noexcept;
      };
}
\end{codeblock}

\pnum
If an \libconcept{operation_state} object is destroyed
during the lifetime of its asynchronous operation\iref{exec.async.ops},
the behavior is undefined.
\begin{note}
The \libconcept{operation_state} concept does not impose requirements
on any operations other than destruction and \tcode{start},
including copy and move operations.
Invoking any such operation on an object
whose type models \libconcept{operation_state} can lead to undefined behavior.
\end{note}

\pnum
The program is ill-formed
if it performs a copy or move construction or assignment operation on
an operation state object created by connecting a library-provided sender.

\rSec2[exec.opstate.start]{\tcode{execution::start}}

\pnum
The name \tcode{start} denotes a customization point object
that starts\iref{exec.async.ops}
the asynchronous operation associated with the operation state object.
For a subexpression \tcode{op},
the expression \tcode{start(op)} is ill-formed
if \tcode{op} is an rvalue.
Otherwise, it is expression-equivalent to
\tcode{\exposid{MANDATE-NOTHROW}(op.start())}.

\pnum
If \tcode{op.start()} does not start\iref{exec.async.ops}
the asynchronous operation associated with the operation state \tcode{op},
the behavior of calling \tcode{start(op)} is undefined.

\rSec1[exec.snd]{Senders}

\rSec2[exec.snd.general]{General}

\pnum
Subclauses \ref{exec.factories} and \ref{exec.adapt} define
customizable algorithms that return senders.
Each algorithm has a default implementation.
Let \tcode{sndr} be the result of an invocation of such an algorithm or
an object equal to the result\iref{concepts.equality}, and
let \tcode{Sndr} be \tcode{decltype((sndr))}.
Let \tcode{rcvr} be a receiver of type \tcode{Rcvr}
with associated environment \tcode{env} of type \tcode{Env}
such that \tcode{\libconcept{sender_to}<Sndr, Rcvr>} is \tcode{true}.
For the default implementation of the algorithm that produced \tcode{sndr},
connecting \tcode{sndr} to \tcode{rcvr} and
starting the resulting operation state\iref{exec.async.ops}
necessarily results in the potential evaluation\iref{basic.def.odr} of
a set of completion operations
whose first argument is a subexpression equal to \tcode{rcvr}.
Let \tcode{Sigs} be a pack of completion signatures corresponding to
this set of completion operations.
Then the type of the expression \tcode{get_completion_signatures(sndr, env)} is
a specialization of
the class template \tcode{completion_signatures}\iref{exec.util.cmplsig},
the set of whose template arguments is \tcode{Sigs}.
If a user-provided implementation of the algorithm
that produced \tcode{sndr} is selected instead of the default,
any completion signature
that is in the set of types
denoted by \tcode{completion_signatures_of_t<Sndr, Env>} and
that is not part of \tcode{Sigs} shall correspond to
error or stopped completion operations,
unless otherwise specified.

\rSec2[exec.snd.expos]{Exposition-only entities}

\pnum
Subclause \ref{exec.snd} makes use of the following exposition-only entities.

\pnum
For a queryable object \tcode{env},
\tcode{\exposid{FWD-ENV}(env)} is an expression
whose type satisfies \exposconcept{queryable}
such that for a query object \tcode{q} and
a pack of subexpressions \tcode{as},
the expression \tcode{\exposid{FWD-ENV}(env).query(q, as...)} is ill-formed
if \tcode{forwarding_query(q)} is \tcode{false};
otherwise, it is expression-equivalent to \tcode{env.query(q, as...)}.

\pnum
For a query object \tcode{q} and \tcode{a} subexpression \tcode{v},
\tcode{\exposid{MAKE-ENV}(q, v)} is an expression \tcode{env}
whose type satisfies \exposconcept{queryable}
such that the result of \tcode{env.query(q)} has
a value equal to \tcode{v}\iref{concepts.equality}.
Unless otherwise stated,
the object to which \tcode{env.query(q)} refers remains valid
while \tcode{env} remains valid.

\pnum
For two queryable objects \tcode{env1} and \tcode{env2},
a query object \tcode{q}, and
a pack of subexpressions \tcode{as},
\tcode{\exposid{JOIN-ENV}(env1, env2)} is an expression \tcode{env3}
whose type satisfies \exposconcept{queryable}
such that \tcode{env3.query(q, as...)} is expression-equivalent to:
\begin{itemize}
\item
\tcode{env1.query(q, as...)} if that expression is well-formed,
\item
otherwise, \tcode{env2.query(q, as...)} if that expression is well-formed,
\item
otherwise, \tcode{env3.query(q, as...)} is ill-formed.
\end{itemize}

\pnum
The results of \exposid{FWD-ENV}, \exposid{MAKE-ENV}, and \exposid{JOIN-ENV}
can be context-dependent;
i.e., they can evaluate to expressions
with different types and value categories
in different contexts for the same arguments.

\pnum
For a scheduler \tcode{sch},
\tcode{\exposid{SCHED-ATTRS}(sch)} is an expression \tcode{o1}
whose type satisfies \exposconcept{queryable}
such that \tcode{o1.query(get_completion_scheduler<Tag>)} is
an expression with the same type and value as \tcode{sch}
where \tcode{Tag} is one of \tcode{set_value_t} or \tcode{set_stopped_t}, and
such that \tcode{o1.query(get_domain)} is expression-equivalent to
\tcode{sch.query(get_domain)}.
\tcode{\exposid{SCHED-ENV}(sch)} is an expression \tcode{o2}
whose type satisfies \exposconcept{queryable}
such that \tcode{o2.query(get_scheduler)} is a prvalue
with the same type and value as \tcode{sch}, and
such that \tcode{o2.query(get_domain)} is expression-equivalent to
\tcode{sch.query(get_domain)}.

\pnum
For two subexpressions \tcode{rcvr} and \tcode{expr},
\tcode{\exposid{SET-VALUE}(rcvr, expr)} is expression-equivalent to
\tcode{(expr, set_value(std::move(rcvr)))}
if the type of \tcode{expr} is \tcode{void};
otherwise, \tcode{set_value(std::move(rcvr), expr)}.
\tcode{\exposid{TRY-EVAL}(rcvr, expr)} is equivalent to:
\begin{codeblock}
try {
  expr;
} catch(...) {
  set_error(std::move(rcvr), current_exception());
}
\end{codeblock}
if \tcode{expr} is potentially-throwing; otherwise, \tcode{expr}.
\tcode{\exposid{TRY-SET-VALUE}(rcvr, expr)} is
\begin{codeblock}
@\exposid{TRY-EVAL}@(rcvr, @\exposid{SET-VALUE}@(rcvr, expr))
\end{codeblock}
except that \tcode{rcvr} is evaluated only once.

\begin{itemdecl}
template<class Default = default_domain, class Sndr>
  constexpr auto @\exposid{completion-domain}@(const Sndr& sndr) noexcept;
\end{itemdecl}

\begin{itemdescr}
\pnum
\tcode{\exposid{COMPL-DOMAIN}(T)} is the type of the expression
\tcode{get_domain(get_completion_scheduler<T>(get_env(sndr)))}.

\pnum
\effects
If all of the types
\tcode{\exposid{COMPL-DOMAIN}(set_value_t)},
\tcode{\exposid{COMPL-DOMAIN}(set_error_t)}, and\linebreak
\tcode{\exposid{COMPL-DOMAIN}(set_stopped_t)} are ill-formed,
\tcode{completion-domain<Default>(sndr)} is
a default-constructed prvalue of type \tcode{Default}.
Otherwise, if they all share a common type\iref{meta.trans.other}
(ignoring those types that are ill-formed),
then \tcode{\exposid{completion-domain}<Default>(sndr)} is
a default-constructed prvalue of that type.
Otherwise, \tcode{\exposid{completion-domain}<Default>(sndr)} is ill-formed.
\end{itemdescr}

\begin{itemdecl}
template<class Tag, class Env, class Default>
  constexpr decltype(auto) @\exposid{query-with-default}@(
    Tag, const Env& env, Default&& value) noexcept(@\seebelow@);
\end{itemdecl}

\begin{itemdescr}
\pnum
Let \tcode{e} be the expression \tcode{Tag()(env)}
if that expression is well-formed;
otherwise, it is \tcode{static_cast<Default>(std::forward<Default>(value))}.

\pnum
\returns
\tcode{e}.

\pnum
\remarks
The expression in the noexcept clause is \tcode{noexcept(e)}.
\end{itemdescr}

\begin{itemdecl}
template<class Sndr>
  constexpr auto @\exposid{get-domain-early}@(const Sndr& sndr) noexcept;
\end{itemdecl}

\begin{itemdescr}
\pnum
\effects
Equivalent to:
\begin{codeblock}
return Domain();
\end{codeblock}
where \tcode{Domain} is
the decayed type of the first of the following expressions that is well-formed:
\begin{itemize}
\item \tcode{get_domain(get_env(sndr))}
\item \tcode{\exposid{completion-domain}(sndr)}
\item \tcode{default_domain()}
\end{itemize}
\end{itemdescr}

\begin{itemdecl}
template<class Sndr, class Env>
  constexpr auto @\exposid{get-domain-late}@(const Sndr& sndr, const Env& env) noexcept;
\end{itemdecl}

\begin{itemdescr}
\pnum
\effects
Equivalent to:
\begin{itemize}
\item
If \tcode{\exposconcept{sender-for}<Sndr, continues_on_t>} is \tcode{true}, then
\begin{codeblock}
return Domain();
\end{codeblock}
where \tcode{Domain} is the type of the following expression:
\begin{codeblock}
[] {
  auto [_, sch, _] = sndr;
  return @\exposid{query-or-default}@(get_domain, sch, default_domain());
}();
\end{codeblock}
\begin{note}
The \tcode{continues_on} algorithm works
in tandem with \tcode{schedule_from}\iref{exec.schedule.from}
to give scheduler authors a way to customize both
how to transition onto (\tcode{continues_on}) and off of (\tcode{schedule_from})
a given execution context.
Thus, \tcode{continues_on} ignores the domain of the predecessor and
uses the domain of the destination scheduler to select a customization,
a property that is unique to \tcode{continues_on}.
That is why it is given special treatment here.
\end{note}
\item
Otherwise,
\begin{codeblock}
return Domain();
\end{codeblock}
where \tcode{Domain} is the first of the following expressions
that is well-formed and whose type is not \tcode{void}:
\begin{itemize}
\item \tcode{get_domain(get_env(sndr))}
\item \tcode{\exposid{completion-domain}<void>(sndr)}
\item \tcode{get_domain(env)}
\item \tcode{get_domain(get_scheduler(env))}
\item \tcode{default_domain()}
\end{itemize}
\end{itemize}
\end{itemdescr}

\pnum
\begin{codeblock}
template<@\exposconcept{callable}@ Fun>
  requires is_nothrow_move_constructible_v<Fun>
struct @\exposid{emplace-from}@ {
  Fun @\exposid{fun}@;                                                      // \expos
  using type = @\exposid{call-result-t}@<Fun>;

  constexpr operator type() && noexcept(@\exposconcept{nothrow-callable}@<Fun>) {
    return std::move(fun)();
  }

  constexpr type operator()() && noexcept(@\exposconcept{nothrow-callable}@<Fun>) {
    return std::move(fun)();
  }
};
\end{codeblock}
\begin{note}
\exposid{emplace-from} is used to emplace non-movable types
into \tcode{tuple}, \tcode{optional}, \tcode{variant}, and similar types.
\end{note}

\pnum
\begin{codeblock}
struct @\exposid{on-stop-request}@ {
  inplace_stop_source& @\exposid{stop-src}@;       // \expos
  void operator()() noexcept { @\exposid{stop-src}@.request_stop(); }
};
\end{codeblock}

\pnum
\begin{codeblock}
template<class T@$_0$@, class T@$_1$@, @...,@ class T@$_n$@>
struct @\exposid{product-type}@ {       // \expos
  T@$_0$@ t@$_0$@;                // \expos
  T@$_1$@ t@$_1$@;                // \expos
    @...@
  T@$_n$@ t@$_n$@;                // \expos

  template<size_t I, class Self>
  constexpr decltype(auto) @\exposid{get}@(this Self&& self) noexcept;      // \expos

  template<class Self, class Fn>
  constexpr decltype(auto) @\exposid{apply}@(this Self&& self, Fn&& fn)     // \expos
    noexcept(@\seebelow@);
};
\end{codeblock}

\pnum
\begin{note}
\exposid{product-type} is presented here in pseudo-code form
for the sake of exposition.
It can be approximated in standard \Cpp{} with a tuple-like implementation
that takes care to keep the type an aggregate
that can be used as the initializer of a structured binding declaration.
\end{note}
\begin{note}
An expression of type \exposid{product-type} is usable as
the initializer of a structured binding declaration\iref{dcl.struct.bind}.
\end{note}

\begin{itemdecl}
template<size_t I, class Self>
constexpr decltype(auto) @\exposid{get}@(this Self&& self) noexcept;
\end{itemdecl}

\begin{itemdescr}
\pnum
\effects
Equivalent to:
\begin{codeblock}
auto& [...ts] = self;
return std::forward_like<Self>(ts...[I]);
\end{codeblock}
\end{itemdescr}

\begin{codeblock}
template<class Self, class Fn>
constexpr decltype(auto) @\exposid{apply}@(this Self&& self, Fn&& fn) noexcept(@\seebelow@);
\end{codeblock}

\begin{itemdescr}
\pnum
\constraints
The expression in the \tcode{return} statement below is well-formed.

\pnum
\effects
Equivalent to:
\begin{codeblock}
auto& [...ts] = self;
return std::forward<Fn>(fn)(std::forward_like<Self>(ts)...);
\end{codeblock}

\pnum
\remarks
The expression in the \tcode{noexcept} clause is \tcode{true}
if the \tcode{return} statement above is not potentially throwing;
otherwise, \tcode{false}.
\end{itemdescr}

\begin{itemdecl}
template<class Tag, class Data = see below, class... Child>
  constexpr auto @\exposid{make-sender}@(Tag tag, Data&& data, Child&&... child);
\end{itemdecl}

\begin{itemdescr}
\pnum
\mandates
The following expressions are \tcode{true}:
\begin{itemize}
\item \tcode{\libconcept{semiregular}<Tag>}
\item \tcode{\exposconcept{movable-value}<Data>}
\item \tcode{(\libconcept{sender}<Child> \&\&...)}
\end{itemize}

\pnum
\returns
A prvalue of
type \tcode{\exposid{basic-sender}<Tag, decay_t<Data>, decay_t<Child>...>}
that has been direct-list-initialized with the forwarded arguments,
where \exposid{basic-sender} is the following exposition-only class template except as noted below.
\end{itemdescr}

\begin{codeblock}
namespace std::execution {
  template<class Tag>
  concept @\defexposconcept{completion-tag}@ =                                      // \expos
    @\libconcept{same_as}@<Tag, set_value_t> || @\libconcept{same_as}@<Tag, set_error_t> || @\libconcept{same_as}@<Tag, set_stopped_t>;

  template<template<class...> class T, class... Args>
  concept @\defexposconcept{valid-specialization}@ =                                // \expos
    requires { typename T<Args...>; };

  struct @\exposid{default-impls}@ {                                        // \expos
    static constexpr auto @\exposid{get-attrs}@ = @\seebelow@;              // \expos
    static constexpr auto @\exposid{get-env}@ = @\seebelow@;                // \expos
    static constexpr auto @\exposid{get-state}@ = @\seebelow@;              // \expos
    static constexpr auto @\exposid{start}@ = @\seebelow@;                  // \expos
    static constexpr auto @\exposid{complete}@ = @\seebelow@;               // \expos
  };

  template<class Tag>
  struct @\exposid{impls-for}@ : @\exposid{default-impls}@ {};       // \expos

  template<class Sndr, class Rcvr>                              // \expos
  using @\exposid{state-type}@ = decay_t<@\exposid{call-result-t}@<
    decltype(@\exposid{impls-for}@<tag_of_t<Sndr>>::@\exposid{get-state}@), Sndr, Rcvr&>>;

  template<class Index, class Sndr, class Rcvr>                 // \expos
  using @\exposid{env-type}@ = @\exposid{call-result-t}@<
    decltype(@\exposid{impls-for}@<tag_of_t<Sndr>>::@\exposid{get-env}@), Index,
    @\exposid{state-type}@<Sndr, Rcvr>&, const Rcvr&>;

  template<class Sndr, size_t I = 0>
  using @\exposid{child-type}@ = decltype(declval<Sndr>().template @\exposid{get}@<I+2>());     // \expos

  template<class Sndr>
  using @\exposid{indices-for}@ = remove_reference_t<Sndr>::@\exposid{indices-for}@;           // \expos

  template<class Sndr, class Rcvr>
  struct @\exposid{basic-state}@ {                                          // \expos
    @\exposid{basic-state}@(Sndr&& sndr, Rcvr&& rcvr) noexcept(@\seebelow@)
      : @\exposid{rcvr}@(std::move(rcvr))
      , @\exposid{state}@(@\exposid{impls-for}@<tag_of_t<Sndr>>::@\exposid{get-state}@(std::forward<Sndr>(sndr), @\exposid{rcvr}@)) { }

    Rcvr @\exposid{rcvr}@;                                                  // \expos
    @\exposid{state-type}@<Sndr, Rcvr> @\exposid{state}@;                               // \expos
  };

  template<class Sndr, class Rcvr, class Index>
    requires @\exposconcept{valid-specialization}@<@\exposid{env-type}@, Index, Sndr, Rcvr>
  struct @\exposid{basic-receiver}@ {                                       // \expos
    using receiver_concept = receiver_t;

    using @\exposid{tag-t}@ = tag_of_t<Sndr>;                               // \expos
    using @\exposid{state-t}@ = @\exposid{state-type}@<Sndr, Rcvr>;                     // \expos
    static constexpr const auto& @\exposid{complete}@ = @\exposid{impls-for}@<@\exposid{tag-t}@>::@\exposid{complete}@;   // \expos

    template<class... Args>
      requires @\exposconcept{callable}@<decltype(@\exposid{complete}@), Index, @\exposid{state-t}@&, Rcvr&, set_value_t, Args...>
    void set_value(Args&&... args) && noexcept {
      @\exposid{complete}@(Index(), op->@\exposid{state}@, op->@\exposid{rcvr}@, set_value_t(), std::forward<Args>(args)...);
    }

    template<class Error>
      requires @\exposconcept{callable}@<decltype(@\exposid{complete}@), Index, @\exposid{state-t}@&, Rcvr&, set_error_t, Error>
    void set_error(Error&& err) && noexcept {
      @\exposid{complete}@(Index(), op->@\exposid{state}@, op->@\exposid{rcvr}@, set_error_t(), std::forward<Error>(err));
    }

    void set_stopped() && noexcept
      requires @\exposconcept{callable}@<decltype(@\exposid{complete}@), Index, @\exposid{state-t}@&, Rcvr&, set_stopped_t> {
      @\exposid{complete}@(Index(), op->@\exposid{state}@, op->@\exposid{rcvr}@, set_stopped_t());
    }

    auto get_env() const noexcept -> @\exposid{env-type}@<Index, Sndr, Rcvr> {
      return @\exposid{impls-for}@<tag-t>::@\exposid{get-env}@(Index(), op->@\exposid{state}@, op->@\exposid{rcvr}@);
    }

    @\exposid{basic-state}@<Sndr, Rcvr>* @\exposid{op}@;                           // \expos
  };

  constexpr auto @\exposid{connect-all}@ = @\seebelow@;                         // \expos

  template<class Sndr, class Rcvr>
  using @\exposid{connect-all-result}@ = @\exposid{call-result-t}@<                     // \expos
    decltype(@\exposid{connect-all}@), @\exposid{basic-state}@<Sndr, Rcvr>*, Sndr, @\exposid{indices-for}@<Sndr>>;

  template<class Sndr, class Rcvr>
    requires @\exposconcept{valid-specialization}@<@\exposid{state-type}@, Sndr, Rcvr> &&
             @\exposconcept{valid-specialization}@<@\exposid{connect-all-result}@, Sndr, Rcvr>
  struct @\exposid{basic-operation}@ : @\exposid{basic-state}@<Sndr, Rcvr> {                // \expos
    using operation_state_concept = operation_state_t;
    using @\exposid{tag-t}@ = tag_of_t<Sndr>;                               // \expos

    @\exposid{connect-all-result}@<Sndr, Rcvr> @\exposid{inner-ops}@;              // \expos

    @\exposid{basic-operation}@(Sndr&& sndr, Rcvr&& rcvr) noexcept(@\seebelow@)  // \expos
      : @\exposid{basic-state}@<Sndr, Rcvr>(std::forward<Sndr>(sndr), std::move(rcvr)),
        @\exposid{inner-ops}@(@\exposid{connect-all}@(this, std::forward<Sndr>(sndr), @\exposid{indices-for}@<Sndr>()))
    {}

    void start() & noexcept {
      auto& [...ops] = @\exposid{inner-ops}@;
      @\exposid{impls-for}@<tag-t>::@\exposid{start}@(this->@\exposid{state}@, this->@\exposid{rcvr}@, ops...);
    }
  };

  template<class Sndr, class Env>
  using @\exposid{completion-signatures-for}@ = @\seebelow@;                   // \expos

  template<class Tag, class Data, class... Child>
  struct @\exposid{basic-sender}@ : @\exposid{product-type}@<Tag, Data, Child...> {    // \expos
    using sender_concept = sender_t;
    using @\exposid{indices-for}@ = index_sequence_for<Child...>;       // \expos

    decltype(auto) get_env() const noexcept {
      auto& [_, data, ...child] = *this;
      return @\exposid{impls-for}@<Tag>::@\exposid{get-attrs}@(data, child...);
    }

    template<@\exposconcept{decays-to}@<@\exposid{basic-sender}@> Self, @\libconcept{receiver}@ Rcvr>
    auto connect(this Self&& self, Rcvr rcvr) noexcept(@\seebelow@)
      -> @\exposid{basic-operation}@<Self, Rcvr> {
      return {std::forward<Self>(self), std::move(rcvr)};
    }

    template<@\exposconcept{decays-to}@<@\exposid{basic-sender}@> Self, class Env>
    auto get_completion_signatures(this Self&& self, Env&& env) noexcept
      -> @\exposid{completion-signatures-for}@<Self, Env> {
      return {};
    }
  };
}
\end{codeblock}

\pnum
The default template argument for the \tcode{Data} template parameter
denotes an unspecified empty trivially copyable class type
that models \libconcept{semiregular}.

\pnum
It is unspecified whether a specialization of \exposid{basic-sender}
is an aggregate.

\pnum
An expression of type \exposid{basic-sender} is usable as
the initializer of a structured binding declaration\iref{dcl.struct.bind}.

\pnum
The expression in the \tcode{noexcept} clause of
the constructor of \exposid{basic-state} is
\begin{codeblock}
is_nothrow_move_constructible_v<Rcvr> &&
@\exposconcept{nothrow-callable}@<decltype(@\exposid{impls-for}@<tag_of_t<Sndr>>::@\exposid{get-state}@), Sndr, Rcvr&> &&
(@\libconcept{same_as}@<@\exposid{state-type}@<Sndr, Rcvr>, @\exposid{get-state-result}@> ||
 is_nothrow_constructible_v<@\exposid{state-type}@<Sndr, Rcvr>, @\exposid{get-state-result}@>)
\end{codeblock}
where \exposid{get-state-result} is
\begin{codeblock}
@\exposid{call-result-t}@<decltype(@\exposid{impls-for}@<tag_of_t<Sndr>>::@\exposid{get-state}@), Sndr, Rcvr&>.
\end{codeblock}

\pnum
The object \exposid{connect-all} is initialized with
a callable object equivalent to the following lambda:
\begin{itemdecl}
[]<class Sndr, class Rcvr, size_t... Is>(
  @\exposid{basic-state}@<Sndr, Rcvr>* op, Sndr&& sndr, index_sequence<Is...>) noexcept(@\seebelow@)
    -> decltype(auto) {
    auto& [_, data, ...child] = sndr;
    return @\exposid{product-type}@{connect(
      std::forward_like<Sndr>(child),
      @\exposid{basic-receiver}@<Sndr, Rcvr, integral_constant<size_t, Is>>{op})...};
  }
\end{itemdecl}

\begin{itemdescr}
\pnum
\constraints
The expression in the \tcode{return} statement is well-formed.

\pnum
\remarks
The expression in the \tcode{noexcept} clause is \tcode{true}
if the \tcode{return} statement is not potentially throwing;
otherwise, \tcode{false}.
\end{itemdescr}

\pnum
The expression in the \tcode{noexcept} clause of
the constructor of \exposid{basic-operation} is:
\begin{codeblock}
is_nothrow_constructible_v<@\exposid{basic-state}@<Self, Rcvr>, Self, Rcvr> &&
noexcept(@\exposid{connect-all}@(this, std::forward<Sndr>(sndr), @\exposid{indices-for}@<Sndr>()))
\end{codeblock}

\pnum
The expression in the \tcode{noexcept} clause of
the \tcode{connect} member function of \exposid{basic-sender} is:
\begin{codeblock}
is_nothrow_constructible_v<@\exposid{basic-operation}@<Self, Rcvr>, Self, Rcvr>
\end{codeblock}

\pnum
The member \tcode{\exposid{default-impls}::\exposid{get-attrs}}
is initialized with a callable object equivalent to the following lambda:
\begin{codeblock}
[](const auto&, const auto&... child) noexcept -> decltype(auto) {
  if constexpr (sizeof...(child) == 1)
    return (@\exposid{FWD-ENV}@(get_env(child)), ...);
  else
    return env<>();
}
\end{codeblock}

\pnum
The member \tcode{\exposid{default-impls}::\exposid{get-env}}
is initialized with a callable object equivalent to the following lambda:
\begin{codeblock}
[](auto, auto&, const auto& rcvr) noexcept -> decltype(auto) {
  return @\exposid{FWD-ENV}@(get_env(rcvr));
}
\end{codeblock}

\pnum
The member \tcode{\exposid{default-impls}::\exposid{get-state}}
is initialized with a callable object equivalent to the following lambda:
\begin{codeblock}
[]<class Sndr, class Rcvr>(Sndr&& sndr, Rcvr& rcvr) noexcept -> decltype(auto) {
  auto& [_, data, ...child] = sndr;
  return std::forward_like<Sndr>(data);
}
\end{codeblock}

\pnum
The member \tcode{\exposid{default-impls}::\exposid{start}}
is initialized with a callable object equivalent to the following lambda:
\begin{codeblock}
[](auto&, auto&, auto&... ops) noexcept -> void {
  (execution::start(ops), ...);
}
\end{codeblock}

\pnum
The member \tcode{\exposid{default-impls}::\exposid{complete}}
is initialized with a callable object equivalent to the following lambda:
\begin{codeblock}
[]<class Index, class Rcvr, class Tag, class... Args>(
  Index, auto& state, Rcvr& rcvr, Tag, Args&&... args) noexcept
    -> void requires @\exposconcept{callable}@<Tag, Rcvr, Args...> {
  static_assert(Index::value == 0);
  Tag()(std::move(rcvr), std::forward<Args>(args)...);
}
\end{codeblock}

\pnum
For a subexpression \tcode{sndr} let \tcode{Sndr} be \tcode{decltype((sndr))}.
Let \tcode{rcvr} be a receiver
with an associated environment of type \tcode{Env}
such that \tcode{\libconcept{sender_in}<Sndr, Env>} is \tcode{true}.
\tcode{\exposid{completion-signatures-for}<Sndr, Env>} denotes
a specialization of \tcode{completion_signatures},
the set of whose template arguments correspond to
the set of completion operations that are potentially evaluated
as a result of starting\iref{exec.async.ops}
the operation state that results from connecting \tcode{sndr} and \tcode{rcvr}.
When \tcode{\libconcept{sender_in}<Sndr, Env>} is \tcode{false},
the type denoted by \tcode{\exposid{completion-signatures-for}<Sndr, Env>},
if any, is not a specialization of \tcode{completion_signatures}.

\recommended
When \tcode{\libconcept{sender_in}<Sndr, Env>} is \tcode{false},
implementations are encouraged to use the type
denoted by \tcode{\exposid{completion-signatures-for}<Sndr, Env>}
to communicate to users why.

\begin{itemdecl}
template<@\libconcept{sender}@ Sndr, @\exposconcept{queryable}@ Env>
  constexpr auto @\exposid{write-env}@(Sndr&& sndr, Env&& env);     // \expos
\end{itemdecl}

\begin{itemdescr}
\pnum
\exposid{write-env} is an exposition-only sender adaptor that,
when connected with a receiver \tcode{rcvr},
connects the adapted sender with a receiver
whose execution environment is the result of
joining the \exposconcept{queryable} argument \tcode{env}
to the result of \tcode{get_env(rcvr)}.

\pnum
Let \exposid{write-env-t} be an exposition-only empty class type.

\pnum
\returns
\begin{codeblock}
@\exposid{make-sender}@(@\exposid{write-env-t}@(), std::forward<Env>(env), std::forward<Sndr>(sndr))
\end{codeblock}

\pnum
\remarks
The exposition-only class template \exposid{impls-for}\iref{exec.snd.general}
is specialized for \exposid{write-env-t} as follows:
\begin{codeblock}
template<>
struct @\exposid{impls-for}@<@\exposid{write-env-t}@> : @\exposid{default-impls}@ {
  static constexpr auto @\exposid{get-env}@ =
    [](auto, const auto& state, const auto& rcvr) noexcept {
      return @\seebelow@;
    };
};
\end{codeblock}
Invocation of
\tcode{\exposid{impls-for}<\exposid{write-env-t}>::\exposid{get-env}}
returns an object \tcode{e} such that
\begin{itemize}
\item
\tcode{decltype(e)} models \exposconcept{queryable} and
\item
given a query object \tcode{q},
the expression \tcode{e.query(q)} is expression-equivalent
to \tcode{state.query(q)} if that expression is valid,
otherwise, \tcode{e.query(q)} is expression-equivalent
to \tcode{get_env(rcvr).que\-ry(q)}.
\end{itemize}
\end{itemdescr}

\rSec2[exec.snd.concepts]{Sender concepts}

\pnum
The \libconcept{sender} concept defines
the requirements for a sender type\iref{exec.async.ops}.
The \libconcept{sender_in} concept defines
the requirements for a sender type
that can create asynchronous operations given an associated environment type.
The \libconcept{sender_to} concept defines
the requirements for a sender type
that can connect with a specific receiver type.
The \tcode{get_env} customization point object is used to access
a sender's associated attributes.
The connect customization point object is used to connect\iref{exec.async.ops}
a sender and a receiver to produce an operation state.

\begin{codeblock}
namespace std::execution {
  template<class Sigs>
    concept @\exposconcept{valid-completion-signatures}@ = @\seebelow@;            // \expos

  template<class Sndr>
    concept @\defexposconcept{is-sender}@ =                                         // \expos
      @\libconcept{derived_from}@<typename Sndr::sender_concept, sender_t>;

  template<class Sndr>
    concept @\defexposconcept{enable-sender}@ =                                     // \expos
      @\exposconcept{is-sender}@<Sndr> ||
      @\exposconcept{is-awaitable}@<Sndr, @\exposid{env-promise}@<env<>>>;                 // \ref{exec.awaitable}

  template<class Sndr>
    concept @\deflibconcept{sender}@ =
      bool(@\exposconcept{enable-sender}@<remove_cvref_t<Sndr>>) &&
      requires (const remove_cvref_t<Sndr>& sndr) {
        { get_env(sndr) } -> @\exposconcept{queryable}@;
      } &&
      @\libconcept{move_constructible}@<remove_cvref_t<Sndr>> &&
      @\libconcept{constructible_from}@<remove_cvref_t<Sndr>, Sndr>;

  template<class Sndr, class Env = env<>>
    concept @\deflibconcept{sender_in}@ =
      @\libconcept{sender}@<Sndr> &&
      @\exposconcept{queryable}@<Env> &&
      requires (Sndr&& sndr, Env&& env) {
        { get_completion_signatures(std::forward<Sndr>(sndr), std::forward<Env>(env)) }
          -> @\exposconcept{valid-completion-signatures}@;
      };

  template<class Sndr, class Rcvr>
    concept @\deflibconcept{sender_to}@ =
      @\libconcept{sender_in}@<Sndr, env_of_t<Rcvr>> &&
      @\libconcept{receiver_of}@<Rcvr, completion_signatures_of_t<Sndr, env_of_t<Rcvr>>> &&
      requires (Sndr&& sndr, Rcvr&& rcvr) {
        connect(std::forward<Sndr>(sndr), std::forward<Rcvr>(rcvr));
      };
}
\end{codeblock}

\pnum
Given a subexpression \tcode{sndr},
let \tcode{Sndr} be \tcode{decltype((sndr))} and
let \tcode{rcvr} be a receiver
with an associated environment whose type is \tcode{Env}.
A completion operation is a \defnadj{permissible}{completion}
for \tcode{Sndr} and \tcode{Env}
if its completion signature appears in the argument list of the specialization of \tcode{completion_signatures} denoted by
\tcode{completion_signatures_of_t<Sndr, Env>}.
\tcode{Sndr} and \tcode{Env} model \tcode{\libconcept{sender_in}<Sndr, Env>}
if all the completion operations
that are potentially evaluated by connecting \tcode{sndr} to \tcode{rcvr} and
starting the resulting operation state
are permissible completions for \tcode{Sndr} and \tcode{Env}.

\pnum
A type models
the exposition-only concept \defexposconcept{valid-completion-signatures}
if it denotes a specialization of
the \tcode{completion_signatures} class template.

\pnum
The exposition-only concepts
\exposconcept{sender-of} and \exposconcept{sender-in-of}
define the requirements for a sender type
that completes with a given unique set of value result types.
\begin{codeblock}
namespace std::execution {
  template<class... As>
    using @\exposid{value-signature}@ = set_value_t(As...);             // \expos

  template<class Sndr, class Env, class... Values>
    concept @\defexposconcept{sender-in-of}@ =
      @\libconcept{sender_in}@<Sndr, Env> &&
      @\exposid{MATCHING-SIG}@(                     // see \ref{exec.general}
        set_value_t(Values...),
        value_types_of_t<Sndr, Env, @\exposid{value-signature}@, type_identity_t>);

  template<class Sndr, class... Values>
    concept @\defexposconcept{sender-of}@ = @\exposconcept{sender-in-of}@<Sndr, env<>, Values...>;
}
\end{codeblock}

\pnum
Let \tcode{sndr} be an expression
such that \tcode{decltype((sndr))} is \tcode{Sndr}.
The type \tcode{tag_of_t<Sndr>} is as follows:
\begin{itemize}
\item
If the declaration
\begin{codeblock}
auto&& [tag, data, ...children] = sndr;
\end{codeblock}
would be well-formed, \tcode{tag_of_t<Sndr>} is
an alias for \tcode{decltype(auto(tag))}.
\item
Otherwise, \tcode{tag_of_t<Sndr>} is ill-formed.
\end{itemize}

\pnum
Let \exposconcept{sender-for} be an exposition-only concept defined as follows:
\begin{codeblock}
namespace std::execution {
  template<class Sndr, class Tag>
  concept @\defexposconcept{sender-for}@ =
    @\libconcept{sender}@<Sndr> &&
    @\libconcept{same_as}@<tag_of_t<Sndr>, Tag>;
}
\end{codeblock}

\pnum
For a type \tcode{T},
\tcode{\exposid{SET-VALUE-SIG}(T)} denotes the type \tcode{set_value_t()}
if \tcode{T} is \cv{} \tcode{void};
otherwise, it denotes the type \tcode{set_value_t(T)}.

\pnum
Library-provided sender types
\begin{itemize}
\item
always expose an overload of a member \tcode{connect}
that accepts an rvalue sender and
\item
only expose an overload of a member \tcode{connect}
that accepts an lvalue sender if they model \libconcept{copy_constructible}.
\end{itemize}

\rSec2[exec.awaitable]{Awaitable helpers}

\pnum
The sender concepts recognize awaitables as senders.
For \ref{exec}, an \defn{awaitable} is an expression
that would be well-formed as the operand of a \tcode{co_await} expression
within a given context.

\pnum
For a subexpression \tcode{c},
let \tcode{\exposid{GET-AWAITER}(c, p)} be expression-equivalent to
the series of transformations and conversions applied to \tcode{c}
as the operand of an \grammarterm{await-expression} in a coroutine,
resulting in lvalue \tcode{e} as described by \ref{expr.await},
where \tcode{p} is an lvalue referring to the coroutine's promise,
which has type \tcode{Promise}.
\begin{note}
This includes the invocation of
the promise type's \tcode{await_transform} member if any,
the invocation of the \tcode{operator co_await}
picked by overload resolution if any, and
any necessary implicit conversions and materializations.
\end{note}

\pnum
Let \exposconcept{is-awaitable} be the following exposition-only concept:
\begin{codeblock}
namespace std {
  template<class T>
  concept @\exposconcept{await-suspend-result}@ = @\seebelow@;                     // \expos

  template<class A, class Promise>
  concept @\defexposconcept{is-awaiter}@ =                                          // \expos
    requires (A& a, coroutine_handle<Promise> h) {
      a.await_ready() ? 1 : 0;
      { a.await_suspend(h) } -> @\exposconcept{await-suspend-result}@;
      a.await_resume();
    };

  template<class C, class Promise>
  concept @\defexposconcept{is-awaitable}@ =                                        // \expos
    requires (C (*fc)() noexcept, Promise& p) {
      { @\exposid{GET-AWAITER}@(fc(), p) } -> @\exposconcept{is-awaiter}@<Promise>;
    };
}
\end{codeblock}

\tcode{\defexposconcept{await-suspend-result}<T>} is \tcode{true}
if and only if one of the following is \tcode{true}:
\begin{itemize}
\item \tcode{T} is \tcode{void}, or
\item \tcode{T} is \tcode{bool}, or
\item \tcode{T} is a specialization of \tcode{coroutine_handle}.
\end{itemize}

\pnum
For a subexpression \tcode{c}
such that \tcode{decltype((c))} is type \tcode{C}, and
an lvalue \tcode{p} of type \tcode{Promise},
\tcode{\exposid{await-result-\newline type}<C, Promise>} denotes
the type \tcode{decltype(\exposid{GET-AWAITER}(c, p).await_resume())}.

\pnum
Let \exposid{with-await-transform} be the exposition-only class template:
\begin{codeblock}
namespace std::execution {
  template<class T, class Promise>
    concept @\defexposconcept{has-as-awaitable}@ =                                  // \expos
      requires (T&& t, Promise& p) {
        { std::forward<T>(t).as_awaitable(p) } -> @\exposconcept{is-awaitable}@<Promise&>;
      };

  template<class Derived>
    struct @\exposid{with-await-transform}@ {                               // \expos
      template<class T>
        T&& await_transform(T&& value) noexcept {
          return std::forward<T>(value);
        }

      template<@\exposconcept{has-as-awaitable}@<Derived> T>
        decltype(auto) await_transform(T&& value)
          noexcept(noexcept(std::forward<T>(value).as_awaitable(declval<Derived&>()))) {
          return std::forward<T>(value).as_awaitable(static_cast<Derived&>(*this));
        }
    };
}
\end{codeblock}

\pnum
Let \exposid{env-promise} be the exposition-only class template:
\begin{codeblock}
namespace std::execution {
  template<class Env>
  struct @\exposid{env-promise}@ : @\exposid{with-await-transform}@<@\exposid{env-promise}@<Env>> { // \expos
    @\unspec@ get_return_object() noexcept;
    @\unspec@ initial_suspend() noexcept;
    @\unspec@ final_suspend() noexcept;
    void unhandled_exception() noexcept;
    void return_void() noexcept;
    coroutine_handle<> unhandled_stopped() noexcept;

    const Env& get_env() const noexcept;
  };
}
\end{codeblock}
\begin{note}
Specializations of \exposid{env-promise} are used only for the purpose of type computation;
its members need not be defined.
\end{note}

\rSec2[exec.domain.default]{\tcode{execution::default_domain}}

\pnum
\begin{codeblock}
namespace std::execution {
  struct @\libglobal{default_domain}@ {
    template<@\libconcept{sender}@ Sndr, @\exposconcept{queryable}@... Env>
        requires (sizeof...(Env) <= 1)
      static constexpr @\libconcept{sender}@ decltype(auto) transform_sender(Sndr&& sndr, const Env&... env)
        noexcept(@\seebelow@);

    template<@\libconcept{sender}@ Sndr, @\exposconcept{queryable}@ Env>
      static constexpr @\exposconcept{queryable}@ decltype(auto) transform_env(Sndr&& sndr, Env&& env) noexcept;

    template<class Tag, @\libconcept{sender}@ Sndr, class... Args>
      static constexpr decltype(auto) apply_sender(Tag, Sndr&& sndr, Args&&... args)
        noexcept(@\seebelow@);
  };
}
\end{codeblock}

\indexlibrarymember{transform_sender}{default_domain}%
\begin{itemdecl}
template<@\libconcept{sender}@ Sndr, @\exposconcept{queryable}@... Env>
  requires (sizeof...(Env) <= 1)
constexpr @\libconcept{sender}@ decltype(auto) transform_sender(Sndr&& sndr, const Env&... env)
  noexcept(@\seebelow@);
\end{itemdecl}

\begin{itemdescr}
\pnum
Let \tcode{e} be the expression
\begin{codeblock}
tag_of_t<Sndr>().transform_sender(std::forward<Sndr>(sndr), env...)
\end{codeblock}
if that expression is well-formed;
otherwise, \tcode{std::forward<Sndr>(sndr)}.

\pnum
\returns
\tcode{e}.

\pnum
\remarks
The exception specification is equivalent to \tcode{noexcept(e)}.
\end{itemdescr}

\indexlibrarymember{transform_env}{default_domain}%
\begin{itemdecl}
template<@\libconcept{sender}@ Sndr, @\exposconcept{queryable}@ Env>
  constexpr @\exposconcept{queryable}@ decltype(auto) transform_env(Sndr&& sndr, Env&& env) noexcept;
\end{itemdecl}

\begin{itemdescr}
\pnum
Let \tcode{e} be the expression
\begin{codeblock}
tag_of_t<Sndr>().transform_env(std::forward<Sndr>(sndr), std::forward<Env>(env))
\end{codeblock}
if that expression is well-formed;
otherwise, \tcode{static_cast<Env>(std::forward<Env>(env))}.

\pnum
\mandates
\tcode{noexcept(e)} is \tcode{true}.

\pnum
\returns
\tcode{e}.
\end{itemdescr}

\indexlibrarymember{apply_sender}{default_domain}%
\begin{itemdecl}
template<class Tag, @\libconcept{sender}@ Sndr, class... Args>
constexpr decltype(auto) apply_sender(Tag, Sndr&& sndr, Args&&... args)
  noexcept(@\seebelow@);
\end{itemdecl}

\begin{itemdescr}
\pnum
Let \tcode{e} be the expression
\begin{codeblock}
  Tag().apply_sender(std::forward<Sndr>(sndr), std::forward<Args>(args)...)
\end{codeblock}

\pnum
\constraints
\tcode{e} is a well-formed expression.

\pnum
\returns
\tcode{e}.

\pnum
\remarks
The exception specification is equivalent to \tcode{noexcept(e)}.
\end{itemdescr}

\rSec2[exec.snd.transform]{\tcode{execution::transform_sender}}

\indexlibraryglobal{transform_sender}%
\begin{itemdecl}
namespace std::execution {
  template<class Domain, @\libconcept{sender}@ Sndr, @\exposconcept{queryable}@... Env>
      requires (sizeof...(Env) <= 1)
    constexpr @\libconcept{sender}@ decltype(auto) transform_sender(Domain dom, Sndr&& sndr, const Env&... env)
      noexcept(@\seebelow@);
}
\end{itemdecl}

\begin{itemdescr}
\pnum
Let \exposid{transformed-sndr} be the expression
\begin{codeblock}
dom.transform_sender(std::forward<Sndr>(sndr), env...)
\end{codeblock}
if that expression is well-formed; otherwise,
\begin{codeblock}
default_domain().transform_sender(std::forward<Sndr>(sndr), env...)
\end{codeblock}
Let \exposid{final-sndr} be the expression \exposid{transformed-sndr}
if \exposid{transformed-sndr} and \exposid{sndr}
have the same type ignoring cv-qualifiers;
otherwise, it is
the expression \tcode{transform_sender(dom, \exposid{transformed-sndr}, env...)}.

\pnum
\returns
\exposid{final-sndr}.

\pnum
\remarks
The exception specification is equivalent to
\tcode{noexcept(\exposid{final-sndr})}.
\end{itemdescr}

\rSec2[exec.snd.transform.env]{\tcode{execution::transform_env}}

\indexlibraryglobal{transform_env}%
\begin{itemdecl}
namespace std::execution {
  template<class Domain, @\libconcept{sender}@ Sndr, @\exposconcept{queryable}@ Env>
    constexpr @\exposconcept{queryable}@ decltype(auto) transform_env(Domain dom, Sndr&& sndr, Env&& env) noexcept;
}
\end{itemdecl}

\begin{itemdescr}
\pnum
Let \tcode{e} be the expression
\begin{codeblock}
dom.transform_env(std::forward<Sndr>(sndr), std::forward<Env>(env))
\end{codeblock}
if that expression is well-formed; otherwise,
\begin{codeblock}
default_domain().transform_env(std::forward<Sndr>(sndr), std::forward<Env>(env))
\end{codeblock}

\pnum
\mandates
\tcode{noexcept(e)} is \tcode{true}.

\pnum
\returns
\tcode{e}.
\end{itemdescr}

\rSec2[exec.snd.apply]{\tcode{execution::apply_sender}}

\indexlibraryglobal{apply_sender}%
\begin{itemdecl}
namespace std::execution {
  template<class Domain, class Tag, @\libconcept{sender}@ Sndr, class... Args>
    constexpr decltype(auto) apply_sender(Domain dom, Tag, Sndr&& sndr, Args&&... args)
      noexcept(@\seebelow@);
}
\end{itemdecl}

\begin{itemdescr}
\pnum
Let $e$ be the expression
\begin{codeblock}
dom.apply_sender(Tag(), std::forward<Sndr>(sndr), std::forward<Args>(args)...)
\end{codeblock}
if that expression is well-formed; otherwise,
\begin{codeblock}
default_domain().apply_sender(Tag(), std::forward<Sndr>(sndr), std::forward<Args>(args)...)
\end{codeblock}

\pnum
\constraints
The expression $e$ is well-formed.

\pnum
\returns
$e$.

\pnum
\remarks
The exception specification is equivalent to \tcode{noexcept($e$)}.
\end{itemdescr}

\rSec2[exec.getcomplsigs]{\tcode{execution::get_completion_signatures}}

\pnum
\tcode{get_completion_signatures} is a customization point object.
Let \tcode{sndr} be an expression
such that \tcode{decltype((sndr))} is \tcode{Sndr}, and
let \tcode{env} be an expression
such that \tcode{decltype((env))} is \tcode{Env}.
Let \tcode{new_sndr} be the expression
\tcode{transform_sender(decltype(\exposid{get-domain-late}(sndr, env))\{\}, sndr, env)}, and
let \tcode{NewSndr} be \tcode{decltype((new_sndr))}.
Then \tcode{get_completion_signatures(sndr, env)} is expression-equiva\-lent to
\tcode{(void(sndr), void(env), CS())}
except that \tcode{void(sndr)} and \tcode{void(env)} are
indeterminately sequenced,
where \tcode{CS} is:
\begin{itemize}
\item
\tcode{decltype(new_sndr.get_completion_signatures(env))}
if that type is well-formed,

\item
Otherwise, \tcode{remove_cvref_t<NewSndr>::completion_signatures}
if that type is well-formed,

\item
Otherwise,
if \tcode{\exposconcept{is-awaitable}<NewSndr, \exposid{env-promise}<Env>>} is \tcode{true},
then:
\begin{codeblock}
completion_signatures<
  @\exposid{SET-VALUE-SIG}@(@\exposid{await-result-type}@<NewSndr, @\exposid{env-promise}@<Env>>),        // \iref{exec.snd.concepts}
  set_error_t(exception_ptr),
  set_stopped_t()>
\end{codeblock}

\item
Otherwise, \tcode{CS} is ill-formed.
\end{itemize}

\pnum
Let \tcode{rcvr} be an rvalue
whose type \tcode{Rcvr} models \libconcept{receiver}, and
let \tcode{Sndr} be the type of a sender
such that \tcode{\libconcept{sender_in}<Sndr, env_of_t<Rcvr>>} is \tcode{true}.
Let \tcode{Sigs...} be the template arguments of
the \tcode{completion_signatures} specialization
named by \tcode{completion_signatures_of_t<Sndr, env_of_t<Rcvr>>}.
Let \tcode{CSO} be a completion function.
If sender \tcode{Sndr} or its operation state cause
the expression \tcode{CSO(rcvr, args...)}
to be potentially evaluated\iref{basic.def.odr}
then there shall be a signature \tcode{Sig} in \tcode{Sigs...}
such that
\begin{codeblock}
@\exposid{MATCHING-SIG}@(@\exposid{decayed-typeof}@<CSO>(decltype(args)...), Sig)
\end{codeblock}
is \tcode{true}\iref{exec.general}.

\rSec2[exec.connect]{\tcode{execution::connect}}

\pnum
\tcode{connect} connects\iref{exec.async.ops} a sender with a receiver.

\pnum
The name \tcode{connect} denotes a customization point object.
For subexpressions \tcode{sndr} and \tcode{rcvr},
let \tcode{Sndr} be \tcode{decltype((sndr))} and
\tcode{Rcvr} be \tcode{decltype((rcvr))},
let \tcode{new_sndr} be the expression
\begin{codeblock}
transform_sender(decltype(@\exposid{get-domain-late}@(sndr, get_env(rcvr))){}, sndr, get_env(rcvr))
\end{codeblock}
and let \tcode{DS} and \tcode{DR} be
\tcode{decay_t<decltype((new_sndr))>} and \tcode{decay_t<Rcvr>}, respectively.

\pnum
Let \exposid{connect-awaitable-promise} be the following exposition-only class:

\begin{codeblock}
namespace std::execution {
  struct @\exposid{connect-awaitable-promise}@ : @\exposid{with-await-transform}@<@\exposid{connect-awaitable-promise}@> {

    @\exposid{connect-awaitable-promise}@(DS&, DR& rcvr) noexcept : @\exposid{rcvr}@(rcvr) {}

    suspend_always initial_suspend() noexcept { return {}; }
    [[noreturn]] suspend_always final_suspend() noexcept { terminate(); }
    [[noreturn]] void unhandled_exception() noexcept { terminate(); }
    [[noreturn]] void return_void() noexcept { terminate(); }

    coroutine_handle<> unhandled_stopped() noexcept {
      set_stopped(std::move(@\exposid{rcvr}@));
      return noop_coroutine();
    }

    @\exposid{operation-state-task}@ get_return_object() noexcept {
      return @\exposid{operation-state-task}@{
        coroutine_handle<@\exposid{connect-awaitable-promise}@>::from_promise(*this)};
    }

    env_of_t<DR> get_env() const noexcept {
      return execution::get_env(@\exposid{rcvr}@);
    }

  private:
    DR& @\exposid{rcvr}@;                           // \expos
  };
}
\end{codeblock}

\pnum
Let \exposid{operation-state-task} be the following exposition-only class:
\begin{codeblock}
namespace std::execution {
  struct @\exposid{operation-state-task}@ {                              // \expos
    using operation_state_concept = operation_state_t;
    using promise_type = @\exposid{connect-awaitable-promise}@;

    explicit @\exposid{operation-state-task}@(coroutine_handle<> h) noexcept : coro(h) {}
    @\exposid{operation-state-task}@(@\exposid{operation-state-task}@&&) = delete;
    ~@\exposid{operation-state-task}@() { @\exposid{coro}@.destroy(); }

    void start() & noexcept {
      @\exposid{coro}@.resume();
    }

  private:
    coroutine_handle<> @\exposid{coro}@;                                    // \expos
  };
}
\end{codeblock}

\pnum
Let \tcode{V} name the type
\tcode{\exposid{await-result-type}<DS, \exposid{connect-awaitable-promise}>},
let \tcode{Sigs} name the type
\begin{codeblock}
completion_signatures<
  @\exposid{SET-VALUE-SIG}@(V),         // see \iref{exec.snd.concepts}
  set_error_t(exception_ptr),
  set_stopped_t()>
\end{codeblock}
and let \exposid{connect-awaitable} be an exposition-only coroutine
defined as follows:
\begin{codeblock}
namespace std::execution {
  template<class Fun, class... Ts>
  auto @\exposid{suspend-complete}@(Fun fun, Ts&&... as) noexcept {    // \expos
    auto fn = [&, fun]() noexcept { fun(std::forward<Ts>(as)...); };

    struct awaiter {
      decltype(@\exposid{fn}@) @\exposid{fn}@;                                     // \expos

      static constexpr bool await_ready() noexcept { return false; }
      void await_suspend(coroutine_handle<>) noexcept { @\exposid{fn}@(); }
      [[noreturn]] void await_resume() noexcept { unreachable(); }
    };
    return awaiter{@\exposid{fn}@};
  }

  @\exposid{operation-state-task}@ @\exposid{connect-awaitable}@(DS sndr, DR rcvr) requires @\libconcept{receiver_of}@<DR, Sigs> {
    exception_ptr ep;
    try {
      if constexpr (@\libconcept{same_as}@<V, void>) {
        co_await std::move(sndr);
        co_await @\exposid{suspend-complete}@(set_value, std::move(rcvr));
      } else {
        co_await @\exposid{suspend-complete}@(set_value, std::move(rcvr), co_await std::move(sndr));
      }
    } catch(...) {
      ep = current_exception();
    }
    co_await @\exposid{suspend-complete}@(set_error, std::move(rcvr), std::move(ep));
  }
}
\end{codeblock}

\pnum
The expression \tcode{connect(sndr, rcvr)} is expression-equivalent to:
\begin{itemize}
\item
\tcode{new_sndr.connect(rcvr)} if that expression is well-formed.

\mandates
The type of the expression above satisfies \libconcept{operation_state}.

\item
Otherwise, \tcode{\exposid{connect-awaitable}(new_sndr, rcvr)}.
\end{itemize}

\mandates
\tcode{\libconcept{sender}<Sndr> \&\& \libconcept{receiver}<Rcvr>} is \tcode{true}.

\rSec2[exec.factories]{Sender factories}

\rSec3[exec.schedule]{\tcode{execution::schedule}}

\pnum
\tcode{schedule} obtains a schedule sender\iref{exec.async.ops}
from a scheduler.

\pnum
The name \tcode{schedule} denotes a customization point object.
For a subexpression \tcode{sch},
the expression \tcode{schedule(sch)} is expression-equivalent to
\tcode{sch.schedule()}.

\pnum
\mandates
The type of \tcode{sch.schedule()} satisfies \libconcept{sender}.

\pnum
If the expression
\begin{codeblock}
get_completion_scheduler<set_value_t>(get_env(sch.schedule())) == sch
\end{codeblock}
is ill-formed or evaluates to \tcode{false},
the behavior of calling \tcode{schedule(sch)} is undefined.

\rSec3[exec.just]{\tcode{execution::just}, \tcode{execution::just_error}, \tcode{execution::just_stopped}}

\pnum
\tcode{just}, \tcode{just_error}, and \tcode{just_stopped} are sender factories
whose asynchronous operations complete synchronously in their start operation
with a value completion operation,
an error completion operation, or
a stopped completion operation, respectively.

\pnum
The names \tcode{just}, \tcode{just_error}, and \tcode{just_stopped} denote
customization point objects.
Let \exposid{just-cpo} be one of
\tcode{just}, \tcode{just_error}, or \tcode{just_stopped}.
For a pack of subexpressions \tcode{ts},
let \tcode{Ts} be the pack of types \tcode{decltype((ts))}.
The expression \tcode{\exposid{just-cpo}(ts...)} is ill-formed if
\begin{itemize}
\item
\tcode{(\exposconcept{movable-value}<Ts> \&\&...)} is \tcode{false}, or
\item
\exposid{just-cpo} is \tcode{just_error} and
\tcode{sizeof...(ts) == 1} is \tcode{false}, or
\item
\exposid{just-cpo} is \tcode{just_stopped} and
\tcode{sizeof...(ts) == 0} is \tcode{false}.
\end{itemize}

Otherwise, it is expression-equivalent to
\tcode{\exposid{make-sender}(\exposid{just-cpo}, \exposid{product-type}\{ts...\})}.

For \tcode{just}, \tcode{just_error}, and \tcode{just_stopped},
let \exposid{set-cpo} be
\tcode{set_value}, \tcode{set_error}, and \tcode{set_stopped}, respectively.
The exposition-only class template \exposid{impls-for}\iref{exec.snd.general}
is specialized for \exposid{just-cpo} as follows:
\begin{codeblock}
namespace std::execution {
  template<>
  struct @\exposid{impls-for}@<@\exposid{decayed-typeof}@<@\exposid{just-cpo}@>> : @\exposid{default-impls}@ {
    static constexpr auto @\exposid{start}@ =
      [](auto& state, auto& rcvr) noexcept -> void {
        auto& [...ts] = state;
        @\exposid{set-cpo}@(std::move(rcvr), std::move(ts)...);
      };
  };
}
\end{codeblock}

\rSec3[exec.read.env]{\tcode{execution::read_env}}

\pnum
\tcode{read_env} is a sender factory for a sender
whose asynchronous operation completes synchronously in its start operation
with a value completion result equal to
a value read from the receiver's associated environment.

\pnum
\tcode{read_env} is a customization point object.
For some query object \tcode{q},
the expression \tcode{read_env(q)} is expression-equivalent to
\tcode{\exposid{make-sender}(read_env, q)}.

\pnum
The exposition-only class template \exposid{impls-for}\iref{exec.snd.general}
is specialized for \tcode{read_env} as follows:
\begin{codeblock}
namespace std::execution {
  template<>
  struct @\exposid{impls-for}@<@\exposid{decayed-typeof}@<read_env>> : @\exposid{default-impls}@ {
    static constexpr auto start =
      [](auto query, auto& rcvr) noexcept -> void {
        @\exposid{TRY-SET-VALUE}@(rcvr, query(get_env(rcvr)));
      };
  };
}
\end{codeblock}

\rSec2[exec.adapt]{Sender adaptors}

\rSec3[exec.adapt.general]{General}

\pnum
Subclause \ref{exec.adapt} specifies a set of sender adaptors.

\pnum
The bitwise inclusive \logop{or} operator is overloaded
for the purpose of creating sender chains.
The adaptors also support function call syntax with equivalent semantics.

\pnum
Unless otherwise specified:
\begin{itemize}
\item
A sender adaptor is prohibited from causing observable effects,
apart from moving and copying its arguments,
before the returned sender is connected with a receiver using \tcode{connect},
and \tcode{start} is called on the resulting operation state.
\item
A parent sender\iref{exec.async.ops} with a single child sender \tcode{sndr} has
an associated attribute object equal to
\tcode{\exposid{FWD-ENV}(get_env(sndr))}\iref{exec.fwd.env}.
\item
A parent sender with more than one child sender has
an associated attributes object equal to \tcode{env<>\{\}}.
\item
When a parent sender is connected to a receiver \tcode{rcvr},
any receiver used to connect a child sender has
an associated environment equal to \tcode{\exposid{FWD-ENV}(get_env(rcvr))}.
\item
These requirements apply to any function
that is selected by the implementation of the sender adaptor.
\end{itemize}

\pnum
If a sender returned from a sender adaptor specified in \ref{exec.adapt}
is specified to include \tcode{set_error_t(Err)}
among its set of completion signatures
where \tcode{decay_t<Err>} denotes the type \tcode{exception_ptr},
but the implementation does not potentially evaluate
an error completion operation with an \tcode{exception_ptr} argument,
the implementation is allowed to omit
the \tcode{exception_ptr} error completion signature from the set.

\rSec3[exec.adapt.obj]{Closure objects}

\pnum
A \defnadj{pipeable}{sender adaptor closure object} is a function object
that accepts one or more \libconcept{sender} arguments and returns a \libconcept{sender}.
For a pipeable sender adaptor closure object \tcode{c} and
an expression \tcode{sndr}
such that \tcode{decltype((sndr))} models \libconcept{sender},
the following expressions are equivalent and yield a \libconcept{sender}:
\begin{codeblock}
c(sndr)
sndr | c
\end{codeblock}
Given an additional pipeable sender adaptor closure object \tcode{d},
the expression \tcode{c | d} produces
another pipeable sender adaptor closure object \tcode{e}:

\tcode{e} is a perfect forwarding call wrapper\iref{func.require}
with the following properties:
\begin{itemize}
\item
Its target object is an object \tcode{d2} of type \tcode{decltype(auto(d))}
direct-non-list-initialized with \tcode{d}.
\item
It has one bound argument entity,
an object \tcode{c2} of type \tcode{decltype(auto(c))}
direct-non-list-initialized with \tcode{c}.
\item
Its call pattern is \tcode{d2(c2(arg))},
where arg is the argument used in a function call expression of \tcode{e}.
\end{itemize}
The expression \tcode{c | d} is well-formed if and only if
the initializations of the state entities\iref{func.def} of \tcode{e}
are all well-formed.

\pnum
An object \tcode{t} of type \tcode{T} is
a pipeable sender adaptor closure object
if \tcode{T} models \tcode{\libconcept{derived_from}<sender_adaptor_closure<T>>},
\tcode{T} has no other base classes
of type \tcode{sender_adaptor_closure<U>} for any other type \tcode{U}, and
\tcode{T} does not satisfy \libconcept{sender}.

\pnum
The template parameter \tcode{D} for \tcode{sender_adaptor_closure} can be
an incomplete type.
Before any expression of type \cv{} \tcode{D} appears as
an operand to the \tcode{|} operator,
\tcode{D} shall be complete and
model \tcode{\libconcept{derived_from}<sender_adaptor_closure<D>>}.
The behavior of an expression involving an object of type \cv{} \tcode{D}
as an operand to the \tcode{|} operator is undefined
if overload resolution selects a program-defined \tcode{operator|} function.

\pnum
A \defnadj{pipeable}{sender adaptor object} is a customization point object
that accepts a \libconcept{sender} as its first argument and
returns a \libconcept{sender}.
If a pipeable sender adaptor object accepts only one argument,
then it is a pipeable sender adaptor closure object.

\pnum
If a pipeable sender adaptor object adaptor accepts more than one argument,
then let \tcode{sndr} be an expression
such that \tcode{decltype((sndr))} models \libconcept{sender},
let \tcode{args...} be arguments
such that \tcode{adaptor(sndr, args...)} is a well-formed expression
as specified below, and
let \tcode{BoundArgs} be a pack that denotes \tcode{decltype(auto(args))...}.
The expression \tcode{adaptor(args...)} produces
a pipeable sender adaptor closure object \tcode{f}
that is a perfect forwarding call wrapper with the following properties:
\begin{itemize}
\item
Its target object is a copy of adaptor.
\item
Its bound argument entities \tcode{bound_args} consist of
objects of types \tcode{BoundArgs...} direct-non-list-initialized with
\tcode{std::forward<decltype((args))>(args)...}, respectively.
\item
Its call pattern is \tcode{adaptor(rcvr, bound_args...)},
where \tcode{rcvr} is
the argument used in a function call expression of \tcode{f}.
\end{itemize}
The expression \tcode{adaptor(args...)} is well-formed if and only if
the initializations of the bound argument entities of the result,
as specified above, are all well-formed.

\rSec3[exec.starts.on]{\tcode{execution::starts_on}}

\pnum
\tcode{starts_on} adapts an input sender into a sender
that will start on an execution agent belonging to
a particular scheduler's associated execution resource.

\pnum
The name \tcode{starts_on} denotes a customization point object.
For subexpressions \tcode{sch} and \tcode{sndr},
if \tcode{decltype((\newline sch))} does not satisfy \libconcept{scheduler}, or
\tcode{decltype((sndr))} does not satisfy \libconcept{sender},
\tcode{starts_on(sch, sndr)} is ill-formed.

\pnum
Otherwise,
the expression \tcode{starts_on(sch, sndr)} is expression-equivalent to:
\begin{codeblock}
transform_sender(
  @\exposid{query-or-default}@(get_domain, sch, default_domain()),
  @\exposid{make-sender}@(starts_on, sch, sndr))
\end{codeblock}
except that \tcode{sch} is evaluated only once.

\pnum
Let \tcode{out_sndr} and \tcode{env} be subexpressions
such that \tcode{OutSndr} is \tcode{decltype((out_sndr))}.
If \tcode{\exposconcept{sender-for}<Out\-Sndr, starts_on_t>} is \tcode{false},
then the expressions \tcode{starts_on.transform_env(out_sndr, env)} and\linebreak
\tcode{starts_on.transform_sender(out_sndr, env)} are ill-formed; otherwise
\begin{itemize}
\item
\tcode{starts_on.transform_env(out_sndr, env)} is equivalent to:
\begin{codeblock}
auto&& [_, sch, _] = out_sndr;
return @\exposid{JOIN-ENV}@(@\exposid{SCHED-ENV}@(sch), @\exposid{FWD-ENV}@(env));
\end{codeblock}
\item
\tcode{starts_on.transform_sender(out_sndr, env)} is equivalent to:
\begin{codeblock}
auto&& [_, sch, sndr] = out_sndr;
return let_value(
  schedule(sch),
  [sndr = std::forward_like<OutSndr>(sndr)]() mutable
    noexcept(is_nothrow_move_constructible_v<decay_t<OutSndr>>) {
    return std::move(sndr);
  });
\end{codeblock}
\end{itemize}

\pnum
Let \tcode{out_sndr} be a subexpression denoting
a sender returned from \tcode{starts_on(sch, sndr)} or one equal to such, and
let \tcode{OutSndr} be the type \tcode{decltype((out_sndr))}.
Let \tcode{out_rcvr} be a subexpression denoting a receiver
that has an environment of type \tcode{Env}
such that \tcode{\libconcept{sender_in}<OutSndr, Env>} is \tcode{true}.
Let \tcode{op} be an lvalue referring to the operation state
that results from connecting \tcode{out_sndr} with \tcode{out_rcvr}.
Calling \tcode{start(op)} shall start \tcode{sndr}
on an execution agent of the associated execution resource of \tcode{sch}.
If scheduling onto \tcode{sch} fails,
an error completion on \tcode{out_rcvr} shall be executed
on an unspecified execution agent.

\rSec3[exec.continues.on]{\tcode{execution::continues_on}}

\pnum
\tcode{continues_on} adapts a sender into one
that completes on the specified scheduler.

\pnum
The name \tcode{continues_on} denotes a pipeable sender adaptor object.
For subexpressions \tcode{sch} and \tcode{sndr},
if \tcode{decltype((sch))} does not satisfy \libconcept{scheduler}, or
\tcode{decltype((sndr))} does not satisfy \libconcept{sender},
\tcode{continues_on(sndr, sch)} is ill-formed.

\pnum
Otherwise,
the expression \tcode{continues_on(sndr, sch)} is expression-equivalent to:
\begin{codeblock}
transform_sender(@\exposid{get-domain-early}@(sndr), @\exposid{make-sender}@(continues_on, sch, sndr))
\end{codeblock}
except that \tcode{sndr} is evaluated only once.

\pnum
The exposition-only class template \exposid{impls-for}
is specialized for \tcode{continues_on_t} as follows:
\begin{codeblock}
namespace std::execution {
  template<>
  struct @\exposid{impls-for}@<continues_on_t> : @\exposid{default-impls}@ {
    static constexpr auto @\exposid{get-attrs}@ =
      [](const auto& data, const auto& child) noexcept -> decltype(auto) {
        return @\exposid{JOIN-ENV}@(@\exposid{SCHED-ATTRS}@(data), @\exposid{FWD-ENV}@(get_env(child)));
      };
  };
}
\end{codeblock}

\pnum
Let \tcode{sndr} and \tcode{env} be subexpressions
such that \tcode{Sndr} is \tcode{decltype((sndr))}.
If \tcode{\exposconcept{sender-for}<Sndr, continues\-_on_t>} is \tcode{false},
then
the expression \tcode{continues_on.transform_sender(sndr, env)} is ill-formed;
otherwise, it is equal to:
\begin{codeblock}
auto [_, data, child] = sndr;
return schedule_from(std::move(data), std::move(child));
\end{codeblock}
\begin{note}
This causes the \tcode{continues_on(sndr, sch)} sender to become
\tcode{schedule_from(sch, sndr)} when it is connected with a receiver
whose execution domain does not customize \tcode{continues_on}.
\end{note}

\pnum
Let \tcode{out_sndr} be a subexpression denoting
a sender returned from \tcode{continues_on(sndr, sch)} or one equal to such, and
let \tcode{OutSndr} be the type \tcode{decltype((out_sndr))}.
Let \tcode{out_rcvr} be a subexpression denoting a receiver
that has an environment of type \tcode{Env}
such that \tcode{\libconcept{sender_in}<OutSndr, Env>} is \tcode{true}.
Let \tcode{op} be an lvalue referring to the operation state
that results from connecting \tcode{out_sndr} with \tcode{out_rcvr}.
Calling \tcode{start(op)} shall
start \tcode{sndr} on the current execution agent and
execute completion operations on \tcode{out_rcvr}
on an execution agent of the execution resource associated with \tcode{sch}.
If scheduling onto \tcode{sch} fails,
an error completion on \tcode{out_rcvr} shall be executed
on an unspecified execution agent.

\rSec3[exec.schedule.from]{\tcode{execution::schedule_from}}

\pnum
\tcode{schedule_from} schedules work dependent on the completion of a sender
onto a scheduler's associated execution resource.
\begin{note}
\tcode{schedule_from} is not meant to be used in user code;
it is used in the implementation of \tcode{continues_on}.
\end{note}

\pnum
The name \tcode{schedule_from} denotes a customization point object.
For some subexpressions \tcode{sch} and \tcode{sndr},
let \tcode{Sch} be \tcode{decltype((sch))} and
\tcode{Sndr} be \tcode{decltype((sndr))}.
If \tcode{Sch} does not satisfy \libconcept{scheduler}, or
\tcode{Sndr} does not satisfy \libconcept{sender},
\tcode{schedule_from(sch, sndr)} is ill-formed.

\pnum
Otherwise,
the expression \tcode{schedule_from(sch, sndr)} is expression-equivalent to:
\begin{codeblock}
transform_sender(
  @\exposid{query-or-default}@(get_domain, sch, default_domain()),
  @\exposid{make-sender}@(schedule_from, sch, sndr))
\end{codeblock}
except that \tcode{sch} is evaluated only once.

\pnum
The exposition-only class template \exposid{impls-for}\iref{exec.snd.general}
is specialized for \tcode{schedule_from_t} as follows:
\begin{codeblock}
namespace std::execution {
  template<>
  struct @\exposid{impls-for}@<schedule_from_t> : @\exposid{default-impls}@ {
    static constexpr auto @\exposid{get-attrs}@ = @\seebelow@;
    static constexpr auto @\exposid{get-state}@ = @\seebelow@;
    static constexpr auto @\exposid{complete}@ = @\seebelow@;
  };
}
\end{codeblock}

\pnum
The member \tcode{\exposid{impls-for}<schedule_from_t>::\exposid{get-attrs}}
is initialized with a callable object equivalent to the following lambda:
\begin{codeblock}
[](const auto& data, const auto& child) noexcept -> decltype(auto) {
  return @\exposid{JOIN-ENV}@(@\exposid{SCHED-ATTRS}@(data), @\exposid{FWD-ENV}@(get_env(child)));
}
\end{codeblock}

\pnum
The member \tcode{\exposid{impls-for}<schedule_from_t>::\exposid{get-state}}
is initialized with a callable object equivalent to the following lambda:
\begin{codeblock}
[]<class Sndr, class Rcvr>(Sndr&& sndr, Rcvr& rcvr) noexcept(@\seebelow@)
    requires @\libconcept{sender_in}@<@\exposid{child-type}@<Sndr>, env_of_t<Rcvr>> {

  auto& [_, sch, child] = sndr;

  using sched_t = decltype(auto(sch));
  using variant_t = @\seebelow@;
  using receiver_t = @\seebelow@;
  using operation_t = connect_result_t<schedule_result_t<sched_t>, receiver_t>;
  constexpr bool nothrow = noexcept(connect(schedule(sch), receiver_t{nullptr}));

  struct @\exposid{state-type}@ {
    Rcvr& @\exposid{rcvr}@;                 // \expos
    variant_t @\exposid{async-result}@;     // \expos
    operation_t @\exposid{op-state}@;       // \expos

    explicit @\exposid{state-type}@(sched_t sch, Rcvr& rcvr) noexcept(nothrow)
      : @\exposid{rcvr}@(rcvr), @\exposid{op-state}@(connect(schedule(sch), receiver_t{this})) {}
  };

  return @\exposid{state-type}@{sch, rcvr};
}
\end{codeblock}

\pnum
Objects of the local class \exposid{state-type} can be used
to initialize a structured binding.

\pnum
Let \tcode{Sigs} be
a pack of the arguments to the \tcode{completion_signatures} specialization
named by \tcode{completion_signatures_of_t<child-type<Sndr>, env_of_t<Rcvr>>}.
Let \exposid{as-tuple} be an alias template
that transforms a completion signature \tcode{Tag(Args...)} into
the tuple specialization \tcode{\exposid{decayed-tuple}<Tag, Args...>}.
Then \tcode{variant_t} denotes
the type \tcode{variant<monostate, \exposid{as-tuple}<Sigs>...>},
except with duplicate types removed.

\pnum
\tcode{receiver_t} is an alias for the following exposition-only class:
\begin{codeblock}
namespace std::execution {
  struct @\exposid{receiver-type}@ {
    using receiver_concept = receiver_t;
    @\exposid{state-type}@* @\exposid{state}@;          // \expos

    void set_value() && noexcept {
      visit(
        [this]<class Tuple>(Tuple& result) noexcept -> void {
          if constexpr (!@\libconcept{same_as}@<monostate, Tuple>) {
            auto& [tag, ...args] = result;
            tag(std::move(@\exposid{state}@->@\exposid{rcvr}@), std::move(args)...);
          }
        },
        @\exposid{state}@->@\exposid{async-result}@);
    }

    template<class Error>
    void set_error(Error&& err) && noexcept {
      execution::set_error(std::move(@\exposid{state}@->@\exposid{rcvr}@), std::forward<Error>(err));
    }

    void set_stopped() && noexcept {
      execution::set_stopped(std::move(@\exposid{state}@->@\exposid{rcvr}@));
    }

    decltype(auto) get_env() const noexcept {
      return @\exposid{FWD-ENV}@(execution::get_env(@\exposid{state}@->@\exposid{rcvr}@));
    }
  };
}
\end{codeblock}

\pnum
The expression in the \tcode{noexcept} clause of the lambda is \tcode{true}
if the construction of the returned \exposid{state-type} object
is not potentially throwing;
otherwise, \tcode{false}.

\pnum
The member \tcode{\exposid{impls-for}<schedule_from_t>::\exposid{complete}}
is initialized with a callable object equivalent to the following lambda:
\begin{codeblock}
[]<class Tag, class... Args>(auto, auto& state, auto& rcvr, Tag, Args&&... args) noexcept
    -> void {
  using result_t = @\exposid{decayed-tuple}@<Tag, Args...>;
  constexpr bool nothrow = is_nothrow_constructible_v<result_t, Tag, Args...>;

  try {
    state.@\exposid{async-result}@.template emplace<result_t>(Tag(), std::forward<Args>(args)...);
  } catch (...) {
    if constexpr (!nothrow) {
      set_error(std::move(rcvr), current_exception());
      return;
    }
  }
  start(state.@\exposid{op-state}@);
};
\end{codeblock}

\pnum
Let \tcode{out_sndr} be a subexpression denoting
a sender returned from \tcode{schedule_from(sch, sndr)} or one equal to such,
and let \tcode{OutSndr} be the type \tcode{decltype((out_sndr))}.
Let \tcode{out_rcvr} be a subexpression denoting a receiver
that has an environment of type \tcode{Env}
such that \tcode{\libconcept{sender_in}<OutSndr, Env>} is \tcode{true}.
Let \tcode{op} be an lvalue referring to the operation state
that results from connecting \tcode{out_sndr} with \tcode{out_rcvr}.
Calling \tcode{start(op)} shall
start \tcode{sndr} on the current execution agent and
execute completion operations on \tcode{out_rcvr}
on an execution agent of the execution resource associated with \tcode{sch}.
If scheduling onto \tcode{sch} fails,
an error completion on \tcode{out_rcvr} shall be executed
on an unspecified execution agent.

\rSec3[exec.on]{\tcode{execution::on}}

\pnum
The \tcode{on} sender adaptor has two forms:
\begin{itemize}
\item
\tcode{on(sch, sndr)},
which starts a sender \tcode{sndr} on an execution agent
belonging to a scheduler \tcode{sch}'s associated execution resource and
that, upon \tcode{sndr}'s completion,
transfers execution back to the execution resource
on which the \tcode{on} sender was started.
\item
\tcode{on(sndr, sch, closure)},
which upon completion of a sender \tcode{sndr},
transfers execution to an execution agent
belonging to a scheduler \tcode{sch}'s associated execution resource,
then executes a sender adaptor closure \tcode{closure}
with the async results of the sender, and
that then transfers execution back to the execution resource
on which \tcode{sndr} completed.
\end{itemize}

\pnum
The name \tcode{on} denotes a pipeable sender adaptor object.
For subexpressions \tcode{sch} and \tcode{sndr},
\tcode{on(sch, sndr)} is ill-formed if any of the following is \tcode{true}:
\begin{itemize}
\item
\tcode{decltype((sch))} does not satisfy \libconcept{scheduler}, or
\item
\tcode{decltype((sndr))} does not satisfy \libconcept{sender} and
\tcode{sndr} is not
a pipeable sender adaptor closure object\iref{exec.adapt.obj}, or
\item
\tcode{decltype((sndr))} satisfies \libconcept{sender} and
\tcode{sndr }is also a pipeable sender adaptor closure object.
\end{itemize}

\pnum
Otherwise, if \tcode{decltype((sndr))} satisfies \libconcept{sender},
the expression \tcode{on(sch, sndr)} is expression-equivalent to:
\begin{codeblock}
transform_sender(
  @\exposid{query-or-default}@(get_domain, sch, default_domain()),
  @\exposid{make-sender}@(on, sch, sndr))
\end{codeblock}
except that \tcode{sch} is evaluated only once.

\pnum
For subexpressions \tcode{sndr}, \tcode{sch}, and \tcode{closure}, if
\begin{itemize}
\item
\tcode{decltype((sch))} does not satisfy \libconcept{scheduler}, or
\item
\tcode{decltype((sndr))} does not satisfy \libconcept{sender}, or
\item
\tcode{closure} is not a pipeable sender adaptor closure object\iref{exec.adapt.obj},
\end{itemize}
the expression \tcode{on(sndr, sch, closure)} is ill-formed;
otherwise, it is expression-equivalent to:
\begin{codeblock}
transform_sender(
  @\exposid{get-domain-early}@(sndr),
  @\exposid{make-sender}@(on, @\exposid{product-type}@{sch, closure}, sndr))
\end{codeblock}
except that \tcode{sndr} is evaluated only once.

\pnum
Let \tcode{out_sndr} and \tcode{env} be subexpressions,
let \tcode{OutSndr} be \tcode{decltype((out_sndr))}, and
let \tcode{Env} be \tcode{decltype((\linebreak env))}.
If \tcode{\exposconcept{sender-for}<OutSndr, on_t>} is \tcode{false},
then the expressions \tcode{on.transform_env(out_sndr, env)} and
\tcode{on.transform_sender(out_sndr, env)} are ill-formed.

\pnum
Otherwise:
Let \exposid{not-a-scheduler} be an unspecified empty class type, and
let \exposid{not-a-sender} be the exposition-only type:
\begin{codeblock}
struct @\exposid{not-a-sender}@ {
  using sender_concept = sender_t;

  auto get_completion_signatures(auto&&) const {
    return @\seebelow@;
  }
};
\end{codeblock}
where the member function \tcode{get_completion_signatures} returns
an object of a type that is not
a specialization of the \tcode{completion_signatures} class template.

\pnum
The expression \tcode{on.transform_env(out_sndr, env)}
has effects equivalent to:
\begin{codeblock}
auto&& [_, data, _] = out_sndr;
if constexpr (@\libconcept{scheduler}@<decltype(data)>) {
  return @\exposid{JOIN-ENV}@(@\exposid{SCHED-ENV}@(std::forward_like<OutSndr>(data)), @\exposid{FWD-ENV}@(std::forward<Env>(env)));
} else {
  return std::forward<Env>(env);
}
\end{codeblock}

\pnum
The expression \tcode{on.transform_sender(out_sndr, env)}
has effects equivalent to:
\begin{codeblock}
auto&& [_, data, child] = out_sndr;
if constexpr (@\libconcept{scheduler}@<decltype(data)>) {
  auto orig_sch =
    @\exposid{query-with-default}@(get_scheduler, env, @\exposid{not-a-scheduler}@());

  if constexpr (@\libconcept{same_as}@<decltype(orig_sch), @\exposid{not-a-scheduler}@>) {
    return @\exposid{not-a-sender}@{};
  } else {
    return continues_on(
      starts_on(std::forward_like<OutSndr>(data), std::forward_like<OutSndr>(child)),
      std::move(orig_sch));
  }
} else {
  auto& [sch, closure] = data;
  auto orig_sch = @\exposid{query-with-default}@(
    get_completion_scheduler<set_value_t>,
    get_env(child),
    @\exposid{query-with-default}@(get_scheduler, env, @\exposid{not-a-scheduler}@()));

  if constexpr (@\libconcept{same_as}@<decltype(orig_sch), @\exposid{not-a-scheduler}@>) {
    return @\exposid{not-a-sender}@{};
  } else {
    return @\exposid{write-env}@(
      continues_on(
        std::forward_like<OutSndr>(closure)(
          continues_on(
            @\exposid{write-env}@(std::forward_like<OutSndr>(child), @\exposid{SCHED-ENV}@(orig_sch)),
            sch)),
        orig_sch),
      @\exposid{SCHED-ENV}@(sch));
  }
}
\end{codeblock}

\pnum
\recommended
Implementations should use
the return type of \tcode{\exposid{not-a-sender}::get_completion_signatures}
to inform users that their usage of \tcode{on} is incorrect
because there is no available scheduler onto which to restore execution.

\pnum
Let \tcode{out_sndr} be a subexpression denoting
a sender returned from \tcode{on(sch, sndr)} or one equal to such, and
let \tcode{OutSndr} be the type \tcode{decltype((out_sndr))}.
Let \tcode{out_rcvr} be a subexpression denoting a receiver
that has an environment of type \tcode{Env}
such that \tcode{\libconcept{sender_in}<OutSndr, Env>} is \tcode{true}.
Let \tcode{op} be an lvalue referring to the operation state
that results from connecting \tcode{out_sndr} with \tcode{out_rcvr}.
Calling \tcode{start(op)} shall
\begin{itemize}
\item
remember the current scheduler, \tcode{get_scheduler(get_env(rcvr))};
\item
start \tcode{sndr} on an execution agent belonging to
sch's associated execution resource;
\item
upon \tcode{sndr}'s completion,
transfer execution back to the execution resource
associated with the scheduler remembered in step 1; and
\item
forward \tcode{sndr}'s async result to \tcode{out_rcvr}.
\end{itemize}
If any scheduling operation fails,
an error completion on \tcode{out_rcvr} shall be executed
on an unspecified execution agent.

\pnum
Let \tcode{out_sndr} be a subexpression denoting
a sender returned from \tcode{on(sndr, sch, closure)} or one equal to such, and
let \tcode{OutSndr} be the type \tcode{decltype((out_sndr))}.
Let \tcode{out_rcvr} be a subexpression denoting a receiver
that has an environment of type \tcode{Env}
such that \tcode{\libconcept{sender_in}<OutSndr, Env>} is \tcode{true}.
Let \tcode{op} be an lvalue referring to the operation state
that results from connecting \tcode{out_sndr} with \tcode{out_rcvr}.
Calling \tcode{start(op)} shall
\begin{itemize}
\item
remember the current scheduler,
which is the first of the following expressions that is well-formed:
\begin{itemize}
\item \tcode{get_completion_scheduler<set_value_t>(get_env(sndr))}
\item \tcode{get_scheduler(get_env(rcvr))};
\end{itemize}
\item
start \tcode{sndr} on the current execution agent;
\item
upon \tcode{sndr}'s completion,
transfer execution to an agent
owned by sch's associated execution resource;
\item
forward \tcode{sndr}'s async result as if by
connecting and starting a sender \tcode{closure(S)},
where \tcode{S} is a sender
that completes synchronously with \tcode{sndr}'s async result; and
\item
upon completion of the operation started in the previous step,
transfer execution back to the execution resource
associated with the scheduler remembered in step 1 and
forward the operation's async result to \tcode{out_rcvr}.
\end{itemize}
If any scheduling operation fails,
an error completion on \tcode{out_rcvr} shall be executed on
an unspecified execution agent.

\rSec3[exec.then]{\tcode{execution::then}, \tcode{execution::upon_error}, \tcode{execution::upon_stopped}}

\pnum
\tcode{then} attaches an invocable as a continuation
for an input sender's value completion operation.
\tcode{upon_error} and \tcode{upon_stopped} do the same
for the error and stopped completion operations, respectively,
sending the result of the invocable as a value completion.

\pnum
The names \tcode{then}, \tcode{upon_error}, and \tcode{upon_stopped}
denote pipeable sender adaptor objects.
Let the expression \exposid{then-cpo} be one of
\tcode{then}, \tcode{upon_error}, or \tcode{upon_stopped}.
For subexpressions \tcode{sndr} and \tcode{f},
if \tcode{decltype((sndr))} does not satisfy \libconcept{sender}, or
\tcode{decltype((f))} does not satisfy \exposconcept{movable-value},
\tcode{\exposid{then-cpo}(\linebreak sndr, f) }is ill-formed.

\pnum
Otherwise,
the expression \tcode{\exposid{then-cpo}(sndr, f)} is expression-equivalent to:
\begin{codeblock}
transform_sender(@\exposid{get-domain-early}@(sndr), @\exposid{make-sender}@(@\exposid{then-cpo}@, f, sndr))
\end{codeblock}
except that \tcode{sndr} is evaluated only once.

\pnum
For \tcode{then}, \tcode{upon_error}, and \tcode{upon_stopped},
let \exposid{set-cpo} be
\tcode{set_value}, \tcode{set_error}, and \tcode{set_stopped}, respectively.
The exposition-only class template \exposid{impls-for}\iref{exec.snd.general}
is specialized for \exposid{then-cpo} as follows:
\begin{codeblock}
namespace std::execution {
  template<>
  struct @\exposid{impls-for}@<@\exposid{decayed-typeof}@<@\exposid{then-cpo}@>> : @\exposid{default-impls}@ {
    static constexpr auto @\exposid{complete}@ =
      []<class Tag, class... Args>
        (auto, auto& fn, auto& rcvr, Tag, Args&&... args) noexcept -> void {
          if constexpr (@\libconcept{same_as}@<Tag, @\exposid{decayed-typeof}@<set-cpo>>) {
            @\exposid{TRY-SET-VALUE}@(rcvr,
                          invoke(std::move(fn), std::forward<Args>(args)...));
          } else {
            Tag()(std::move(rcvr), std::forward<Args>(args)...);
          }
        };
  };
}
\end{codeblock}

\pnum
The expression \tcode{\exposid{then-cpo}(sndr, f)} has undefined behavior
unless it returns a sender \tcode{out_sndr} that
\begin{itemize}
\item
invokes \tcode{f} or a copy of such
with the value, error, or stopped result datums of \tcode{sndr}
for \tcode{then}, \tcode{upon_error}, and \tcode{upon_stopped}, respectively,
using the result value of \tcode{f} as \tcode{out_sndr}'s value completion, and
\item
forwards all other completion operations unchanged.
\end{itemize}

\rSec3[exec.let]{\tcode{execution::let_value}, \tcode{execution::let_error}, \tcode{execution::let_stopped}}

\pnum
\tcode{let_value}, \tcode{let_error}, and \tcode{let_stopped} transform
a sender's value, error, and stopped completions, respectively,
into a new child asynchronous operation
by passing the sender's result datums to a user-specified callable,
which returns a new sender that is connected and started.

\pnum
For \tcode{let_value}, \tcode{let_error}, and \tcode{let_stopped},
let \exposid{set-cpo} be
\tcode{set_value}, \tcode{set_error}, and \tcode{set_stopped}, respectively.
Let the expression \exposid{let-cpo} be one of
\tcode{let_value}, \tcode{let_error}, or \tcode{let_stopped}.
For a subexpression \tcode{sndr},
let \tcode{\exposid{let-env}(sndr)} be expression-equivalent to
the first well-formed expression below:
\begin{itemize}
\item
\tcode{\exposid{SCHED-ENV}(get_completion_scheduler<\exposid{decayed-typeof}<\exposid{set-cpo}>>(get_env(sndr)))}
\item
\tcode{\exposid{MAKE-ENV}(get_domain, get_domain(get_env(sndr)))}
\item
\tcode{(void(sndr), env<>\{\})}
\end{itemize}

\pnum
The names \tcode{let_value}, \tcode{let_error}, and \tcode{let_stopped} denote
pipeable sender adaptor objects.
For subexpressions \tcode{sndr} and \tcode{f},
let \tcode{F} be the decayed type of \tcode{f}.
If \tcode{decltype((sndr))} does not satisfy \libconcept{sender} or
if \tcode{decltype((f))} does not satisfy \exposconcept{movable-value},
the expression \tcode{\exposid{let-cpo}(sndr, f)} is ill-formed.
If \tcode{F} does not satisfy \libconcept{invocable},
the expression \tcode{let_stopped(sndr, f)} is ill-formed.

\pnum
Otherwise,
the expression \tcode{\exposid{let-cpo}(sndr, f)} is expression-equivalent to:
\begin{codeblock}
transform_sender(@\exposid{get-domain-early}@(sndr), @\exposid{make-sender}@(@\exposid{let-cpo}@, f, sndr))
\end{codeblock}
except that \tcode{sndr} is evaluated only once.

\pnum
The exposition-only class template \exposid{impls-for}\iref{exec.snd.general}
is specialized for \exposid{let-cpo} as follows:
\begin{codeblock}
namespace std::execution {
  template<class State, class Rcvr, class... Args>
  void @\exposid{let-bind}@(State& state, Rcvr& rcvr, Args&&... args);      // \expos

  template<>
  struct @\exposid{impls-for}@<@\exposid{decayed-typeof}@<@\exposid{let-cpo}@>> : @\exposid{default-impls}@ {
    static constexpr auto @\exposid{get-state}@ = @\seebelow@;
    static constexpr auto @\exposid{complete}@ = @\seebelow@;
  };
}
\end{codeblock}

\pnum
Let \exposid{receiver2} denote the following exposition-only class template:
\begin{codeblock}
namespace std::execution {
  template<class Rcvr, class Env>
  struct @\exposid{receiver2}@ {
    using receiver_concept = receiver_t;

    template<class... Args>
    void set_value(Args&&... args) && noexcept {
      execution::set_value(std::move(@\exposid{rcvr}@), std::forward<Args>(args)...);
    }

    template<class Error>
    void set_error(Error&& err) && noexcept {
      execution::set_error(std::move(@\exposid{rcvr}@), std::forward<Error>(err));
    }

    void set_stopped() && noexcept {
      execution::set_stopped(std::move(@\exposid{rcvr}@));
    }

    decltype(auto) get_env() const noexcept {
      return @\seebelow@;
    }

    Rcvr& @\exposid{rcvr}@;                 // \expos
    Env @\exposid{env}@;                    // \expos
  };
}
\end{codeblock}
Invocation of the function \tcode{\exposid{receiver2}::get_env}
returns an object \tcode{e} such that
\begin{itemize}
\item
\tcode{decltype(e)} models \exposconcept{queryable} and
\item
given a query object \tcode{q},
the expression \tcode{e.query(q)} is expression-equivalent
to \tcode{\exposid{env}.query(q)} if that expression is valid,
otherwise \tcode{e.query(q)} is expression-equivalent
to \tcode{get_env(\exposid{rcvr}).query(q)}.
\end{itemize}

\pnum
\tcode{\exposid{impls-for}<\exposid{decayed-typeof}<\exposid{let-cpo}>>::\exposid{get-state}}
is initialized with a callable object equivalent to the following:
\begin{codeblock}
[]<class Sndr, class Rcvr>(Sndr&& sndr, Rcvr& rcvr) requires @\seebelow@ {
  auto& [_, fn, child] = sndr;
  using fn_t = decay_t<decltype(fn)>;
  using env_t = decltype(@\exposid{let-env}@(child));
  using args_variant_t = @\seebelow@;
  using ops2_variant_t = @\seebelow@;

  struct @\exposid{state-type}@ {
    fn_t @\exposid{fn}@;                    // \expos
    env_t @\exposid{env}@;                  // \expos
    args_variant_t @\exposid{args}@;        // \expos
    ops2_variant_t @\exposid{ops2}@;        // \expos
  };
  return @\exposid{state-type}@{std::forward_like<Sndr>(fn), @\exposid{let-env}@(child), {}, {}};
}
\end{codeblock}

\pnum
Let \tcode{Sigs} be a pack of the arguments
to the \tcode{completion_signatures} specialization named by
\tcode{completion_signatures_of_t<\exposid{child-type}<Sndr>, env_of_t<Rcvr>>}.
Let \tcode{LetSigs} be a pack of those types in \tcode{Sigs}
with a return type of \tcode{\exposid{decayed-typeof}<\exposid{set-cpo}>}.
Let \exposid{as-tuple} be an alias template
such that \tcode{\exposid{as-tuple}<\linebreak Tag(Args...)>} denotes
the type \tcode{\exposid{decayed-tuple}<Args...>}.
Then \tcode{args_variant_t} denotes
the type \tcode{variant<monostate, \exposid{as-tuple}<LetSigs>...>}
except with duplicate types removed.

\pnum
Given a type \tcode{Tag} and a pack \tcode{Args},
let \exposid{as-sndr2} be an alias template
such that \tcode{\exposid{as-sndr2}<Tag(Args...)>} denotes
the type \tcode{\exposid{call-result-t}<Fn, decay_t<Args>\&...>}.
Then \tcode{ops2_variant_t} denotes
the type
\begin{codeblock}
variant<monostate, connect_result_t<@\exposid{as-sndr2}@<LetSigs>, @\exposid{receiver2}@<Rcvr, Env>>...>
\end{codeblock}
except with duplicate types removed.

\pnum
The \grammarterm{requires-clause} constraining the above lambda is satisfied
if and only if
the types \tcode{args_variant_t} and \tcode{ops2_variant_t} are well-formed.

\pnum
The exposition-only function template \exposid{let-bind}
has effects equivalent to:
\begin{codeblock}
using args_t = @\exposid{decayed-tuple}@<Args...>;
auto mkop2 = [&] {
  return connect(
    apply(std::move(state.fn),
          state.args.template emplace<args_t>(std::forward<Args>(args)...)),
    @\exposid{receiver2}@{rcvr, std::move(state.env)});
};
start(state.ops2.template emplace<decltype(mkop2())>(@\exposid{emplace-from}@{mkop2}));
\end{codeblock}

\pnum
\tcode{\exposid{impls-for}<\exposid{decayed-typeof}<let-cpo>>::\exposid{complete}}
is initialized with a callable object equivalent to the following:
\begin{codeblock}
[]<class Tag, class... Args>
  (auto, auto& state, auto& rcvr, Tag, Args&&... args) noexcept -> void {
    if constexpr (@\libconcept{same_as}@<Tag, @\exposid{decayed-typeof}@<@\exposid{set-cpo}@>>) {
      @\exposid{TRY-EVAL}@(rcvr, @\exposid{let-bind}@(state, rcvr, std::forward<Args>(args)...));
    } else {
      Tag()(std::move(rcvr), std::forward<Args>(args)...);
    }
  }
\end{codeblock}

\pnum
Let \tcode{sndr} and \tcode{env} be subexpressions, and
let \tcode{Sndr} be \tcode{decltype((sndr))}.
If
\tcode{\exposconcept{sender-for}<Sndr, \exposid{decayed-\linebreak typeof}<\exposid{let-cpo}>>}
is \tcode{false},
then the expression \tcode{\exposid{let-cpo}.transform_env(sndr, env)}
is ill-formed.
Otherwise, it is equal to
\tcode{\exposid{JOIN-ENV}(\exposid{let-env}(sndr), \exposid{FWD-ENV}(env))}.

\pnum
Let the subexpression \tcode{out_sndr} denote
the result of the invocation \tcode{\exposid{let-cpo}(sndr, f)} or
an object equal to such, and
let the subexpression \tcode{rcvr} denote a receiver
such that the expression \tcode{connect(out_sndr, rcvr)} is well-formed.
The expression \tcode{connect(out_sndr, rcvr)} has undefined behavior
unless it creates an asynchronous operation\iref{exec.async.ops} that,
when started:
\begin{itemize}
\item
invokes \tcode{f} when \exposid{set-cpo} is called
with \tcode{sndr}'s result datums,
\item
makes its completion dependent on
the completion of a sender returned by \tcode{f}, and
\item
propagates the other completion operations sent by \tcode{sndr}.
\end{itemize}

\rSec3[exec.bulk]{\tcode{execution::bulk}}

\pnum
\tcode{bulk} runs a task repeatedly for every index in an index space.

The name \tcode{bulk} denotes a pipeable sender adaptor object.
For subexpressions \tcode{sndr}, \tcode{shape}, and \tcode{f},
let \tcode{Shape} be \tcode{decltype(auto(shape))}.
If
\begin{itemize}
\item
\tcode{decltype((sndr))} does not satisfy \libconcept{sender}, or
\item
\tcode{Shape} does not satisfy \libconcept{integral}, or
\item
\tcode{decltype((f))} does not satisfy \exposconcept{movable-value},
\end{itemize}
\tcode{bulk(sndr, shape, f)} is ill-formed.

\pnum
Otherwise,
the expression \tcode{bulk(sndr, shape, f)} is expression-equivalent to:

\begin{codeblock}
transform_sender(@\exposid{get-domain-early}@(sndr), @\exposid{make-sender}@(bulk, @\exposid{product-type}@{shape, f}, sndr))
\end{codeblock}
except that \tcode{sndr} is evaluated only once.

\pnum
The exposition-only class template \exposid{impls-for}\iref{exec.snd.general}
is specialized for \tcode{bulk_t} as follows:
\begin{codeblock}
namespace std::execution {
  template<>
  struct @\exposid{impls-for}@<bulk_t> : @\exposid{default-impls}@ {
    static constexpr auto @\exposid{complete}@ = @\seebelow@;
  };
}
\end{codeblock}

\pnum
The member \tcode{\exposid{impls-for}<bulk_t>::\exposid{complete}}
is initialized with a callable object equivalent to the following lambda:
\begin{codeblock}
[]<class Index, class State, class Rcvr, class Tag, class... Args>
  (Index, State& state, Rcvr& rcvr, Tag, Args&&... args) noexcept -> void requires @\seebelow@ {
    if constexpr (@\libconcept{same_as}@<Tag, set_value_t>) {
      auto& [shape, f] = state;
      constexpr bool nothrow = noexcept(f(auto(shape), args...));
      @\exposid{TRY-EVAL}@(rcvr, [&]() noexcept(nothrow) {
        for (decltype(auto(shape)) i = 0; i < shape; ++i) {
          f(auto(i), args...);
        }
        Tag()(std::move(rcvr), std::forward<Args>(args)...);
      }());
    } else {
      Tag()(std::move(rcvr), std::forward<Args>(args)...);
    }
  }
\end{codeblock}

\pnum
The expression in the \grammarterm{requires-clause} of the lambda above
is \tcode{true} if and only
if \tcode{Tag} denotes a type other than \tcode{set_value_t} or
if the expression \tcode{f(auto(shape), args...)} is well-formed.

\pnum
Let the subexpression \tcode{out_sndr} denote
the result of the invocation \tcode{bulk(sndr, shape, f)} or
an object equal to such, and
let the subexpression \tcode{rcvr} denote a receiver
such that the expression \tcode{connect(out_sndr, rcvr)} is well-formed.
The expression \tcode{connect(out_sndr, rcvr)} has undefined behavior
unless it creates an asynchronous operation\iref{exec.async.ops} that,
when started,
\begin{itemize}
\item
on a value completion operation,
invokes \tcode{f(i, args...)}
for every \tcode{i} of type \tcode{Shape} in \range{\tcode{0}}{\tcode{shape}},
where \tcode{args} is a pack of lvalue subexpressions
referring to the value completion result datums of the input sender, and
\item
propagates all completion operations sent by \tcode{sndr}.
\end{itemize}

\rSec3[exec.split]{\tcode{execution::split}}

\pnum
\tcode{split} adapts an arbitrary sender
into a sender that can be connected multiple times.

\pnum
Let \exposid{split-env} be the type of an environment
such that, given an instance \tcode{env},
the expression \tcode{get_stop_token(env)} is well-formed and
has type \tcode{inplace_stop_token.}

\pnum
The name \tcode{split} denotes a pipeable sender adaptor object.
For a subexpression \tcode{sndr}, let \tcode{Sndr} be \tcode{decltype((sndr))}.
If \tcode{\libconcept{sender_in}<Sndr, \exposid{split-env}>} is \tcode{false},
\tcode{split(sndr)} is ill-formed.

\pnum
Otherwise, the expression \tcode{split(sndr)} is expression-equivalent to:
\begin{codeblock}
transform_sender(@\exposid{get-domain-early}@(sndr), @\exposid{make-sender}@(split, {}, sndr))
\end{codeblock}
except that \tcode{sndr} is evaluated only once.
\begin{note}
The default implementation of \tcode{transform_sender}
will have the effect of connecting the sender to a receiver.
It will return a sender with a different tag type.
\end{note}

\pnum
Let \exposid{local-state} denote the following exposition-only class template:

\begin{codeblock}
namespace std::execution {
  struct @\exposid{local-state-base}@ {                                     // \expos
    virtual ~@\exposid{local-state-base}@() = default;
    virtual void @\exposid{notify}@() noexcept = 0;                         // \expos
  };

  template<class Sndr, class Rcvr>
  struct @\exposid{local-state}@ : @\exposid{local-state-base}@ {                       // \expos
    using @\exposid{on-stop-callback}@ =                                    // \expos
      stop_callback_for_t<stop_token_of_t<env_of_t<Rcvr>>, @\exposid{on-stop-request}@>;

    @\exposid{local-state}@(Sndr&& sndr, Rcvr& rcvr) noexcept;
    ~@\exposid{local-state}@();

    void @\exposid{notify}@() noexcept override;

  private:
    optional<@\exposid{on-stop-callback}@> @\exposid{on_stop}@;                         // \expos
    @\exposid{shared-state}@<Sndr>* @\exposid{sh_state}@;                               // \expos
    Rcvr* @\exposid{rcvr}@;                                                 // \expos
  };
}
\end{codeblock}

\begin{itemdecl}
@\exposid{local-state}@(Sndr&& sndr, Rcvr& rcvr) noexcept;
\end{itemdecl}

\begin{itemdescr}
\pnum
\effects
Equivalent to:
\begin{codeblock}
auto& [_, data, _] = sndr;
this->@\exposid{sh_state}@ = data.sh_state.get();
this->@\exposid{sh_state}@->@\exposid{inc-ref}@();
this->@\exposid{rcvr}@ = addressof(rcvr);
\end{codeblock}
\end{itemdescr}

\begin{itemdecl}
~@\exposid{local-state}@();
\end{itemdecl}

\begin{itemdescr}
\pnum
\effects
Equivalent to:
\begin{codeblock}
sh_state->@\exposid{dec-ref}@();
\end{codeblock}
\end{itemdescr}

\begin{itemdecl}
void @\exposid{notify}@() noexcept override;
\end{itemdecl}

\begin{itemdescr}
\pnum
\effects
Equivalent to:
\begin{codeblock}
@\exposid{on_stop}@.reset();
visit(
  [this](const auto& tupl) noexcept -> void {
    apply(
      [this](auto tag, const auto&... args) noexcept -> void {
        tag(std::move(*@\exposid{rcvr}@), args...);
      },
      tupl);
  },
  @\exposid{sh_state}@->result);
\end{codeblock}
\end{itemdescr}

\pnum
Let \exposid{split-receiver} denote
the following exposition-only class template:
\begin{codeblock}
namespace std::execution {
  template<class Sndr>
  struct @\exposid{split-receiver}@ {                                       // \expos
    using receiver_concept = receiver_t;

    template<class Tag, class... Args>
    void @\exposid{complete}@(Tag, Args&&... args) noexcept {               // \expos
      using tuple_t = @\exposid{decayed-tuple}@<Tag, Args...>;
      try {
        @\exposid{sh_state}@->result.template emplace<tuple_t>(Tag(), std::forward<Args>(args)...);
      } catch (...) {
        using tuple_t = tuple<set_error_t, exception_ptr>;
        @\exposid{sh_state}@->result.template emplace<tuple_t>(set_error, current_exception());
      }
      @\exposid{sh_state}@->notify();
    }

    template<class... Args>
    void set_value(Args&&... args) && noexcept {
      @\exposid{complete}@(execution::set_value, std::forward<Args>(args)...);
    }

    template<class Error>
    void set_error(Error&& err) && noexcept {
      @\exposid{complete}@(execution::set_error, std::forward<Error>(err));
    }

    void set_stopped() && noexcept {
      @\exposid{complete}@(execution::set_stopped);
    }

    struct @\exposid{env}@ {                                                // \expos
      @\exposid{shared-state}@<Sndr>* @\exposid{sh-state}@;                             // \expos

      inplace_stop_token query(get_stop_token_t) const noexcept {
        return @\exposid{sh-state}@->stop_src.get_token();
      }
    };

    @\exposid{env}@ get_env() const noexcept {
      return @\exposid{env}@{@\exposid{sh_state}@};
    }

    @\exposid{shared-state}@<Sndr>* @\exposid{sh_state}@;                               // \expos
  };
}
\end{codeblock}

\pnum
Let \exposid{shared-state} denote the following exposition-only class template:
\begin{codeblock}
namespace std::execution {
  template<class Sndr>
  struct @\exposid{shared-state}@ {
    using @\exposid{variant-type}@ = @\seebelow@;                             // \expos
    using @\exposid{state-list-type}@ = @\seebelow@;                          // \expos

    explicit @\exposid{shared-state}@(Sndr&& sndr);

    void @\exposid{start-op}@() noexcept;                                   // \expos
    void @\exposid{notify}@() noexcept;                                     // \expos
    void @\exposid{inc-ref}@() noexcept;                                    // \expos
    void @\exposid{dec-ref}@() noexcept;                                    // \expos

    inplace_stop_source @\exposid{stop_src}@{};                             // \expos
    @\exposid{variant-type}@ @\exposid{result}@{};                                      // \expos
    @\exposid{state-list-type}@ @\exposid{waiting_states}@;                             // \expos
    atomic<bool> @\exposid{completed}@{false};                              // \expos
    atomic<size_t> @\exposid{ref_count}@{1};                                // \expos
    connect_result_t<Sndr, @\exposid{split-receiver}@<Sndr>> @\exposid{op_state}@;      // \expos
  };
}
\end{codeblock}

\pnum
Let \tcode{Sigs} be a pack of the arguments
to the \tcode{completion_signatures} specialization
named by \tcode{completion_signatures_of_t<Sndr>}.
For type \tcode{Tag} and pack \tcode{Args},
let \exposid{as-tuple} be an alias template
such that \tcode{\exposid{as-tuple}<Tag(Args...)>} denotes
the type \tcode{\exposid{decayed-tuple}<Tag, Args...>}.
Then \exposid{variant-type} denotes the type
\begin{codeblock}
variant<tuple<set_stopped_t>, tuple<set_error_t, exception_ptr>, @\exposid{as-tuple}@<Sigs>...>
\end{codeblock}
but with duplicate types removed.

\pnum
Let \exposid{state-list-type} be a type
that stores a list of pointers to \exposid{local-state-base} objects and
that permits atomic insertion.

\begin{itemdecl}
explicit @\exposid{shared-state}@(Sndr&& sndr);
\end{itemdecl}

\begin{itemdescr}
\pnum
\effects
Initializes \exposid{op_state} with the result of
\tcode{connect(std::forward<Sndr>(sndr), \exposid{split-re\-ceiver}\{this\})}.

\pnum
\ensures
\exposid{waiting_states} is empty, and \exposid{completed} is \tcode{false}.
\end{itemdescr}

\begin{itemdecl}
void @\exposid{start-op}@() noexcept;
\end{itemdecl}

\begin{itemdescr}
\pnum
\effects
Evaluates \tcode{\exposid{inc-ref}()}.
If \tcode{stop_src.stop_requested()} is \tcode{true},
evaluates \tcode{\exposid{notify}()};
otherwise, evaluates \tcode{start(\exposid{op_state})}.
\end{itemdescr}

\begin{itemdecl}
void @\exposid{notify}@() noexcept;
\end{itemdecl}

\begin{itemdescr}
\pnum
\effects
Atomically does the following:
\begin{itemize}
\item
Sets \tcode{completed} to \tcode{true}, and
\item
Exchanges \tcode{waiting_states} with an empty list,
storing the old value in a local \tcode{prior_states}.
\end{itemize}
Then, for each pointer \tcode{p} in \tcode{prior_states},
evaluates \tcode{p->\exposid{notify}()}.
Finally, evaluates \tcode{\exposid{dec-ref}()}.
\end{itemdescr}

\begin{itemdecl}
void @\exposid{inc-ref}@() noexcept;
\end{itemdecl}

\begin{itemdescr}
\pnum
\effects
Increments \exposid{ref_count}.
\end{itemdescr}

\begin{itemdecl}
void @\exposid{dec-ref}@() noexcept;
\end{itemdecl}

\begin{itemdescr}
\pnum
\effects
Decrements \exposid{ref_count}.
If the new value of \exposid{ref_count} is \tcode{0},
calls \tcode{delete this}.

\pnum
\sync
If an evaluation of \tcode{\exposid{dec-ref}()} does not
decrement the \tcode{ref_count} to \tcode{0} then
synchronizes with the evaluation of \tcode{dec-ref()}
that decrements \tcode{ref_count} to \tcode{0}.
\end{itemdescr}

\pnum
Let \exposid{split-impl-tag} be an empty exposition-only class type.
Given an expression \tcode{sndr},
the expression \tcode{split.transform_sender(sndr)} is equivalent to:
\begin{codeblock}
auto&& [tag, _, child] = sndr;
auto* sh_state = new @\exposid{shared-state}@{std::forward_like<decltype((sndr))>(child)};
return @\exposid{make-sender}@(@\exposid{split-impl-tag}@(), @\exposid{shared-wrapper}@{sh_state, tag});
\end{codeblock}
where \exposid{shared-wrapper} is an exposition-only class
that manages the reference count of the \exposid{shared-state} object
pointed to by sh_state.
\exposid{shared-wrapper} models \libconcept{copyable}
with move operations nulling out the moved-from object,
copy operations incrementing the reference count
by calling \tcode{sh_state->\exposid{inc-ref}()}, and
assignment operations performing a copy-and-swap operation.
The destructor has no effect if sh_state is null;
otherwise, it decrements the reference count
by evaluating \tcode{sh_state->\exposid{dec-ref}()}.

\pnum
The exposition-only class template \exposid{impls-for}\iref{exec.snd.general}
is specialized for \exposid{split-impl-tag} as follows:
\begin{codeblock}
namespace std::execution {
  template<>
  struct @\exposid{impls-for}@<@\exposid{split-impl-tag}@> : @\exposid{default-impls}@ {
    static constexpr auto @\exposid{get-state}@ = @\seebelow@;
    static constexpr auto @\exposid{start}@ = @\seebelow@;
  };
}
\end{codeblock}

\pnum
The member
\tcode{\exposid{impls-for}<\exposid{split-impl-tag}>::\exposid{get-state}}
is initialized with a callable object equivalent to
the following lambda expression:
\begin{codeblock}
[]<class Sndr>(Sndr&& sndr, auto& rcvr) noexcept {
  return @\exposid{local-state}@{std::forward<Sndr>(sndr), rcvr};
}
\end{codeblock}

\pnum
The member
\tcode{\exposid{impls-for}<\exposid{split-impl-tag}>::\exposid{start}}
is initialized with a callable object
that has a function call operator equivalent to the following:
\begin{codeblock}
template<class Sndr, class Rcvr>
void operator()(@\exposid{local-state}@<Sndr, Rcvr>& state, Rcvr& rcvr) const noexcept;
\end{codeblock}

\effects
If \tcode{state.\exposid{sh_state}->\exposid{completed}} is \tcode{true},
evaluates \tcode{state.\exposid{notify}()} and returns.
Otherwise, does the following in order:
\begin{itemize}
\item
Evaluates
\begin{codeblock}
state.@\exposid{on_stop}@.emplace(
  get_stop_token(get_env(rcvr)),
  @\exposid{on-stop-request}@{state.@\exposid{sh_state}@->@\exposid{stop_src}@});
\end{codeblock}
\item
Then atomically does the following:
\begin{itemize}
\item
Reads the value \tcode{c} of
\tcode{state.\exposid{sh_state}->\exposid{completed}}, and
\item
Inserts \tcode{addressof(state)} into
\tcode{state.\exposid{sh_state}->\exposid{waiting_states}}
if \tcode{c} is \tcode{false}.
\end{itemize}
\item
If \tcode{c} is \tcode{true},
calls \tcode{state.\exposid{notify}()} and returns.
\item
Otherwise,
if \tcode{addressof(state)} is the first item added to
\tcode{state.\exposid{sh_state}->\exposid{waiting_states}},
evaluates \tcode{state.\exposid{sh_state}->\exposid{start-op}()}.
\end{itemize}

\rSec3[exec.when.all]{\tcode{execution::when_all}}

\pnum
\tcode{when_all} and \tcode{when_all_with_variant}
both adapt multiple input senders into a sender
that completes when all input senders have completed.
\tcode{when_all} only accepts senders
with a single value completion signature and
on success concatenates all the input senders' value result datums
into its own value completion operation.
\tcode{when_all_with_variant(sndrs...)} is semantically equivalent to
w\tcode{hen_all(into_variant(sndrs)...)},
where \tcode{sndrs} is a pack of subexpressions
whose types model \libconcept{sender}.

\pnum
The names \tcode{when_all} and \tcode{when_all_with_variant} denote
customization point objects.
Let \tcode{sndrs} be a pack of subexpressions,
let \tcode{Sndrs} be a pack of the types \tcode{decltype((sndrs))...}, and
let \tcode{CD} be
the type \tcode{common_type_t<decltype(\exposid{get-domain-early}(sndrs))...>}.
The expressions \tcode{when_all(sndrs...)} and
\tcode{when_all_with_variant(sndrs...)} are ill-formed
if any of the following is \tcode{true}:
\begin{itemize}
\item
\tcode{sizeof...(sndrs)} is \tcode{0}, or
\item
\tcode{(\libconcept{sender}<Sndrs> \&\& ...)} is \tcode{false}, or
\item
\tcode{CD} is ill-formed.
\end{itemize}

\pnum
The expression \tcode{when_all(sndrs...)} is expression-equivalent to:
\begin{codeblock}
transform_sender(CD(), @\exposid{make-sender}@(when_all, {}, sndrs...))
\end{codeblock}

\pnum
The exposition-only class template \exposid{impls-for}\iref{exec.snd.general}
is specialized for \tcode{when_all_t} as follows:
\begin{codeblock}
namespace std::execution {
  template<>
  struct @\exposid{impls-for}@<when_all_t> : @\exposid{default-impls}@ {
    static constexpr auto @\exposid{get-attrs}@ = @\seebelow@;
    static constexpr auto @\exposid{get-env}@ = @\seebelow@;
    static constexpr auto @\exposid{get-state}@ = @\seebelow@;
    static constexpr auto @\exposid{start}@ = @\seebelow@;
    static constexpr auto @\exposid{complete}@ = @\seebelow@;
  };
}
\end{codeblock}

\pnum
The member \tcode{\exposid{impls-for}<when_all_t>::\exposid{get-attrs}}
is initialized with a callable object
equivalent to the following lambda expression:
\begin{codeblock}
[](auto&&, auto&&... child) noexcept {
  if constexpr (@\libconcept{same_as}@<CD, default_domain>) {
    return env<>();
  } else {
    return @\exposid{MAKE-ENV}@(get_domain, CD());
  }
}
\end{codeblock}

\pnum
The member \tcode{\exposid{impls-for}<when_all_t>::\exposid{get-env}}
is initialized with a callable object
equivalent to the following lambda expression:
\begin{codeblock}
[]<class State, class Rcvr>(auto&&, State& state, const Receiver& rcvr) noexcept {
  return @\seebelow@;
}
\end{codeblock}
Returns an object \tcode{e} such that
\begin{itemize}
\item
\tcode{decltype(e)} models \exposconcept{queryable}, and
\item
\tcode{e.query(get_stop_token)} is expression-equivalent to
\tcode{state.\exposid{stop-src}.get_token()}, and
\item
given a query object \tcode{q} with type other than \cv{} \tcode{stop_token_t},
\tcode{e.query(q)} is expression-equivalent to \tcode{get_env(rcvr).query(q)}.
\end{itemize}

\pnum
The member \tcode{\exposid{impls-for}<when_all_t>::\exposid{get-state}}
is initialized with a callable object
equivalent to the following lambda expression:
\begin{codeblock}
[]<class Sndr, class Rcvr>(Sndr&& sndr, Rcvr& rcvr) noexcept(@$e$@) -> decltype(@$e$@) {
  return @$e$@;
}
\end{codeblock}
where $e$ is the expression
\begin{codeblock}
std::forward<Sndr>(sndr).apply(@\exposid{make-state}@<Rcvr>())
\end{codeblock}
and where \exposid{make-state} is the following exposition-only class template:
\begin{codeblock}
template<class Sndr, class Env>
concept @\defexposconcept{max-1-sender-in}@ = @\libconcept{sender_in}@<Sndr, Env> &&                // \expos
  (tuple_size_v<value_types_of_t<Sndr, Env, tuple, tuple>> <= 1);

enum class @\exposid{disposition}@ { @\exposid{started}@, @\exposid{error}@, @\exposid{stopped}@ };             // \expos

template<class Rcvr>
struct @\exposid{make-state}@ {
  template<@\exposconcept{max-1-sender-in}@<env_of_t<Rcvr>>... Sndrs>
  auto operator()(auto, auto, Sndrs&&... sndrs) const {
    using values_tuple = @\seebelow@;
    using errors_variant = @\seebelow@;
    using stop_callback = stop_callback_for_t<stop_token_of_t<env_of_t<Rcvr>>, @\exposid{on-stop-request}@>;

    struct @\exposid{state-type}@ {
      void @\exposid{arrive}@(Rcvr& rcvr) noexcept {                        // \expos
        if (0 == --count) {
          @\exposid{complete}@(rcvr);
        }
      }

      void @\exposid{complete}@(Rcvr& rcvr) noexcept;                       // \expos

      atomic<size_t> @\exposid{count}@{sizeof...(sndrs)};                   // \expos
      inplace_stop_source @\exposid{stop_src}@{};                           // \expos
      atomic<@\exposid{disposition}@> disp{@\exposidnc{disposition}@::@\exposidnc{started}@};           // \expos
      errors_variant @\exposid{errors}@{};                                  // \expos
      values_tuple @\exposid{values}@{};                                    // \expos
      optional<stop_callback> @\exposid{on_stop}@{nullopt};                 // \expos
    };

    return @\exposid{state-type}@{};
  }
};
\end{codeblock}

\pnum
Let \exposid{copy-fail} be \tcode{exception_ptr}
if decay-copying any of the child senders' result datums can potentially throw;
otherwise, \exposid{none-such},
where \exposid{none-such} is an unspecified empty class type.

\pnum
The alias \tcode{values_tuple} denotes the type
\begin{codeblock}
tuple<value_types_of_t<Sndrs, env_of_t<Rcvr>, @\exposid{decayed-tuple}@, optional>...>
\end{codeblock}
if that type is well-formed; otherwise, \tcode{tuple<>}.

\pnum
The alias \tcode{errors_variant} denotes
the type \tcode{variant<\exposid{none-such}, \exposid{copy-fail}, Es...>}
with duplicate types removed,
where \tcode{Es} is the pack of the decayed types
of all the child senders' possible error result datums.

\pnum
The member
\tcode{void \exposid{state-type}::\exposid{complete}(Rcvr\& rcvr) noexcept}
behaves as follows:
\begin{itemize}
\item
If \tcode{disp} is equal to \tcode{\exposid{disposition}::\exposid{started}},
evaluates:
\begin{codeblock}
auto tie = []<class... T>(tuple<T...>& t) noexcept { return tuple<T&...>(t); };
auto set = [&](auto&... t) noexcept { set_value(std::move(rcvr), std::move(t)...); };

@\exposid{on_stop}@.reset();
apply(
  [&](auto&... opts) noexcept {
    apply(set, tuple_cat(tie(*opts)...));
  },
  values);
\end{codeblock}
\item
Otherwise,
if \tcode{disp} is equal to \tcode{\exposid{disposition}::\exposid{error}},
evaluates:
\begin{codeblock}
@\exposid{on_stop}@.reset();
visit(
  [&]<class Error>(Error& error) noexcept {
    if constexpr (!@\libconcept{same_as}@<Error, @\exposid{none-such}@>) {
      set_error(std::move(rcvr), std::move(error));
    }
  },
  errors);
\end{codeblock}
\item
Otherwise, evaluates:
\begin{codeblock}
@\exposid{on_stop}@.reset();
set_stopped(std::move(rcvr));
\end{codeblock}
\end{itemize}

\pnum
The member \tcode{\exposid{impls-for}<when_all_t>::\exposid{start}}
is initialized with a callable object
equivalent to the following lambda expression:
\begin{codeblock}
[]<class State, class Rcvr, class... Ops>(
    State& state, Rcvr& rcvr, Ops&... ops) noexcept -> void {
  state.@\exposid{on_stop}@.emplace(
    get_stop_token(get_env(rcvr)),
    @\exposid{on-stop-request}@{state.@\exposid{stop_src}@});
  if (state.@\exposid{stop_src}@.stop_requested()) {
    state.@\exposid{on_stop.}@reset();
    set_stopped(std::move(rcvr));
  } else {
    (start(ops), ...);
  }
}
\end{codeblock}

\pnum
The member \exposid{\tcode{impls-for}<when_all_t>::\exposid{complete}}
is initialized with a callable object
equivalent to the following lambda expression:
\begin{codeblock}
[]<class Index, class State, class Rcvr, class Set, class... Args>(
    this auto& complete, Index, State& state, Rcvr& rcvr, Set, Args&&... args) noexcept -> void {
  if constexpr (@\libconcept{same_as}@<Set, set_error_t>) {
    if (@\exposid{disposition}@::@\exposid{error}@ != state.disp.exchange(@\exposid{disposition}@::@\exposid{error}@)) {
      state.@\exposid{stop_src}@.request_stop();
      @\exposid{TRY-EMPLACE-ERROR}@(state.errors, std::forward<Args>(args)...);
    }
  } else if constexpr (@\libconcept{same_as}@<Set, set_stopped_t>) {
    auto expected = @\exposid{disposition}@::@\exposid{started}@;
    if (state.disp.compare_exchange_strong(expected, @\exposid{disposition}@::@\exposid{stopped}@)) {
      state.@\exposid{stop_src}@.request_stop();
    }
  } else if constexpr (!@\libconcept{same_as}@<decltype(State::values), tuple<>>) {
    if (state.disp == @\exposid{disposition}@::@\exposid{started}@) {
      auto& opt = get<Index::value>(state.values);
      @\exposid{TRY-EMPLACE-VALUE}@(complete, opt, std::forward<Args>(args)...);
    }
  }
  state.@\exposid{arrive}@(rcvr);
}
\end{codeblock}
where \tcode{\exposid{TRY-EMPLACE-ERROR}(v, e)},
for subexpressions \tcode{v} and \tcode{e}, is equivalent to:
\begin{codeblock}
try {
  v.template emplace<decltype(auto(e))>(e);
} catch (...) {
  v.template emplace<exception_ptr>(current_exception());
}
\end{codeblock}
if the expression \tcode{decltype(auto(e))(e)} is potentially throwing;
otherwise, \tcode{v.template emplace<decl\-type(auto(e))>(e)};
and where \tcode{\exposid{TRY-EMPLACE-VALUE}(c, o, as...)},
for subexpressions \tcode{c}, \tcode{o}, and pack of subexpressions \tcode{as},
is equivalent to:
\begin{codeblock}
try {
  o.emplace(as...);
} catch (...) {
  c(Index(), state, rcvr, set_error, current_exception());
  return;
}
\end{codeblock}
if the expression \tcode{\exposid{decayed-tuple}<decltype(as)...>\{as...\}}
is potentially throwing;
otherwise, \tcode{o.emplace(\linebreak as...)}.

\pnum
The expression \tcode{when_all_with_variant(sndrs...)}
is expression-equivalent to:
\begin{codeblock}
transform_sender(CD(), @\exposid{make-sender}@(when_all_with_variant, {}, sndrs...));
\end{codeblock}

\pnum
Given subexpressions \tcode{sndr} and \tcode{env},
if
\tcode{\exposconcept{sender-for}<decltype((sndr)), when_all_with_variant_t>}
is \tcode{false},
then the expression \tcode{when_all_with_variant.transform_sender(sndr, env)}
is ill-formed;
otherwise, it is equivalent to:
\begin{codeblock}
auto&& [_, _, ...child] = sndr;
return when_all(into_variant(std::forward_like<decltype((sndr))>(child))...);
\end{codeblock}
\begin{note}
This causes the \tcode{when_all_with_variant(sndrs...)} sender
to become \tcode{when_all(into_variant(sndrs)...)}
when it is connected with a receiver
whose execution domain does not customize \tcode{when_all_with_variant}.
\end{note}

\rSec3[exec.into.variant]{\tcode{execution::into_variant}}

\pnum
\tcode{into_variant} adapts a sender with multiple value completion signatures
into a sender with just one value completion signature
consisting of a \tcode{variant} of \tcode{tuple}s.

\pnum
The name \tcode{into_variant} denotes a pipeable sender adaptor object.
For a subexpression \tcode{sndr}, let \tcode{Sndr} be \tcode{decltype((sndr))}.
If \tcode{Sndr} does not satisfy \libconcept{sender},
\tcode{into_variant(sndr)} is ill-formed.

\pnum
Otherwise, the expression \tcode{into_variant(sndr)}
is expression-equivalent to:
\begin{codeblock}
transform_sender(@\exposid{get-domain-early}@(sndr), @\exposid{make-sender}@(into_variant, {}, sndr))
\end{codeblock}
except that \tcode{sndr} is only evaluated once.

\pnum
The exposition-only class template \exposid{impls-for}\iref{exec.snd.general}
is specialized for \tcode{into_variant} as follows:
\begin{codeblock}
namespace std::execution {
  template<>
  struct @\exposid{impls-for}@<into_variant_t> : @\exposid{default-impls}@ {
    static constexpr auto @\exposid{get-state}@ = @\seebelow@;
    static constexpr auto @\exposid{complete}@ = @\seebelow@;
  };
}
\end{codeblock}

\pnum
The member \tcode{\exposid{impls-for}<into_variant_t>::\exposid{get-state}}
is initialized with a callable object equivalent to the following lambda:
\begin{codeblock}
[]<class Sndr, class Rcvr>(Sndr&& sndr, Rcvr& rcvr) noexcept
  -> type_identity<value_types_of_t<@\exposid{child-type}@<Sndr>, env_of_t<Rcvr>>> {
  return {};
}
\end{codeblock}

\pnum
The member \tcode{\exposid{impls-for}<into_variant_t>::\exposid{complete}}
is initialized with a callable object equivalent to the following lambda:
\begin{codeblock}
[]<class State, class Rcvr, class Tag, class... Args>(
    auto, State, Rcvr& rcvr, Tag, Args&&... args) noexcept -> void {
  if constexpr (@\libconcept{same_as}@<Tag, set_value_t>) {
    using variant_type = typename State::type;
    @\exposid{TRY-SET-VALUE}@(rcvr, variant_type(@\exposid{decayed-tuple}@<Args...>{std::forward<Args>(args)...}));
  } else {
    Tag()(std::move(rcvr), std::forward<Args>(args)...);
  }
}
\end{codeblock}

\rSec3[exec.stopped.opt]{\tcode{execution::stopped_as_optional}}

\pnum
\tcode{stopped_as_optional} maps a sender's stopped completion operation
into a value completion operation as a disengaged \tcode{optional}.
The sender's value completion operation
is also converted into an \tcode{optional}.
The result is a sender that never completes with stopped,
reporting cancellation by completing with a disengaged \tcode{optional}.

\pnum
The name \tcode{stopped_as_optional} denotes a pipeable sender adaptor object.
For a subexpression \tcode{sndr}, let \tcode{Sndr} be \tcode{decltype((sndr))}.
The expression \tcode{stopped_as_optional(sndr)} is expression-equivalent to:
\begin{codeblock}
transform_sender(@\exposid{get-domain-early}@(sndr), @\exposid{make-sender}@(stopped_as_optional, {}, sndr))
\end{codeblock}
except that \tcode{sndr} is only evaluated once.

\pnum
Let \tcode{sndr} and \tcode{env} be subexpressions
such that \tcode{Sndr} is \tcode{decltype((sndr))} and
\tcode{Env} is \tcode{decltype((env))}.
If \tcode{\exposconcept{sender-for}<Sndr, stopped_as_optional_t>}
is \tcode{false}, or
if the type \tcode{\exposid{single-sender-value-type}<Sndr, Env>}
is ill-formed or \tcode{void},
then the expression \tcode{stopped_as_optional.transform_sender(sndr, env)}
is ill-formed;
otherwise, it is equivalent to:
\begin{codeblock}
auto&& [_, _, child] = sndr;
using V = @\exposid{single-sender-value-type}@<Sndr, Env>;
return let_stopped(
    then(std::forward_like<Sndr>(child),
         []<class... Ts>(Ts&&... ts) noexcept(is_nothrow_constructible_v<V, Ts...>) {
           return optional<V>(in_place, std::forward<Ts>(ts)...);
         }),
    []() noexcept { return just(optional<V>()); });
\end{codeblock}

\rSec3[exec.stopped.err]{\tcode{execution::stopped_as_error}}

\pnum
\tcode{stopped_as_error} maps an input sender's stopped completion operation
into an error completion operation as a custom error type.
The result is a sender that never completes with stopped,
reporting cancellation by completing with an error.

\pnum
The name \tcode{stopped_as_error} denotes a pipeable sender adaptor object.
For some subexpressions \tcode{sndr} and \tcode{err},
let \tcode{Sndr} be \tcode{decltype((sndr))} and
let \tcode{Err} be \tcode{decltype((err))}.
If the type \tcode{Sndr} does not satisfy \libconcept{sender} or
if the type \tcode{Err} does not satisfy \exposconcept{movable-value},
\tcode{stopped_as_error(sndr, err)} is ill-formed.
Otherwise, the expression \tcode{stopped_as_error(sndr, err)}
is expression-equivalent to:
\begin{codeblock}
transform_sender(@\exposid{get-domain-early}@(sndr), @\exposid{make-sender}@(stopped_as_error, err, sndr))
\end{codeblock}
except that \tcode{sndr} is only evaluated once.

\pnum
Let \tcode{sndr} and \tcode{env} be subexpressions
such that \tcode{Sndr} is \tcode{decltype((sndr))} and
\tcode{Env} is \tcode{decltype((env))}.
If \tcode{\exposconcept{sender-for}<Sndr, stopped_as_error_t>} is \tcode{false},
then the expression \tcode{stopped_as_error.transform_sender(sndr, env)}
is ill-formed;
otherwise, it is equivalent to:
\begin{codeblock}
auto&& [_, err, child] = sndr;
using E = decltype(auto(err));
return let_stopped(
    std::forward_like<Sndr>(child),
    [err = std::forward_like<Sndr>(err)]() mutable noexcept(is_nothrow_move_constructible_v<E>) {
      return just_error(std::move(err));
    });
\end{codeblock}

\rSec2[exec.consumers]{Sender consumers}

\rSec3[exec.sync.wait]{\tcode{this_thread::sync_wait}}

\pnum
\tcode{this_thread::sync_wait} and \tcode{this_thread::sync_wait_with_variant}
are used
to block the current thread of execution
until the specified sender completes and
to return its async result.
\tcode{sync_wait} mandates
that the input sender has exactly one value completion signature.

\pnum
Let \exposid{sync-wait-env} be the following exposition-only class type:
\begin{codeblock}
namespace std::this_thread {
  struct @\exposid{sync-wait-env}@ {
    execution::run_loop* @\exposid{loop}@;                                  // \expos

    auto query(execution::get_scheduler_t) const noexcept {
      return @\exposid{loop}@->get_scheduler();
    }

    auto query(execution::get_delegation_scheduler_t) const noexcept {
      return @\exposid{loop}@->get_scheduler();
    }
  };
}
\end{codeblock}

\pnum
Let \exposid{sync-wait-result-type} and
\exposid{sync-wait-with-variant-result-type}
be exposition-only alias templates defined as follows:
\begin{codeblock}
namespace std::this_thread {
  template<execution::@\libconcept{sender_in}@<@\exposid{sync-wait-env}@> Sndr>
    using @\exposid{sync-wait-result-type}@ =
      optional<execution::value_types_of_t<Sndr, @\exposid{sync-wait-env}@, @\exposid{decayed-tuple}@,
               type_identity_t>>;

  template<execution::@\libconcept{sender_in}@<@\exposid{sync-wait-env}@> Sndr>
    using @\exposid{sync-wait-with-variant-result-type}@ =
      optional<execution::value_types_of_t<Sndr, @\exposid{sync-wait-env}@>>;
}
\end{codeblock}

\pnum
The name \tcode{this_thread::sync_wait} denotes a customization point object.
For a subexpression \tcode{sndr}, let \tcode{Sndr} be \tcode{decltype((sndr))}.
If \tcode{\libconcept{sender_in}<Sndr, \exposid{sync-wait-env}>}
is \tcode{false},
the expression \tcode{this_thread::sync_wait(sndr)} is ill-formed.
Otherwise, it is expression-equivalent to the following,
except that \tcode{sndr} is evaluated only once:
\begin{codeblock}
apply_sender(@\exposid{get-domain-early}@(sndr), sync_wait, sndr)
\end{codeblock}
\mandates
\begin{itemize}
\item
The type \tcode{\exposid{sync-wait-result-type}<Sndr>} is well-formed.
\item
\tcode{\libconcept{same_as}<decltype($e$), \exposid{sync-wait-result-type}<Sndr>>}
is \tcode{true}, where $e$ is the \tcode{apply_sender} expression above.
\end{itemize}

\pnum
Let \exposid{sync-wait-state} and \exposid{sync-wait-receiver}
be the following exposition-only class templates:
\begin{codeblock}
namespace std::this_thread {
  template<class Sndr>
  struct @\exposid{sync-wait-state}@ {                                      // \expos
    execution::run_loop @\exposid{loop}@;                                   // \expos
    exception_ptr @\exposid{error}@;                                        // \expos
    @\exposid{sync-wait-result-type}@<Sndr> @\exposidnc{result}@;                         // \expos
  };

  template<class Sndr>
  struct @\exposid{sync-wait-receiver}@ {                                   // \expos
    using receiver_concept = execution::receiver_t;
    @\exposidnc{sync-wait-state}@<Sndr>* @\exposid{state}@;                               // \expos

    template<class... Args>
    void set_value(Args&&... args) && noexcept;

    template<class Error>
    void set_error(Error&& err) && noexcept;

    void set_stopped() && noexcept;

    @\exposid{sync-wait-env}@ get_env() const noexcept { return {&@\exposid{state}@->@\exposid{loop}@}; }
  };
}
\end{codeblock}

\begin{itemdecl}
template<class... Args>
void set_value(Args&&... args) && noexcept;
\end{itemdecl}

\begin{itemdescr}
\pnum
\effects
Equivalent to:
\begin{codeblock}
try {
  @\exposid{state}@->@\exposid{result}@.emplace(std::forward<Args>(args)...);
} catch (...) {
  @\exposid{state}@->@\exposid{error}@ = current_exception();
}
@\exposid{state}@->@\exposid{loop}@.finish();
\end{codeblock}
\end{itemdescr}

\begin{itemdecl}
template<class Error>
void set_error(Error&& err) && noexcept;
\end{itemdecl}

\begin{itemdescr}
\pnum
\effects
Equivalent to:
\begin{codeblock}
@\exposid{state}@->@\exposid{error}@ = @\exposid{AS-EXCEPT-PTR}@(std::forward<Error>(err));    // see \ref{exec.general}
@\exposid{state}@->@\exposid{loop}@.finish();
\end{codeblock}
\end{itemdescr}

\begin{itemdecl}
void set_stopped() && noexcept;
\end{itemdecl}

\begin{itemdescr}
\pnum
\effects
Equivalent to \tcode{\exposid{state}->\exposid{loop}.finish()}.
\end{itemdescr}

\pnum
For a subexpression \tcode{sndr}, let \tcode{Sndr} be \tcode{decltype((sndr))}.
If \tcode{\libconcept{sender_to}<Sndr, \exposid{sync-wait-receiver}<\linebreak Sndr>>}
is \tcode{false},
the expression \tcode{sync_wait.apply_sender(sndr)} is ill-formed;
otherwise, it is equivalent to:
\begin{codeblock}
@\exposid{sync-wait-state}@<Sndr> state;
auto op = connect(sndr, @\exposid{sync-wait-receiver}@<Sndr>{&state});
start(op);

state.@\exposid{loop}@.run();
if (state.@\exposid{error}@) {
  rethrow_exception(std::move(state.@\exposid{error}@));
}
return std::move(state.@\exposid{result}@);
\end{codeblock}

\pnum
The behavior of \tcode{this_thread::sync_wait(sndr)} is undefined unless:
\begin{itemize}
\item
It blocks the current thread of execution\iref{defns.block}
with forward progress guarantee delegation\iref{intro.progress}
until the specified sender completes.
\begin{note}
The default implementation of \tcode{sync_wait} achieves
forward progress guarantee delegation by providing a \tcode{run_loop} scheduler
via the \tcode{get_delegation_scheduler} query
on the \exposid{sync-wait-receiver}'s environment.
The \tcode{run_loop} is driven by the current thread of execution.
\end{note}
\item
It returns the specified sender's async results as follows:
\begin{itemize}
\item
For a value completion,
the result datums are returned in
a \tcode{tuple} in an engaged \tcode{optional} object.
\item
For an error completion, an exception is thrown.
\item
For a stopped completion, a disengaged \tcode{optional} object is returned.
\end{itemize}
\end{itemize}

\rSec3[exec.sync.wait.var]{\tcode{this_thread::sync_wait_with_variant}}

\pnum
The name \tcode{this_thread::sync_wait_with_variant} denotes
a customization point object.
For a subexpression \tcode{sndr},
let \tcode{Sndr} be \tcode{decltype(into_variant(sndr))}.
If \tcode{\libconcept{sender_in}<Sndr, \exposid{sync-wait-env}>}
is \tcode{false},
\tcode{this_thread::sync_wait_with_variant(sndr)} is ill-formed.
Otherwise, it is expression-equivalent to the following,
except \tcode{sndr} is evaluated only once:
\begin{codeblock}
apply_sender(@\exposid{get-domain-early}@(sndr), sync_wait_with_variant, sndr)
\end{codeblock}
\mandates
\begin{itemize}
\item
The type \tcode{\exposid{sync-wait-with-variant-result-type}<Sndr>}
is well-formed.
\item
\tcode{\libconcept{same_as}<decltype($e$), \exposid{sync-wait-with-variant-result-type}<Sndr>>}
is \tcode{true}, where $e$ is the \tcode{ap\-ply_sender} expression above.
\end{itemize}

\pnum
If \tcode{\exposconcept{callable}<sync_wait_t, Sndr>} is \tcode{false},
the expression \tcode{sync_wait_with_variant.apply_sender(\linebreak sndr)} is ill-formed.
Otherwise, it is equivalent to:
\begin{codeblock}
using result_type = @\exposid{sync-wait-with-variant-result-type}@<Sndr>;
if (auto opt_value = sync_wait(into_variant(sndr))) {
  return result_type(std::move(get<0>(*opt_value)));
}
return result_type(nullopt);
\end{codeblock}

\pnum
The behavior of \tcode{this_thread::sync_wait_with_variant(sndr)}
is undefined unless:
\begin{itemize}
\item
It blocks the current thread of execution\iref{defns.block}
with forward progress guarantee delegation\iref{intro.progress}
until the specified sender completes.
\begin{note}
The default implementation of \tcode{sync_wait_with_variant} achieves
forward progress guarantee delegation by relying on
the forward progress guarantee delegation provided by \tcode{sync_wait}.
\end{note}
\item
It returns the specified sender's async results as follows:
\begin{itemize}
\item
For a value completion,
the result datums are returned in an engaged \tcode{optional} object
that contains a \tcode{variant} of \tcode{tuple}s.
\item
For an error completion, an exception is thrown.
\item
For a stopped completion, a disengaged \tcode{optional} object is returned.
\end{itemize}
\end{itemize}

\rSec1[exec.util]{Sender/receiver utilities}

\rSec2[exec.util.cmplsig]{\tcode{execution::completion_signatures}}

\pnum
\tcode{completion_signatures} is a type
that encodes a set of completion signatures\iref{exec.async.ops}.

\pnum
\begin{example}
\begin{codeblock}
struct my_sender {
  using sender_concept = sender_t;
  using completion_signatures =
    execution::completion_signatures<
      set_value_t(),
      set_value_t(int, float),
      set_error_t(exception_ptr),
      set_error_t(error_code),
      set_stopped_t()>;
};
\end{codeblock}
Declares \tcode{my_sender} to be a sender
that can complete by calling one of the following
for a receiver expression \tcode{rcvr}:
\begin{itemize}
\item \tcode{set_value(rcvr)}
\item \tcode{set_value(rcvr, int\{...\}, float\{...\})}
\item \tcode{set_error(rcvr, exception_ptr\{...\})}
\item \tcode{set_error(rcvr, error_code\{...\})}
\item \tcode{set_stopped(rcvr)}
\end{itemize}
\end{example}

\pnum
This subclause makes use of the following exposition-only entities:
\begin{codeblock}
template<class Fn>
  concept @\defexposconcept{completion-signature}@ = @\seebelow@;
\end{codeblock}

\pnum
A type \tcode{Fn} satisfies \exposconcept{completion-signature}
if and only if it is a function type with one of the following forms:
\begin{itemize}
\item
\tcode{set_value_t(Vs...)},
where \tcode{Vs} is a pack of object or reference types.
\item
\tcode{set_error_t(Err)},
where \tcode{Err} is an object or reference type.
\item
\tcode{set_stopped_t()}
\end{itemize}

\pnum
\begin{codeblock}
template<bool>
  struct @\exposid{indirect-meta-apply}@ {
    template<template<class...> class T, class... As>
      using @\exposid{meta-apply}@ = T<As...>;                              // \expos
  };

template<class...>
  concept @\defexposconcept{always-true}@ = true;                                   // \expos

template<class Tag,
         @\exposconcept{valid-completion-signatures}@ Completions,
         template<class...> class Tuple,
         template<class...> class Variant>
  using @\exposid{gather-signatures}@ = @\seebelow@;
\end{codeblock}

\pnum
Let \tcode{Fns} be a pack of the arguments of
the \tcode{completion_signatures} specialization named by \tcode{Completions},
let \tcode{TagFns} be a pack of the function types in \tcode{Fns}
whose return types are \tcode{Tag}, and
let $\tcode{Ts}_n$ be a pack of the function argument types
in the $n$-th type in \tcode{TagFns}.
Then, given two variadic templates \tcode{Tuple} and \tcode{Variant},
the type \tcode{\exposid{gather-signatures}<Tag, Completions, Tuple, Variant>}
names the type
\begin{codeblock}
@\exposid{META-APPLY}@(Variant, @\exposid{META-APPLY}@(Tuple, Ts@$_0$@...),
                    @\itcorr[1]\exposid{META-APPLY}@(Tuple, Ts@$_1$@...),
                    @\itcorr[1]\ldots@,
                    @\itcorr[1]\exposid{META-APPLY}@(Tuple, Ts@$_{m-1}$@...))
\end{codeblock}
where $m$ is the size of the pack \tcode{TagFns} and
\tcode{META-APPLY(T, As...)} is equivalent to:
\begin{codeblock}
typename @\exposid{indirect-meta-apply}@<@\exposid{always-true}@<As...>>::template @\exposid{meta-apply}@<T, As...>
\end{codeblock}

\pnum
\begin{note}
The purpose of \exposid{META-APPLY} is to make it valid
to use non-variadic templates as \tcode{Variant} and \tcode{Tuple} arguments
to \exposid{gather-signatures}.
\end{note}

\pnum
\begin{codeblock}
namespace std::execution {
  template<@\exposconcept{completion-signature}@... Fns>
    struct completion_signatures {};

  template<class Sndr, class Env = env<>,
           template<class...> class Tuple = @\exposid{decayed-tuple}@,
           template<class...> class Variant = @\exposid{variant-or-empty}@>
      requires @\libconcept{sender_in}@<Sndr, Env>
    using value_types_of_t =
      @\exposid{gather-signatures}@<set_value_t, completion_signatures_of_t<Sndr, Env>, Tuple, Variant>;

  template<class Sndr, class Env = env<>,
           template<class...> class Variant = @\exposid{variant-or-empty}@>
      requires @\libconcept{sender_in}@<Sndr, Env>
    using error_types_of_t =
      @\exposid{gather-signatures}@<set_error_t, completion_signatures_of_t<Sndr, Env>,
                        type_identity_t, Variant>;

  template<class Sndr, class Env = env<>>
      requires @\libconcept{sender_in}@<Sndr, Env>
    constexpr bool sends_stopped =
      !@\libconcept{same_as}@<@\exposid{type-list}@<>,
               @\exposid{gather-signatures}@<set_stopped_t, completion_signatures_of_t<Sndr, Env>,
                                 @\exposid{type-list}@, @\exposid{type-list}@>>;
}
\end{codeblock}

\rSec2[exec.util.cmplsig.trans]{\tcode{execution::transform_completion_signatures}}

\pnum
\tcode{transform_completion_signatures} is an alias template
used to transform one set of completion signatures into another.
It takes a set of completion signatures and
several other template arguments
that apply modifications to each completion signature in the set
to generate a new specialization of \tcode{completion_signature}s.
\pnum
\begin{example}
Given a sender \tcode{Sndr} and an environment \tcode{Env},
adapt the completion signatures of \tcode{Sndr} by
lvalue-ref qualifying the values,
adding an additional \tcode{exception_ptr} error completion
if it is not already there, and
leaving the other completion signatures alone.
\begin{codeblock}
template<class... Args>
  using my_set_value_t =
    completion_signatures<
      set_value_t(add_lvalue_reference_t<Args>...)>;

using my_completion_signatures =
  transform_completion_signatures<
    completion_signatures_of_t<Sndr, Env>,
    completion_signatures<set_error_t(exception_ptr)>,
    my_set_value_t>;
\end{codeblock}
\end{example}

\pnum
This subclause makes use of the following exposition-only entities:
\begin{codeblock}
template<class... As>
  using @\exposid{default-set-value}@ =
    completion_signatures<set_value_t(As...)>;

template<class Err>
  using @\exposid{default-set-error}@ =
    completion_signatures<set_error_t(Err)>;
\end{codeblock}

\pnum
\begin{codeblock}
namespace std::execution {
  template<@\exposconcept{valid-completion-signatures}@ InputSignatures,
           @\exposconcept{valid-completion-signatures}@ AdditionalSignatures = completion_signatures<>,
           template<class...> class SetValue = @\exposid{default-set-value}@,
           template<class> class SetError = @\exposid{default-set-error}@,
           @\exposconcept{valid-completion-signatures}@ SetStopped = completion_signatures<set_stopped_t()>>
  using transform_completion_signatures = completion_signatures<@\seebelow@>;
}
\end{codeblock}

\pnum
\tcode{SetValue} shall name an alias template
such that for any pack of types \tcode{As},
the type \tcode{SetValue<As...>} is either ill-formed or else
\tcode{\exposconcept{valid-completion-signatures}<SetValue<As...>>} is satisfied.
\tcode{SetError} shall name an alias template
such that for any type \tcode{Err},
\tcode{SetError<Err>} is either ill-formed or else
\tcode{\exposconcept{valid-completion-signatures}<SetError<Err>>} is satisfied.

\pnum
Let \tcode{Vs} be a pack of the types in the \exposid{type-list} named by
\tcode{\exposid{gather-signatures}<set_value_t, InputSigna\-tures, SetValue, \exposid{type-list}>}.

\pnum
Let \tcode{Es} be a pack of the types in the \exposid{type-list} named by
\tcode{\exposid{gather-signatures}<set_error_t, InputSigna\-tures, type_identity_t, \exposid{error-list}>},
where \exposid{error-list} is an alias template
such that \tcode{\exposid{error-list}<\linebreak Ts...>} is
\tcode{\exposid{type-list}<SetError<Ts>...>}.

\pnum
Let \tcode{Ss} name the type \tcode{completion_signatures<>} if
\tcode{\exposid{gather-signatures}<set_stopped_t, InputSigna\-tures, \exposid{type-list}, \exposid{type-list}>}
is an alias for the type \tcode{\exposid{type-list}<>};
otherwise, \tcode{SetStopped}.

\pnum
If any of the above types are ill-formed,
then
\begin{codeblock}
transform_completion_signatures<InputSignatures, AdditionalSignatures,
                                SetValue, SetError, SetStopped>
\end{codeblock}
is ill-formed.
Otherwise,
\begin{codeblock}
transform_completion_signatures<InputSignatures, AdditionalSignatures,
                                SetValue, SetError, SetStopped>
\end{codeblock}
is the type \tcode{completion_signatures<Sigs...>}
where \tcode{Sigs...} is the unique set of types in all the template arguments
of all the \tcode{completion_signatures} specializations in the set
\tcode{AdditionalSignatures}, \tcode{Vs...}, \tcode{Es...}, \tcode{Ss}.

\rSec1[exec.envs]{Queryable utilities}

\rSec2[exec.prop]{Class template \tcode{prop}}

\begin{codeblock}
namespace std::execution {
  template<class QueryTag, class ValueType>
  struct @\libglobal{prop}@ {
    QueryTag @\exposid{query_}@;            // \expos
    ValueType @\exposid{value_}@;           // \expos

    constexpr const ValueType& query(QueryTag) const noexcept {
      return @\exposid{value_}@;
    }
  };

  template<class QueryTag, class ValueType>
    prop(QueryTag, ValueType) -> prop<QueryTag, unwrap_reference_t<ValueType>>;
}
\end{codeblock}

\pnum
Class template \tcode{prop} is for building a queryable object
from a query object and a value.

\pnum
\mandates
\tcode{\exposconcept{callable}<QueryTag, \exposid{prop-like}<ValueType>>}
is modeled,
where \exposid{prop-like} is the following exposition-only class template:
\begin{codeblock}
template<class ValueType>
struct @\exposid{prop-like}@ {              // \expos
  const ValueType& query(auto) const noexcept;
};
\end{codeblock}

\pnum
\begin{example}
\begin{codeblock}
template<@\libconcept{sender}@ Sndr>
sender auto parameterize_work(Sndr sndr) {
  // Make an environment such that \tcode{get_allocator(env)} returns a reference to a copy of \tcode{my_alloc\{\}}.
  auto e = prop(get_allocator, my_alloc{});

  // Parameterize the input sender so that it will use our custom execution environment.
  return write_env(sndr, e);
}
\end{codeblock}
\end{example}

\pnum
Specializations of \tcode{prop} are not assignable.

\rSec2[exec.env]{Class template \tcode{env}}

\begin{codeblock}
namespace std::execution {
  template<@\exposconcept{queryable}@... Envs>
  struct @\libglobal{env}@ {
    Envs@$_0$@ @$\exposid{envs}_0$@;               // \expos
    Envs@$_1$@ @$\exposid{envs}_1$@;               // \expos
      @\vdots@
    Envs@$_{n-1}$@ @$\exposid{envs}_{n-1}$@;           // \expos

    template<class QueryTag>
      constexpr decltype(auto) query(QueryTag q) const noexcept(@\seebelow@);
  };

  template<class... Envs>
    env(Envs...) -> env<unwrap_reference_t<Envs>...>;
}
\end{codeblock}

\pnum
The class template \tcode{env} is used to construct a queryable object
from several queryable objects.
Query invocations on the resulting object are resolved
by attempting to query each subobject in lexical order.

\pnum
Specializations of \tcode{env} are not assignable.

\pnum
It is unspecified
whether \tcode{env} supports initialization
using a parenthesized \grammarterm{expression-list}\iref{dcl.init},
unless the \grammarterm{expression-list} consist of
a single element of type (possibly const) \tcode{env}.

\pnum
\begin{example}
\begin{codeblock}
template<@\libconcept{sender}@ Sndr>
sender auto parameterize_work(Sndr sndr) {
  // Make an environment such that:
  //   \tcode{get_allocator(env)} returns a reference to a copy of \tcode{my_alloc\{\}}
  //   \tcode{get_scheduler(env)} returns a reference to a copy of \tcode{my_sched\{\}}
  auto e = env{prop(get_allocator, my_alloc{}),
               prop(get_scheduler, my_sched{})};

  // Parameterize the input sender so that it will use our custom execution environment.
  return write_env(sndr, e);
}
\end{codeblock}
\end{example}

\indexlibrarymember{query}{env}%
\begin{itemdecl}
template<class QueryTag>
constexpr decltype(auto) query(QueryTag q) const noexcept(@\seebelow@);
\end{itemdecl}

\begin{itemdescr}
\pnum
Let \exposconcept{has-query} be the following exposition-only concept:
\begin{codeblock}
template<class Env, class QueryTag>
  concept @\defexposconcept{has-query}@ =                   // \expos
    requires (const Env& env) {
      env.query(QueryTag());
    };
\end{codeblock}

\pnum
Let \exposid{fe} be the first element of
$\exposid{envs}_0$, $\exposid{envs}_1$, $\dotsc$, $\exposid{envs}_{n-1}$
such that the expression \tcode{\exposid{fe}.query(q)} is well-formed.

\pnum
\constraints
\tcode{(\exposconcept{has-query}<Envs, QueryTag> || ...)} is \tcode{true}.

\pnum
\effects
Equivalent to: \tcode{return \exposid{fe}.query(q);}

\pnum
\remarks
The expression in the \tcode{noexcept} clause is equivalent
to \tcode{noexcept(\exposid{fe}.query(q))}.
\end{itemdescr}

\rSec1[exec.ctx]{Execution contexts}

\rSec2[exec.run.loop]{\tcode{execution::run_loop}}

\rSec3[exec.run.loop.general]{General}

\pnum
A \tcode{run_loop} is an execution resource on which work can be scheduled.
It maintains a thread-safe first-in-first-out queue of work.
Its \tcode{run} member function removes elements from the queue and
executes them in a loop on the thread of execution that calls \tcode{run}.

\pnum
A \tcode{run_loop} instance has an associated \defn{count}
that corresponds to the number of work items that are in its queue.
Additionally, a \tcode{run_loop} instance has an associated state
that can be one of
\defn{starting}, \defn{running}, \defn{finishing}, or \defn{finished}.

\pnum
Concurrent invocations of the member functions of \tcode{run_loop}
other than \tcode{run} and its destructor do not introduce data races.
The member functions
\exposid{pop-front}, \exposid{push-back}, and \tcode{finish}
execute atomically.

\pnum
\recommended
Implementations should use an intrusive queue of operation states
to hold the work units to make scheduling allocation-free.

\begin{codeblock}
namespace std::execution {
  class @\libglobal{run_loop}@ {
    // \ref{exec.run.loop.types}, associated types
    class @\exposid{run-loop-scheduler}@;                                   // \expos
    class @\exposid{run-loop-sender}@;                                      // \expos
    struct @\exposid{run-loop-opstate-base}@ {                              // \expos
      virtual void @\exposid{execute}@() = 0;                               // \expos
      run_loop* @\exposid{loop}@;                                           // \expos
      run-loop-opstate-base* @\exposid{next}@;                              // \expos
    };
    template<class Rcvr>
      using @\exposid{run-loop-opstate}@ = @\unspec@;                     // \expos

    // \ref{exec.run.loop.members}, member functions
    @\exposid{run-loop-opstate-base}@* @\exposid{pop-front}@();                         // \expos
    void @\exposid{push-back}@(@\exposid{run-loop-opstate-base}@*);                     // \expos

  public:
    // \ref{exec.run.loop.ctor}, constructor and destructor
    run_loop() noexcept;
    run_loop(run_loop&&) = delete;
    ~run_loop();

    // \ref{exec.run.loop.members}, member functions
    @\exposid{run-loop-scheduler}@ get_scheduler();
    void run();
    void finish();
  };
}
\end{codeblock}

\rSec3[exec.run.loop.types]{Associated types}

\begin{itemdecl}
class @\exposid{run-loop-scheduler}@;
\end{itemdecl}

\pnum
\exposid{run-loop-scheduler} is an unspecified type
that models \libconcept{scheduler}.

\pnum
Instances of \exposid{run-loop-scheduler} remain valid
until the end of the lifetime of the \tcode{run_loop} instance
from which they were obtained.

\pnum
Two instances of \exposid{run-loop-scheduler} compare equal
if and only if they were obtained from the same \tcode{run_loop} instance.

\pnum
Let \exposid{sch} be an expression of type \exposid{run-loop-scheduler}.
The expression \tcode{schedule(\exposid{sch})}
has type \exposid{run-loop-\newline sender} and
is not potentially-throwing if \exposid{sch} is not potentially-throwing.

\begin{itemdecl}
class @\exposid{run-loop-sender}@;
\end{itemdecl}

\pnum
\exposid{run-loop-sender} is an exposition-only type
that satisfies \libconcept{sender}.
For any type \tcode{Env},
\tcode{completion_signatures_of_t<\exposid{run-loop-sender}, Env>} is
\begin{codeblock}
completion_signatures<set_value_t(), set_error_t(exception_ptr), set_stopped_t()>
\end{codeblock}

\pnum
An instance of \exposid{run-loop-sender} remains valid
until the end of the lifetime of its associated \tcode{run_loop} instance.

\pnum
Let \exposid{sndr} be an expression of type \exposid{run-loop-sender},
let \exposid{rcvr} be an expression
such that \tcode{\libconcept{receiver_of}<decltype((\exposid{rcvr})), CS>} is \tcode{true}
where \tcode{CS} is the \tcode{completion_signatures} specialization above.
Let \tcode{C} be either \tcode{set_value_t} or \tcode{set_stopped_t}.
Then:
\begin{itemize}
\item
The expression \tcode{connect(\exposid{sndr}, \exposid{rcvr})}
has type \tcode{\exposid{run-loop-opstate}<decay_t<decltype((\exposid{rcvr}))>>}
and is potentially-throwing if and only if
\tcode{(void(\exposid{sndr}), auto(\exposid{rcvr}))} is potentially-throwing.
\item
The expression \tcode{get_completion_scheduler<C>(get_env(\exposid{sndr}))}
is potentially-throwing if and only if \exposid{sndr} is potentially-throwing,
has type \exposid{run-loop-scheduler}, and
compares equal to the \exposid{run-loop-\newline scheduler} instance
from which \exposid{sndr} was obtained.
\end{itemize}

\begin{itemdecl}
template<class Rcvr>
  struct @\exposid{run-loop-opstate}@;
\end{itemdecl}

\pnum
\tcode{\exposid{run-loop-opstate}<Rcvr>}
inherits privately and unambiguously from \exposid{run-loop-opstate-base}.

\pnum
Let $o$ be a non-const lvalue of type \tcode{\exposid{run-loop-opstate}<Rcvr>},
and let \tcode{REC($o$)} be a non-const lvalue reference to an instance of type \tcode{Rcvr}
that was initialized with the expression \exposid{rcvr}
passed to the invocation of connect that returned $o$.
Then:
\begin{itemize}
\item
The object to which \tcode{\exposid{REC}($o$)} refers
remains valid for the lifetime of the object to which $o$ refers.
\item
The type \tcode{\exposid{run-loop-opstate}<Rcvr>} overrides
\tcode{\exposid{run-loop-opstate-base}::\exposid{execute}()}
such that \tcode{$o$.\exposid{exe\-cute}()} is equivalent to:
\begin{codeblock}
if (get_stop_token(@\exposid{REC}@(@$o$@)).stop_requested()) {
  set_stopped(std::move(@\exposid{REC}@(@$o$@)));
} else {
  set_value(std::move(@\exposid{REC}@(@$o$@)));
}
\end{codeblock}
\item
The expression \tcode{start($o$)} is equivalent to:
\begin{codeblock}
try {
  @$o$@.@\exposid{loop}@->@\exposid{push-back}@(addressof(@$o$@));
} catch(...) {
  set_error(std::move(@\exposid{REC}@(@$o$@)), current_exception());
}
\end{codeblock}
\end{itemize}

\rSec3[exec.run.loop.ctor]{Constructor and destructor}

\indexlibraryctor{run_loop}%
\begin{itemdecl}
run_loop() noexcept;
\end{itemdecl}

\begin{itemdescr}
\pnum
\ensures
\exposid{count} is \tcode{0} and \exposid{state} is \exposid{starting}.
\end{itemdescr}

\indexlibrarydtor{run_loop}%
\begin{itemdecl}
~run_loop();
\end{itemdecl}

\begin{itemdescr}
\pnum
\effects
If \exposid{count} is not \tcode{0} or if \exposid{state} is \exposid{running},
invokes \tcode{terminate}\iref{except.terminate}.
Otherwise, has no effects.
\end{itemdescr}

\rSec3[exec.run.loop.members]{Member functions}

\begin{itemdecl}
@\exposid{run-loop-opstate-base}@* @\exposid{pop-front}@();
\end{itemdecl}

\begin{itemdescr}
\pnum
\effects
Blocks\iref{defns.block} until one of the following conditions is \tcode{true}:
\begin{itemize}
\item
\exposid{count} is \tcode{0} and \exposid{state} is \exposid{finishing},
in which case \exposid{pop-front} sets \exposid{state} to \exposid{finished}
and returns \tcode{nullptr}; or
\item
\exposid{count} is greater than \tcode{0},
in which case an item is removed from the front of the queue,
\exposid{count} is decremented by \tcode{1}, and
the removed item is returned.
\end{itemize}
\end{itemdescr}

\begin{itemdecl}
void @\exposid{push-back}@(@\exposid{run-loop-opstate-base}@* item);
\end{itemdecl}

\begin{itemdescr}
\pnum
\effects
Adds \tcode{item} to the back of the queue and
increments \exposid{count} by \tcode{1}.

\pnum
\sync
This operation synchronizes with
the \exposid{pop-front} operation that obtains \tcode{item}.
\end{itemdescr}

\indexlibrarymember{get_scheduler}{run_loop}%
\begin{itemdecl}
@\exposid{run-loop-scheduler}@ get_scheduler();
\end{itemdecl}

\begin{itemdescr}
\pnum
\returns
An instance of \exposid{run-loop-scheduler}
that can be used to schedule work onto this \tcode{run_loop} instance.
\end{itemdescr}

\indexlibrarymember{run}{run_loop}%
\begin{itemdecl}
void run();
\end{itemdecl}

\begin{itemdescr}
\pnum
\expects
\exposid{state} is either \exposid{starting} or \exposid{finishing}.

\pnum
\effects
If \exposid{state} is \exposid{starting},
sets the \exposid{state} to \exposid{running},
otherwise leaves \exposid{state} unchanged.
Then, equivalent to:
\begin{codeblock}
while (auto* op = @\exposid{pop-front}@()) {
  op->@\exposid{execute}@();
}
\end{codeblock}

\pnum
\remarks
When \exposid{state} changes, it does so without introducing data races.
\end{itemdescr}

\indexlibrarymember{finish}{run_loop}%
\begin{itemdecl}
void finish();
\end{itemdecl}

\begin{itemdescr}
\pnum
\expects
\exposid{state} is either \exposid{starting} or \exposid{running}.

\pnum
\effects
Changes \exposid{state} to \exposid{finishing}.

\pnum
\sync
\tcode{finish} synchronizes with the \exposid{pop-front} operation
that returns \tcode{nullptr}.
\end{itemdescr}

\rSec1[exec.coro.util]{Coroutine utilities}

\rSec2[exec.as.awaitable]{\tcode{execution::as_awaitable}}

\pnum
\tcode{as_awaitable} transforms an object into one
that is awaitable within a particular coroutine.
Subclause \ref{exec.coro.util} makes use of
the following exposition-only entities:
\begin{codeblock}
namespace std::execution {
  template<class Sndr, class Promise>
    concept @\defexposconcept{awaitable-sender}@ =
      @\exposconcept{single-sender}@<Sndr, env_of_t<Promise>> &&
      @\libconcept{sender_to}@<Sndr, @\exposid{awaitable-receiver}@> &&    // \seebelow
      requires (Promise& p) {
        { p.unhandled_stopped() } -> @\libconcept{convertible_to}@<coroutine_handle<>>;
      };

  template<class Sndr, class Promise>
    class @\exposidnc{sender-awaitable}@;                                     // \expos
}
\end{codeblock}

\pnum
The type \tcode{\exposid{sender-awaitable}<Sndr, Promise>} is equivalent to:

\begin{codeblock}
namespace std::execution {
  template<class Sndr, class Promise>
  class @\exposidnc{sender-awaitable}@ {
    struct @\exposidnc{unit}@ {};                                             // \expos
    using @\exposidnc{value-type}@ =                                          // \expos
      @\exposidnc{single-sender-value-type}@<Sndr, env_of_t<Promise>>;
    using @\exposidnc{result-type }@=                                         // \expos
      conditional_t<is_void_v<@\exposid{value-type}@>, unit, @\exposid{value-type}@>;
    struct @\exposidnc{awaitable-receiver}@;                                  // \expos

    variant<monostate, @\exposidnc{result-type}@, exception_ptr> @\exposidnc{result}@{};    // \expos
    connect_result_t<Sndr, @\exposidnc{awaitable-receiver}@> @\exposidnc{state}@;           // \expos

  public:
    @\exposid{sender-awaitable}@(Sndr&& sndr, Promise& p);
    static constexpr bool await_ready() noexcept { return false; }
    void await_suspend(coroutine_handle<Promise>) noexcept { start(@\exposid{state}@); }
    @\exposid{value-type}@ await_resume();
  };
}
\end{codeblock}

\pnum
\exposid{awaitable-receiver} is equivalent to:
\begin{codeblock}
struct @\exposid{awaitable-receiver}@ {
  using receiver_concept = receiver_t;
  variant<monostate, @\exposidnc{result-type}@, exception_ptr>* @\exposidnc{result-ptr}@;   // \expos
  coroutine_handle<Promise> @\exposidnc{continuation}@;                       // \expos
  // \seebelow
};
\end{codeblock}

\pnum
Let \tcode{rcvr} be an rvalue expression of type \exposid{awaitable-receiver},
let \tcode{crcvr} be a const lvalue that refers to \tcode{rcvr},
let \tcode{vs} be a pack of subexpressions, and
let \tcode{err} be an expression of type \tcode{Err}. Then:
\begin{itemize}
\item
If \tcode{\libconcept{constructible_from}<\exposid{result-type}, decltype((vs))...>}
is satisfied,
the expression \tcode{set_value(\newline rcvr, vs...)} is equivalent to:
\begin{codeblock}
try {
  rcvr.@\exposid{result-ptr}@->template emplace<1>(vs...);
} catch(...) {
  rcvr.@\exposid{result-ptr}@->template emplace<2>(current_exception());
}
rcvr.@\exposid{continuation}@.resume();
\end{codeblock}
Otherwise, \tcode{set_value(rcvr, vs...)} is ill-formed.
\item
The expression \tcode{set_error(rcvr, err)} is equivalent to:
\begin{codeblock}
rcvr.@\exposid{result-ptr}@->template emplace<2>(@\exposid{AS-EXCEPT-PTR}@(err));    // see \ref{exec.general}
rcvr.@\exposid{continuation}@.resume();
\end{codeblock}
\item
The expression \tcode{set_stopped(rcvr)} is equivalent to:
\begin{codeblock}
static_cast<coroutine_handle<>>(rcvr.@\exposid{continuation}@.promise().unhandled_stopped()).resume();
\end{codeblock}
\item
For any expression \tcode{tag}
whose type satisfies \exposconcept{forwarding-query} and
for any pack of subexpressions \tcode{as},
\tcode{get_env(crcvr).query(tag, as...)} is expression-equivalent to:
\begin{codeblock}
tag(get_env(as_const(crcvr.@\exposid{continuation}@.promise())), as...)
\end{codeblock}
\end{itemize}

\begin{itemdecl}
@\exposid{sender-awaitable}@(Sndr&& sndr, Promise& p);
\end{itemdecl}

\begin{itemdescr}
\pnum
\effects
Initializes \exposid{state} with
\begin{codeblock}
connect(std::forward<Sndr>(sndr),
        @\exposid{awaitable-receiver}@{addressof(result), coroutine_handle<Promise>::from_promise(p)})
\end{codeblock}
\end{itemdescr}

\begin{itemdecl}
@\exposid{value-type}@ await_resume();
\end{itemdecl}

\begin{itemdescr}
\pnum
\effects
Equivalent to:
\begin{codeblock}
if (@\exposid{result}@.index() == 2)
  rethrow_exception(get<2>(@\exposid{result}@));
if constexpr (!is_void_v<@\exposid{value-type}@>)
  return std::forward<@\exposid{value-type}@>(get<1>(@\exposid{result}@));
\end{codeblock}
\end{itemdescr}

\pnum
\tcode{as_awaitable} is a customization point object.
For subexpressions \tcode{expr} and \tcode{p}
where \tcode{p} is an lvalue,
\tcode{Expr} names the type \tcode{decltype((expr))} and
\tcode{Promise} names the type \tcode{decay_t<decltype((p))>},
\tcode{as_awaitable(expr, p)} is expression-equivalent to,
except that the evaluations of \tcode{expr} and \tcode{p}
are indeterminately sequenced:
\begin{itemize}
\item
\tcode{expr.as_awaitable(p)} if that expression is well-formed.

\mandates
\tcode{\exposconcept{is-awaitable}<A, Promise>} is \tcode{true},
where \tcode{A} is the type of the expression above.
\item
Otherwise, \tcode{(void(p), expr)}
if \tcode{\exposconcept{is-awaitable}<Expr, U>} is \tcode{true},
where \tcode{U} is an unspecified class type
that is not \tcode{Promise} and
that lacks a member named \tcode{await_transform}.

\expects
\tcode{\exposconcept{is-awaitable}<Expr, Promise>} is \tcode{true} and
the expression \tcode{co_await expr}
in a coroutine with promise type \tcode{U} is expression-equivalent to
the same expression in a coroutine with promise type \tcode{Promise}.
\item
Otherwise, \tcode{\exposid{sender-awaitable}\{expr, p\}}
if \tcode{\exposconcept{awaitable-sender}<Expr, Promise>} is \tcode{true}.
\item
Otherwise, \tcode{(void(p), expr)}.
\end{itemize}

\rSec2[exec.with.awaitable.senders]{\tcode{execution::with_awaitable_senders}}

\pnum
\tcode{with_awaitable_senders},
when used as the base class of a coroutine promise type,
makes senders awaitable in that coroutine type.

In addition, it provides a default implementation of \tcode{unhandled_stopped}
such that if a sender completes by calling \tcode{set_stopped},
it is treated as if an uncatchable "stopped" exception were thrown
from the \grammarterm{await-expression}.
\begin{note}
The coroutine is never resumed, and
the \tcode{unhandled_stopped} of the coroutine caller's promise type is called.
\end{note}

\begin{codeblock}
namespace std::execution {
  template<@\exposconcept{class-type}@ Promise>
    struct @\libglobal{with_awaitable_senders}@ {
      template<class OtherPromise>
        requires (!@\libconcept{same_as}@<OtherPromise, void>)
      void set_continuation(coroutine_handle<OtherPromise> h) noexcept;

      coroutine_handle<> @\libmember{continuation}{with_awaitable_senders}@() const noexcept { return @\exposid{continuation}@; }

      coroutine_handle<> @\libmember{unhandled_stopped}{with_awaitable_senders}@() noexcept {
        return @\exposid{stopped-handler}@(@\exposid{continuation}@.address());
      }

      template<class Value>
      @\seebelow@ await_transform(Value&& value);

    private:
      [[noreturn]] static coroutine_handle<>
        @\exposid{default-unhandled-stopped}@(void*) noexcept {             // \expos
        terminate();
      }
      coroutine_handle<> @\exposid{continuation}@{};                        // \expos
      coroutine_handle<> (*@\exposid{stopped-handler}@)(void*) noexcept =   // \expos
        &@\exposid{default-unhandled-stopped}@;
    };
}
\end{codeblock}

\indexlibrarymember{set_continuation}{with_awaitable_senders}%
\begin{itemdecl}
template<class OtherPromise>
  requires (!@\libconcept{same_as}@<OtherPromise, void>)
void set_continuation(coroutine_handle<OtherPromise> h) noexcept;
\end{itemdecl}

\begin{itemdescr}
\pnum
\effects
Equivalent to:
\begin{codeblock}
@\exposid{continuation}@ = h;
if constexpr ( requires(OtherPromise& other) { other.unhandled_stopped(); } ) {
  @\exposid{stopped-handler}@ = [](void* p) noexcept -> coroutine_handle<> {
    return coroutine_handle<OtherPromise>::from_address(p)
      .promise().unhandled_stopped();
  };
} else {
  @\exposid{stopped-handler}@ = &@\exposid{default-unhandled-stopped}@;
}
\end{codeblock}
\end{itemdescr}

\indexlibrarymember{await_transform}{with_awaitable_senders}%
\begin{itemdecl}
template<class Value>
@\exposid{call-result-t}@<as_awaitable_t, Value, Promise&> await_transform(Value&& value);
\end{itemdecl}

\begin{itemdescr}
\pnum
\effects
Equivalent to:
\begin{codeblock}
return as_awaitable(std::forward<Value>(value), static_cast<Promise&>(*this));
\end{codeblock}
\end{itemdescr}
